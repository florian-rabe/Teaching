After lists, and trees, graphs are the most important data structure in computer science.
In fact, just like lists are a special of trees, trees are a special of graphs.

Data structures for lists and trees are of course used to represent lists and trees.
But they are also used to represent other data.
For example, we can represent a set as a list (Sect.~\ref{sec:ad:listset}) or as a tree (Sect.~\ref{sec:ad:redblacktree}) or a list as a tree (Sect.~\ref{sec:ad:heaplists}).
That is because choosing the more complex data structure (i.e., a tree instead of a list) can allow for more efficient algorithms.

Data structures for graphs on the other hand are almost exclusively used to represent graphs.
That is because they are rather difficult to work with.
But they are needed because graph-like data occurs very frequently.

\section{Specification}

\begin{definition}[Graph]
A \textbf{graph} consists is a pair $G=(N,E)$ such that $E$ is a binary relation on $N$.

If $E$ is symmetric, $G$ is called \textbf{undirected}, otherwise \textbf{directed}.
\end{definition}

The set $N$ is usually but not necessarily finite.

Like for trees, there are many definitions to talk about the parts of a graph:

\begin{definition}[Parts of a Graph]
Consider a graph $G=(N,E)$.

An element $n\in N$ is called a \textbf{node} or a \textbf{vertex}.
An element $(m,n)\in E$ is called an \textbf{edge} from $m$ to $n$.
It is also called an \textbf{incoming edge} of $n$ and an \textbf{outgoing edge} of $m$.

For every node $n$, the number of incoming edges is called the \textbf{in-degree} of $n$, and the number of outgoing edges is called the \textbf{out-degree} of $n$.
If $G$ is undirected graph, incoming and outgoing edges are the same, and we simply speak of the \textbf{degree} of $n$.

A \textbf{path} from $a_0$ to $a_n$ is a list $[a_0,\ldots,a_n]\in N^*$ such that there is an edge from $a_{i-1}$ to $a_i$ for $i=1\,ldots,n$.

$n$ is called the \textbf{length} of the path.
If $n=0$ (and thus $a_0=a_n$), the path is called \textbf{empty}.
If there is a path from $a_0$ to $a_n$, then $a_n$ is called \textbf{reachable} from $a_0$.

A \textbf{cycle} is a non-empty path from $a$ to itself.
If $G$ has (no) cycles, it is called \textbf{(a)cyclic}.

Let us write $\ov{G}$ for the undirected graph $(N, E\cup E^{-1})$ in which all edges go both ways.
Then $G$ is called \textbf{connected} if all nodes in $\ov{G}$ are reachable from each other.

A \textbf{clique} is a subset $C$ of $N$ such that there is an edge from every $a\in C$ to every other $b\in C$.
$G$ is called \textbf{complete} if $N$ is a clique.
\end{definition}


\paragraph{Visualization}
A good intuition to think of graph is to imagine the nodes as places and the edges as streets between them.
In a directed graph, all edges are one-way streets.

All concepts about graphs also have very intuitive visual aspects:

\begin{ctabular}{|l|l|l|}
\hline
& \multicolumn{2}{c|}{Visual Intuition}\\
Concept & undirected &  directed \\
\hline
node & \multicolumn{2}{c|}{point}  \\
edge from $a$ to $b$ & line from $a$ to $b$ & arrow from $a$ to $b$\\
incoming edge of $a$ & &  arrow pointing at $a$ \\
outgoing edge of $a$ & &  arrow pointing away from $a$ \\
$b$ reachable from $a$ & we can walk from $a$ to $b$ along edges & \ldots in arrow direction \\
path from $a$ to $b$ of length $n$ & a walk from $a$ to $b$ in $n$ steps & \ldots in arrow direction \\
weight\footnotemark of an edge & \multicolumn{2}{c|}{cost intuition: length of the line}\\ 
                               & \multicolumn{2}{c|}{capacity intuition: width of the line}\\ 
complete & we can walk everywhere in $1$ step &  \ldots in arrow direction \\
cycle & we can walk in a circle & \ldots in arrow direction \\
connected & \multicolumn{2}{c|}{graph can be drawn in one stroke} \\
\hline
\end{ctabular}
\footnotetext{See below for weighted graphs.}

\paragraph{Reachability Relation}
Many graph properties are just rephrasings of or closely related to relation properties.
Most importantly:

\begin{theorem}[Reachability]
For every graph $G$, the relation ``$b$ is reachable from $a$'' is
\begin{compactitem}
 \item reflexive and transitive
 \item symmetric iff $G$ is undirected
 \item anti-symmetric iff $G$ is acyclic
\end{compactitem}
\end{theorem}
\begin{proof}
Exercise.
\end{proof}

\paragraph{Labeled Graphs}
Like for trees, graphs are only useful for compuation, if we can store data in them.
Contrary to trees, we often need to store data in the nodes \emph{and} the edges.

\begin{definition}[Labeled Graph]
A $A$-$B$-\textbf{labeled} graph is a triple of
\begin{compactitem}
 \item a graph $G=(N,E)$
 \item a function $nodeLabel:N\to A$
 \item a function $edgeLabel:E\to B$
\end{compactitem}
$Graph[A,B]$ is the set of $A$-$B$-labeled graphs.
\end{definition}

The most important special case arises when the nodes are not labeled (i.e., we put $A=\Unit$) and the edges are labeled with numbers, i.e., $B=\Z$:

\begin{definition}
A \textbf{weighted} graph is a $\Unit$-$\N$-labeled graph.
The label of an edge from is called its \textbf{weight}.
\end{definition}

There are two important applications of weigthed graphs that use different interpreations of the weights:
\begin{itemize}
 \item Cost intuition: The weight of an edge is the cost of moving along the edge.
 For example, if the nodes represent cities and the edges flight routes, the weight can be the distance.
 \item Capacity intuition: The weight of an edge is the capacity for moving objects along the edge.
 For example, if the nodes represent cities and the edges flight routes, the weight can be the number of flights per day.
\end{itemize}

Correspondingly, we define:
\begin{definition}
Consider a weighted graph.
We write $weight(i,j)$ for the weight of the edge from $i$ to $j$.

We make $weight:N\times N\to \N^\infty$ a total function by using a default value whenever there is no edge from $i$ to $j$:
\begin{compactitem}
 \item A \textbf{cost-weighted} graph uses the default $weight(i,j)=\infty$.
 \item A \textbf{capacity-weighted} graph uses the default $weight(i,j)=0$.
\end{compactitem}

In a cost-weighted graph, the \textbf{cost of a path} is the sum of the weights of all edges.

In a capacity-weighted graph, the \textbf{capacity of a path} is the minimal weight of any edge in it.
\end{definition}

\section{Data Structures}

Graphs $G=(N,E)$ are among the trickiest data structures to design.
It is straightforward to represent $N$ as a set, e.g., the set $\Z_m$ if there are $m$ nodes.

But there are many options to represent $E$, all with different advantages:
For example,
\begin{compactitem}
 \item a function $N\times N\to Bool$\\
 This makes it easy to check whether an edge exists but very hard to enumerate all edges.
 \item a set $e$ with functions $from: e\to N$ and $out:e\to N$\\
 This makes it easy to enumerate all edges but hard to navigate in the graph.
 \item a function $outgoing:N\to Set[N]$ or $incoming:N\to Set[N]$\\
 The former makes it easy to navigate forwards (in arrow direction) but hard to navigate backwards.
 The latter has the dual problem.\\
 \item two functions $outgoing:N\to Set[N]$ and $incoming:N\to Set[N]$ \\
 This makes navigation easy in both directions. But the representation is redundant: Every time we add/remove an edge, we have to update both $outgoing$ and $incoming$.
\end{compactitem}

\subsection{Adjacency Matrix}

An often useful representation is via a matrix, called the \textbf{adjacency matrix} of $G$.

\begin{definition}[Adjacency Matrix]
Given a graph $G=(N,E)$ with $|N|=m$.
The adjacency matrix of $G$ is the matrix $A\in \Bool^{mm}$ where $A_{ij}==\true$ iff there is an edge from $i$ to $j$ in $G$.
\end{definition}

Adjacency matrices have the nice property that we can multiply them.
Here matrix multiplication is computed using conjunction and disjunction instead of multiplication and addition:
\begin{definition}
$A,B\in \Bool^{mm}$, we define $(A\cdot B)_{ik}:= \bigvee_{j=0,\ldots,m-1}A_{ij}\wedge B_{jk}$.
\end{definition}
This is useful because:
\begin{theorem}
If $A$ is the adjacency matrix of $G$, then $(A^n){ij}$ iff there is a path of length $n$ from $i$ to $j$ in $G$.
\end{theorem}

This is advantageous because it lets us compute all paths of a given length efficiently $A^n$ using square-and-multiply (Sect.~\ref{sec:ad:exp:sqmult}).

Moreover, in acyclic graph, there are only finitely many paths.
Thus, we eventually have $A^n=A^{n+1}=\ldots$, at which point we have computed all paths.

A drawback of the adjacency matrix is that its size $m^2$.
In particular, for a undirected graph, half the space is wasted because we always have $A_{ij}=A_{ji})$.

\subsection{Adjacency Lists}

For graphs with many nodes and few edges, it is better to store adjacency lists:

\begin{definition}[Adjacency List]
Given a graph $G=(N,E)$ with $|N|=m$.
The adjacency list of a node $i$ is the sorted list $A_i$ of all $j$ such that there is an edge from $i$ to $j$ in $G$.

The adjacency list--representation of $G$ consists of an list $[(0,A_0),\ldots,(m-1,A_{m-1})]$ pairing every node with its adjacency list.
\end{definition}

The size of the adjacency list--representation is $|N|+|E|$, which is usually much smaller than $|N|^2$.

%\subsection{Adjacency Matrix for Weighted Graphs}
%
%For a weighted graph, the adjacency matrix $W\in (\N^\infty)^{mm}$ of a weighted graph is the matrix such that $W_{ij}=weight(i,j)$.
%
%By picking appropriate operations for matrix multiplication, we can use $W^n$ to compute the capacity between two nodes.
%We use minimum instead of addition, and addition instead of multiplication.
%% semi-ring on \N

\section{Algorithms}

%\subsection{Search}
%
%\subsection{Minimal Spanning Tree}
%
%\subsection{Shortest Path}
%
%\subsection{Maximal Flow}

