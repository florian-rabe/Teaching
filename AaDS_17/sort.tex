\section{Specification}\label{sec:ad:listsort:spec}

Lists are the most important non-primitive data structure in computer science.

\subsection{Lists}\label{sec:ad:list:spec}

For a set $A$, the set $A^*$ contains all lists $[a_0,\ldots,a_{l-1}]$ with elements $a_i\in A$ for some $l\in\N$.
$l$ is called the length of the list.

Because $A^*$ is a set for an arbitrary set $A$, data structures for lists must be polymorphic with a type parameter $A$.

\paragraph{Immutable Lists}
The following table specifies the most important functions involving lists:

\begin{ctabular}{|l|l|l|}
\hline
function & returns & abbreviation\\
\hline
$nil[A]\in A^*$ & $[]$ & \\
$range(m\in\N,n\in\N)\in\N^*$ & $[m,\ldots,n-1]$ or $[]$ if $m\geq n$ & \\
\hline
\multicolumn{3}{|c|}{below, let $l\in A^*$ be of the form $[a_0,\ldots,a_{l-1}]$ and assume $n<l$} \\
$length[A](x\in A^*)\in \N$ & $l$ & \\
$get[A](x\in A^*, n\in\N)\in A^*$ & $a_n$ & $x_n$ or $x[n]$\\
$prepend[A](a\in A, x\in A^*)\in A^*$ & $[a,a_0,\ldots,a_{l-1}]$ &\\
$append[A](x\in A^*, a\in A)\in A^*$ & $[a_0,\ldots,a_{l-1},a]$ &\\
$revert[A](x\in A^*)\in A^*$ & $[a_{l-1},\ldots,a_0]$ & \\
$concat[A](x\in A^*, y\in A^*)\in A^*$ & $[a_0,\ldots,a_{l-1},b_0,\ldots,b_{k-1}]$ if $y=[b_0,\ldots,b_{k-1}]$ &  $x+y$\\
$map[A,B](x\in A^*, f\in A\to B)\in B^*$ & $[f(a_0),\ldots,f(a_{l-1})]$ & $l\;map\;f$\\
$fold[A,B](x\in A^*, b\in B, f\in A\times B\to B)\in B$ & $f(a_1,f(a_2,\ldots,f(a_n,b))\ldots)$ & \\ 
\hline
$delete[A](x\in A^*, n\in\N)\in A^*$ & $[a_0,\ldots,a_{n-1},a_{n+1},\ldots,a_{l-1}]$ & \\
$insert[A](x\in A^*, a\in A, n\in\N)\in A^*$ & $[a_0,\ldots,a_{n-1},a,a_n,a_{n+1},\ldots,a_{l-1}]$ & \\
$update[A](x\in A^*, a\in A, n\in\N)\in A^*$ & $[a_0,\ldots,a_{n-1},a,a_{n+1},\ldots,a_{l-1}]$ & \\ % $insert(delete(l,n),a,n)
\hline
\end{ctabular}

Most of them are polymorphic.
$map$ and $fold$ even take a second type parameter for the return type of the function.

These are split into three groups:
\begin{compactitem}
\item The first group contains functions to create new lists. These are important to have any lists.
\item The second group contains functions that take a list $l\in A^*$ as their first argument and return data about $l$ or use $l$ to build new data.
\item The third group also takes a list $l\in A^*$ but also returns an element of $A^*$.
 This distinction is irrelevant in mathematics but critical in computer science: These functions may be implemented using in-place-updates.
 With in-place update, the list $l$ is changed to become the intended result. The original value of $l$ is lost in the process.
 If this is the case, we speak of \textbf{mutable} lists.
\end{compactitem}

\paragraph{Mutable Lists}
The following table specifies the most important functions on mutable lists that differ from immutable lists.
Instead of returning a new list, they have the effect of assigning a new value to the first argument.

\begin{ctabular}{|l|l|l|l|}
\hline
function & returns & effect & abbreviation\\
\hline
\multicolumn{4}{|c|}{below, let $l\in A^*$ be of the form $[a_0,\ldots,a_{l-1}]$ and assume $n<l$} \\
$delete[A](x\in A^*, n\in\N)$ & nothing & $x:=[a_0,\ldots,a_{n-1},a_{n+1},\ldots,a_{l-1}]$ & \\
$insert[A](x\in A^*, a\in A, n\in\N)$ & nothing & $x:=[a_0,\ldots,a_{n-1},a,a_n,a_{n+1},\ldots,a_{l-1}]$ & \\
$update[A](x\in A^*, a\in A, n\in\N)$ & nothing & $x:=[a_0,\ldots,a_{n-1},a,a_{n+1},\ldots,a_{l-1}]$ & $x_n := a$ or $x[n]:= a$\\ % $insert(delete(l,n),a,n)
\hline
\end{ctabular}

The other functions such as $length$ and $get$ are not affected.


\subsection{Sorting}\label{sec:ad:sort:spec}

Sorting a list is intuitively straightforward.
We need a function that takes a list and returns a list with the same elements in a different order, namely such that all elements occur according to their size.

\begin{example}
Consider $x=[4,6,5,3,5,0]\in\N^*$.
Then $sort(x)$ must yield $[0,3,4,5,5,6]$.

Here we made the implicit assumption that we want to sort with respect to the $\leq$-order on $\N$.
We could also use the $\geq$-order.
Then $sort(x)$ should return $[6,5,5,4,3,0]$.

Thus, sorting always depends on the chosen order.
\end{example}

\begin{definition}[Sorting]\label{def:ad:sort:spec}
Fix a set $A$ and a total order $\leq$ on $A$.

A list $x=[a_0,\ldots,a_l]\in A^*$ is called $\leq$-\textbf{sorted} if $a_0\leq a_1 \leq \ldots \leq a_{l-1}\leq a_l$.

Let $count(x\in A^*,a\in A)\in\N$ be the number of times that $a$ occurs in $x$.
Two list $x,y\in A^*$ are a \textbf{permutation} of each other if $count(x,a)=count(y,a)$ for all $a\in A$.

$sort:A^*\to A^*$ is called a $\leq$-\textbf{sorting} function if for all $x\in A^*$, the list $sort(x)$ is a $\leq$-sorted permutation of $x$.
\end{definition}

As usual we check that the specification indeed defines a function:

\begin{theorem}[Uniqueness]
The function $sort$ from Def.~\ref{def:ad:sort:spec} exists uniquely.
\end{theorem}
\begin{proof}
Because $\leq$ is assumed to be total, every list $x$ has a unique least element, which must occur first in $sort(x)$.
By induction on the length of $x$, we show that all elements of $sort(x)$ are determined.
\end{proof}

For immutable lists, the above definition is all the specification we need.
For mutable lists, we specify an alternative sorting function that does not create a new list:

\begin{definition}[In-place Sorting]
An effectful function $sort$ that takes an argument $x\in A^*$ and has the side-effect of modifying the value $v$ of $x$ to $v'$ is called an \textbf{in-place} $\leq$-\textbf{sorting} function if $v'=s(v)$ for a $\leq$-sorting function $s$.
\end{definition}

\subsection{Sorting by a Property}\label{sec:ad:sort:stable}

Often we do not have a total order on $A$, and we want to sort according to a certain property.
The property must be given by a function $p:A\to P$ such that we have a total order $\leq$ on $P$.

For example, we may want to sort a list of students by age.
Then $A=Student$, $P=\N$, and $p:(s\in Student)\mapsto age(s)$.

However, there may be ties: A list may contain multiple different elements that agree in the value of $p$.
To break, we require that the order in the original list should be preserved.
Formally:

\begin{definition}[Sorting by Property]\label{def:ad:sort:stable}
Fix sets $A$ and $P$, a function $p:A\to P$, and a total order $\leq$ on $P$.

Given a list $x\in A^*$, we define a total order $\leq^p$ on the elements of $x$ as follows:
 \[x_i \leq^p x_j \tb\miff\tb p(x_i) < p(x_j) \tb \mor \tb p(x_i)=p(x_j) \mand i\leq j\]

$sort:A^*\to A^*$ is called a \textbf{stable sorting} function for $p$ and $\leq$ if it is a sort function for $\leq^p$.
\end{definition}

Note that normal sorting becomes a special case of sorting by property using $P=A$ and $p(a)=a$.

\subsection{Why Do We Care About Sorting?}

Thus, a good, modern programmer might respond as follows:
\begin{compactenum}
\item How do you implement sorting a list? --- I call the $\mathit{sort}$ function of my programming language's basic library.
\item OK, but what if there is no $\mathit{sort}$ function? --- I import a library that provides it.
\item OK, but what if there is no such library? --- I use a different programming language.
\item OK, but what if circumstances beyond your control prevent you from using third-party libraries? --- I copy-paste a definition from the internet.\footnote{Nowadays an internet search for elementary problems almost always finds a solution for every programming language, usually on \url{http://www.stackexchange.org}.}
\end{compactenum}

Thus, for most people the only realistic situations in which to implement sorting algorithms is in exams, job interviews, or similar situations.
Then the question is never actually about sorting---it just uses sorting as an example to see whether the programmer understands how to design algorithms, analyze their complexity, and verify their correctness.

In any case, sorting is an extremely good subject for an introductory computer science class because it
\begin{compactitem}
 \item is an elementary problem that is easy to understand for students,
 \item is complex enough to exhibit many important general principles in interesting ways,
 \item is simple enough for all analysis to be doable manually,
 \item has multiple solutions, none of which is better than all the others,
 \item is extremely well-studied,
 \item is widely taught so that the internet is full of good visualizations that help learners.
\end{compactitem}


%%%%%%%%%%%%%%%%%%%%%%%%%%%%%%%%%%%%%%%%%%%%%%%%%%%%%%%%%%%%%%%%%%%%%%%%
\section{Design: Data Structures for Lists}\label{sec:ad:list:ds}


Besides natural numbers, the most important examples of a data structure are lists.
There are many different data structures for lists that differ subtly in how simply and/or efficiently the various functions can be implemented.
We will write $List[A]$ whenever we mean an arbitrary data structure for lists.

\subsection{Immutable Lists}

For immutable lists, functions like $delete$, $insert$, and $update$ (see Sect.~\ref{sec:math:sets:derivfun}) always return new lists.
That requires copying (parts of) the old list, which takes more time and memory.

Without further qualification, this is usually what $List[A]$ refers to.

\subsubsection{Functional Style: Lists as an Inductive Type}

Functional languages usually implement lists an in inductive data type:
\begin{acode}
\adata{{IndList[A]}}{nil,{cons(head: A,\, tail:IndList[A])}}
\end{acode}
Now the list $[1,2,3]$ is built as $cons(1,cons(2,cons(3,nil)))$.

Then functions on lists are implemented using recursion and pattern-matching.
For example:
\begin{acode}
\afun[{IndList[B]}]{map}{x:IndList[A],f:A\to B}{\amatch{x}{\acase{nil}{nil},\acase{cons(h,t)}{cons(f(h),map(t,f))}}}
\end{acode}

\subsubsection{Object-Oriented Style: Linked Lists}

Every inductive data type can also be systematically realized in an object-oriented language.
The correspondence is as follows:

\begin{ctabular}{|l|l|l|}
\hline
inductive type & class & example: lists\\
\hline
name of the type & abstract class & $IndList$ \\
parameters of the type & parameters of the class & $A$ \\
constructor & concrete subclass & e.g., $cons$\\
constructor arguments & constructor arguments & $head:A,tail:IndList[A]$ \\
\hline
\end{ctabular}

A basic realization looks as follows:
\begin{acode}
\aclassA{{IndList[A]}}{}{}{}\\
\aclass{{nil[A]}}{}{IndList[A]()}{}\\
\aclass{{cons[A]}}{head:A,tail:List[A]}{IndList[A]()}{}
\end{acode}
Now the list $[1,2,3]$ is built as $\anew{cons}{1, \anew{cons}{2, \anew{cons}{3, \anew{nil}{}}}}$.

Instead of pattern-matching, we have to use instance-checking to split cases.
For example:
\begin{acode}
\afun[{IndList[B]}]{map}{x:IndList[A],f:A\to B}{
  \aifelse{\aisinst{x}{nil}}
    {\anew{nil}{}}
    {xc := \aasinst{x}{cons} \\
     \anew{cons}{f(xc.head), map(x.tail,f)}
    }
}
\end{acode}

Moreover, we have to override equality so that, e.g., two instances of $cons$ are equal iff they used equal constructor arguments.

\subsubsection{Complexity}

Complexity of lists is measured in the lenght $n$ of the list.

Most operations on lists are linear because the algorithm must traverse the whole list.
For example, the straightforward implementation of $length$ takes $\Theta(n)$.

Similarly, $get(x,i)$ takes $i$ steps to find the element. This is $n$ in the worst case and $n/2$ on average.
So it also takes $\Theta(n)$.

In general, immutable lists require copying the list, whenever we insert, delete, or update elements.
These algorithms must traverse the list.
Therefore, they usually take $\Theta(n)$ time for the traversal \emph{and} $\Theta(n)$ memory for the result list.

In the case of $map(x,f)$ and $fold(x,a,f)$, the complexity depends on the passed function $f$.
However, in the typical case where the run time of $f$ does not depend on the length of the list, we can assume it takes constant time $c$.
Thus, the overall run time is $\Theta(cn)=\Theta(n)$.

However, there is one important exception: $prepend$ takes $\Theta(1)$.
This is because we can implement $prepend(a,x)$ simply by calling $cons(a,x)$.
Correspondingly, removing the first element takes $\Theta(1)$.

\subsection{Mutable Lists}

Mutable lists allow assignments to the individual elements of the list.
This allows updating an element without copying the list, thus allowing for many operations with $\Theta(1)$ time or memory complexity.

Because we can update the list in place, it becomes critical for efficiency how exactly the list is stored in memory.
Three cases are of great importance, all with advantages and disadvantages:

\begin{ctabular}{|l|l|p{6cm}|}
\hline
data structure & memory layout & remark \\
\hline
array & all in a row & easy to find elements but difficult to insert/delete \\
(singly-)linked list & every element points to next one & easy to insert/delete but traversal needed \\
doubly-linked list &  every element points to next and previous one & traversal in both directions possible, more overhead\\
growable array & linked list of arrays & compromise between the above \\
\hline
\end{ctabular}

\subsubsection{Arrays}

The data structure $Array[A]$ stores all elements in a row in memory.
Arrays must be a primitive feature of the programming language and are so in most languages.

For example, the list $x=[1,2,5]$ is stored in $3$ consecutive memory locations:
\begin{amemory}
\avar{x}{\N^*}{P}
\alocations
\aloc{P}{1}
\aloc{P+1}{3}
\aloc{P+2}{5}
\end{amemory}

That allows implementing $get$ and $update$ in $\Theta(1)$.
$get(x,n)$ is evaluated by retrieving the element in memory location $P+n$.
That takes one step to retrieve $x$, one step for the addition, and one step to retrieve the element at $P+n$.
$update(x,a,n)$ works accordingly.

Inserting and deleting elements still takes $\Theta(n)$.
For example, we can implement deleting by:
\begin{acode}
\afun{delete}{x:Array[A],n:\N}{
  \afor{i}{n}{length(x)-1}{
    x[n] := x[n+1]
  }
}
\end{acode}

Inserting an element into an array is difficult though: The memory location behind the array may not be available because it was already used for something else.
Therefore, arrays are often realized in such a way that the programmer chooses in advance the maximal length of the array.
Thus, technically this data structure does not realize the set $A^*$ but the set $A^n$ for some length $n$.
This may waste memory if $n$ is chosen too large.
But arrays are unbeatable in the common situation where we know that we will never call $insert$ anyway.

\subsubsection{Linked Lists}

Mutable linked list consist of a reference to the first element.
Each element consists of a value and a reference to its successor.
We can implement that using classes (or similar primitives like structs in C):
\begin{acode}
\aclass{{LinkedList[A]}}{head:Elem[A]}{}{}\\
\aclass{{Elem[A]}}{value:A, next:Elem[A]}{}{}
\end{acode}

Technically, $head$ and $next$ should have the type $Elem(A)^?$ to allow for empty lists and the end of the list, respectively.
However, object-oriented programmers usually use a trick where the built-in value $null$ is used:
\begin{compactitem}
 \item If $head$ is null, we have the empty list.
 \item If $next$ is null, we have the last element of the list.
\end{compactitem} those cases.

Now the list $[1,2,5]$ is built as $x:=\anew{LinkedList}{\anew{Elem}{1, \anew{Elem}{2, \anew{Elem}{5, null}}}}$.
It is stored in memory as
\begin{amemory}
\avar{x}{\N^*}{P}
\alocations
\aloc{P.head}{Q}
\hline
\aloc{Q.value}{1}
\aloc{Q.next}{R}
\hline
\aloc{R.value}{2}
\aloc{R.next}{S}
\hline
\aloc{S.value}{5}
\aloc{S.next}{null}
\end{amemory}

Deletion can now be realized in-place as follows
\begin{acode}
\afun{delete}{x:LinkedList[A],n:\N}{
 \aifelse{n==0}{x.head := x.head.next}{
  previous := x.head \\
  current := x.head.next\\
  \afor{i}{1}{n-1}{
    previous := current\\
    current := current.next\\
  }
  previous.next := current.next
 }
}
\end{acode}

Like immutable lists, linked lists take $\Theta(n)$ time for most operations.
However, they still perform better because changes can be done in-place.
Moreover, many operations can be done in $\Theta(1)$ memory whereas immutable lists often require $\Theta(n)$ memory.

An interesting exception is the following variant of $insert$:
Instead of taking the position $n$ at which to insert (which takes linear time to find), it take the element after which to insert:
\begin{acode}
\afun{insert}{x:LinkedList[A],after:Elem[A],a:A}{
 after.next := \anew{Elem}{a, after.next}
}
\end{acode}

A similar trick for deleting does not work so well: We can implement $delete(x:LinkedList[A], after:Elem[A])$ in $\Theta(1)$ if we know after which element to delete.
But a function $delete(x:LinkedList[A], e:Elem[A])$ where $e$ is to be deleted would be useful to delete---that still requires $\Theta(n)$ to find $e$ in the linked list.

\subsubsection{Doubly-Linked Lists}

Doubly-linked linked list are the same as linked lists except that each element also knows its predecessor ($null$ for the first element).
Moreover, the list knows its first and last element.

\begin{acode}
\aclass{{DoubleLinkedList[A]}}{head:Elem[A], last:Elem[A]}{}{}\\
\aclass{{Elem[A]}}{value:A, previous: Elem[A], next:Elem[A]}{}{}
\end{acode}

Now the list $x=[1,2,5]$ is stored in memory as
\begin{amemory}
\avar{x}{\N^*}{P}
\alocations
\aloc{P.head}{Q}
\aloc{P.last}{S}
\hline
\aloc{Q.value}{1}
\aloc{Q.previous}{null}
\aloc{Q.next}{R}
\hline
\aloc{R.value}{2}
\aloc{R.previous}{Q}
\aloc{R.next}{S}
\hline
\aloc{S.value}{5}
\aloc{S.previous}{R}
\aloc{S.next}{null}
\end{amemory}

Operations on doubly-linked lists are usually in the same complexity class as the corresponding ones for singly-linked lists.

A doubly-linked list has more memory overhead and thus copying and update operations have more time overhead.
But doubly-linked lists can be traversed efficiently in \emph{both} directions.
For example, processing the elements of a singly-linked list in reverse order requires two traversals: one to find the find element, one to process.
The same operation on a doubly-linked list requires only one traversal.
Both are $\Theta(n)$ though, but the latter may be twice as fast.

In a double-linked list, we can also define nice constant-time variants for both $insert$ and $delete$.
For example:
\begin{acode}
\afun{delete}{x:DoubleLinkedList[A],e:Elem[A]}{
 \aifelse{e.previous == null}{x.head := e.next}{e.previous.next := e.next} \\
 \aifelse{e.next == null}{x.last := e.previous}{e.next.previous := e.previous}
}
\end{acode}

The following table summarizes the complexity of some operations on arrays, linked lists and doubly-linked lists in terms of the length $l$: 
\begin{ctabular}{|l|l|l|l|l|l|l|l|l|} 
\hline
\multirow{2}{*}{}  & \multirow{2}{*}{$length[A]$} & $get[A]$ & $update[A]$ & $insert[A]$ & $delete[A]$ & \multirow{2}{*}{$prepend[A]$} & \multirow{2}{*}{$append[A]$} & \multirow{2}{*}{$reverse[A]$} \\ \cline{3-6}
&       & \multicolumn{4}{c|}{at position $n$}  &         &        &         \\ \hline
Array & $\Theta(1)$ & \multicolumn{2}{c|}{$\Theta(1)$} & \multicolumn{2}{c|}{$\Theta(l-n)$} & $\Theta(l)$ & $\Theta(1)$ & $\Theta(l)$ \\
Linked list & $\Theta(l)$ & \multicolumn{2}{c|}{$\Theta(n)$} & \multicolumn{2}{c|}{$\Theta(n)$} & $\Theta(1)$ & $\Theta(l)$ & $\Theta(l)$ \\
Doubly-linked List & $\Theta(l)$ & \multicolumn{2}{c|}{$\Theta(n)$} & \multicolumn{2}{c|}{$\Theta(n)$} & $\Theta(1)$ & $\Theta(1)$ & $\Theta(l)$  \\ \hline
\end{ctabular}


\subsubsection{Growable Arrays}

Growable arrays are a compromise between arrays and linked lists.
Initially, they behave like an array with a fixed length $l$.
However, when inserting an element that increases the length beyond $l$, we create a second array of length $l$ (elsewhere in memory) and connect the two.
Thus, a growable array is a linked list of fixed-length arrays.
The choice of $l$ is up to the data structure designer, who may allow the programmer to tweak it.

Retrieval and update technically are linear now.
To access the element in position $n$, we have to make $n/l$ retrievals to jump to the needed array.
Because $l$ is constant, that yields $\Theta(n)$ retrievals.
However, $l$ is usually large so that element access is only a little slower than for an array and much faster than for a linked list.

%%%%%%%%%%%%%%%%%%%%%%%%%%%%%%%%%%%%%%%%%%%%%%%%%%%%%%%%%%%%%%%%%%%%%%%%
\section{Design: Algorithms for Sorting}\label{sec:ad:sort:algo}

We assume a fixed set $A$ and a fixed comparison function $\leq:A\times A \to \B$.
For $x\in A^*$, we write $Sorted(x)$ if $x$ is $\leq$-sorted.

\paragraph{Auxiliary Functions}
Many in-place sorting algorithms have to swap two elements in a mutable list at some point.
Therefore, we define an auxiliary function

\begin{acode}
\afun{swap}{x:MutableList[A], i:\N, j:\N}{
  h := x[i]\\
  x[i] := x[j]\\
  x[j] := h
}
\end{acode}
Here $MutableList$ is any of the mutable data structures from above.

It is easy to see that this function indeed has the effect of swapping two elements in $x$.
For arrays, the time complexity of $swap$ is $\Theta(1)$.
For linked lists, it is $\Theta(n)$.


\subsection{Bubblesort}

Bubblesort is a stable in-place sorting algorithm that closely follows the natural way how a human would sort.
The idea is to find two elements that are not in order and swap them.
If no such elements exist, the list is sorted.

\begin{acode}
\afun{bubblesort}{x:Array[A]}{
 sorted := \false \\
 \awhile{!sorted}{
   sorted:=\true \\
   \afor{i}{0}{length(x)-2}{
     \aif{! x[i]\leq x[i+1]}{
       sorted := \false \\
       swap(x,i,i+1)
     }
   }
 }
}
\end{acode}

\paragraph{Correctness}
The for-loop compares all $length(x)-1$ pairs of neighboring elements.
It sets $sorted$ to $\false$ if the list is not sorted.
Thus, we obtain the loop invariant $F(x,sorted)=sorted==Sorted(x)$, which immediately yields partial correctness.

Total correctness follows from the termination ordering
 \[T(x,sorted)=\text{number of pairs $i,j$ such that $! x_i\leq x_j$} + \cas{1\mifc sorted==\false\\ 0\mifc sorted==\true}\]
Indeed, this number decreases in every iteration of the loop in which $x$ is not sorted.
The second summand is necessary to make $T(x,sorted)$ also decreases if $x$ is already sorted (which happens exactly one in the last iteration).

\paragraph{Complexity}
If $n$ is the length of $x$, each iteration of the while-loop has complexity $\Theta(n)$.
Moreover, the while-loop iterates at most $n$ times.
That happens in the worst-case: when $x$ is reversely sorted initially.
Thus, the complexity is $\Theta(n^2)$.

In the best-case, when $x$ is already sorted initially, the complexity is $\Theta(n)$.
That is already optimal because it requires $n-1$ comparisons to determine that a list is sorted.

\subsection{Insertionsort}

Insertion is also a stable in-place algorithm.

The idea is to sort increasingly large prefixes of a list $x$.
If $[x_0,\ldots,x_{i-1}]$ is sorted already, the element $x_i$ is inserted among them.

\begin{acode}
\afun{insertionsort}{x:Array[A]}{
  \afor{i}{1}{length(x)-1}{
    current := x[i] \\
    pos := i \\
    \awhile[shift elements to the right to make space for $current$]{pos > 0 \aand ! current \leq x[pos-1]}{
       x[pos] := x[pos-1] \\
       pos := pos - 1
    }\\
    x[pos] := current
  }
}
\end{acode}

\paragraph{Correctness}
We use a loop-invariant for the for-loop: $F(x,i)=Sorted([x_0,\ldots,x_{i-1}])$.
The preservation of the loop-invariant is non-obvious but straightforward to verify.
It holds initially because the empty list is trivially sorted.
That yields partial correctness.

Termination is easy to show using the termination ordering $T(x,i,current,pos)=pos$ for the while-loop.

\paragraph{Complexity}
If $n$ is the length of $x$, the for-loop runs $n$ times with $i=0,\ldots,n-1$
Inside, the while-loop runs $i$ times in the worst-case: if $x$ is reversely sorted, all $i$ elements before $current$ must be shifted to the right.
That sums up to $0+1+\ldots+n-1\in \Theta(n^2)$.

Everything else is $O(n)$.
Thus, the worst-case complexity is $\Theta(n^2)$.

In the best-case, if $x$ is already sorted, the while-loop never runs, and the complexity is $\Theta(n)$.

\subsection{Mergesort}

Mergesort is based on the observation that
\begin{compactitem}
  \item sorting smaller lists is much easier than sorting larger lists (because the number of pairs that have to be compared in $\Theta(n^2)$,
  \item merging two sorted lists is easy (linear time).
\end{compactitem}
Thus, we can divide a list into two halves, sort them recursively, then merge the results.
This is similar to the idea of square-and-multiply (Sect.~\ref{sec:ad:exp:sqmult}) and an example of the family of divide-and-conquer algorithms.

Because it needs auxiliary memory to do the merging of two half lists into one, it is easiest to implement as non-in-place algorithm.
Then the input data structure does not matter and can be assumed to be immutable.
The following is a straightforward realization:

\begin{acode}
\afun[{List[A]}]{mergesort}{x:List[A]}{
  n := length(x) \\
  \aifelse{n<2}{x}{
    k := n\divop 2\\
    l := mergesort([x_0,\ldots,x_{k-1}]) \\
    r := mergesort([x_k,\ldots,x_{n-1}]) \\
    \areturn{merge(l,r)}
  }
}\\
\\
\afun[{List[A]}]{merge}{x:List[A], y:List[A]}{
  xRest := x\\
  yRest := y\\
  res = [] \\
  \awhile{nonempty(xRest) \aor nonempty(yRest)}{
    takefromX := empty(yRest) \aor (nonempty(xRest) \aand xRest.head \leq yRest.head)\\
    \aifelse{takefromX}{
      res := cons(xRest.head, res) \\
      xRest := xRest.tail
    }{
      res := cons(yRest.head, res) \\
      yRest := yRest.tail
    }
  }\\
  \areturn{reverse(res)}
}
\end{acode}

\paragraph{Correctness}
Because the function $merge$ is not part of the specification, we have to first specify which property we want to prove about it.
The needed property for $z:=merge(x,y)$ is:
 \begin{compactitem}
   \item precondition: $Sorted(x)$ and $Sorted(y)$
   \item postcondition: $Sorted(z)$ and $z$ is a permutation of $x+y$
 \end{compactitem}

Now we can prove each function correct.
\medskip

First we consider $mergesort$.
Partial correctness means to prove $Sorted(mergesort(x))$.
That is very easy:
\begin{compactitem}
  \item If $n<2$, $x$ is trivially sorted.
  \item Otherwise:
   \begin{compactitem}
     \item $Sorted(a)$ and $Sorted(b)$ follow from the postcondition of the recursive call.
     \item Then the postcondition of $merge$ yields $Sorted(merge(a,b))$.
   \end{compactitem}
\end{compactitem}

Relative termination is immediate (assuming that $merge$ always terminates, which we prove below).
A termination ordering is given by $T(x)=length(x)$.
Indeed, $mergesort$ recurses only into strictly shorted lists.
\medskip

Second we consider $merge$.
We use a loop invariant $F(x,y,xRest,yRest,res)$ that states that
 \begin{compactitem}
  \item $Sorted(reverse(res))$ and $Sorted(xRest)$ and $Sorted(yRest)$
  \item All elements in $res$ are in $\leq$-relation to all elements in $xRest+yRest$.
  \item $res+xRest+yRest$ is a permutation of $x+y$
 \end{compactitem}
It is non-obvious but it is straightforward to see that this indeed a loop invariant:
 \begin{compactitem}
   \item $reverse(res)$ remains sorted because we always take the smallest element in $yRest+xRight$ and prepend it to $res$.
    In particular, because $xRest$ and $yRest$ are sorted, the smallest element must be $xRest.head$ or $yRest.head$.
   \item For the same reason, all elements of $res$ remain smaller than the ones of $xRest$ and $yRest$.
   \item Because we only remove elements from $xRest$ and $yRest$, they remain sorted.
   \item Because every element that is removed from $xRest$ or $yRest$ is immediately added to $res$, they remain a permutation.
 \end{compactitem}

To show partial correctness, we see that
\begin{compactitem}
  \item The loop invariant holds initially, which is obvious.
  \item After completing the loop, $xRest$ and $yRest$ are empty.
  \item Then, using the loop invariant, it is easy to show that $reverse(res)$ is sorted and a permutation of $x+y$.
\end{compactitem}

To show termination, we use $T(x,y,xRest,yRest,res)=length(xRest)+length(yRest)$.
It is easy to see that $T$ is a the termination ordering for the while-loop.

\paragraph{Complexity}
We have to analyze the complexity of both functions.

First we consider $merge$.
Let $n=length(x)+length(y)$.
\begin{compactitem}
 \item The three assignments in the beginning are $O(1)$.
 \item The while-loop is repeated once for every element of $x$ and $y$, which requires $\Theta(n)$ steps.
 The body of the loop takes $O(1)$. So $\Theta(n)$ in total.
 \item The last step requires reverting $res$, which has $n$ elements at this point.
 Reverting a list requires building a new list by traversing the old one. That is $\Theta(n)$ as well.
\end{compactitem}
Thus, the total complexity of $merge$ is $\Theta(n)=\Theta(length(x)+length(y))$.
\medskip

Second we consider $mergesort$.
Let $n=length(x)$.
We compute the time complexity $C(n)$:
\begin{compactitem}
 \item The assignments and the if-statement are in $O(1)$.
 \item The recursive calls to $mergesort$ take $C(n/2)$ each.
 \item The call to $merge$ takes $\Theta(length(a)+length(b))=\Theta(n)$.
\end{compactitem}
That yields
 \[C(n)=2\cdot C(n/2)+\Theta(n) = \ldots = 2^k\cdot C(n/2^k) + k\cdot \Theta(n)\]
 By choosing $k=\log_2 n$ and $C(1)=C(0)\in O(1)$, we obtain
 \[C(n)=n\cdot O(1)+\log_2 n\cdot \Theta(n)=\Theta(n\log_2 n)\]
\medskip

Thus, mergesort is quasilinear and thus strictly more efficient than bubblesort and insertionsort.

Contrary to bubblesort and insertionsort, mergesort takes the same amount of time no matter how sorted the input already is.
The recursion and the merging happen in essentially the same way independent of the input list.
Thus, its best-case complexity is also $\Theta(n\log_2 n)$.

\begin{remark}[Building the list reversely in $merge$]
$merge$ could be simplified by always adding the element $xLeft.head$ or $yLeft.head$ to the \emph{end} of $res$ instead of the beginning.
However, as discussed in Sect.~\ref{sec:ad:list:ds}, adding an element to the beginning of an immutable list takes constant time whereas adding to the end takes linear time.

Therefore, if we added elements to the end of $res$ would become quadratic instead of linear.
Then mergesort as a whole would also be quadratic.
\end{remark}

\subsection{Quicksort}

Quicksort is similar to mergesort in that two sublists are sorted recursively.
The main differences are:
\begin{compactitem}
 \item It does not divide the list $x$ in half.
  Instead if picks some element $a$ from the list (called the \emph{pivot}).
  Then it divides $x$ into sublists $a$ and $b$ containing the elements smaller and greater than $x$ respectively.\\
  No merging is necessary because all elements in $a$ are smaller than all elements in $b$.
  Thus the sorted list is $quicksort(a)+x+quicksort(b)$.
 \item To divide the list, quicksort has to traverse and reorder the list anyway.
 Therefore, it can easily be implemented in-place avoiding the use of auxiliary memory.
\end{compactitem}

When implemented as an in-place sorting algorithm, the recursive call takes two additional arguments: two numbers $first$ and $last$ that describe the sublist that should be sorted.

\begin{remark}[Additional Arguments in a Recursion]
Carrying along auxiliary information is very typical for recursive algorithms.
Therefore, we often find pairs of function:
 \begin{compactitem}
  \item A recursive function that takes additional arguments.\\
   That is $quicksortSublist$ below, which takes the entire list and the information about which sublist to sort.
  \item A non-recursive function that does nothing but call the other function with the initial arguments.\\
   That is $quicksort$ below, which calls $quicksortSublist$ on the entire list (e.g., on the sublist from $0$ to the end of $x$).
 \end{compactitem}
\end{remark}


\begin{acode}
\afun{quicksort}{x:Array[A]}{
  quicksortSublist(x,0,length(x)-1)
}\\
\\
\afun{quicksortSublist}{x:Array[A], first:\N, last: \N}{
  \aifelse{first \geq last}{
    \areturn{}
  }{
    pivot := x[last]\\
    pivotPos := first\\ %%    // place for swapping
    \aloopinv{x[k]\leq pivot \mfor k=first,\ldots,pivotPos-1 \mand pivot \leq x[k] \mfor k=pivotPos,\ldots,j-1}
    \afor{j}{first}{last - 1}{
      \aif{x[j] \leq pivot}{
         swap(x,pivotPos,j) \\
         pivotPos := pivotPos + 1
      }
    }\\
    swap(x,pivotPos,last)\\
    \\
    quicksortSubList(x, first, pivotPos - 1)\\
    quicksortSubList(x, pivotPos + 1, last)
  }
}
\end{acode}

\paragraph{Correctness}
Before proving correctness we have to specify the behavior of the auxiliary function $quicksortSublist$:
\begin{compactitem}
 \item precondition: none
 \item postcondition: $Sorted([x_{first},\ldots,x_{last}])$
\end{compactitem}
Then the correctness of $quicksort$ follows immediately from that of $quicksortSublist$.
\medskip

Now we prove the partial correctness of $quicksortSublist$.
First, the base case is trivially correct: It does nothing for lists of length $0$ or $1$.
For the recursive case, we prove that the following two properties holds just before the two recursive calls:
\begin{compactitem}
 \item The sublist $[x_{first},\ldots,x_{last}]$ is a permutation of its original value, and no other elements of $x$ have changed.
  That is easy to because we only change $x$ by calling $swap$ on positions between $first$ and $last$.
 \item All values $x_k$ are
  \begin{compactitem}
    \item smaller than $pivot$ for $k=first,\ldots,pivotPos-1$,
    \item equal to $pivot$ for $k=pivotPos$,
    \item greater than $pivot$ for $k=pivotPos+1,\ldots,last$.
  \end{compactitem}
  We prove that by using the indicated loop invariant for the for-loop.
  It is trivially true before the for-loop because $first=pivotPos$ and $pivotPos=j$.
  It is straightforward to check that it is preserved by the for-loop.
  Therefore, it holds after the for-loop for the value $j=last-1$.
  The last call to $swap$ moves the pivot element into $x_{pivotPos}$ so that the loop invariant is now also true for $j=last$.
  Then the needed properties can be seen easily.
\end{compactitem}
\medskip

To prove the termination of $quicksortSublist$, we use the termination ordering $T(x,first,last)=last-first+1$ (which is the length of the sublist).
That value always decreases because the pivot element is never part of the recursive call.

\paragraph{Complexity}
Let $n=last-first-1$ be the length of the sublist.
It is easy to see that, apart from the recursion, $quicksortSublist$ takes $\Theta(n)$ steps because the for-loop traverses the sublist.
Thus, the complexity of quicksort depends entirely on the lengths of the sublists in the recursive calls.
However, the pivot position and therefore those lengths are hard to predict.

The best-case complexity arises if the pivot always happens to be in the middle.
Then the same reasoning as for mergesort, yields best-case complexity $\Theta(n\log_2 n)$.
The worst-case arises if the list is already sorted: then the pivot position will always be the last one, and the two sublists have sizes $n-1$ and $0$.
That results in $n$ recursive calls on sublists of length $n$, $n-1$, \ldots, $1$ as well as $n$ calls on empty sublists.
Consequently, the worst-case complexity is $\Theta(n^2)$.

However, the worst-case complexity does not do quicksort justice because it is much higher than its average-case complexity.
Because there are only finitely many permutations for a list of fixed length, the average-case complexity can be worked out systematically.
The result is $\Theta(n\log_2 n)$.
\medskip

It may seem that quicksort is less attractive than mergesort because of its higher worst-case complexity.
However, that is a minor effect because the algorithms have the same best-case and average-case complexity.
Instead, the constant factors, which are rounded away by using $\Theta$-classes, become important to compare two algorithms with such similar complexity.

Here quicksort is superior to mergesort.
Moreover, quicksort can be optimized in many ways.
In particular, the choice of the pivot can be tuned in order to increase the likelihood that the two sublists end up having the same size.
For example, we can randomly pick $3$ elements of the sublist and use the middle-size one as the pivot.
With such optimizations, quicksort can become substantially faster than mergesort.

\subsection{Other Algorithms}

There is a number of other sorting algorithms that we will not go into here.
Examples include counting sort, radix sort, and bucket sort.

One particularly important sorting algorithm is heap sort, which we discuss in Sect.~\ref{sec:ad:heapsort} (after introducing heaps).

\subsection{In Programming Languages}

Most programming languages come with a standard library that includes efficient sorting algorithms.
Moreover, other libraries for other algorithms may be around.
In some cases, languages only specify the interface and leave the implementation (and thus the choice of algorithm )to individual implementations of compilers/interpreters.

The following gives some examples.

Python uses Timsort (named after the programmer), which is a hybrid of mergesort and insertionsort with various optimizations.
It is written directly in C.

Java used to use just quicksort.
Java 7 uses either Timsort (ported to Java) or a variant of quicksort that uses two pivot elements.

Scala defers to Java's implementation.

C++'s std library specification does not prescribe a sorting algorithm but requires $O(n\log_2 n)$ worst-case complexity (average-case in earlier versions).
Implementations vary in their choice of algorithm, e.g., using hybrid algorithms that perform some iterations of quicksort before switching to insertionsort for the resulting small lists.

For Javascript, the choice is up to the browser (because every browser is a separate implementation of Javascript).
