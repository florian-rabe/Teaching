\documentclass[a4paper]{article}

\usepackage[course={Algorithms and Data Structures},number=2,date=2017-02-14,duedate=2017-02-23]{../../myhomeworks}

\newcounter{chapter} % needed for dependencies of mylecturenotes
\usepackage[root=../..]{../../mylecturenotes}
\usepackage{../../macros/algorithm}

\begin{document}

\header

\begin{problem}{Fibonacci}{4+4+2}
In the programming language of your choice:

\begin{compactenum}
 \item Implement the exponential, the linear, and the inexact algorithm given in the lecture notes for the function $fib(n)$.
 Use arbitrary precision integer arithmetic for the exponential and the linear algorithm, and use floating point numbers for the inexact one.
 \item Write a program to determine the largest $n$ for which $fib(n)$ returns a result in at most $10$ seconds.
  Determine the value $n$ for the exponential and for the linear algorithm.
  % for me in Python: around 35 for the exponential one, around 250000 for the linear one
 \item Write a program to determine the smallest $n$ for which the inexact algorithm returns an incorrect result.
  Run your program to find that value.
  % for me in Python: 71
\end{compactenum}

(Of course, the answers to the last two questions will depend on your programming language and computer.)
\end{problem}

\begin{problem}{Asymptotic Notation: Examples}{(2+1+1)+2+2+2}
For each of the following functions $f$ and $g$, determine whether $f\in O(g)$, $g\in O(f)$, neither, or both.

\begin{compactenum}
 \item $f(n)=7n$ and $g(n)=n^7$
 \item $f(n)=n^2/\log_2 n$ and $g(n)=n$
 \item $f(n)=27\log_2 n$ and $g(n)=(\log_2(4n))^2$
 \item $f(n)=\sqrt{n}$ and $g(n)=5n^{0.5}+7n^{0.3}+11\log_2 n$
\end{compactenum}

Show your work for the first problem, i.e.,
\begin{compactitem}
 \item give $k$ and $N$ such that the property in the definition of $\Oleq{}$ holds, or
 \item show that no such $k$ and $N$ exist.
\end{compactitem}
\end{problem}

\begin{problem}{Asymptotic Notation: Theory}{2+3}
Prove that $\Oleq{}{}$ is
\begin{compactenum}
 \item reflexive
 \item transitive
\end{compactenum}

\begin{hint}
To prove $\Oleq{f}{g}$, you have to give $N$ and $k$ such that $f(n)\leq k\cdot g(n)$ for all $n>N$.

For reflexivity, that is easy.

For transitivity, assume $k_1$ and $N_1$ exist for $\Oleq{f}{g}$ and $k_2$ and $N_2$ exist for $\Oleq{g}{h}$, then give $k$ and $N$ for $\Oleq{f}{h}$.
\end{hint}
\end{problem}


\end{document}
