\documentclass[a4paper]{article}

\usepackage[course={Algorithms and Data Structures},number=2,date=2017-02-14,due=2017-02-23,unpublished]{../../myhomeworks}

\newcounter{chapter} % needed for dependencies of mylecturenotes
\usepackage[root=../..]{../../mylecturenotes}
\usepackage{../../macros/algorithm}

\begin{document}

\header

The following is an (highly simplified variant of an) example from the author's research:

\begin{example}[String Processing]\label{ex:ad:quadraticstringprocessor}
Consider the following function in the Scala programming language
\begin{lstlisting}
def processString(s: String) {
  var rest = s
  while (s != "") {
    if (rest.startsWith("foo")) {
      // does not matter, assume this takes constant time
    } else {
      // also does not matter
    }
    rest = s.substring(1)
  }
}
\end{lstlisting}
Here \lstinline|startsWith| and \lstinline|substring| are methods on strings from the Java library (which Scala can call with only constant-time overhead).

Let $C(n)$ be the run time of this function where $n$ is the length of the input string.
What is the complexity class of $C(n)$?

You can run the program yourself on increasingly large inputs to find out.
In the author's case, the surprising effect was noticeable when $n>10^7$, e.g., when reading a $10$ MB text file.
\end{example}


\end{document}
