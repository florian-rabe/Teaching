Union types provide data structures for the disjoint union $A+B$ of sets.

They are dual to product types and behave in very much the same way.

\section{Positional Unions}

The simplest data structure directly implements the disjoint union.

\subsection{Primitive Unions}

Functional programming languages could include disjoint union as a primitive feature.
However, this is rare because disjoint unions turn out to be much less practically relevant than products.

Instead, languages almost exclusively define them as inductive data types.

\subsection{Unions as an Inductive Data Type}

Disjoint unions can also easily be defined as an inductive data type:
\begin{acode}
\adata{DisjointUnion[A,B]}{InjectionLeft(A), InjectionRight(B)}\\
\\
\afun[C]{matchUnion[A,B]}{u: DisjointUnion[A,B], f: A\to C, g: B\to C}{\amatch{u}{\acase{InjectionLeft(a)}{f(a)},\acase{InjectionRight(b)}{g(b)}}}\\
\end{acode}

\section{Labeled Unions}

Labeled unions define the disjoint union of multiple types.

\subsection{Specification}

Let $l_1,\ldots,l_n$ be some fresh unique names.

The set $U=l_1(A_1)|\ldots|l_n(:A_n)$ is isomorphic to the set $A_1+ \ldots +A_n$.

Its elements are of the form $l_1(a_1)$ or \ldots or $l_n(a_n)$ for $a_i\in A_i$.

Given $u\in U$, we use pattern-matching to distinguish between the $n$ possible cases for $u$.

\subsection{Data Structures}

Data structures for labeled unions are rarely used.

Instead, they can easily be implemented as special cases of inductive data types as in
\begin{acode}
\acomment{$l_1(A_1)|\ldots|l_n(A_n)$}\\
\adata{U}{l_1(A_1),\ldots,l_n(A_n)}
\end{acode}


\section{Recursive Disjoint Unions: Inductive Data Types}

Finally inductive data types arise as the most general case of disjoint unions.
Here the components of the union may already refer to the whole disjoint union.

We have seen examples already, e.g., in the inductive data type for lists.
For example, we can understand $List[A]$ as the disjoint union of the unit type (which contains a single element corresponding to the empty list) and the product of $A$ (for the head) and $List[A]$ (for the tail).

Inductive data types are part of all functional programming languages and can be mimicked using classes in object-oriented programming languages (as described in Sect.~\ref{sec:ad:list:ds}).