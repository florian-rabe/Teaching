\documentclass{book}

\usepackage{graphicx}
\usepackage{xkeyval}
\usepackage{multirow}
%\usepackage{bm} %% bold face math symbols
\usepackage{listings}
\usepackage{mytikz}
%\usepackage{stmaryrd}%\newcommand{\contra}{\lightning}
%\usepackage{rotating} \newcommand{\sw}[1]{\begin{sideways}#1\end{sideways}}

\usepackage{../macros/algorithm}
%\usepackage{ded}

\usepackage{../mylecturenotes}

\title{Lectures Notes on Data Structures and Algorithms}
\author{Florian Rabe}
\date{2017}

\begin{document}
\maketitle

\tableofcontents
\newpage

\part{Introduction and Foundations}

\chapter{Meta-Remarks}
 \begin{center}
\textbf{Important stuff that you should read carefully!}
\end{center}

\paragraph{State of these notes}
I constantly work on my lecture notes.
Therefore, keep in mind that:
\begin{compactitem}
\item I am developing these notes in parallel with the lecture---they can grow or change throughout the semester.
\item These notes are neither a subset nor a superset of the material discussed in the lecture.
\item Unless mentioned otherwise, all material in these notes is exam-relevant (in addition to all material discussed in the lectures).
\end{compactitem}
\medskip

\paragraph{Collaboration on these notes}
I am writing these notes using LaTeX and storing them in a git repository on GitHub at \url{https://github.com/florian-rabe/Teaching}.
Familiarity with LaTeX as well as Git and GitHub is not part of this lecture. But it is essential skill for you.
Ask in the lecture if you have difficulty figuring it out on your own.
\medskip

As an experiment in teaching, I am inviting all of you to collaborate on these lecture notes with me.
\medskip

By forking and by submitting pull requests for this repository, you can suggest changes to these notes.
For example, you are encouraged to:
\begin{compactitem}
\item Fix typos and other errors.
\item Add examples and diagrams that I develop on the board during lectures.
\item Add solutions for the homeworks if I did not provide any (of course, I will only integrate solutions after the deadline).
\item Add additional examples, exercises, or explanations that you came up or found in other sources.
 If you use material from other sources (e.g., by copying an diagram from some website), make sure that you have the license to use it and that you acknowledge sources appropriately!
\end{compactitem}
The TAs and I will review and approve or reject the changes.
If you make substantial contributions, I will list you as a contributor (i.e., something you can put in your CV).
\medskip

Any improvement you make will not only help your fellow students, it will also increase your own understanding of the material.
Therefore, I can give you up to $10\%$ bonus credit for such contributions.
(Make sure your git commits carry a user name that I can connect to you.)
Because this is an experiment, I will have to figure out the details along the way.

\paragraph{Other Advice}
I maintain a list of useful advice for students at \url{https://svn.kwarc.info/repos/frabe/Teaching/general/advice_for_students.pdf}.
It is mostly targeted at older students who work in individual projects with me (e.g., students who work on their BSc thesis).
But much of it is useful for you already now or will become useful soon.
So have a look.

\paragraph{Skipped Chapters}
These notes were originally prepared for my 2nd semester CS course at Jacobs University in Spring 2017.
In that course, the following chapters were skipped or treated only very superficially: \ref{sec:ad:finiteds}, \ref{sec:ad:numbers}, \ref{sec:ad:option}, \ref{sec:ad:functions}, \ref{sec:ad:unions}, \ref{sec:ad:parallel}, \ref{sec:ad:prot}, \ref{sec:ad:random}, \ref{sec:ad:quantum}.
In the other chapters, almost all material was covered; only a few subsections were skipped.

\chapter{Basic Concepts}\label{sec:ad:fund}
  These lecture notes do not follow a particular textbook.

Students interested in additional literature may safely use \cite{cormen_algorithms} (available online), one of the most widely used textbooks.
Knuth's book series on the Art of Computer Programming \cite{knuth_art}, although not usually used as a modern textbook, is also interesting as the most famous and historically significant book on the topic.

\section{What are Data Structures and Algorithms?}\label{sec:ad:whatare}

Data structures and algorithms are among the most fundamental concepts in computer science.

\subsection{Static vs. Dynamic}\label{sec:ad:static}

In all areas of life and science, we often find a pair of concept such that one concept captures static and the other one dynamic aspects.
This is best understood by example:

\begin{center}
\begin{tabular}{|l||l|l|}
\hline
area & static & dynamic \\
\hline
\hline
\multicolumn{3}{|c|}{in life}\\
\hline
existence & be & become \\
\hline
events & situation & development \\
\hline
food & ingredients & cooking \\
\hline
\hline
\multicolumn{3}{|c|}{in science}\\
\hline
mathematics & sets & functions \\
\hline
physics & space & time \\
\hline
chemistry & molecules & reactions \\
\hline
engineering & materials & construction \\
\hline
\hline
\multicolumn{3}{|c|}{\textbf{in computer science}}\\
\hline
hardware & memory & processing \\
\hline
abstract machines & states & transitions \\
\hline
programming  & types & functions \\
\hline
\textbf{software design} & \textbf{data structures} & \textbf{algorithms} \\
\hline
\end{tabular}
\end{center}

The static aspects describes things as they are at one point in time.
The dynamic aspects describes how they change over time.

Data structures and algorithms have this role in software design.
Data structures are sets of objects (the data) that describe the domain that our software is meant to be used for.
Algorithms are operations that describe how the objects in that domain change.

\subsection{Basic Definition and Examples}\label{sec:ad:basicdef}

\begin{definition}[Data Structure]\label{def:ad:ds}
Assume some set of effective objects.

A data structure defines a subset of these objects by providing effective methods for determining
\begin{compactitem}
 \item whether an object is in the data structure or not,
 \item whether two objects are equal.
\end{compactitem}
\medskip

In practice, a data structure is often bundled with several algorithms for it.
\end{definition}

\begin{definition}[Algorithm]\label{def:ad:algo}
An algorithm consists of
\begin{compactitem}
\item a data structure that defines the possible input objects
\item a data structure that defines the possible output objects
\item an effective method for transforming an input object into an output object
\end{compactitem}
\end{definition}

These definitions are not very helpful---they define the words ``data structure'' and ''algorithm'' by using other not-defined words, namely ``effective object'' and ``effective method''.
Let us look at some examples before discussing effective objects and methods in Sect.~\ref{sec:ad:effective}.

\begin{example}[Natural Numbers]
The most important data structure are the natural numbers.

It is defined as follows:
\begin{compactitem}
 \item The string $0$ is a natural number.
 \item If $n$ is a natural number, then the string $s(n)$ is a natural number.
 \item All natural numbers are obtained by applying the previous step finitely many times, and these are all different.
\end{compactitem}

We immediately define the usual abbreviations $1,2,\ldots,$.
It is also straightforward to define algorithms for the basic functions on natural numbers such as $m+n$, $m-n$, $m*n$, etc.
\end{example}

\begin{example}[Euclidean Algorithm]\label{ex:ad:euclid}
The Euclidean algorithm (see also Sect.~\ref{sec:ad:history}) computes the greatest common divisor $\gcd(m,n)$ of two natural numbers $m,n\in\N$.
It consists of the following components:
\begin{compactitem}
\item input: $\N\times\N$
\item output: $\N$
\item effective method:
\begin{acode}
\afun[\N]{gcd}{m:\N,n:\N}{
x := m \acomment{introduce variables, initialize with input data}\\
y := n \\
%% loop invariant gcd(m,n)=gcd(x,y)
\awhile[repeat as long as $\gcd(x,y)\neq x$]{x\neq y}{
  \aifelse[subtract the smaller number from the bigger one, which does not affect $\gcd(x,y)$]{x<y}{y := y-x}{x := x-y}
}\\
\areturn[now trivially $\gcd(x,y)=x$]{x}
}
\end{acode}
\end{compactitem}

The algorithm starts by introducing variables $x$ and $y$ and initializes them with the input data $m$ and $n$.
Then it repeatedly subtracts the smaller number from the greater one until both are equal.
This works because $\gcd(x,y)=\gcd(x-y,y)$.
If $x$ and $y$ are equal, we can return the output because $\gcd(x,x)=x$.
\medskip

This algorithm has a subtle bug (Can you see it?) that we will fix in Ex.~\ref{ex:ad:euclid:term}.
\end{example}

For a simpler example, consider the definition of the factorial $n!=1\cdot \ldots n$ for $n\in\N$.

\begin{example}[Factorial]\label{ex:ad:factorial}
The factorial can be defined as follows:
\begin{compactitem}
\item input: $\N$
\item output: $\N$
\item effective method:
\begin{acode}
\afun[\N]{fact}{n:\N}{
product := 1\\
factor  := 1 \\
%% loop invariant n! = product * factor * ... * n
\awhile{factor \leq n}{
  product := product \cdot factor\\
  factor := factor+1
}\\
\areturn{product}
}
\end{acode}

Here the variable $factor$ runs through all values from $1$ to $n$ and the variable $product$ collects the product of those values.
\end{compactitem}
\end{example}

\begin{notation}
It is convenient to give the effective method of an algorithm as a function definition using pseudo-code.
That way the input and output do not have to be spelled out separately because they are clear from the data structures used in the header of the function definition.
\end{notation}

\subsection{Effective Objects and Methods}\label{sec:ad:effective}

It is now a central task in computer science to define data structures and algorithms that correspond to given sets and functions.
This question that was first asked by David Hilbert in 1920, one of the most influential mathematicians at the same time.
In modern terminology, he wanted to define data structures for all sets and algorithms for all functions and then machines to mechanize all mathematics.

In the 1930s, several scientist worked on this problem and eventually realized that it cannot be done.
These scientists included Alonzo Church, Kurt G\"odel, John von Neumann, and Alan Turing.
Their work provided partial solutions and theoretical limits to the problem.
In retrospect, this was the birth of computer science.

Not every set and not every function can be represented by a data structure or an algorithm (see Sect.~\ref{sec:ad:computable} for the reason why not).
That limitations bring us back to the question of effective objects and methods:

\begin{definition}[Effective Object]
An effective object is any object that can be stored, manipulated, and communicated by a physical machine.

Here, \emph{physical} means any machine that we can build in the physical world.\footnotemark
\end{definition}
\footnotetext{Sometimes we use hypothetical machines. For example, quantum computers are physical machines that we think we can build but have not been able to build in practice at useful scales yet.}

Thus, every physical machine defines its own kind of effective objects.
All digital machines (which includes all modern computers) use the same effective objects: lists of bits.
These are stored in memory or on hard drives, which provide essentially one very, very long list of bits.

Data structures use fragments of these lists to represents sets.
For example, the set $\Z_{2^{32}}$ of $32$-bit-integers is represented by a list of $32$ bits.

\begin{definition}[Effective Method]
An effective method consists of a sequence of instructions such that
\begin{compactitem}
 \item any reasonably intelligent human can carry out the instructions
 \item and all such humans will carry out the instructions in exactly the same way (in particular reaching the same result).
\end{compactitem}
\end{definition}

The first condition makes sure that any prior knowledge needed to understand the instructions is be explicitly stated or referenced.
The second condition makes sure that an effective method has a well-defined result: There may be no ambiguity, randomness, or unspecified choice.

\begin{example}
The second condition excludes for example the following instructions
\begin{compactitem}
 \item ``Let $x$ be the factorial of $5$.'': Different humans could compute the factorial differently because it is not clear which algorithm to use for the factorial.
 \item ``Let $x$ be a random integer.'': Randomness is not allowed.
 \item ``Let $x$ be an element of the list $l$.'': It is not specified which element should be chosen.
\end{compactitem} 
\end{example}

\subsection{History}\label{sec:ad:history}

One of the earliest and most famous (arguably \emph{the} earliest) algorithms is Euclid's algorithm for computing the greatest common divisor (see Ex.~\ref{ex:ad:euclid}).
It is given around 300 BC in Euclid's Elements \cite[Book VII, Proposition 2]{elements}, maybe the most influential textbook of all time.
\medskip

The word \emph{algorithm} is much younger.
It is derived from the name of the 9th century scientist al-Khwarizmi.
He was one of the most important scientists of his millennium but is relatively unknown in the Western world because he was an Arab and wrote in Arabic.
Translations of his work on arithmetic in the 12th century spread several new results to the Western world.

This included the use of numbers as abstract objects as opposed to geometric distances that had dominated Europe since the work of the Greek mathematicians (such as Euclid).
It also included the positional number system and the base-$10$ digits that are still in use today.
The corresponding arithmetical operations on numbers were named \emph{algorismus} after him in Latin, which developed into the modern word.
He also worked on algorithms for solving linear and quadratic equations, and one of his basic operations called \emph{al-jabr} gave rise to the word \emph{algebra}.
\medskip

The modern \emph{meaning} of the word \emph{algorithm} is even younger: Its formalization was effected by a major development in the 1920s and 1930s that eventually gave to modern computer science itself.
Hilbert was the most influential mathematician in the early 20th century.
One of his legacies was to call for solutions to certain fundamental problems \cite{hilbertsproblems}.
Another legacy was his program \cite{hilbertsprogram}, a call for the formalization of mathematics that (among other things) should yield an algorithm for determining whether any given mathematical formula is a theorem.

Hilbert's program inspired seminal work by (among others) Alonzo Church, Kurt G\"odel, and Alan Turing.
This led to several concrete definitions of \emph{algorithm}, including Turing-machines and the $\lambda$-calculus, from which all modern programming languages are derived.
It also led to an understanding of the limits of what algorithms can do (see Sect.~\ref{sec:ad:computable}), which has led to the modern theory of computation.

\subsection{The Limits of Data Structures and Algorithms}\label{sec:ad:computable}

\subsubsection{Countability of Data Structures and Algorithms}

We can now see immediately why not all mathematical objects are effective in digital machines: There are only countably many lists of bits.
Therefore, there can only be countably many effective objects.

Similarly, any data structure we define must be defined as a list of characters in some language.
But there are only countably many such lists.
Therefore, there can only be countably many data structures.
For the same reason, there can only be countably many algorithms.
\medskip

Inspecting the sizes of the constructed sets from Sect.~\ref{sec:math:sets}, we can observe that
\begin{compactitem}
\item If all arguments are finite, so is the constructed set---except for lists.
\item If all arguments are at most countable, so is the constructed set---except for function and power sets.
\end{compactitem}
Because of these exceptions, we cannot restrict attention to finite or countable sets only---working with them invariably leads to uncountable sets.

\subsubsection{Computability}

At best, we can hope to give data structures for all countable sets.
But not even that is possible.
Because countable sets have uncountably many subsets, we cannot give data structures for every subset of every countable set.

Therefore, we give the sets that have data structures a special name:

\begin{definition}[Decidable]
A set is called \textbf{decidable} if we can give a data structure for it.
\end{definition}

Similarly, at best we can hope to give algorithms for all functions between decidable sets.
Again that is not possible.
Because countable sets have uncountably many functions between them, we cannot give algorithms for all functions between decidable sets.

Therefore, we give the functions that have algorithms a special name:

\begin{definition}[Computable]
A function between decidable sets is called \textbf{computable} if we can give an algorithm for it.
\end{definition}

Decidability and computability are discussed in detail in---typically---a second year course in theoretical computer science.

\subsubsection{The Role of Programming Languages}

\paragraph{Vagueness of the Definitions}
It is not possible to precisely define effective objects and methods---every definition eventually uses not-defined concepts like ``machine'' or ``instruction''.
Thus, it impossible to precisely define what data structures and algorithms are.
Instead, we must assume those concepts to exist a priori.

That may seem flawed---but it is actually very normal.
We can compare the situation to physics where there is also no precise definition of \emph{space} and \emph{time}.
In fact, the question what space and time are is among the difficult of all of physics.\footnotemark
\footnotetext{For example, even today physicists have no agreed-upon answer to the question why time moves forwards but not backwards.}

Similarly, the question of what data structures and algorithms are is among the most fundamental of computer science.
Every machine and evey programming language give their own answers to the question.

\paragraph{Data Description and Programming Languages}
The only way to have a precise definition of \emph{data structure} and \emph{algorithm} is to choose a concrete formal language.

\begin{definition}[Languages]
A \textbf{data description language} is a formal language for writing objects and data structures.

A \textbf{programming language} is a formal language for writing algorithms.
\end{definition}

Because algorithms require data structures, every programming language includes a data description language.
And because all data structures usually come with specific algorithms, we are mostly interested in programming languages.

But there are some languages that are pure data description languages.
These are useful when storing data on hard drives or when exchanging data between programs and computers (e.g., on the internet).
Examples of pure data description languages are JSON, XML, HTML, and UML.

\paragraph{Types of Programming Languages}
Programming languages can vary widely in how they represent data structures.

We can roughly distinguish the following groups:
\begin{compactitem}
\item Untyped languages avoid explicit definitions of data structures.
Instead, they use algorithms such as $\mathit{isNat}$ to check, e.g., if an object is a natural number.\\
Examples are Javascript and Python.
\item Functional languages primarily use inductive data types to write data structures and recursive functions to write algorithms.\\
Examples are SML and Haskell.
\item Object-oriented languages primarily use classes to write data structures and imperative loops to write algorithms.\\
Examples are Java and C++.
\item Multi-paradigm languages combine functional and object-oriented features.\\
Examples are Scala and F\#.
\end{compactitem}


\paragraph{Independence of the Choice of Language}
Above we have seen that the concrete meaning of \emph{data structure} and \emph{algorithm} seems to depend on the choice of programming language.
Thus, it seems that whether a set is decidable or a function computable also depends on the choice of programming language.

One of the most amazing and deepest results of theoretical computer science is that this is not the case:

\begin{theorem}[Church-Turing Thesis]
All known machines and programming languages (including theoretical ones such as Turing machines)
\begin{compactitem}
\item can define data structures for exactly the same sets,
\item can define algorithms for exactly the same functions.
\end{compactitem}

Thus, it does not depend on the chosen machine or programming language
\begin{compactitem}
\item whether a set is decidable,
\item whether a function is computable.
\end{compactitem}
\end{theorem}
\begin{proof}
The proof is very complex.
For every program of every language, we must provide an equivalent program in every other language.

However, this can be done (and has been done) for all languages.
\end{proof}
%This should only hold for Turing complete languages (and some especially nice programming languages like e.g. Agda are not Turing complete). Non Turing complete languages can not nessecarily construct all the set a Turing complete language can. For instance in Agda (and some other total functional programming languages) you can only implement algorithms (including constructors of data types) for programs proven to be terminating. According to the halting problem this implies, that there are some algorithms that you can define on a Turing machine, but not in Agda. 

A related (stronger) theorem is that every programming language $P$ allows defining for every programming language $Q$ a program that executes $Q$-programs.

We can only prove these theorems for all \emph{known} languages.
It is generally believed to be true also for all \emph{possible} languages, but that is impossible to prove.

\section{Specification vs. Design vs. Implementation}

Above we have seen sets and functions as well as data structures and algorithms.
Moreover, we have already mentioned programs consisting of types and functions.

The following table gives a overview of the relation between these concepts:

\begin{ctabular}{|l|l|l|}
\hline
Specification & Design/Architecture & Implementation \\
\hline
\hline
sets          & data structures & types    \\
functions     & algorithms      & functions\\
\hline
\end{ctabular}

Software development consists of $3$ steps:
\begin{enumerate}
\item The \textbf{specification} describes the intended behavior in terms of mathematical sets and functions.\\
It does not prescribe in any way how these sets and functions are realized.
The same specification can have multiple different correct realizations differing among others in size, maintainability, or efficiency.\\
A good specification should be:
\begin{compactitem}
  \item adequate: actually describe the problem that needs solving
  \item simple: easy to understand
  \item unambiguous: impossible to misunderstand
  \item consistent: possible to realize
  \item (optionally) complete: no freedom in what it means (An incomplete specification is not necessarily a flaw. For example, one might specify a function on integers without saying what should happen for negative input.)
\end{compactitem}
\item The \textbf{architecture} makes concrete choices for the data structures and algorithms that realize the needed sets and functions.\\
It usually defines many auxiliary data structures and algorithms that are not part of the specification.\\
The architecture does not prescribe a programming language. It can be correctly realized in any programming language.
\item The \textbf{implementation} chooses a programming languages and then writes a \textbf{program} in it that realizes the architecture.
The program includes concrete choices for the type and function definitions that realize the needed data structures and algorithms.\\
It usually defines many auxiliary types and functions that are not part of the architecture.
\end{enumerate}

\begin{terminology}
\emph{Design} and \emph{architecture} can usually be used synonymously.

The words \emph{specification}, \emph{design}, and \emph{implementation} can refer to both the process and the result.
For example, we can say that the result of implementation is one implementation.
\end{terminology}

It is critical to distinguish the three steps in software development:
\begin{itemize}
 \item Specification changes are much more expensive than design changes.
 Changing the specification may completely change, which design is appropriate.
 Therefore, every single design decision must be revisited and checked for appropriateness.
 \item Design changes are much more expensive than implementation changes.
 Changing the design may completely change which components of the implementation are needed and how they interact.\\
 Therefore, every part of the program must potentially be revisited. \\
 In particular, whenever the design of component $X$ is changed, we have to revisit every place of the program that uses $X$. 
 This often introduces bugs.
\end{itemize}
Typically any specification change entails bigger design changes, and any design change entails bigger implementation changes.
Moreover, specification changes require
\begin{compactitem}
	\item re-verification (i.e., checking that the implementation still correctly implements the specification)
	\item re-certification by regulatory agencies (if applicable to the specific software)
	\item changes to documentation, manuals, and tutorials, re-training of users, etc.
	\item distribution of software updates, which confuses and disrupts their workflows
	\item need for other software projects to adapt to the updated software
\end{compactitem}

An ideal programmer proceeds in the order specification-design-implementation.
However, it is often necessary to loop back: The design phase may reveal problems in the specification, and the implementation phase may reveal problems in the design.
Therefore, we usually have to work on all $3$ parts in parallel---but with a strong preference against changing specification and design.

Many self-taught or not-well-taught programmer do not understand the difference between the $3$ steps or do not systematically apply it.
There are many such programmers, who never studied CS or got a degree without taking a rigorous foundations course.
Their programs are typically awful because:
\begin{compactitem}
 \item They begin programming without writing down the specification.
 Consequently, they do not realize that they have not actually understood the specification.
 This results in programs that do not meet the specification, which then leads to retroactive changes to the design.
 Over time the program becomes (sometimes called ``spaghetti code'') that is unmaintainable and cannot be understood by other programmers, often not even by the programmer herself.
 \item They begin programming without consciously choosing a design.
 Consequently, they end up with a random design that may or may not be appropriate for the task.
 Over time they change the design multiple times (without being aware that they are changing the design).
 Each change introduces new bugs and more mess.
\end{compactitem}

\begin{example}[Greatest Common Divider]\label{ex:ad:euclid2}
The specification of the greatest common divider function $\gcd$ is as follows:
Given natural numbers $m$ and $n$, return a natural number $g$ such that
\begin{compactitem}
\item $g|m$ and $g|n$
\item for every number $h$ such that $h|m$ and $h|n$ we have that $h|g$
\end{compactitem}
\medskip

Before we design an algorithm, we should check whether $\gcd$ is indeed a function:
\begin{compactitem}
  \item Consistency: Does such a $g=\gcd(m,n)$ always exists?
  \item Uniqueness: Could there be more than one such $g$?
\end{compactitem}
Using mathematics, we can prove that $g$ indeed exists uniquely.
\medskip

Now we design an algorithm.
Let us assume that we have already designed data structures for the natural numbers with the usual operations.
There are many reasonable algorithms, among them the one from Ex.~\ref{ex:ad:euclid}.
For the sake of example, we use a different one here:
\medskip

\begin{acode}
\afun[\N]{gcdRec}{m:\N, n:\N}{
  \aifelse{n==0}{m}{\gcd(n,m\modop n)}
}
\end{acode}

\medskip
This is a recursive algorithm: The instructions may recursive call the algorithm itself with new input.
\medskip

Finally, we implement the algorithm.
We choose SML as the programming language.
First we implement the data structure for natural numbers and the function $mod:nat*nat \to nat$ that were assumed by the specification.
Note that this requires some auxiliary functions that were not part of the algorithm:
\begin{lstlisting}
datatype nat = zero | succ of nat

fun leq(m: nat, n: nat): bool = case (m,n) of
  (zero,    zero)    => true
| (zero,    succ(y)) => true
| (succ(x), zero)    => false
| (succ(x), succ(y)) => leq(x,y)

fun minus(m: nat, n: nat): nat = case (m,n) of
  (zero,    zero)    => zero
| (zero,    succ(y)) => zero (* error case, should not happen *)
| (succ(x), zero)    => succ(x)
| (succ(x), succ(y)) => minus(x,y)
  
fun mod(m:nat, n:nat):nat =
  if m = n then zero
  else if leq(m,n) then m
  else mod(minus(m,n), n)
\end{lstlisting}
 
Then we define
\begin{lstlisting}
fun gcdRec(m:nat, n: nat): nat = if n = zero then m else gcdRec(n,mod(m,n))
\end{lstlisting}
\end{example}

\section{Stateful Aspects}

\subsection{Immutable vs. Mutable Data Structures}

Consider a data structure for the set $\N^*$ of lists of natural number and assume we have a variable $x:\N^*$.

\subsubsection{Immutable Data Structures and Call-by-Value}

We can always assign a new value to $x$ as a whole.
For example, after executing $x:=[1,3,5]$, we have the following data stored in memory:
\begin{amemory}
\avar{x}{\N^*}{[1,3,5]}
\alocations
\aloc{P}{[1,3,5]}
\end{amemory}
Here the left part shows the variables as seen by the programmer.
The right part shows the objects as they are maintained in memory by the programming language.
$P$ is some name for the memory location holding the value of $x$.
Importantly, the programmer is completely unaware of the organization of the data in memory and only sees the value of $x$.

In particular, $x$ is just an abbreviation for the value $[1,3,5]$.
If we pass $x$ to a function $f$, there is no difference between saying $f(x)$ and $f([1,3,5])$.
That is called \textbf{call-by-value}.

For example, if we execute the instruction $y = delete(x,2)$, we obtain:
\begin{amemory}
\avar{x}{\N^*}{[1,3,5]}
\avar{y}{\N^*}{[1,3]}
\alocations
\aloc{P}{[1,3,5]}
\aloc{Q}{[1,3]}
\end{amemory}
All old data is as before.
For the new variable $y$, a new memory location $Q$ has been allocated and filled with the result of the operation.
This has the drawback that the entire list was duplicated, and we now use twice as much memory as before.

Immutable data structures and call-by-value are the usual way how functions work in mathematics.
Such data structures are closely related to their specification and make writing, understanding, and analyzing algorithms very easy.

\subsubsection{Mutable Data Structures and Call-by-Name}

If our data structure is mutable, the value of a variable $x$ is just a reference to the memory location where the value is stored.

For example, after executing $x:=[1,3,5]$, we have the following data stored in memory:
\begin{amemory}
\avar{x}{\N^*}{P}
\alocations
\aloc{P}{[1,3,5]}
\end{amemory}
The value of $x$ is now the reference to the memory location.
The programmer still cannot see $P$ directly.%
\footnote{Some programming languages allow explicitly creating and manipulating these references.
The most notable example is $C$ (where the references are called \emph{pointers}).
With a few caveats (most importantly that it can allow for maximal optimization), that can be considered a design flaw in the programming language.}

But there are two carefully-designed ways how $P$ can be accessed indirectly.
Firstly, we can assign new values to each component of $x$.
For example, after $x.1:=4$, the memory looks like
\begin{amemory}
\avar{x}{\N^*}{P}
\alocations
\aloc{P}{[1,4,5]}
\end{amemory}
The old value at location $P$ is gone and has been replaced by the new value.

Secondly, when we pass $x$ to a function $f$, we pass the reference to the value, not the value itself.
This is called \textbf{call-by-name} or \textbf{call-by-reference}.

For example, after executing $delete(x,2)$, we have
\begin{amemory}
\avar{x}{\N^*}{P}
\alocations
\aloc{P}{[1,4]}
\end{amemory}
No additional memory location has been allocated for the result, and no copying took place.
That makes the operation much more time- and memory-efficient.
But from a mathematical perspective, this is very odd: The function call $delete(x,2)$ \emph{changed} the value of $x$ under the hood.
\medskip

In many programming languages (in particular object-oriented ones), mutable data structures are called \emph{classes}.
Some functions involving a mutable data structure will make use of mutability, some will not.
This must be part of the specification of each function.

\subsection{Environments and Side Effects}

So far we have said that algorithms realize mathematical functions.
That makes algorithms very close to the specification and makes writing, understanding, and analyzing them very easy.
But it is not the whole picture in computer science---computer science needs a generalization:

\begin{definition}[Stateful Functions]
Let $E$ be the set of environments.
An \textbf{effectful function} from $A$ to $B$ is a function $A\times E\to B\times E$.
\end{definition}

Again this is a vague definition because the word ``environment'' is not defined.
That is normal---there is no universally recognized definition for it.
Intuitively, an object $e\in E$ represents the state of the environment.
$e$ contains all information that is visible from the outside of our algorithms and that can be acted on by the algorithm.
These usually include the global variables, all kinds of input/output, threads, and exceptions.

An effectful function $f$ from $A$ to $B$ can do two things besides returning a result of type $B$:
\begin{compactitem}
 \item It can use the environment (because $E$ occurs in its input type).
   Thus, calling $f$ twice on the same $a\in A$ may return different results if the environment has changed in between.\\
   Formally, if $f(a,e_1)=(b_1,e_1')$ and $f(a,e_2)=(b_2,e_2')$ always implies $b_1=b_2$, we say that $f$ is \textbf{environment-independent}.
 \item It can change the environment (because $E$ occurs in its output type).
   Thus, programmers must be careful when to call $f$ and how often to call $f$ because every call may have an effect that can be observed by the user.\\
   Formally, if $f(a,e)=(b,e')$ always implies $e=e'$, we say that $f$ is \textbf{side-effect-free}.
\end{compactitem}
If $f$ is both environment-independent and side-effect-free, $f$ is called \textbf{pure}.
In that case, we always have $f(a,e)=(g(a),e)$ for some function $g:A\to B$, i.e., we can ignore environments entirely.
Thus, pure functions are essentially the same as the usual mathematical functions.

An environment $e\in E$ is usually a big tuple containing among others
\begin{compactitem}
 \item the current values of all accessible mutable variables
 \item console input/output:
   \begin{compactitem}
      \item the list of characters to be printed out to the user
      \item the list of characters typed by the user that are available for reading
   \end{compactitem}
 \item file and peripheral network input/output: for every open file, network connection or similar
   \begin{compactitem}
      \item the list of data to be written to the connection
      \item the list of data that is are available for reading
   \end{compactitem}
 \item information about exceptions
   \begin{compactitem}
      \item by depending on this aspect of the environment, effectful functions can handle exceptions
      \item by effecting this aspect of the environment, effectful functions can raise exceptions
   \end{compactitem}
 \item the set of currently active threads
 \item additional components depending on the features of the respective programming language
\end{compactitem}

Environment-dependency and side effects are important.
Without input/output side effect, the user could never provide input for algorithms and could never find out what the output is.
Moreover, computers could not be used to read sensor data or control peripheral devices.

But they also present major challenges to algorithm design.
Because the precise definition of $E$ depends on the details of the programming language, it is very difficult to precisely specify effectful functions.
And without a precise specification, the programmer never knows whether an algorithm is designed and implemented correctly.
Therefore, some programming languages such as Haskell try to systematically restrict environment-dependency and side-effects as much as possible.

\section{Parametric Polymorphism}

Many important data structures and algorithms are polymorphic in the following sense:

\begin{definition}[Polymorphism]
A \textbf{polymorphic data structure} $D$ is an operator that maps data structures $D_1,\ldots,D_n$ to a data structure $D[D_1,\ldots,D_n]$.

A \textbf{polymorphic algorithm} $F$ is an operator that maps data structures $D_1,\ldots,D_n$ to an algorithm $F[D_1,\ldots,D_n]$.

The $D_i$ are called the \textbf{type parameters} or \textbf{type arguments} of the data structure/algorithm.
\end{definition}

This is best understood by example:

\begin{example}[Lists]
Lists are a polymorphic data structure.
$A^*$ is the set of lists whose elements have type $A$.
Any data structure for $A^*$ should take $A$ as a type parameter.

For example, $List[A]$ may be a data structure such that $List[\Int]$ is the type of lists of integers.

Most algorithms about lists are polymorphic as well.
For example, reversing a list can be realized using an algorithm
\begin{acode}
\afun[{List[A]}]{reverse[A]}{x: List[A]}{
  \ldots
}
\end{acode}
\end{example}

\begin{terminology}
There are many different concepts of polymorphism that are (correctly, confusingly, or even wrongly) called \emph{polymorphism}.
The special kind described here is usually called \emph{parametric polymorphism}.

Both terminology and notations vary across programming languages, communities, and textbooks.
\end{terminology}

A more difficult example arises if we want to sort a list: To sort a list over $A$, we need a comparison function $\leq(x:A,y:A):\Bool$.
Moreover, $\leq$ has to be a total order.
We can handle that using abstract classes:

\begin{example}
Consider the following polymorphic abstract class for total orders:
\begin{acode}
\aclassA{TotOrd[A]}{}{}{}{
  \afun[\Bool]{lessOrEqual}{x:A,y:A}{}
}
\end{acode}
It requires a function $lessOrEqual$ that provides the comparison $\leq$.
The axioms for being a total order can usually not be programmed---they can only be added as part of the documentation.

Then a polymorphic sorting algorithm could look like
\begin{acode}
\afun[{List[A]}]{sort[A]}{ord: TotOrd[A], x:List[A]}{
\ldots
}
\end{acode}
\end{example}

\begin{notation}[Omitting Type Parameters]
Most of the time it is possible to omit the type parameters when calling a polymorphic function without ambiguity.
For example, if $l: List[\Int]$, we can simply say $revert(l)$ instead of $revert[\Int](l)$---both human readers and compiler can infer the type argument.

Most programming languages that allow polymorphism also allow omitting parameters if they can be inferred uniquely.
It is also allowed to do so in examples and pseudo-code.
\end{notation}

\subsubsection{In Programming Languages}

Even though polymorphism is relatively simple mathematically, not all programming languages do a good job of implementing it.
Therefore, we will often gloss over issues of polymorphism when giving algorithms.

But we give a few examples of polymorphism in a few typed programming languages.

\paragraph{Scala}
Scala's syntax is very similar to te pseudo-code used in these notes:

\begin{lstlisting}
abstract class TotOrd[A] {
  def lessOrEqual(x:A, y:A): Boolean
}

object IntSmaller extends TotOrd[Int] {
  def lessOrEqual(x:Int, y:Int): Boolean = x <= y
}

object Sort {
  def sort[A](ord: TotOrd[A], x: List[A]): List[A] = {
    ...
  }
}

object Test {
  def main(args: Array[String]) {
    sort[Int](IntSmaller, List(4,3,5))
  }
}
\end{lstlisting}

\paragraph{Java}
In Java, polymorphic data structures are called \emph{generics}.
It uses angular instead of square brackets and puts the parameter types of a polymorphic algorithm before the return type instead of after the name:

\begin{lstlisting}
interface TotOrd<A> {
  public Boolean lessOrEqual(A x, A y);
}

class Sort {
  static <A> List<A> sort(TotOrd<A> ord, List<A> x) {
    ...
  }
}

class IntSmaller implements TotOrd<Integer> {
  public Boolean lessOrEqual(Integer x, Integer y) {
    return x <= y;
  }
  public static IntSmaller it = new IntSmaller();
}

class Test {
  public static void main (String[] args) {
    Sort.sort(IntSmaller.it, Arrays.asList(3,5,4));
  }
}
\end{lstlisting}

\paragraph{C++}
In C++, we can use templates to implement polymorphism.
C++ also uses angular brackets, and the parameter types of classes and functions must be declared using the \lstinline|template| keyword.

\begin{lstlisting}
using namespace std;
#include <list>

template <class A>
class TotOrd {
  bool lessOrEqual(A x, A y);
};

class IntSmaller: public TotOrd<int> {
  bool lessOrEqual(int x, int y) {return x <= y;}
};
IntSmaller* is = new IntSmaller();
  
template <class A>
list<A> sort(TotOrd<A> ord, list<A> x) {
  ...
};

int test() {
    sort<int>(*is, {3,5,4});  
}
\end{lstlisting}

\paragraph{SML}
In SML, we do not have abstract classes, but we can use a datatype instead.
The type parameters of polymorphic types and functions are not declared explicitly.
Instead, they are implicit given as variables like \lstinline|'a|.

\begin{lstlisting}
datatype 'a TotOrder = TotOrder of 'a * 'a -> bool
fun lessOrEqual(ord: 'a TotOrder): 'a * 'a -> bool = case ord of TotOrder(f) => f

val IntSmaller: int TotOrder = TotOrder(fn (x,y) => x <= y)

fun sort(ord: 'a TotOrder, x: 'a list) = ...

fun test() = sort(IntSmaller, [3,5,4])
\end{lstlisting}


\chapter{Design Goals}\label{sec:ad:goals}
 % exercise: Ex.~\ref{ex:ad:quadraticstringprocessor}
  \section{Correctness}

\subsection{General Definition}

The most important goal of design is \emph{correctness}:

\begin{definition}\label{def:ad:correct}
We say that:
\begin{compactitem}
 \item A data structure $D$ is correct for a set $S$ if the objects of $D$ correspond exactly to the elements of $S$.
 \item An algorithm $A$ is correct for a function $F$ if for every possible input $x$ the result of running $A$ on $x$ has output $F(x)$.
\end{compactitem}
\end{definition}

\subsubsection{Data Structures}

Obviously, an incorrect algorithm is simply a bug.%
\footnote{However, there are advanced areas of computer science that study approximation algorithms.
For example, we may want to use a fast algorithm that is almost correct for a function for which no fast algorithm exists.}

However, incorrect data structures are often used.

\begin{example}
The data structure $\Int$ is not correct for the sets $\N$ or the $\Z$.
In both cases, $\Int$ has not enough objects.
$\Int$ even has objects that are not in $\N$ at all (namely negative numbers).

However, $\Int$ is routinely used in practice as if it were a correct data structure for $\N$ and $\Z$.
If $\Int$ uses $32$ bits, it only covers the numbers between $-2^{31}$ and $2^{31}-1$.
As long as all involved numbers are between $-2^{31}$ and $2^{31}$, this is no problem.

It is possible to define correct data structure for $\N$ and $\Z$.
But that can be disadvantageous because
\begin{compactitem}
\item operations on $\Int$ are much faster,
\item interfacing with other program components may be difficult if they use different data structures.
\end{compactitem}
\end{example}

\begin{example}
There is no data structure that is correct for $\R$.

Therefore, the data structure $\Float$ is used in practice as if it were a correct data structure for $\R$.
This always leads to rounding errors so that all results about $\Float$ are only approximate.

$\Float$ is often also used as if it were a correct data structure for $\Q$.
That is a bad habit because computations on $\Float$ are only approximate even if the inputs are exact.
For example, there is no guarantee that $1.0/2.0$ returns $0.5$ and not $0.49999999999$.
\end{example}

\begin{example}
Object-oriented languages use class types.
Because of the $null$ pointer, a class $A$ that implements a set $S$ actually implements the set $S^?$---a value of type $A$ can be $null$ or an instance of $A$.

Therefore, many good programmers systematically avoid ever using $null$.
Still, the use of $null$ is wide-spread in practice.
\end{example}

\begin{example}
Assume we have a correct data structure for $A$.
\medskip

Then we can give a correct data structure for $\{x\in A|P(x)\}$ if $P\in A \to \B$ is computable.
However, because the set of computable functions is itself not decidable, programming languages usually do not allow defining correct data structures for $\{x\in A|P(x)\}$.

More severely, we cannot in general give a correct data structure for $\{F(x):x\in A\}$ at all.
Even if $F$ is computable, we cannot give an algorithm that determines whether a given object is in that set.

Neither can we give a correct data structure for $A/r$ for $r\in A\times A\to \B$.
Even if $r$ is computable, we cannot give an algorithm for equality of elements of $A/r$.
\end{example}

\subsubsection{Algorithms}

The process of making sure that an algorithm is correct is called \emph{verification}.
Verification is very difficult.
In particular, the function that determines whether a data structure or algorithm is correct is itself not computable.
Therefore, we have to prove the correctness of each data structure or algorithm individually.

Good programmers design algorithms that are close to the specification.
That makes it easier to verify the design.

To make verification more systematic, we usually split the specification into two parts: precondition and postcondition.
Independently, we split the verification arguments into two independent steps: termination and partial correctness.
The definitions are as follows:

\begin{definition}\label{def:ad:correct2}
Consider an algorithm $A$ for a function $f(x_1\in I_1,\ldots,x_n\in I_n)\in O$.

We define:
\begin{compactitem}
\item A \textbf{precondition} for $A$ is a formula $Pre(x_1,\ldots,x_n)$ about the inputs.
\item A \textbf{postcondition} for $A$ is a formula $Post(x_1,\ldots,x_n,r)$ about the inputs and the output.
\item $A$ \textbf{terminates for} $v_1,\ldots,v_n$ if running $A$ with these inputs finishes in finitely many steps.
\item $A$ \textbf{terminates} if it terminates whenever $Pre(v_1,\ldots,v_n)$.
\item $A$ is \textbf{partially correct} if for all $v_1,\ldots,v_n$
\begin{compactitem}
 \item if $Pre(v_1,\ldots,v_n)$ and
 \item $A$ terminates for $v_1,\ldots,v_n$ with return value $r$, then
 \item $Post(v_1,\ldots,v_n,r)$
\end{compactitem}
\item $A$ is \textbf{totally correct} it is partially correct and terminates.
\end{compactitem}

Finally we can recover Def.~\ref{def:ad:correct} by saying that $A$ is a correct algorithm for a function $f$ it is totally correct with 
\begin{compactitem}
 \item precondition: nothing (always true)
 \item postcondition: $r==f(x_1,\ldots,x_n)$
\end{compactitem}
\end{definition}

The reason for splitting correctness up is that partial correctness and termination are often proved separately in very different ways.
 So it is good to have separate definitions for them.
 Sect.~\ref{sec:ad:loopinv} and~\ref{sec:ad:termord} describe the most important techniques.

The reason for splitting the specification into pre- and postcondition is to make fine-granular statements about what input an algorithm expects and what output it provides.
They can be seen as a trade between the programmer W who writes function F and the programmer C who calls F.
The precondition is the price that C has to pay (by making sure the precondition holds before calling F).
And the postcondition in the service that W provides in exchange (by returning a value that satisfies the postcondition).

In particular, W may assume that the precondition holds---she does not have to check it.
Instead, C has to check it.
Vice versa, C may assume that the postcondition holds afterwards.

\begin{example}[Pre/Postcondition]
Consider a variant $gcd32(x:\Int,y:\Int):\Int$ of the Euclidean algorithm that uses $32$-bit integers.
This can never be correct because it cannot handle arbitrarily large natural numbers.
Moreover, the input and output type now allow negative values, which we want to exclude.
 
So we could use the following:
\begin{compactitem}
\item precondition: $Pre(x,y)\;=\;0\leq x \leq MaxInt \;\wedge\; 0\leq y \leq MaxInt$
\item postcondition: $Post(x,y,r)\;=\;0\leq r\leq MaxInt\;\wedge\; r==\gcd(x,y)$
\end{compactitem}
where $MaxInt$ is the maximal value of the type $\Int$.

Note that this specification makes the strong requirement that there will be no overflows.
That works out for the Euclidean algorithm because all its intermediate results are smaller than the input.

For other algorithms, like a 32-bit algorithm $fib32(n:\Int):\Int$ for Fibonacci numbers, the input has to be much smaller than $MaxInt$ to make sure the output fits into a $32$-bit integer.
So we might use:
\begin{compactitem}
\item precondition: $Pre(n)\;=\;0\leq x \leq 46$
\item postcondition: $Post(n,r)\;=\;r==fib(n)$
\end{compactitem}
\end{example}

\subsection{Partial Correctness}\label{sec:ad:loopinv}

\subsubsection{Loop Invariants for while-Loops}

Many algorithms use while-loops.
Verifying the correctness of while-loops is notoriously difficult.

Therefore, many good programmers try to avoid while-loops altogether.
Instead, they prefer operations on lists (like $map$, $fold$, and $foreach$) or recursive algorithms.
\medskip

The central method for verifying the partial correctness of a while-loop is the \emph{loop invariant}:

\begin{definition}[Loop Invariant]\label{def:ad:loopinv}
Consider a loop of the form $\awhileI{C(\vec{x})}{code}$.
Here $\vec{x}=(x_1,\ldots,x_n)$ are the names that are in scope before entering the loop (i.e., excluding any names declared only in $code$).

A formula $F(\vec{x})$ is a \textbf{loop invariant} for this loop if $F$ is preserved by the loop: if $F$ holds before executing $code$, it also holds afterwards.
 Specifically, for all $\vec{v}$, the following must hold
   \[C(\vec{v}) \mand F(\vec{v}) \tb \mimplies\tb F(code(\vec{v}))\]
   where $code(v)=(v'_1,\ldots,v'_n)$ contains the values of the $x_i$ after executing $x_1:=v_1; \ldots; x_n:=v_n; code$.
\end{definition}

If we have a loop invariant, we can use it as follows:
\begin{theorem}\label{thm:ad:loopinv}
Consider a loop $\awhileI{C(\vec{x})}{code}$ with a loop invariant $F(\vec{x})$.\\
Assume that $F(\vec{v})$ holds where $v_i$ is the value of $x_i$ before executing the while-loop.
\medskip

Then $\neg C(\vec{x}) \wedge F(\vec{x})$ holds if and when the while-loop has been executed.\footnotemark
\end{theorem}
\begin{proof}
After the while-loop $C(\vec{x})$ cannot hold---otherwise, the while-loop would continue.
Because $F(\vec{x})$ held before executing the loop and is preserved by every iteration of $code$, it also holds after executing the loop.
\end{proof}
\footnotetext{We assume here that the evaluation of $C(\vec{x})$ has no side-effects and thus may not change the values of the $x_i$.
In most programming languages, that would be allowed, but is a very bad practice precisely because it makes loop-invariant arguments more complicated.}

Note that Thm.~\ref{thm:ad:loopinv} says \emph{if and when} the while-loop has been executed.
That is because it is not guaranteed that the while-loop terminates.
We still have to prove termination separately.

\begin{example}[Euclidean Algorithm]\label{ex:ad:euclid:partcorr}
We prove partial correctness of the algorithm from Ex.~\ref{ex:ad:euclid}.
We proceed statement-by-statement.
\medskip

The first two statements are easy to handle: Their effect is that $x==m$ and $y==n$.
\medskip

But now we reach a while-loop.
We have $\vec{x}=(m,n,x,y)$ and $C(m,n,x,y)=x\neq y$.
A loop invariant is given by $F(m,n,x,y)\;=\;\gcd(m,n)==\gcd(x,y)$.
The intuition of this loop-invariant is that we only apply operations to $x$ and $y$ that do not change their $\gcd$.
\medskip

To work with the while-loop, we prove that $F$ is a loop invariant:
\begin{compactitem}
 \item We show that $F$ holds before the loop. \\ Before reaching the loop, we have $x==m$ and $y==n$. Thus, immediately $gcd(m,n)==gcd(x,y)$.
 \item We show that $F$ is preserved by the loop. \\ Let us assume that $C(m,n,x,y)$ holds, i.e., $x\neq y$ (i).\\
  Moreover, let us assume that $F(m,n,x,y)$ holds, i.e., $\gcd(m,n)==\gcd(x,y)$ (ii).\\
  Let $code(m,n,x,y)=(m',n',x',y')$.\\
  We have to prove $F(m',n',x',y')$, i.e., $\gcd(m,n)=\gcd(x',y')$.\\
  To do that, we have to distinguish two cases according to the if-statement:
  \begin{compactitem}
   \item $x<y$: Then $(m',n',x',y') = (m,n,x,y-x)$.
   Thus we have to prove that $\gcd(m,n)=\gcd(x,y-x)$.\\
   Because of (ii), it is sufficient to prove $\gcd(x,y)=\gcd(x,y-x)$.
   That follows from the mathematical properties of $\gcd$.
   \item $y<x$: Then $(m',n',x',y') = (m,n,x-y,y)$.
   We have to prove that $\gcd(m,n)=\gcd(x-y,x)$.\\
   That follows in the same way as in the first case.
   \item We do not need a case for $x==y$ because that is excluded by (i).
  \end{compactitem}
\end{compactitem}
\medskip

Now we can continue.
The next statement is $\areturn{x}$.
Using Thm.~\ref{thm:ad:loopinv}, we obtain that $\neg C(m,n,x,y)\wedge F(m,n,x,y)$ holds, i.e., $\neg x\neq y \wedge \gcd(m,n)==\gcd(x,y)$.
That yields $x==y$ and therefore $\gcd(m,n)==\gcd(x,x)==x$.
Thus, the returned value is indeed $\gcd(m,n)$.

To prove total correctness, we still have to show that the while-loop terminates, which we do in Ex.~\ref{ex:ad:euclid:term}
\end{example}

\subsubsection{Induction for Recursive Functions}

Proving partial correctness of recursive functions is very easy because we can simply use the postcondition about the recursive call.
Formally, this means we do an induction proof on the number of recursive calls.

\begin{example}[Recursive Euclidean Algorithm]\label{ex:ad:euclid2:partcorr}
We prove partial correctness for the algorithm $gcdRec(m:\N,n:\N)$ from Ex.~\ref{ex:ad:euclid2}.

We have to prove the postcondition $gcdRec(m,n)==\gcd(m,n)$ where $r$ is the return value.
We proceed by induction, i.e., we assume that the property holds for all recursive calls.
Then we have to handle two cases for the two branches of the if-statement:
\begin{compactitem}
 \item $n==0$: Then $gcdRec(m,n)=m$, and the postcondition follows from $\gcd(m,0)==m$.
 \item $n\neq 0$: Then, by using the induction hypothesis, $gcdRec(n, m\modop n)==\gcd(n, m\modop n)$.
   Then the postcondition follows from $\gcd(m,n)==\gcd(n,m\modop n)$.
\end{compactitem}

To prove total correctness, we still have to show that the recursion terminates which we in Ex.~\ref{ex:ad:euclid2:term}.
\end{example}

\subsection{Termination}\label{sec:ad:termord}

Verifying the termination of an algorithm is also very hard.
The halting function is the function that takes as input an algorithm $A$ and an object $I$ and returns as output the following boolean: $\true$ if $A$ terminates with input $I$ and $\false$ otherwise.
One of the most important results of theoretical computer science is that the halting function is not computable, i.e., there is no algorithm for it.

Thus, even if do not care what our algorithm actually does and only want to know if it terminates at all, all we can do is prove it manually for each input.

Termination is trivial for assignment, for-loop\footnote{In some programming languages, it is possible to write non-terminating for-loops by explicitly assigning to the counter variable in the body of the loop. That is a very bad practice precisely because it endangers termination.}, if-statement, and the return-statement.
Only while-loops and recursion are tricky.
The most important technique to prove termination is to use a termination ordering.

\subsubsection{Termination Orderings for While-Loops}

\begin{definition}[Termination Ordering]\label{sec:def:termord}
Consider a while-loop of the form $\awhileI{C(\vec{x})}{code}$.

A \textbf{termination ordering} for it is a function $T(\vec{x})\in\N$ such that for all $\vec{v}$ we have that
\[C(\vec{v}) \tb\mimplies\tb T(\vec{v})>T(code(\vec{v})).\]
\end{definition}

The intuition behind a termination ordering is that $T(\vec{x})$ strictly decreases in every iteration of the loop.
Because it cannot decrease indefinitely, there can only be finitely many iterations, i.e., the loop must terminate.
The following theorem makes that precise:

\begin{theorem}[Termination Ordering]\label{sec:thm:termord}
Consider a the loop $\awhileI{C(\vec{x})}{code}$ and a termination ordering $T(\vec{x})$ for it.

Then the while-loop terminates for all initial values $\vec{v}$ of $\vec{x}$.
\end{theorem}
\begin{proof}
We define a sequence $\vec{v}^0, \vec{v}^1, \ldots$ such that $\vec{v}^i$ contains the values of $\vec{x}$ after $i$ iterations of executing $code$:
\[\vec{v}^0=\vec{v}\]
\[\vec{v}^{i+1}=code(\vec{v}^i) \tb \mfor i>0\]
\medskip

We use an indirect proof: We assume the while-loop does not terminate and show a contradiction.\\
If the loop does not terminate, the condition must always be true, i.e., $C(\vec{v}^i)$ for all $i\in\N$.\\
Then the termination ordering yields $T(\vec{v}^i)>T(\vec{v}^{i+1})$ for all $i\in\N$.\\
That yields an infinite sequence $T(\vec{v}^0) > T(\vec{v}^1) > \ldots $ of natural numbers.\\
But such a sequence cannot exist, which yields the needed contradiction.
\end{proof}


\begin{example}[Euclidean Algorithm]\label{ex:ad:euclid:term}
We prove that the algorithm from Ex.~\ref{ex:ad:euclid} terminates for all inputs.
Only the while-loop presents a problem.

A termination ordering for the while-loop is given by $T(m,n,x,y)=x+y$.
The intuition of this termination ordering is that the loop makes either $x$ or $y$ smaller.
Therefore, it must make their sum smaller.
\medskip

We show that $T$ is indeed a termination ordering.\\
As when proving the loop-invariant, we put $(m',n',x',y')=code(m,n,x,y)$.\\
We have to show that $T(m,n,x,y)>T(m',n',x',y')$, i.e., $x+y>x'+y'$.\\
We again distinguish two cases according to the if-statement:
\begin{compactitem}
 \item $x<y$ and thus $(m',n',x',y')=(m,n,x,y-x)$: We have to show $x+y>x+y-x$.
 \item $x>y$ and thus $(m',n',x',y')=(m,n,x-y,y)$: We have to show $x+y>x-y+y$.
\end{compactitem}
Both cases are trivially true for all $x,y\in\N\sm\{0\}$.

But what happens if $x==0$ or $y==0$?
Indeed, the proof of the termination ordering property does not go through.\\
Inspecting the algorithm again, we realize that we have found a bug: If exactly one of the two inputs is $0$, the algorithm never terminates.

We can fix the algorithm in two ways:
\begin{compactitem}
 \item We change the specification to match the behavior of the algorithm.
 That means to change the input data structure such that $m,n\in\N\sm\{0\}$.
 \item We change the algorithm to match the specification.
  We can do that by adding the lines
  \begin{acode}
    \aifI{x==0}{\areturn{y}}\\
    \aifI{y==0}{\areturn{x}}
  \end{acode}
  Now the loop can be analyzed with the assumption that $x\neq 0$ and $y\neq 0$.
\end{compactitem}
\end{example}

\subsubsection{Termination Orderings for Recursion}

Termination orderings for recursion work in essentially the same way.
But the precise definition is a little bit trickier.

\begin{definition}[Termination Ordering for Recursion]\label{sec:def:termord:rec}
Consider a recursive function $f(\vec{x})$.

A \textbf{termination ordering} for $f$ is a function $T(\vec{x})\in\N$ such that: whenever $f$ is called with arguments $\vec{v}$ and recursively calls itself with arguments $\vec{v'}$, then $T(\vec{v})>T(\vec{v'})$.
\end{definition}

Then we can prove the corresponding theorem:

\begin{definition}[Relative Termination]\label{sec:def:termord:relterm}
Consider a recursive function $f(\vec{x})$.

We say that $f$ \textbf{terminates relatively} if the following holds: $f$ terminates for all arguments under the assumption that all recursive calls terminate.
\end{definition}

\begin{theorem}[Termination Ordering for Recursion]\label{sec:thm:termord:rec}
Consider a recursive function $f(\vec{x})$ with a termination ordering $T$ for it.

If $f$ terminates relatively, then it terminates for all arguments.
\end{theorem}
\begin{proof}
This is proved in the same way as for while-loops.
\end{proof}

\begin{example}[Recursive Euclidean Algorithm]\label{ex:ad:euclid2:term}
Consider the recursive algorithm from Ex.~\ref{ex:ad:euclid2}.

It is easy to see that the arguments never get bigger during the recursion.
So we might try $T(m,n)=m+n$ as a termination ordering.
But that does not work because if $m<n$, the recursive call is to $gcd(n,m)$, which just flips the arguments.
In that case, $T(m,n)=m+n$ does not become strictly smaller.

It becomes easier to show termination if we expand the recursive call once.
That yields the equivalent function:
\begin{acode}
\afun[\N]{gcd}{m:\N, n:\N}{
  \aifelse{n==0}{m}{
    \aifelse{m\modop n==0}{n}{\gcd(m\modop n, n\modop(m\modop n))}
  }
}
\end{acode}

Relative termination is trivial either way: Under the assumption that the recursive call returns, the function consists only of if-statements and therefore terminates.

And for the expanded function, $T(m,n)=m+n$ is a termination ordering.
We have to prove $m+n>(m\modop n)+(n\modop(m\modop n))$, which is easy to see.
\end{example}

\subsection{Implementing Loop Invariants and Termination Orderings}

Loop invariants and termination orderings an be tricky to understand for beginners.
Therefore, the following gives a more concrete explanation.

Consider an arbitrary algorithm that uses a while-loop, e.g.,
\begin{acode}
\afun[\N]{fact}{n:\N}{
product := 1\\
factor  := 1 \\
%% loop invariant n! = product * factor * ... * n
\awhile{factor \leq n}{
  product := product \cdot factor\\
  factor := factor+1
}\\
\areturn{product}
}
\end{acode}

To exemplify the role of a termination ordering, we modify it as follows:
\begin{acode}
\afun[\N]{T}{n:\N,product:\N,factor:\N}{
???
}\\
\\
\afun[\N]{fact}{n:\N}{
product := 1\\
factor  := 1 \\
\fbox{$\aprint{T(n,product,factor)}$}\\
\awhile{factor \leq n}{
  product := product \cdot factor\\
  factor := factor+1\\
  \fbox{$\aprint{T(n,product,factor)}$}
}\\
\areturn{product}
}
\end{acode}

Our goal is to implement $T$ such that running the algorithm prints strictly decreasing natural numbers.
Any such implementation of $T$ is a termination ordering and proves that the while loop terminates.
\medskip

To exemplify the role of a loop invariant, we modify the algorithm in a very similar way:
\begin{acode}
\afun[\Bool]{F}{n:\N,product:\N,factor:\N}{
???
}\\
\\
\afun[\N]{fact}{n:\N}{
product := 1\\
factor  := 1 \\
\fbox{$\aprint{F(n,product,factor)}$}\\
\awhile{factor \leq n}{
  product := product \cdot factor\\
  factor := factor+1\\
  \fbox{$\aprint{F(n,product,factor)}$}
}\\
\areturn{product}
}
\end{acode}

Our goal is to implement $F$ such that running the algorithm prints only $\true$.
In that case, 
\begin{compactitem}
 \item $F$ is true before the loop
 \item $F$ is a loop invariant, i.e., if it is true before, it is also true after executing the body of the loop.
\end{compactitem}
Thus, if the while-loop should terminate, afterwards $F$ must be true and the condition of the loop must be false.

There are many possible ways to implement $F$---already $\areturn{\true}$ trivially satisfies the requirements.
A practically useful implementation of $F$ should tell us something that helps establish the postcondition (which in this case is $fact(n)==n!$).


\section{Efficiency}\label{sec:ad:complex}

An algorithm is efficient if it can be run with low cost.
\emph{Complexity} measures that cost.\footnote{At Jacobs University, complexity is discussed in detail in a special course in the $2nd$ year.}
Thus, an efficient algorithm has low complexity and vice versa.

There are two kinds of complexity: \emph{time} and \emph{space} complexity.
Time complexity measures how long it takes for an algorithm to terminate.
Space complexity measures how much temporary memory is needed along the way.
Without qualification, the word \emph{complexity} usually but not always means \emph{time complexity}

In this section, we focus on time complexity.
While termination describes whether an algorithm $A$ terminates at all, its time complexity describes how long it takes to terminate.
The time complexity of $A$ is a function $C:\N\to\N$ such that $C(n)$ is the number of steps needed until $A$ terminates for input of size $n$.

\subsection{Exact Complexity}\label{sec:ad:complex:general}

Exact complexity is tricky because the number of steps and the sizes of inputs depend on the programming language and the physical machine that is used.
For example, we might try to use the following definitions for a simple programming language:

\begin{example}[Counting Steps Exactly]\label{ex:ad:complex:steps}
For a typical programming language implemented on a digital machine, the following definition is roughly right:\\

For the execution of a statement:
\begin{itemize}
 \item $\compl(\aseq{C,D})=\compl(C)+\compl(D)$
 \item $\compl(x:=E)=\compl(E)+1$
   \begin{compactitem}
     \item $\compl(E)$ steps to evaluate the expression $E$
     \item $1$ step to make the assignment
   \end{compactitem}
 \item $\compl(\areturn{E})=\compl(E)+1$
   \begin{compactitem}
     \item $\compl(E)$ steps to evaluate the expression $E$
     \item $1$ step to return
   \end{compactitem}
 \item $\compl(\aifelseI{C}{T}{E})=\compl(C)+1+\cas{\compl(T) \mifc C==\true \\ \compl(E)\mifc C==\false}$
   \begin{compactitem}
     \item $\compl(C)$ steps to evaluate the condition
     \item $1$ step to branch
     \item $\compl(T)$ or $\compl(E)$ steps depending on the branch
   \end{compactitem}
 \item $\compl(\awhileI{C}{B})=(n+1)\cdot\compl(C)+n\cdot\compl(B)$ where $n$ is the number of times that the loop is repeated
   \begin{compactitem}
     \item $\compl(C)$ steps to evaluate the condition $n+1$ times
     \item $1$ step to branch after each evaluation of the condition
     \item $\compl(B)$ steps to execute the body
   \end{compactitem}
\end{itemize}
\medskip

For the evaluation of an expression:
\begin{compactitem}
 \item Retrieving a variable: $\compl(x)=1$
 \item Applying built-in operators $O$ such as $+$ or $\&\&$: $\compl(O(E_1,\ldots, E_n)=\compl(E_1)+\ldots+\compl(E_n)+1$
  \begin{compactitem}
    \item $\compl(E_i)$ steps to evaluate the arguments
    \item $1$ step to apply the operator
  \end{compactitem}  
 \item Calling a function: $\compl(f(E_1,\ldots,E_n))=\compl(E_1)+\ldots+\compl(E_n)+1+n$
  \begin{compactitem}
    \item $\compl(E_i)$ steps to evaluate the arguments
    \item $1$ step to create jump into the definition of $f$
    \item $1$ step each to pass the arguments to $f$
  \end{compactitem}
\end{compactitem}

The size of an object depends on the data structure:
\begin{compactitem}
  \item For $\Int$, $\Float$, $\Char$, and $\B$, the size is $1$.
  \item For $\String$, the size is the length of the string.
  \item For lists, the size is the sum of the sizes of the elements plus $1$ more for each element.
   The ``$1$ more'' is needed because each element needs a pointer to the next element of the list.
\end{compactitem}
\end{example}

In actuality however, a number of subtleties about the implementation of the programming language, its compiler, and the physical machine can affect the run-time of a program.
For example:
\begin{compactitem}
 \item We usually assume that all arithmetic operations take $1$ step.
   But actually, that only applies to arithmetic operations on the type $\Int$ of $32$ or $64$-bit integers.
  \begin{compactitem}
    \item Any arithmetic operation that can handle arbitrarily large numbers takes longer for larger numbers.
     Most such arithmetic operations have complexity closely related to the number of digits needed to represent the arguments.
     That number is logarithmic in the size of the arguments.
    \item Multiplication and related operations usually take longer than addition and related operations.
    Similarly, exponentiation usually takes longer than multiplication.
    \item Any operation not built into the hardware must be implemented using software, which makes it take longer.
     Operations on larger numbers may take longer even if they are of type $\Int$.
  \end{compactitem} 
 \item Floating point operations may take more than $1$ step.
 \item The programming language may provide built-in operations that are actually just abbreviations for non-trivial functions.
  For example, concatenation of strings usually require copying one or both of the strings, which takes at least $1$ step for each character.
  In that case, concatenating longer strings takes longer.
 \item The programming language's compiler may perform arbitrary optimizations in order to make execution faster.
  For example, we may have $\compl(\aifI{\false}{E})=0$ because the compiler removes the statement entirely.
  On the other hand, optimization may occasionally use a bad trade-off and make execution slower.
 \item A smart compiler may generate code that is optimize for multi-core machines, such that, e.g., $2$ steps are executed in $1$ step.
 \item Calling a function may take much more than $1$ step to jump to the function.
  Usually, it requires memory allocation, which can be a complex operation.
 \item For advanced operations, like instantiating a class, it is essentially unpredictable how many steps are required.
 \item From a complexity perspective, IO-operations (printing, networking, file access, etc.) take as many steps as the size of the sent data.
 But they take much more time than anything else.
\end{compactitem}

The dependency of exact complexity on programming language, implementation, and physical machine is awkward because it precludes analyzing an algorithm independent of its implementation.
Therefore, it is common to work with asymptotic complexity instead.

The idea is that dependencies are usually harmless in the sense that they can be ``rounded away''.
For example, it does not matter much whether $\compl(x:=E)=\compl(E)+1$ or $\compl(x:=E)=\compl(E)+2$.
It just means that every program takes a little longer.
It would matter more if $\compl(x:=E)=2\cdot\compl(E)+1$, which is unlikely.

We introduce the formal definitions in Sect.~\ref{sec:ad:onot} and apply them in Sect.~\ref{sec:ad:asympana}.

\subsection{Asymptotic Notation}\label{sec:ad:onot}

The field of complexity theory usually works with with Bachmann-Landau notations.%
\footnote{In the definition below, only $O$, $\Omega$, and $\Theta$ are the standard Bachmann–Landau notations. The symbols $\Oleq{}{}$ and $\Oeq{}{}$ are specific to these lecture notes.}
The basic idea is to focus on the rough shape of the function $C(n)$ instead of its details.
For example, $C(n)=an+b$ is linear, and $C(n)=2^{an+b}$ is exponential.
The distinction linear vs. exponential is often much more important than the distinction $an+b$ vs. $a'n+b'$. 

Therefore, we define classes of functions like linear, exponential, etc.:

\begin{definition}[O-Notation]\label{def:ad:onot}
Let $\R^+$ be the set of positive-or-zero real numbers.

We define a relation on functions $f,g:\N\to\R^+$ by
\[\Oleq{f}{g} \tb\miff\tb \exists N\in\N.\;\exists k>0.\;  \forall n>N.\; f(n)\leq k\cdot g(n)\]
If $\Oleq{f}{g}$, we say that $f$ is \textbf{asymptotically smaller} than $g$.

We write $\Oeq{f}{g}$ if $\Oleq{f}{g}$ and $\Oleq{g}{f}$.

Moreover, for a function $g:\N\to\R^+$, we define the following sets of functions
\[O(g) = \{f:\N\to\R^+\,|\, \Oleq{f}{g}\}\]
\[\Omega(g) = \{h:\N\to\R^+\,|\, \Oleq{g}{h}\}\]
\[\Theta(g) = \{f:\N\to\R^+\,|\, \Oeq{f}{g} \} = O(g)\cap\Omega(g)\]
\end{definition}

Intuitively, $\Oleq{f}{g}$ means that $f$ is essentially smaller than $g$.
More precisely, $f$ is smaller than $g$ \emph{for sufficiently large arguments} and \emph{up to a constant factor}.
The other definitions are straightforward: $O(g)$ is the set of everything smaller than $g$, $\Omega(g)$ is the set of everything larger than $g$, and
$\Theta(g)$ is the set of everything essentially as great as $g$ (i.e., both smaller and larger).

\begin{remark}[A Slightly Simpler Definition]
The following statement is not true in general.
However, it is easier to remember and true for all functions that come up when analyzing algorithms:
$\Oleq{f}{g}$ iff $\exists a>0.\exists b>0.\forall n. f(n)\leq a\cdot g(n)+b$.

We can verbalize that condition as ``$f$ is smaller than $g$ except for a constant factor and a constant summand''.
Those are the two aspects of run time that we can typically make up for by building faster machines.
\end{remark}

\begin{example}[Complexity Classes]\label{ex:ad:onot}
Now we can easily define some important classes of functions grouped by their rough shape:
\begin{compactitem}
\item $\Theta(1)$ is the set of $(\ast)$ constant functions
\item $\Theta(n)$ is the set of $(\ast)$ linear functions
\item $\Theta(n^2)$ is the set of $(\ast)$ quadratic functions
\item and so on
\end{compactitem}
Technically, we should always insert ``asymptotically'' at $(\ast)$.
For example, $\Theta(n)$ contains not only the linear functions but also all functions whose shape is similar to linear when we go to infinity.
But that word is often omitted for brevity.

If we use $O$ instead of $\Theta$, we obtain the sets of \emph{at most} constant/linear/quadratic/etc. functions.
For example, $O(n)$ includes the constant functions whereas $\Theta(n)$ does not.

Similarly, if we use $\Omega$ instead of $\Theta$, we obtain the sets of \emph{at least} constant/linear/quadratic/etc. functions.
For example, $\Omega(n)$ includes the quadratic functions whereas $\Theta(n)$ does not.

Of particular importance in complexity analysis is the set of polynomial functions:
It includes all functions whose shape is similar to a polynomial.

The following table introduces a few more classes and arranges them by increasing size:
\begin{ctabular}{|c|l|}
\hline
$O(1)$ & constant\\
$O(\log_c\log_c n)$ & doubly logarithmic \\
$O(\log_c n)$ & logarithmic \\
$O(n)$ & linear \\
$O(n\log_c n)$ & quasi-linear \\
$O(n^2)$ & quadratic \\
$O(n^3)$ & cubic \\
\vdots & \vdots \\
$\Poly=\bigcup_{k\in\N} O(n^k)$ & polynomial \\
$\Exp=\bigcup_{f\in\Poly}O(c^{f(n)})$ & exponential \\
$\bigcup_{f\in\Exp}O(c^{f(n)})$ & doubly exponential \\
\hline
\end{ctabular}
Here $c>1$ is arbitrary---all choices yield the same classes of functions.

We also say sub-X for strictly lower and super-X for strictly greater complexity than $X$.
For example $\log_cn$ is sub-linear, and $n^2$ is super-linear.
\end{example}

The following theorem collects the basic properties of asymptotic notation:

\begin{theorem}[Asymptotic Notation]\label{thm:ad:onot}
We have the following properties for all $f,g,h,f',g'$:
\begin{compactitem}
 \item $\Oleq{}{}$ is
	\begin{compactitem}
	 \item reflexive: $\Oleq{f}{f}$
	 \item transitive: if $\Oleq{f}{g}$ and $\Oleq{g}{h}$, then $\Oleq{f}{h}$
	\end{compactitem}
  Thus, it is a preorder.
 \item If $\Oleq{f}{f'}$ and $\Oleq{g}{g'}$, then $\Oleq{}{}$ is preserved by
	\begin{compactitem}
	 \item addition: $\Oleq{f+g}{f'+g'}$
	 \item multiplication: $\Oleq{f\cdot g}{f'\cdot g'}$
	\end{compactitem}
 \item $\Oeq{}{}$ is
	\begin{compactitem}
	 \item reflexive: $\Oeq{f}{f}$
	 \item transitive: if $\Oeq{f}{g}$ and $\Oeq{g}{h}$, then $\Oeq{f}{h}$
	 \item symmetric: if $\Oeq{f}{g}$, then $\Oeq{g}{f}$
	\end{compactitem}
  Thus, it is an equivalence relation.
 \item The following are equivalent:
 	\begin{compactitem}
 	 \item $\Oleq{f}{g}$
 	 \item $O(f)\sq O(g)$
 	 \item $\Omega(f)\supseteq \Omega(g)$
 	 \item $f\in O(g)$
 	 \item $g\in \Omega(f)$
 	\end{compactitem}
  All statements express that $f$ is essentially smaller than $g$.
 \item The following are equivalent:
 	\begin{compactitem}
 	 \item $f\in\Theta(g)$
 	 \item $g\in\Theta(f)$
 	 \item $\Theta(f)=\Theta(g)$
 	\end{compactitem}
  All statements express that $f$ is essentially as great as $g$.  
\end{compactitem}
\end{theorem}
\begin{proof}
Exercise.
\end{proof}

% $\Oleq{f}{g}$ implies \exists ab.\forall n.f(n) \leq ag(n)+b
% the reverse almost holds: We have to assume $\Oleq{1}{g}$, i.e., that $g$ does not become arbitrarily small
% that is always true in CS in f,g are complexity functions---the only exception is the constant 0 functions

\begin{notation}\label{not:ad:onot}
The community has gotten used to using $O(f(n))$ as if it were a function.
If $f(n)-g(n)\in O(r(n))$, it is common to write $f(n)=g(n)+O(r(n))$.
The intuition is that $f$ arises by adding some function in $O(r(n))$ to $g$.
This is usually done when $r$ is smaller than $g$, i.e., $r$ is a rest that can be discarded.

Similarly, people often write $f=O(r(n))$ instead of $f\in O(r(n))$ to express that $f$ is equal to some function in $O(r(n))$.
\medskip

These notations are not technically correct and should generally be avoided.
But they are often convenient.
\end{notation}

\begin{example}\label{ex:ad:onot2}
Using Not.~\ref{not:ad:onot}, we can write
 $2^n+5n^2+3=2^n+O(n^2)$.
This expresses that $2^n$ is the dominating term and the polynomial rest can be rounded away.

Or we can write $6n^3+5n^2+\log n=O(n^3)$.
\end{example}

\begin{remark}[Other Notations]
There are a few more notations like $O$, $\Omega$, and $\Theta$.
They include $o$ and $\omega$.
They are less important and are omitted here to avoid confusion.
\end{remark}

\subsection{Asymptotic Complexity}\label{sec:ad:asympana}

Equipped with asymptotic notations, we can now compute the run time of algorithms in a way that is mostly independent of the implementation and the machine.

\begin{example}\label{ex:ad:factorial:complex}
Consider the algorithm from Ex.~\ref{ex:ad:factorial}.
Let $C(n)$ be the number of steps it takes with input $n$.

Because we are only interested in the complexity class of $C$, this is quite simple:
\begin{compactenum}
\item The while-loop must be repeated $n$-times. So the algorithm is at least linear.
\item Each iteration of the while-loop requires one comparison, one multiplication, and two assignments.
 These operations take a constant number $c$ of steps.\footnote{Actually, of course, that depends on how they are implemented and whether $product$ becomes larger than the largest $\Int$. In general, arithmetic of larger numbers may take longer.}\\
 So the entire loop takes $c\cdot n$ steps. The value of $c$ does not matter because we can ignore all constant factors. Thus, the entire loop takes $\Theta(n)$ steps.
\item The assignments in the first two lines and the return statement take constant time each.
Because $C(n)$ is at least linear, we can ignore them entirely.
\item Thus, we obtain $C(n)\in \Theta(n)$ as the complexity class of the algorithm.
\end{compactenum}
\end{example}

Note how all the subtleties described in Sect.~\ref{sec:ad:complex:general} are rounded away by looking at $\Theta$-classes.
\medskip

In many cases, the run time primarily depends on the \emph{size} of the input, i.e., the exact choice of input does not matter as long as we know its size.
Therefore, complexity of an algorithm $A$ is usually measured as a function $C(n)$ that returns the number that $A$ will taken when called on input of size $n$.
\begin{compactitem}
  \item If the input is an integer $x\in \Z$, its size is $n=\log |x|$, which is the number of bits needed to represent $x$.
  \item If the input is a list, its size is usually the length of the list. But sometimes it may matter how big the elements of the list are.
  \item If there are multiple inputs, the size is often the sum of the sizes. But we may also use a function $C(m,n)$ that takes two sizes as arguments.
\end{compactitem}

But sometimes different inputs of the same size make lead to very different run times.
For example, Ex.~\ref{ex:ad:euclid} happens to terminate immediately if the inputs are equal (no matter what the size they have) but takes very long in other cases (specifically when they are consecutive Fibonacci numbers).\\
Thus, we have to distinguish between:
 \begin{compactitem}
  \item worst-case complexity $C_w(n)$: This is the maximal possible number of steps for input of size $n$. If there is no additional information, this is usually what the author means.
  \item average-case complexity $C_a(n)$: This is the average number of steps for input of size $n$. This is more useful in practice, but it is more difficult because we need a probabilistic analysis to compute the average.
  \item best-case complexity: This is the minimal possible numbeer of steps for input of size $n$. This is rarely useful but occasionally helps put a lower bound on the complexity.
\end{compactitem}
There is no universal convention how these details are formalized.
Instead, we often have to consider the context to understand what the author means.

\begin{example}[Euclidean Algorithm]\label{ex:ad:euclid:complex}
Consider the algorithm from Ex.~\ref{ex:ad:euclid}.
Let $n=\max(a,b)$ and let $C(n)$ be the worst-case number of steps the algorithm takes for input $a,b$ (i.e., we use the maximum value of the inputs as the argument of the complexity function).

It is not that easy to see what the worst case actually is.
But we can immediately see that the loop is repeated at most $n$ times.
Each iteration requires one comparison, one subtraction, and one assignment, which we can sum up to a constant factor.{\footnotemark}
Thus, the critical question is how often the loop can be repeated.

We can answer that question by going backwards.
Because $x$ and $y$ are constantly decreased but stay positive, the worst case must arise if they are both decreased all the way down to $1$.
Then computing through the loop backwards, we obtain $1,1,2,3,5,8,13$ as the previous values, i.e., the Fibonacci numbers.

Indeed, the worst-case of the Euclidean algorithm arises if $m$ and $n$ are consecutive Fibonacci numbers.
By applying some general math (see Sect.~\ref{sec:ad:fib}), we obtain that $Fib(k)\in\Theta(2^k)$.
Thus, if $n$ is a Fibonacci number, the number of repetitions of the loop is in $\Theta(\log n)$.

Thus, $C(n)\in \Theta(\log n)$.
\end{example}
\footnotetext{Again we assume that all arithmetic operations take constant time.}

\subsection{Discussion}

\subsubsection{Asymptotic Analysis}

Asymptotic analysis is the dominant form of assessing the complexity of algorithms.
It has the huge advantages that it
\begin{compactitem}
 \item is mostly largely independent of the implementation and the physical machine,
 \item abstract away from minor details that do not significantly affect the quality of the algorithms.
\end{compactitem}

But it has some disadvantages.
Most importantly, the terms that it ignores can be huge.
For example, $n+2^{(2^10000)}\in O(n)$ is linear.
But the constant term is so huge that an algorithm with that complexity will never terminate in practice.

More formally, $\Oleq{f}{g}$ only means that $f$ is smaller than $g$ for \emph{sufficiently large} input.
Thus, $\Oleq{f}{g}$ does not mean that $f$ is better than $g$.
It only means that $f$ is better than $g$ if we need the results for sufficiently large inputs.

\subsubsection{Judging Complexity}

$\Theta$-classes for complexity are usually a very reliable indicator of the performance of an algorithm.
If two algorithms were designed naturally without extreme focus on complexity, we can usually assume that:
\begin{compactitem}
 \item For small inputs, they are both fast, and it does not matter which one we use.
 \item For large inputs, the one in the smaller complexity class will outperform the other.
\end{compactitem}

Note that large inputs are usually not encountered by the programmer: the programmer often only tests his programs with small test cases and examples.
Instead, large input is encountered by users.
Therefore, complexity analysis is an important tool for the programmer to judge algorithms.
Most of the time this boils down to relatively simple rules of thumb:
\begin{compactitem} 
 \item Avoid doing something linearly if you can do it logarithmically or in constant time.
 \item Avoid doing something quadratically if you do it quasi-linearly or linearly.
 \item Avoid doing something exponentially if you can do it polynomially.
\end{compactitem}

The distinction between exponential and polynomial has received particularly much attention in complexity theory.
For example, in cryptography, as a rule of thumb, polynomial is considered easy in the sense that anything that takes only polynomial amount of time to hack is considered insecure.
Exponential on the other hand is considered hard and therefore secure.
For example, the time needed to break a password through brute force is exponential in the length of the password.
So increasing the length and variety of characters from time to time is enough to stay ahead of brute force attacks.

\subsubsection{Algorithm Complexity vs. Specification Complexity}

Note that we have only considered the complexity of \emph{algorithms} here.

We can also define the \textbf{complexity of a specification}: Given a mathematical function $f$, its complexity is that of the most efficient correct algorithm $A$ for it.
In this context, $f$ is usually called the problem and $A$ a solution.

It is generally much harder to analyze the complexity of a problem than that of an algorithm.
It is easy to establish an upper bound for the complexity of a problem: Every algorithm for $f$ defines an upper bound for the complexity of $f$.
But to give a lower bound, we have to prove that there is no better algorithm for $f$ (on any physical machine we might be able to build).
Proving the absence of something is generally quite difficult.

An example is the $P\neq NP$ conjecture, which is the most famous open question in computer science.
$P$ is the class of all problems that have polynomial complexity, and $NP$ is a related class that contains $P$.
It is generally assumed that $NP$ is strictly larger than $P$.
But to prove that, one has to show that there is no polynomial algorithm for some problem in $NP$.

\subsubsection{Algorithm Complexity vs. Implementation Complexity}

The \textbf{complexity of an implementation} is its actual run-time.
It is usually assumed that this corresponds to the complexity of an algorithm.

But occasionally, the subtleties discussed in see Sect.~\ref{sec:ad:complex:general} have to be considered because they do not get rounded away.
These subtleties can usually not make the implementation less complex than the algorithm, but they may make it more complex.
Most importantly, when analyzing the complexity of algorithms, we often assume that arithmetic operations can be performed in $O(1)$.
In practice, that is only true for numbers within the limits of underlying CPU, e.g., $64$-bit numbers.
If we implement the data structures for numbers correctly (i.e., for arbitrarily large numbers), the complexity of the arithmetic operations will be greater.

More generally, when analyzing algorithm complexity, we must make assumptions about the complexity of the primitive operations used in the algorithm.
Then the complexity of the implementation is equal to complexity of the algorithm only if the implementation of the primitive operations satisfies these assumptions.

\begin{example}[Euclidean Algorithm]\label{ex:ad:euclid3}
The implementation in Ex.~\ref{ex:ad:euclid2} uses a very inefficient implementation for the data structure $\N$.
It does not satisfy the assumption that arithmetic operations are done in $O(1)$.
In fact, already the function implementing $\leq$ is in $\Theta(n)$.
Consequently, the complexity of this particular implementation of $\gcd$ is higher than $\Theta(n)$.
\medskip

But there are efficient correct implementations of $\N$, which we could use instead.
For example, if we use base-$2$ representation, we can implement natural numbers as lists of bits.
Because the number of bits of $n$ is $\Theta(\log_2 n)$, most arithmetic operations end up being $O(p(\log_2 n))$ for a polynomial $p$.
For example, addition and subtraction take time linear in the number of bits.
Multiplication and related operations such as $\modop$ are super-linear.
That is more than $O(1)$ but still small enough to often be inessential.

With an efficient implementation of $\N$ and its arithmetic operations, the implementation of $\gcd$, which uses $\Theta(\log_2 n)$ steps and applies $\modop$ at every step, has a complexity somewhat bigger than $O((\log_2 n)^2)$.
The details depend on how we implement $\modop$.
\end{example}

\section{Simplicity}

An important and often under-estimated design goal is simplicity.

An algorithm should be elegant in the sense that it is very close to its mathematical specification.
That makes it easy to understand, verify, document, and maintain.

Often simplicity is much more important than efficiency.
The enemy of simplicity is optimization: Optimization increases efficiency usually at the cost of simplicity.
\medskip

In practice, programmers must balance these two conflicting goals carefully.

\begin{example}[Building a List]
A frequent problem is to read a bunch of values and store them in a list.
This usually requires appending every value to the end of the list as in:

\begin{acode}
data := []\\
\awhile{moreData}{
  d := getData\\
  data := append(data, d)
}\\
\areturn{data}
\end{acode}

But appending to $data$ may take linear time in the length of the list.
This is because $data$ points to the beginning of the list, and the append operation must traverse the entire list to reach the end.
Thus, traversal takes $1$ step for the first element that is appended, $2$ for the second, and so on.
The total time for appending $n$ elements in a row is $1+2+\ldots+n=n(n+1)/2\in \Theta(n^2)$.
Thus, we implement a linear problem with a quadratic algorithm.
\medskip

A common solution is the following:

\begin{acode}
data := []\\
\awhile{moreData}{
  d := getData\\
  data := prepend(d, data)
}\\
\areturn{reverse(data)}
\end{acode}

This \emph{prepends} all elements to the list.
Because no traversal is required, each prepend operation takes $O(1)$.
So the whole loop takes $\Theta(n)$ steps.

But we build the list in the wrong order.
Therefore, we revert it before returning it.
Reversal must traverse and copy the entire list once, which takes linear time again.

Thus, the second algorithm runs in $\Theta(n)$ overall.

But it requires an additional function call, i.e., it is less simple.
In a very large program, it is possible that the calls to $prepend$ and $reverse$ occur in two different program locations that are far away from each other.
A programmer who joins the project may not realize that these two calls are related and may introduce a bug.
\medskip

It is non-obvious which algorithm should be preferred.
The decision has to be made on a case-by-case basis keeping all goals in mind.
For example, if the data is ultimately read from or written to a hard drive, that will be linear.
But it will usually be much slower than building the list in memory, no matter whether the list is built in linear or quadratic time.
\end{example}

\section{Advanced Goals}

There are a number of additional properties that algorithms should have.
These can be formally part of the specification, in which case they are subsumed by the correctness properties.
But often they are deliberately or accidentally ignored when writing the specification.

\paragraph{Reliability}
An algorithm is \textbf{reliable} if it minimizes the damage that can be caused by external factors.
For example, power outages, network failures, user error, available memory and CPU, communication with peripherals (printers, hard drive, etc.) can all introduce problems even if all data structures and algorithms are correct.

\paragraph{Safety}
A system is safe if it cannot cause any harm to property or humans.
For example, an algorithm governing a self-driving car must make sure not to hit a human.

Often safety involves interpreting signals received from and sending signals to external devices that operate in the real world, e.g., the cameras and the engine of the car.
This introduces additional uncertainty (not to mention the other cars and pedestrians) that can be difficult to anticipate in the specification.

\paragraph{Security}
A system is secure if it cannot be maliciously influenced from the outside.
This includes all defenses against hacking.

Security is often not part of the specification.
In fact, attacking a system often requires intentionally violating the specification in order to call algorithms with input that the programmer did not anticipate.

Secure algorithms must catch all such invalid input.

\paragraph{Privacy}
Privacy is the requirement that only the output of an algorithm is visible to the user.
Perfect privacy is impossible to realize because all computation leaks some information other than the output: This reaches from runtime and resource use to obscure effects like the development of heat due to CPU activity.

More critically, badly designed systems may expose intermediate data that occurred during execution but is not specified to be part of the output.
For example, when choosing a password, the output should only the cryptographic hash of the password, not the password itself.

Additionally, a system may behave according to its specification, but the user may be unaware of it.
For example, a user may not be aware that her word document stored its previous revision, thus accidentally exposing an early draft.

\paragraph{Maintainability}
An often-underestimated goal being able to maintain a program.
Software usually lives for years, often decades, and programmers will come and go during its life time.
One of the biggest sources of problems can be unclear or undocumented code---even if it is well-designed, correct, and efficient.

Simple data structures and elegant algorithms that are derived systematically from the specification help here.
It leads to implementations that are easier to understand, which allows new programmers to take over seamlessly.

Minor optimizations should generally be avoided because they make the implementation less maintainable.
Even major optimizations (e.g., linear instead of quadratic) must be weighed against the danger of introducing bugs in the long run.


\chapter{Arithmetic Examples}\label{sec:ad:arithex}
  \section{Exponentiation}\label{sec:ad:exp}

\subsection{Specification}\label{sec:ad:exp:spec}

The function $power(x\in\Z,n\in\N)\in\N$ (also written as $x^n$) returns the $n$-the power of $x$ defined by
\[x^0=1\]
\[x^{n}=x\cdot x^{n-1} \tb\mif n > 0\]

By induction on $n$, we show this indeed specifies a unique function.

\subsection{Naive Algorithm}\label{sec:ad:exp:naive}

It is straightforward to give an algorithm for exponentiation.
For example,
\begin{acode}
\afun[\N]{power}{x:\Z,n:\N}{
  \aifelse{n==0}{1}{x\cdot power(x,n-1)}
}
\end{acode}

\paragraph{Correctness}
The correctness of this algorithm is immediate because it follows the specification literally.
For example, $T(x,n)=n$ is already a termination ordering.

\paragraph{Complexity}
Assuming that all multiplications take $O(1)$ no matter how big $x$ is, the complexity of this algorithm is $\Theta(n)$ because we need $n$ multiplications and recursive calls.

\subsection{Square-and-Multiply Algorithm}\label{sec:ad:exp:sqmult}

It is easy to think that $\Theta(n)$ is also the complexity of the specification, i.e., that there is no sub-linear algorithm for it.
But that is not true.

Consider the square-and-multiply algorithm:
\begin{acode}
\afun[\N]{sqmult}{x:\Z,n:\N}{
  \aifelse{n==0}{1}{
	  r := sqmult(x, n\divop 2)\\
	  sq := r\cdot r\\
	  \aifelseI{n \modop 2==0}{sq}{x\cdot sq}
  }
}
\end{acode}

\paragraph{Correctness}
To prove the correctness of this algorithm, we note that
\[x^{2i+0}=(x^i)^2\]
\[x^{2i+1}=x\cdot(x^i)^2\]
Moreover, we know that $n=2(n\divop 2)+(n\modop 2)$.
Partial correctness of $sqmult$ follows immediately.

To prove termination, we observe that $T(x,n)=n$ is a termination ordering: $n\divop 2$ always decreases (because $n\neq 0$) and remains positive.

\paragraph{Complexity}
Computing the run time of a recursive function often leads to a recurrence relation: The function occurs on both sides with different arguments.
In this case, we get:
 \[C(n)=C(n\divop 2) + c\]
where $c\in O(1)$ is the constant-time effort needed in each iteration.
We systematically expand this further
 \[C(n)=C(n\divop 2) + c = C(n\divop 2\divop 2) + 2\cdot c=\ldots=C(n\overbrace{\divop 2\ldots\divop 2}^{k+1\,\text{times}})+(k+1)\cdot c\]

Now let $n=(b_k\,\ldots\,b_0)_2$ be the binary representation of the exponent.
We know that $k=\lfloor\log_2 n\rfloor$ and $n\overbrace{\divop 2\ldots\divop 2}^{k+1\,\text{times}}=0$.
Moreover, we know from the base case that $C(0)=1$.

Substituting these above yield
\[C(n)\in O(1)+\Theta(\log_2 n)\cdot O(1)=\Theta(\log_2 n)\]

Thus, we can compute $power$ in logarithmic time.

\section{Fibonacci Numbers}\label{sec:ad:fib}

\subsection{Specification}\label{sec:ad:fib:spec}

The Fibonacci numbers $Fib(n\in\N)\in\N$ are defined by
\[fib(0)=0\]
\[fib(1)=1\]
\[fib(n)=fib(n-1)+fib(n-2) \tb\mif n>1\]

By induction on $n$, we prove that this indeed specifies a unique function.

Moreover, we can prove the non-obvious result that
 \[fib(n)=\frac{\phi^n-(1-\phi)^n}{\sqrt{5}} \tb\mfor \phi=\frac{1+\sqrt{5}}{2}\]
($\phi$ is also called the golden ratio.)
That can be further simplified to
 \[fib(n)=round\left(\frac{\phi^n}{\sqrt{5}}\right)\]
where we round to the nearest integer.

\subsection{Naive Algorithm}\label{sec:ad:fib:naive}

It is straightforward to give an algorithm for computing Fibonacci numbers.
For example:
\begin{acode}
\afun[\N]{fib}{n:\N}{
  \aifelse{n\leq 1}{n}{fib(n-1)+fib(n-2)}
}
\end{acode}

\paragraph{Correctness}
The correctness of this algorithm is immediate because it follows the specification literally.
For example, $T(n)=n$ is a termination ordering.

\paragraph{Complexity}
We obtain the recurrence relation $C(n)=C(n-1)+C(n-2)+c$ where $c\in O(1)$ is the constant-time effort of the recursion.
That is the same recurrence as for the definition of the Fibonacci numbers themselves, thus $C(n)\in O(fib(n))=\Exp$.

This naive approach is exponential because every functions spawns $2$ further calls.
Each time $n$ is reduced only by $1$ or $2$, so we have to double the number of calls about $n$ times to $\Theta(2^n)$ calls.

\subsection{Linear Algorithm}\label{sec:ad:fib:linear}

It is straightforward to improve on the naive algorithm, turning an exponential into a linear solution.
For example:
\begin{acode}
\afun[\N]{fib}{n:\N}{
  \aifelse{n\leq 1}{n}{
    prev := 0\\
    current := 1 \\
    i = 1 \\
    \awhile{i<n}{
      next := current + prev \\
      prev := current \\
      current := next\\
      i := i+1
    }\\
    \areturn{current}
   }
}
\end{acode}

\paragraph{Correctness}
As a loop invariant, we can use
\[F(n,prev,current,i) = prev==fib(i-1)\wedge current==fib(i)\]
which is straightforward to verify.
After the loop, we have $i==n$ and thus $current=fib(n)$, which yields partial correctness.

As a termination ordering, we can use $T(n,prev,current,i)=n-i$.
Again this is straightforward to verify.

\paragraph{Complexity}
Both the code before and inside the loop take $O(1)$, and the loop is repeated $n-1$ times.
Thus, the complexity is $O(n)$.

\subsection{Inexact Algorithm}\label{sec:ad:fib:inexact}

It is tempting to compute $fib(n)$ directly using $fib(n)=round(\phi^n/\sqrt{5})$.
Because we can precompute $1/\sqrt{5}$, that requires $n+1$ floating point multiplications, i.e., also $O(n)$.

However, it is next to impossible to verify the correctness of the algorithm.
While termination is trivial, partial correctness does not hold.
We know that the formula $fib(n)=round(\phi^n/\sqrt{5})$ is true, but that has no immediate use for floating point arithmetic.
Rounding errors will accumulate over time and may eventually lead to a false result.

\subsection{Sublinear Algorithm}\label{sec:ad:fib:sublinear}

Maybe surprisingly, we can still do better.
Inspecting the body of the while loop in the linear algorithm, we see that we can rewrite the assignments as
\[(current,prev):=(current+prev, current)\]
which we can write in matrix form as
\[(current,prev):=(current,prev)\cdot\begin{pmatrix}1&1\\1&0\end{pmatrix}\]

Thus, we obtain
\[(fib(n),fib(n-1))= (1,0)\cdot\begin{pmatrix}1&1\\1&0\end{pmatrix}^n \tb\mfor n>0\]

We can now pick any algorithm for computing the $n$-power of a matrix, e.g., by using square-and-multiply from Sect.~\ref{sec:ad:exp:sqmult} for matrices.

\paragraph{Correctness}
Correctness follows from the correctness of square-and-multiply.

\paragraph{Complexity}
Square-and-multiply has complexity $O(\log n)$.
Thus, we can compute $fib(n)$ with logarithmic complexity.

%\section{Multiplication}
% Karatsuba


\section{Matrices}\label{sec:ad:matrix}

\subsection{Specification}

We write $\Z^{mn}$ for the set $(\Z^n)^m$ of vectors over vectors (i.e., matrices) over integers.

We define two operations on matrices:
\begin{compactitem}
\item Addition: For of $x,y\in \Z^{mn}$, we define $x+y\in\Z^{mn}$ by
\[(x+y)_{ij}=x_{ij}+y_{ij}\]

\item Multiplication: For $x\in \Z^{lm}$ and $y\in \Z^{mn}$, we define $x\cdot y\in\Z^{ln}$ by
\[(x\cdot y)_{ij}=x_{i1}\cdot y_{1j} +\ldots +x_{im}\cdot y_{mj}\]
\end{compactitem}


\subsection{Naive Algorithms}

Vectors and matrices are best stored using arrays.
We assume that
\begin{compactitem}
 \item $Mat$ is the data structure of two-dimensional arrays of integers (i.e., arrays of arrays of the same length),
 \item if $x$ is an object of $Mat$, then $x.rows$ is the length of the array and $x.columns$ is the length of the inner arrays,
 \item $\anew{Mat}{m,n}$ produces a new array of length $m$ of arrays of length $n$ in which all fields are initialized as $0$.
\end{compactitem}

Then we have the straightforward algorithms
\begin{acode}
\afun[Mat]{add}{x:Mat,y:Mat}{
  r = \anew{Mat}{x.rows,x.columns}\\
  \afor{i}{1}{x.rows}{
    \afor{j}{1}{x.columns}{
      r.i.j := x.i.j+y.i.j
     }
  }\\
  \areturn{r}
} \\
\\
\afun[Mat]{mult}{x:Mat,y:Mat}{
  r = \anew{Mat}{x.rows,y.columns}\\
  \afor{i}{1}{x.rows}{
    \afor{j}{1}{y.columns}{
      \afor{k}{1}{x.columns}{
        r.i.j := r.i.j + x.i.k\cdot y.k.j
      }
    }
  }\\
  \areturn{r}
}
\end{acode}

\paragraph{Correctness}
The algorithms directly implement the definitions.
Thus, correctness---seemingly---obvious.

But there is one subtlety: The functions take two arbitrary matrices---there is no way to force the user to pass matrices of the correct dimensions.
Therefore, we have to state correctness a bit more carefully:
\begin{compactitem}
 \item for $z:=add(x,y)$
   \begin{compactitem}
     \item[precondition:] $x.rows==y.rows$ and $x.columns==y.columns$,
     \item[postcondition:] $z==x+y$ and $z.rows==x.rows$ and $z.columns==x.columns$.
   \end{compactitem}
 \item for $z:=mult(x,y)$
   \begin{compactitem}
     \item[precondition:]  $x.columns==y.rows$
     \item[postcondition:] $z:=mult(x,y)$ is $x\cdot y$ and $z.rows==x.rows$ and $z.columns==y.columns$
   \end{compactitem}
\end{compactitem}
Then we can easily show that $add$ and $mult$ are correct in the sense that the precondition implies the postcondition.

\paragraph{Complexity}
Assuming that all additions and multiplications take constant time, the complexity is easy to analyze.
For addition it is $\Theta(mn)$ and for multiplication $\Theta(lmn)$ where $l$, $m$, and $n$ are the dimensions of the respective matrices.

For addition, we can immediately see that we cannot improve on $\Theta(mn)$: Just creating the new array and returning it already takes $\Theta(mn)$ steps.
Thus, $\Theta(mn)$ is the complexity of the specification, and the naive algorithm is optimal.

This is not obvious for multiplication.
Using the same argument, we can say that the complexity of multiplication is $\Omega(ln)$.
But there cannot be an $\Theta(ln)$-algorithm because $m$ must matter---if $m$ increases, it must take longer.

\subsection{Strassen's Multiplication Algorithm}\label{sec:ad:matrix:strassen}

Inspecting the definition of matrix multiplication, we see that we can split up matrices into rectangular areas of submatrices, for example, like so:
\[\begin{pmatrix}x_{11} & x_{12} & x_{13} & x_{14} \\ x_{21} & x_{22} & x_{23} & x_{24} \\ x_{31} & x_{32} & x_{33} & x_{34} \\ x_{41} & x_{42} & x_{43} & x_{44}\end{pmatrix}
= \begin{pmatrix}
    \begin{pmatrix}x_{11} & x_{12}\\ x_{21} & x_{22}\end{pmatrix} & \begin{pmatrix} x_{13} & x_{14} \\ x_{23} & x_{24} \end{pmatrix} \\
    \begin{pmatrix}x_{31} & x_{32}\\ x_{41} & x_{42}\end{pmatrix} & \begin{pmatrix} x_{33} & x_{34} \\ x_{43} & x_{44} \end{pmatrix}
  \end{pmatrix}
\]
Moreover, if matrices are split up like that, we can still obtain their product in the same way using recursive matrix multiplication:
\[\begin{pmatrix} a & b \\ c & d\end{pmatrix}\cdot \begin{pmatrix} e & f \\ g & h\end{pmatrix}=
  \begin{pmatrix} ae+bg & af+bh \\ ce+dg & cf+dh\end{pmatrix}=\begin{pmatrix} p & q \\ r & s\end{pmatrix}\]

Strassen's algorithm works in the general.
But for simplicity, we only consider the case $l=m=n$, i.e., we are multiplying square matrices.
Then the naive algorithm has complexity $\Theta(n^3)$, and we know the specification has complexity $\Omega(n^2)$.
The question is to find a solution in between.

We further simplify to $n=2^k$, i.e., we can recursively subdivide our $2^k$-matrices to $4$ $2^{k-1}$-matrices.
Then we can design a recursive algorithm that only needs $k$ nested recursions.

The complexity depends on the details of the implementation.
Naively, computing $p,q,r,s$ requires $8$ recursive calls to multiplications and $4$ additions of $2^{k-1}$-matrices.
That yields
 \[C(n)=8\cdot C(n/2) + \Theta(n^2) = \ldots = 8^k\cdot C(1)+\Theta(n^2)\]
Because $k=\log_2 n$ and $C(1)\in O(1)$, that yields $C(n)\in\Theta(n^{\log_2 8})=\Theta(n^3)$.

However, Strassen observed that we can do better.
With some fiddling around, we can replace the $8$ multiplications and $4$ additions with $7$ multiplications and $18$ additions:
\[M_1 = a(f-h)\]
\[M_2 = (a+b)h\]
\[M_3 = (c+d)e\]
\[M_4 = d(g-e)\]
\[M_5 = (a+d)(e+h)\]
\[M_6 = (b-d)(g+h)\]
\[M_7 = (a-c)(e+f)\]
\[\begin{pmatrix} a & b \\ c & d\end{pmatrix}\cdot \begin{pmatrix} e & f \\ g & h\end{pmatrix}=
  \begin{pmatrix} M_5 + M_4 + M_2 + M_6 & M_1 + M_2 \\ M_3 + M_4 & M_1 + M_5 - M_3 - M_7 \end{pmatrix}\]

The extra additions do not harm because they are $\Theta(n^2)$.
But turning the $8$ into a $7$ yields $C(n)=\Theta(n^{\log_2 7})$.
Thus, Strassen's algorithm reduces $n^3$ to $n^{2.81\ldots}$, which can yield practically relevant improvements for relatively small $n$, e.g., $n\approx 30$.
\medskip

Even more efficient algorithms are found regularly.
The current record is $\Theta(n^{2.37\ldots})$.
However, the sufficiently large $n$ for which these algorithms are actually faster than Strassen's algorithm is so large that they have no practical relevance at the moment.

% multiplication: polynomials, n-digit numbers (optimal solution unknown)

  
\chapter{Example: Lists and Sorting}\label{sec:ad:sort}
  \section{Specification}\label{sec:ad:listsort:spec}

Lists are the most important non-primitive data structure in computer science, and sorting is not only the most important problem about lists but also one of the historically most important algorithmic problems of computer science.

\subsection{Lists}\label{sec:ad:list:spec}

For a set $A$, the set $A^*$ contains all lists $[a_0,\ldots,a_{l-1}]$ with elements $a_i\in A$ for some $l\in\N$.
$l$ is called the length of the list.

Because $A^*$ is a set for an arbitrary set $A$, data structures for lists must be polymorphic with a type parameter $A$.

\paragraph{Immutable Lists}
The following table specifies the most important functions involving lists:

\begin{ctabular}{|l|l|l|}
\hline
function & returns & abbreviation\\
\hline
$nil[A]\in A^*$ & $[]$ & \\
$range(m\in\N,n\in\N)\in\N^*$ & $[m,\ldots,n-1]$ or $[]$ if $m\geq n$ & \\
\hline
\multicolumn{3}{|c|}{below, let $l\in A^*$ be of the form $[a_0,\ldots,a_{l-1}]$ and assume $n<l$} \\
$length[A](x\in A^*)\in \N$ & $l$ & \\
$get[A](x\in A^*, n\in\N)\in A$ & $a_n$ & $x_n$ or $x[n]$\\
$concat[A](x\in A^*, y\in A^*)\in A^*$ & $[a_0,\ldots,a_{l-1},b_0,\ldots,b_{k-1}]$ if $y=[b_0,\ldots,b_{k-1}]$ &  $x+y$\\
$map[A,B](x\in A^*, f\in A\to B)\in B^*$ & $[f(a_0),\ldots,f(a_{l-1})]$ & $l\;map\;f$\\
$fold[A,B](x\in A^*, b\in B, f\in A\times B\to B)\in B$ & $f(a_1,f(a_2,\ldots,f(a_n,b))\ldots)$ & \\ 
\hline
$prepend[A](a\in A, x\in A^*)\in A^*$ & $[a,a_0,\ldots,a_{l-1}]$ &\\
$append[A](x\in A^*, a\in A)\in A^*$ & $[a_0,\ldots,a_{l-1},a]$ &\\
$revert[A](x\in A^*)\in A^*$ & $[a_{l-1},\ldots,a_0]$ & \\
$delete[A](x\in A^*, n\in\N)\in A^*$ & $[a_0,\ldots,a_{n-1},a_{n+1},\ldots,a_{l-1}]$ & \\
$insert[A](x\in A^*, a\in A, n\in\N)\in A^*$ & $[a_0,\ldots,a_{n-1},a,a_n,a_{n+1},\ldots,a_{l-1}]$ & \\
$update[A](x\in A^*, a\in A, n\in\N)\in A^*$ & $[a_0,\ldots,a_{n-1},a,a_{n+1},\ldots,a_{l-1}]$ & \\ % $insert(delete(l,n),a,n)
\hline
\end{ctabular}

Most of them are polymorphic.
$map$ and $fold$ even take a second type parameter for the return type of the function.

These are split into three groups:
\begin{compactitem}
\item The first group contains functions to create new lists. These are important to have any lists.
\item The second group contains functions that take a list $x\in A^*$ and return data about $l$ or use $l$ to build new data.
\item The third group also takes a list $l\in A^*$ but also returns an element of $A^*$.
 This distinction is irrelevant in mathematics but critical in computer science: These functions may be implemented using in-place-updates.
 With in-place update, the list $l$ is changed to become the intended result. The original value of $l$ is lost in the process.
 If this is the case, we speak of \textbf{mutable} lists.
\end{compactitem}

\paragraph{Mutable Lists}
The following table specifies the most important functions on mutable lists that differ from immutable lists.
Instead of returning a new list, they have the effect of assigning a new value to the first argument.

\begin{ctabular}{|l|l|l|l|}
\hline
function & returns & effect & abbreviation\\
\hline
\multicolumn{4}{|c|}{below, let $l\in A^*$ be of the form $[a_0,\ldots,a_{l-1}]$ and assume $n<l$} \\
$delete[A](x\in A^*, n\in\N)$ & nothing & $x:=[a_0,\ldots,a_{n-1},a_{n+1},\ldots,a_{l-1}]$ & \\
$insert[A](x\in A^*, a\in A, n\in\N)$ & nothing & $x:=[a_0,\ldots,a_{n-1},a,a_n,a_{n+1},\ldots,a_{l-1}]$ & \\
$update[A](x\in A^*, a\in A, n\in\N)$ & nothing & $x:=[a_0,\ldots,a_{n-1},a,a_{n+1},\ldots,a_{l-1}]$ & $x_n := a$ or $x[n]:= a$\\ % $insert(delete(l,n),a,n)
\hline
\end{ctabular}

The other functions such as $length$ and $get$ are not affected.


\subsection{Sorting}\label{sec:ad:sort:spec}

Sorting a list is intuitively straightforward.
We need a function that takes a list and returns a list with the same elements in a different order, namely such that all elements occur according to their size.

\begin{example}
Consider $x=[4,6,5,3,5,0]\in\N^*$.
Then $sort(x)$ must yield $[0,3,4,5,5,6]$.

Here we made the implicit assumption that we want to sort with respect to the $\leq$-order on $\N$.
We could also use the $\geq$-order.
Then $sort(x)$ should return $[6,5,5,4,3,0]$.

Thus, sorting always depends on the chosen order.
\end{example}

\begin{definition}[Sorting]\label{def:ad:sort:spec}
Fix a set $A$ and a total order $\leq$ on $A$.

A list $x=[a_0,\ldots,a_l]\in A^*$ is called $\leq$-\textbf{sorted} if $a_0\leq a_1 \leq \ldots \leq a_{l-1}\leq a_l$.

Let $count(x\in A^*,a\in A)\in\N$ be the number of times that $a$ occurs in $x$.
Two list $x,y\in A^*$ are a \textbf{permutation} of each other if $count(x,a)=count(y,a)$ for all $a\in A$.

$sort:A^*\to A^*$ is called a $\leq$-\textbf{sorting} function if for all $x\in A^*$, the list $sort(x)$ is a $\leq$-sorted permutation of $x$.
\end{definition}

As usual we check that the specification indeed defines a function:

\begin{theorem}[Uniqueness]
The function $sort$ from Def.~\ref{def:ad:sort:spec} exists uniquely.
\end{theorem}
\begin{proof}
Because $\leq$ is assumed to be total, every list $x$ has a unique least element, which must occur first in $sort(x)$.
By induction on the length of $x$, we show that all elements of $sort(x)$ are determined.
\end{proof}

For immutable lists, the above definition is all the specification we need.
For mutable lists, we specify an alternative sorting function that does not create a new list:

\begin{definition}[In-place Sorting]
An effectful function $sort$ that takes an argument $x\in A^*$ and has the side-effect of modifying the value $v$ of $x$ to $v'$ is called an \textbf{in-place} $\leq$-\textbf{sorting} function if $v'=s(v)$ for a $\leq$-sorting function $s$.
\end{definition}

\subsection{Sorting by a Property}\label{sec:ad:sort:stable}

Often we do not have a total order on $A$, and we want to sort according to a certain property.
The property must be given by a function $p:A\to P$ such that we have a total order $\leq$ on $P$.

For example, we may want to sort a list of students by age.
Then $A=Student$, $P=\N$, and $p:(s\in Student)\mapsto age(s)$.

However, there may be ties: A list may contain multiple different elements that agree in the value of $p$.
To break, we require that the order in the original list should be preserved.
Formally:

\begin{definition}[Sorting by Property]\label{def:ad:sort:stable}
Fix sets $A$ and $P$, a function $p:A\to P$, and a total order $\leq$ on $P$.

Given a list $x\in A^*$, we define a total order $\leq^p$ on the elements of $x$ as follows:
 \[x_i \leq^p x_j \tb\miff\tb p(x_i) < p(x_j) \tb \mor \tb p(x_i)=p(x_j) \mand i\leq j\]

$sort:A^*\to A^*$ is called a \textbf{stable sorting} function for $p$ and $\leq$ if it is a sort function for $\leq^p$.
\end{definition}

Note that normal sorting becomes a special case of sorting by property using $P=A$ and $p(a)=a$.

\subsection{Why Do We Care About Sorting?}

Nowadays, sorting is a solved problem.
Computer scientists almost never need to implement sorting because all programming languages come with sophisticated ready-to-use solutions.

This is captured in the following exchange where a good, modern programmer is quizzed on sorting:
\begin{compactenum}
\item How do you implement sorting a list? --- I call the $\mathit{sort}$ function of my programming language's basic library.
\item OK, but what if there is no $\mathit{sort}$ function? --- I import a library that provides it.
\item OK, but what if there is no such library? --- I use a different programming language.
\item OK, but what if circumstances beyond your control prevent you from using third-party libraries? --- I copy-paste a definition from the internet.\footnote{Nowadays an internet search for elementary problems almost always finds a solution for every programming language, usually on \url{http://www.stackexchange.org}.}
\end{compactenum}

Thus, for most people the only realistic situations in which to implement sorting algorithms is in exams, job interviews, or similar situations.
Then the question is never actually about sorting---it just uses sorting as an example to see whether the programmer understands how to design algorithms, analyze their complexity, and verify their correctness.

In any case, sorting is an extremely good subject for an introductory computer science class because it
\begin{compactitem}
 \item is an elementary problem that is easy to understand for students,
 \item is complex enough to exhibit many important general principles in interesting ways,
 \item is simple enough for all analysis to be doable manually,
 \item has multiple, very different solutions, none of which is better than all the others,
 \item is extremely well-studied,
 \item is widely taught so that the internet is full of good tutorials, examples, and visualizations that help learners.
\end{compactitem}


%%%%%%%%%%%%%%%%%%%%%%%%%%%%%%%%%%%%%%%%%%%%%%%%%%%%%%%%%%%%%%%%%%%%%%%%
\section{Design: Data Structures for Lists}\label{sec:ad:list:ds}


Besides natural numbers, the most important examples of a data structure are lists.
There are many different data structures for lists that differ subtly in how simply and/or efficiently the various functions can be implemented.
We will write $List[A]$ whenever we mean an arbitrary data structure for lists.

\subsection{Immutable Lists}

For immutable lists, functions like $delete$, $insert$, and $update$ (see Sect.~\ref{sec:math:sets:derivfun}) always return new lists.
That requires copying (parts of) the old list, which takes more time and memory.

Without further qualification, this is usually what $List[A]$ refers to.

\subsubsection{Functional Style: Lists as an Inductive Type}

Functional languages usually implement lists an in inductive data type:
\begin{acode}
\adata{{IndList[A]}}{nil,{cons(head: A,\, tail:IndList[A])}}
\end{acode}
Now the list $[1,2,3]$ is built as $cons(1,cons(2,cons(3,nil)))$.

Then functions on lists are implemented using recursion and pattern-matching.
For example:
\begin{acode}
\afun[{IndList[B]}]{map}{x:IndList[A],f:A\to B}{\amatch{x}{\acase{nil}{nil},\acase{cons(h,t)}{cons(f(h),map(t,f))}}}
\end{acode}

\subsubsection{Object-Oriented Style: Linked Lists}

Every inductive data type can also be systematically realized in an object-oriented language.
The correspondence is as follows:

\begin{ctabular}{|l|l|l|}
\hline
inductive type & class & example: lists\\
\hline
name of the type & abstract class & $IndList$ \\
parameters of the type & parameters of the class & $A$ \\
constructor & concrete subclass & e.g., $cons$\\
constructor arguments & constructor arguments & $head:A,tail:IndList[A]$ \\
\hline
\end{ctabular}

A basic realization looks as follows:
\begin{acode}
\aclassA{{IndList[A]}}{}{}{}\\
\aclass{{nil[A]}}{}{IndList[A]()}{}\\
\aclass{{cons[A]}}{head:A,tail:List[A]}{IndList[A]()}{}
\end{acode}
Now the list $[1,2,3]$ is built as $\anew{cons}{1, \anew{cons}{2, \anew{cons}{3, \anew{nil}{}}}}$.

Instead of pattern-matching, we have to use instance-checking to split cases.
For example:
\begin{acode}
\afun[{IndList[B]}]{map}{x:IndList[A],f:A\to B}{
  \aifelse{\aisinst{x}{nil}}
    {\anew{nil}{}}
    {xc := \aasinst{x}{cons} \\
     \anew{cons}{f(xc.head), map(x.tail,f)}
    }
}
\end{acode}

Moreover, we have to override equality so that, e.g., two instances of $cons$ are equal iff they used equal constructor arguments.

\subsubsection{Complexity}

Complexity of lists is measured in the lenght $n$ of the list.

Most operations on lists are linear because the algorithm must traverse the whole list.
For example, the straightforward implementation of $length$ takes $\Theta(n)$.

Similarly, $get(x,i)$ takes $i$ steps to find the element. This is $n$ in the worst case and $n/2$ on average.
So it also takes $\Theta(n)$.

In general, immutable lists require copying the list whenever we insert, delete, or update elements.
These algorithms must traverse the list.
Therefore, they usually take $\Theta(n)$ time for the traversal \emph{and} $\Theta(n)$ space to store the result list.

In the case of $map(x,f)$ and $fold(x,a,f)$, the complexity depends on the passed function $f$.
However, in the typical case where the run time of $f$ does not depend on the length of the list, we can assume it takes constant time $c$.
Thus, the overall run time is $\Theta(cn)=\Theta(n)$.

However, there is one important exception: $prepend$ takes $\Theta(1)$.
This is because we can implement $prepend(a,x)$ simply by calling $cons(a,x)$.
Correspondingly, removing the first element takes $\Theta(1)$.

\subsection{Mutable Lists}

Mutable lists allow assignments to the individual elements of the list.
This allows updating an element without copying the list, thus allowing for many operations with $\Theta(1)$ time or space complexity.

Because we can update the list in place, it becomes critical for efficiency how exactly the list is stored in memory.
Several cases are of great importance, all with advantages and disadvantages:

\begin{ctabular}{|l|l|p{6cm}|}
\hline
data structure & memory layout & remark \\
\hline
array & all in a row & easy to find elements but difficult to insert/delete \\
(singly-)linked list & every element points to next one & easy to insert/delete but traversal needed \\
doubly-linked list &  every element points to next and previous one & traversal in both directions possible, more overhead\\
growable array & linked list of arrays & compromise between the above \\
\hline
\end{ctabular}

\subsubsection{Arrays}

The data structure $Array[A]$ stores all elements in a row in memory.
Arrays must be a primitive feature of the programming language and are so in most languages.

For example, the list $x=[1,2,5]$ is stored in $3$ consecutive memory locations:
\begin{amemory}
\avar{x}{\N^*}{P}
\alocations
\aloc{P}{1}
\aloc{P+1}{3}
\aloc{P+2}{5}
\end{amemory}

That allows implementing $get$ and $update$ in $\Theta(1)$.
$get(x,n)$ is evaluated by retrieving the element in memory location $P+n$.
That takes one step to retrieve $x$, one step for the addition, and one step to retrieve the element at $P+n$.
$update(x,a,n)$ works accordingly.

Inserting and deleting elements still takes $\Theta(n)$.
For example, we can implement deleting by:
\begin{acode}
\afun{delete}{x:Array[A],n:\N}{
  \afor{i}{n}{length(x)-1}{
    x[i] := x[i+1]
  }
}
\end{acode}

Inserting an element into an array is difficult though: The memory location behind the array may not be available because it may have already been used for something else.
Therefore, arrays are often realized in such a way that the programmer chooses in advance the maximal length of the array.
Thus, technically this data structure does not realize the set $A^*$ but the set $A^n$ for some length $n$.
This may waste memory if $n$ is chosen too large.
But arrays are unbeatable in the common situation where we know that we will never call $insert$ anyway.

\subsubsection{Linked Lists}

Mutable linked lists consist of a reference to the first element.
Each element consists of a value and a reference to its successor.
We can implement that using classes (or similar primitives like structs in C):
\begin{acode}
\aclass{{LinkedList[A]}}{head:Elem[A]}{}{}\\
\aclass{{Elem[A]}}{value:A, next:Elem[A]}{}{}
\end{acode}

Technically, $head$ and $next$ should have the type $Elem(A)^?$ to allow for empty lists and the end of the list, respectively.
However, object-oriented programmers usually use a trick where the built-in value $null$ is used:
\begin{compactitem}
 \item If $head$ is null, we have the empty list.
 \item If $next$ is null, we have the last element of the list.
\end{compactitem}

Now the list $[1,2,5]$ is built as $x:=\anew{LinkedList}{\anew{Elem}{1, \anew{Elem}{2, \anew{Elem}{5, null}}}}$.
It is stored in memory as
\begin{amemory}
\avar{x}{\N^*}{P}
\alocations
\aloc{P.head}{Q}
\hline
\aloc{Q.value}{1}
\aloc{Q.next}{R}
\hline
\aloc{R.value}{2}
\aloc{R.next}{S}
\hline
\aloc{S.value}{5}
\aloc{S.next}{null}
\end{amemory}

Deletion can now be realized in-place as follows:
\begin{acode}
\afun{delete}{x:LinkedList[A],n:\N}{
 \aifelse{n==0}{x.head := x.head.next}{
  previous := x.head \\
  current := x.head.next\\
  \afor{i}{1}{n-1}{
    previous := current\\
    current := current.next\\
  }
  previous.next := current.next
 }
}
\end{acode}

Like immutable lists, linked lists take $\Theta(n)$ time for most operations.
However, they still perform better because changes can be done in-place.
Moreover, many operations can be done in $\Theta(1)$ memory whereas immutable lists often require $\Theta(n)$ memory.

An interesting exception is the following variant of $insert$:
Instead of taking the position $n$ at which to insert (which takes linear time to find), it takes the element after which to insert:
\begin{acode}
\afun{insert}{x:LinkedList[A],after:Elem[A],a:A}{
 after.next := \anew{Elem}{a, after.next}
}
\end{acode}

A similar trick for deleting does not work so well: We can implement $delete(x:LinkedList[A], after:Elem[A])$ in $\Theta(1)$ if we know after which element to delete.
But a function $delete(x:LinkedList[A], e:Elem[A])$ where $e$ is to be deleted still requires $\Theta(n)$ to find $e$ in the linked list.

\subsubsection{Doubly-Linked Lists}

Doubly-linked linked list are the same as linked lists except that each element also knows its predecessor ($null$ for the first element).
Moreover, the list knows its first and last element.

\begin{acode}
\aclass{{DoubleLinkedList[A]}}{head:Elem[A], last:Elem[A]}{}{}\\
\aclass{{Elem[A]}}{value:A, previous: Elem[A], next:Elem[A]}{}{}
\end{acode}

Now the list $x=[1,2,5]$ is stored in memory as
\begin{amemory}
\avar{x}{\N^*}{P}
\alocations
\aloc{P.head}{Q}
\aloc{P.last}{S}
\hline
\aloc{Q.value}{1}
\aloc{Q.previous}{null}
\aloc{Q.next}{R}
\hline
\aloc{R.value}{2}
\aloc{R.previous}{Q}
\aloc{R.next}{S}
\hline
\aloc{S.value}{5}
\aloc{S.previous}{R}
\aloc{S.next}{null}
\end{amemory}

Operations on doubly-linked lists are usually in the same complexity class as the corresponding ones for singly-linked lists.

A doubly-linked list has more memory overhead and thus copying and update operations have more time overhead.
But doubly-linked lists can be traversed efficiently in \emph{both} directions.
For example, processing the elements of a singly-linked list in reverse order requires two traversals: one to find the last element, one to process.
The same operation on a doubly-linked list requires only one traversal.
Both are $\Theta(n)$, but the latter may be twice as fast.

In a double-linked list, we can also define nice constant-time variants for both $insert$ and $delete$.
For example:
\begin{acode}
\afun{delete}{x:DoubleLinkedList[A],e:Elem[A]}{
 \aifelse{e.previous == null}{x.head := e.next}{e.previous.next := e.next} \\
 \aifelse{e.next == null}{x.last := e.previous}{e.next.previous := e.previous}
}
\end{acode}

The following table summarizes the complexity of some operations on arrays, linked lists and doubly-linked lists in terms of the length $l$: 
\begin{ctabular}{|l|l|l|l|l|l|l|l|l|} 
\hline
\multirow{2}{*}{}  & \multirow{2}{*}{$length[A]$} & $get[A]$ & $update[A]$ & $insert[A]$ & $delete[A]$ & \multirow{2}{*}{$prepend[A]$} & \multirow{2}{*}{$append[A]$} & \multirow{2}{*}{$reverse[A]$} \\ \cline{3-6}
&       & \multicolumn{4}{c|}{at position $n$}  &         &        &         \\ \hline
Array & $\Theta(1)$ & \multicolumn{2}{c|}{$\Theta(1)$} & \multicolumn{2}{c|}{$\Theta(l-n)$} & $\Theta(l)$ & $\Theta(1)$ & $\Theta(l)$ \\
Linked list & $\Theta(l)$ & \multicolumn{2}{c|}{$\Theta(n)$} & \multicolumn{2}{c|}{$\Theta(n)$} & $\Theta(1)$ & $\Theta(l)$ & $\Theta(l)$ \\
Doubly-linked List & $\Theta(l)$ & \multicolumn{2}{c|}{$\Theta(n)$} & \multicolumn{2}{c|}{$\Theta(n)$} & $\Theta(1)$ & $\Theta(1)$ & $\Theta(l)$  \\ \hline
\end{ctabular}


\subsubsection{Growable Arrays}

Growable arrays are a compromise between arrays and linked lists.
Initially, they behave like an array with a fixed length $l$.
However, when inserting an element increases the length beyond $l$, we create a second array of length $l$ (elsewhere in memory) and remember their connection by storing a list containing the two elements.
Thus, a growable array is a linked list of fixed-length arrays.
The choice of $l$ is up to the data structure designer, who may allow the programmer to tweak it.

Retrieval and update technically are linear now.
To access the element in position $n$, we have to make $n/l$ retrievals to jump to the needed array.
Because $l$ is constant, that yields $\Theta(n)$ retrievals.
However, $l$ is usually large so that element access is only a little slower than for an array and much faster than for a linked list.

%%%%%%%%%%%%%%%%%%%%%%%%%%%%%%%%%%%%%%%%%%%%%%%%%%%%%%%%%%%%%%%%%%%%%%%%
\section{Design: Algorithms for Sorting}\label{sec:ad:sort:algo}

We assume a fixed set $A$ and a fixed comparison function $\leq:A\times A \to \B$.
For $x\in A^*$, we write $Sorted(x)$ if $x$ is $\leq$-sorted.

\paragraph{Auxiliary Functions}
Many in-place sorting algorithms have to swap two elements in a mutable list at some point.
Therefore, we define an auxiliary function

\begin{acode}
\afun{swap}{x:MutableList[A], i:\N, j:\N}{
  h := x[i]\\
  x[i] := x[j]\\
  x[j] := h
}
\end{acode}
Here $MutableList$ is any of the mutable data structures from above.

It is easy to see that this function indeed has the effect of swapping two elements in $x$.
For arrays, the time complexity of $swap$ is $\Theta(1)$.
For linked lists, it is $\Theta(n)$.


\subsection{Bubblesort}\label{sec:ad:sort:bubble}

Bubblesort is a stable in-place sorting algorithm that closely follows the natural way how a human would sort.
The idea is to find two elements that are not in order and swap them.
If no such elements exist, the list is sorted.

\begin{acode}
\afun{bubblesort}{x:Array[A]}{
 sorted := \false \\
 \awhile{!sorted}{
   sorted:=\true \\
   \afor{i}{0}{length(x)-2}{
     \aif{! x[i]\leq x[i+1]}{
       sorted := \false \\
       swap(x,i,i+1)
     }
   }
 }
}
\end{acode}

\paragraph{Correctness}
The for-loop compares all $length(x)-1$ pairs of neighboring elements.
It sets $sorted$ to $\false$ if the list is not sorted.
Thus, we obtain the loop invariant $F(x,sorted)=sorted==Sorted(x)$, which immediately yields partial correctness.

Total correctness follows from the termination ordering
 \[T(x,sorted)=\text{number of pairs $i,j$ such that $! x_i\leq x_j$} + \cas{1\mifc sorted==\false\\ 0\mifc sorted==\true}\]
Indeed, this number decreases in every iteration of the loop in which $x$ is not sorted.
The second summand is necessary to make sure $T(x,sorted)$ also decreases when $x$ is already sorted (which happens exactly once in the last iteration).

\paragraph{Complexity}
If $n$ is the length of $x$, each iteration of the while-loop has complexity $\Theta(n)$.
Moreover, the while-loop iterates at most $n$ times.
That happens in the worst-case: when $x$ is reversely sorted initially.
Thus, the complexity is $\Theta(n^2)$.

In the best-case, when $x$ is already sorted initially, the complexity is $\Theta(n)$.
That is already optimal because it requires $n-1$ comparisons to determine that a list is sorted.

\subsection{Insertionsort}\label{sec:ad:sort:insertion}

Insertion is also a stable in-place algorithm.

The idea is to sort increasingly large prefixes of a list $x$.
If $[x_0,\ldots,x_{i-1}]$ is sorted already, the element $x_i$ is inserted among them.

\begin{acode}
\afun{insertionsort}{x:Array[A]}{
  \afor{i}{1}{length(x)-1}{
    current := x[i] \\
    pos := i \\
    \awhile[shift elements to the right to make space for $current$]{pos > 0 \aand current < x[pos-1]}{
       x[pos] := x[pos-1] \\
       pos := pos - 1
    }\\
    x[pos] := current
  }
}
\end{acode}

\paragraph{Correctness}
We use a loop-invariant for the for-loop: $F(x,i)=Sorted([x_0,\ldots,x_{i-1}])$.
The preservation of the loop-invariant is non-obvious but straightforward to verify.
It holds initially because the empty list is trivially sorted.
That yields partial correctness.

Termination is easy to show using the termination ordering $T(x,i,current,pos)=pos$ for the while-loop.

\paragraph{Complexity}
If $n$ is the length of $x$, the for-loop runs $n$ times with $i=0,\ldots,n-1$
Inside, the while-loop runs $i$ times in the worst-case: if $x$ is reversely sorted, all $i$ elements before $current$ must be shifted to the right.
That sums up to $0+1+\ldots+n-1\in \Theta(n^2)$.

Everything else is $O(n)$.
Thus, the worst-case complexity is $\Theta(n^2)$.

In the best-case, if $x$ is already sorted, the while-loop never runs, and the complexity is $\Theta(n)$.

\subsection{Mergesort}\label{sec:ad:sort:merge}

Mergesort is based on the observation that
\begin{compactitem}
  \item sorting smaller lists is much easier than sorting larger lists (because the number of pairs that have to be compared in $\Theta(n^2)$,
  \item merging two sorted lists is easy (linear time).
\end{compactitem}
Thus, we can divide a list into two halves, sort them recursively, then merge the results.
This is similar to the idea of square-and-multiply (Sect.~\ref{sec:ad:exp:sqmult}) and an example of the family of divide-and-conquer algorithms.

Because it needs auxiliary memory to do the merging of two half lists into one, it is easiest to implement as non-in-place algorithm.
Then the input data structure does not matter and can be assumed to be immutable.
The following is a straightforward realization:

\begin{acode}
\afun[{List[A]}]{mergesort}{x:List[A]}{
  n := length(x) \\
  \aifelse{n<2}{x}{
    k := n\divop 2\\
    l := mergesort([x_0,\ldots,x_{k-1}]) \\
    r := mergesort([x_k,\ldots,x_{n-1}]) \\
    \areturn{merge(l,r)}
  }
}\\
\\
\afun[{List[A]}]{merge}{x:List[A], y:List[A]}{
  xRest := x\\
  yRest := y\\
  res = [] \\
  \awhile{nonempty(xRest) \aor nonempty(yRest)}{
    takefromX := empty(yRest) \aor (nonempty(xRest) \aand xRest.head \leq yRest.head)\\
    \aifelse{takefromX}{
      res := cons(xRest.head, res) \\
      xRest := xRest.tail
    }{
      res := cons(yRest.head, res) \\
      yRest := yRest.tail
    }
  }\\
  \areturn{reverse(res)}
}
\end{acode}

\paragraph{Correctness}
Because the function $merge$ is not part of the specification, we have to first specify which property we want to prove about it.
The needed property for $z:=merge(x,y)$ is:
 \begin{compactitem}
   \item precondition: $Sorted(x)$ and $Sorted(y)$
   \item postcondition: $Sorted(z)$ and $z$ is a permutation of $x+y$
 \end{compactitem}

Now we can prove each function correct.
\medskip

First we consider $mergesort$.
Partial correctness means to prove $Sorted(mergesort(x))$.
That is very easy:
\begin{compactitem}
  \item If $n<2$, $x$ is trivially sorted.
  \item Otherwise:
   \begin{compactitem}
     \item $Sorted(a)$ and $Sorted(b)$ follow from the postcondition of the recursive call.
     \item Then the postcondition of $merge$ yields $Sorted(merge(a,b))$.
   \end{compactitem}
\end{compactitem}

Relative termination is immediate (assuming that $merge$ always terminates, which we prove below).
A termination ordering is given by $T(x)=length(x)$.
Indeed, $mergesort$ recurses only into strictly shorted lists.
\medskip

Second we consider $merge$.
We use a loop invariant $F(x,y,xRest,yRest,res)$ that states that
 \begin{compactitem}
  \item $Sorted(reverse(res))$ and $Sorted(xRest)$ and $Sorted(yRest)$
  \item All elements in $res$ are in $\leq$-relation to all elements in $xRest+yRest$.
  \item $res+xRest+yRest$ is a permutation of $x+y$
 \end{compactitem}
It is non-obvious but it is straightforward to see that this is indeed a loop invariant:
 \begin{compactitem}
   \item $reverse(res)$ remains sorted because we always take the smallest element in $yRest+xRight$ and prepend it to $res$.
    In particular, because $xRest$ and $yRest$ are sorted, the smallest element must be $xRest.head$ or $yRest.head$.
   \item For the same reason, all elements of $res$ remain smaller than the ones of $xRest$ and $yRest$.
   \item Because we only remove elements from $xRest$ and $yRest$, they remain sorted.
   \item Because every element that is removed from $xRest$ or $yRest$ is immediately added to $res$, they remain a permutation.
 \end{compactitem}

To show partial correctness, we see that
\begin{compactitem}
  \item The loop invariant holds initially, which is obvious.
  \item After completing the loop, $xRest$ and $yRest$ are empty.
  \item Then, using the loop invariant, it is easy to show that $reverse(res)$ is sorted and a permutation of $x+y$.
\end{compactitem}

To show termination, we use $T(x,y,xRest,yRest,res)=length(xRest)+length(yRest)$.
It is easy to see that $T$ is a the termination ordering for the while-loop.

\paragraph{Complexity}
We have to analyze the complexity of both functions.

First we consider $merge$.
Let $n=length(x)+length(y)$.
\begin{compactitem}
 \item The three assignments in the beginning are $O(1)$.
 \item The while-loop is repeated once for every element of $x$ and $y$, which requires $\Theta(n)$ steps.
 The body of the loop takes $O(1)$. So $\Theta(n)$ in total.
 \item The last step requires reverting $res$, which has $n$ elements at this point.
 Reverting a list requires building a new list by traversing the old one. That is $\Theta(n)$ as well.
\end{compactitem}
Thus, the total complexity of $merge$ is $\Theta(n)=\Theta(length(x)+length(y))$.
\medskip

Second we consider $mergesort$.
Let $n=length(x)$.
We compute the time complexity $C(n)$:
\begin{compactitem}
 \item The assignments and the if-statement are in $O(1)$.
 \item The recursive calls to $mergesort$ take $C(n/2)$ each.
 \item The call to $merge$ takes $\Theta(length(a)+length(b))=\Theta(n)$.
\end{compactitem}
That yields
 \[C(n)=2\cdot C(n/2)+\Theta(n) = \ldots = 2^k\cdot C(n/2^k) + k\cdot \Theta(n)\]
 By choosing $k=\log_2 n$ and $C(1)=C(0)\in O(1)$, we obtain
 \[C(n)=n\cdot O(1)+\log_2 n\cdot \Theta(n)=\Theta(n\log_2 n)\]
\medskip

Thus, mergesort is quasilinear and thus strictly more efficient than bubblesort and insertionsort.

Contrary to bubblesort and insertionsort, mergesort takes the same amount of time no matter how sorted the input already is.
The recursion and the merging happen in essentially the same way independent of the input list.
Thus, its best-case complexity is also $\Theta(n\log_2 n)$.

\begin{remark}[Building the list reversely in $merge$]
$merge$ could be simplified by always adding the element $xLeft.head$ or $yLeft.head$ to the \emph{end} of $res$ instead of the beginning.
However, as discussed in Sect.~\ref{sec:ad:list:ds}, adding an element to the beginning of an immutable list takes constant time whereas adding to the end takes linear time.

Therefore, if we added elements to the end of $res$ would become quadratic instead of linear.
Then mergesort as a whole would also be quadratic.
\end{remark}

\subsection{Quicksort}\label{sec:ad:sort:quick}

Quicksort is similar to mergesort in that two sublists are sorted recursively.
The main differences are:
\begin{compactitem}
 \item It does not divide the list $x$ in half.
  Instead it picks some element $a$ from the list (called the \emph{pivot}).
  Then it divides $x$ into sublists $a$ and $b$ containing the elements smaller and greater than $x$ respectively.\\
  No merging is necessary because all elements in $a$ are smaller than all elements in $b$.
  Thus the sorted list is $quicksort(a)+x+quicksort(b)$.
 \item To divide the list, quicksort has to traverse and reorder the list anyway.
 Therefore, it can easily be implemented in-place avoiding the use of auxiliary memory.
\end{compactitem}

When implemented as an in-place sorting algorithm, the recursive call takes two additional arguments: two numbers $first$ and $last$ that describe the sublist that should be sorted.

\begin{remark}[Additional Arguments in a Recursion]
Carrying along auxiliary information is very typical for recursive algorithms.
Therefore, we often find pairs of function:
 \begin{compactitem}
  \item A recursive function that takes additional arguments.\\
   That is $quicksortSublist$ below, which takes the entire list and the information about which sublist to sort.
  \item A non-recursive function that does nothing but call the other function with the initial arguments.\\
   That is $quicksort$ below, which calls $quicksortSublist$ on the entire list (e.g., on the sublist from $0$ to the end of $x$).
 \end{compactitem}
\end{remark}


\begin{acode}
\afun{quicksort}{x:Array[A]}{
  quicksortSublist(x,0,length(x)-1)
}\\
\\
\afun{quicksortSublist}{x:Array[A], first:\N, last: \N}{
  \aifelse{first \geq last}{
    \areturn{}
  }{
    pivot := x[last]\\
    pivotPos := first\\ %%    // place for swapping
    \aloopinv{x[k]\leq pivot \mfor k=first,\ldots,pivotPos-1 \mand pivot \leq x[k] \mfor k=pivotPos,\ldots,j-1}
    \afor{j}{first}{last - 1}{
      \aif{x[j] \leq pivot}{
         swap(x,pivotPos,j) \\
         pivotPos := pivotPos + 1
      }
    }\\
    swap(x,pivotPos,last)\\
    \\
    quicksortSubList(x, first, pivotPos - 1)\\
    quicksortSubList(x, pivotPos + 1, last)
  }
}
\end{acode}

\paragraph{Correctness}
Before proving correctness we have to specify the behavior of the auxiliary function $quicksortSublist$:
\begin{compactitem}
 \item precondition: none
 \item postcondition: $Sorted([x_{first},\ldots,x_{last}])$
\end{compactitem}
Then the correctness of $quicksort$ follows immediately from that of $quicksortSublist$.
\medskip

Now we prove the partial correctness of $quicksortSublist$.
First, the base case is trivially correct: It does nothing for lists of length $0$ or $1$.
For the recursive case, we prove that the following two properties hold just before the two recursive calls:
\begin{compactitem}
 \item The sublist $[x_{first},\ldots,x_{last}]$ is a permutation of its original value, and no other elements of $x$ has changed.
  That is easy to see because we only change $x$ by calling $swap$ on positions between $first$ and $last$.
 \item All values $x_k$ are
  \begin{compactitem}
    \item smaller than $pivot$ for $k=first,\ldots,pivotPos-1$,
    \item equal to $pivot$ for $k=pivotPos$,
    \item greater than $pivot$ for $k=pivotPos+1,\ldots,last$.
  \end{compactitem}
  We prove that by using the indicated loop invariant for the for-loop.
  It is trivially true before the for-loop because $first=pivotPos$ and $pivotPos=j$.
  It is straightforward to check that it is preserved by the for-loop.
  Therefore, it holds after the for-loop for the value $j=last-1$.
  The last call to $swap$ moves the pivot element into $x_{pivotPos}$ so that the loop invariant is now also true for $j=last$.
  Then the needed properties can be seen easily.
\end{compactitem}
\medskip

To prove the termination of $quicksortSublist$, we use the termination ordering $T(x,first,last)=last-first+1$ (which is the length of the sublist).
That value always decreases because the pivot element is never part of the recursive call.

\paragraph{Complexity}
Let $n=last-first-1$ be the length of the sublist.
It is easy to see that, apart from the recursion, $quicksortSublist$ takes $\Theta(n)$ steps because the for-loop traverses the sublist.
Thus, the complexity of quicksort depends entirely on the lengths of the sublists in the recursive calls.
However, the pivot position and therefore those lengths are hard to predict.

The best-case complexity arises if the pivot always happens to be in the middle.
Then the same reasoning as for mergesort yields best-case complexity $\Theta(n\log_2 n)$.
The worst-case arises if the list is already sorted: then the pivot position will always be the last one, and the two sublists have sizes $n-1$ and $0$.
That results in $n$ recursive calls on sublists of length $n$, $n-1$, \ldots, $1$ as well as $n$ calls on empty sublists.
Consequently, the worst-case complexity is $\Theta(n^2)$.

However, the worst-case complexity does not do quicksort justice because it is much higher than its average-case complexity.
Because there are only finitely many permutations for a list of fixed length, the average-case complexity can be worked out systematically.
The result is $\Theta(n\log_2 n)$.
\medskip

It may seem that quicksort is less attractive than mergesort because of its higher worst-case complexity.
However, that is a minor effect because the algorithms have the same best-case and average-case complexity.
Instead, the constant factors, which are rounded away by using $\Theta$-classes, become important to compare two algorithms with such similar complexity.

Here quicksort is superior to mergesort.
Moreover, quicksort can be optimized in many ways.
In particular, the choice of the pivot can be tuned in order to increase the likelihood that the two sublists end up having the same size.
For example, we can randomly pick $3$ elements of the sublist and use the middle-size one as the pivot.
With such optimizations, quicksort can become substantially faster than mergesort.

\subsection{Other Algorithms}

There is a number of other sorting algorithms that we will not go into here.
Examples include counting sort, radix sort, and bucket sort.

One particularly important sorting algorithm is heap sort, which we discuss in Sect.~\ref{sec:ad:heapsort} (after introducing heaps).

\subsection{In Programming Languages}

Most programming languages come with a standard library that includes efficient sorting algorithms.
Moreover, other libraries for other algorithms may be around.
In some cases, languages only specify the interface and leave the implementation (and thus the choice of algorithm) to individual implementations of compilers/interpreters.

The following gives some examples.

Python uses Timsort (named after the programmer), which is a hybrid of mergesort and insertionsort with various optimizations.
It is written directly in C.

Java used to use just quicksort.
Java 7 uses either Timsort (ported to Java) or a variant of quicksort that uses two pivot elements.

Scala defers to Java's implementation.

C++'s std library specification does not prescribe a sorting algorithm but requires $O(n\log_2 n)$ worst-case complexity (average-case in earlier versions).
Implementations vary in their choice of algorithm, e.g., using hybrid algorithms that perform some iterations of quicksort before switching to insertionsort for the resulting small lists.

For Javascript, the choice is up to the browser (because every browser is a separate implementation of Javascript).


\part{Important Data Structures}\label{sec:ad:ds}

\chapter{Finite Data Structures}\label{sec:ad:finiteds}
% \section{Void}

The set $\Void$ contains no elements.

Not surprisingly, it is rarely used.
However, it is nice to have when dealing with operations that do not return.
For example, we say that throwing an exception or terminating the program returns an element of $\Void$.

Most programs do not need the type void.
And most programming language either do not have it or only have it under the hood.

\section{Unit}

The set $\Unit$ contains exactly one element, which we write $()$.

It is rarely used because if we know that $x\in\Unit$, we already know the value of $x$.
Thus, having a value of type $\Unit$ gives us no information.

However, $\Unit$ is nice to have when dealing with operations that do return, but do not return a value.
In that case, we say that the operation returns type $\Unit$.

For example, assignments, loops, and print statements return $\Unit$.
Many methods of mutable data structures also return $\Unit$.
For example, using $\Unit$, we can specify $insert$ for a mutable list from Sect.~\ref{sec:ad:list:spec} as $insert(x\in A^*,a\in A,n\in\N)\in\Unit$.

Functional programming languages usually have a built-in type $\Unit$.
That way, in a functional programming langauge, every operation has a return type.

\section{Booleans}

The set $\Bool$ contains exactly two elements, which we call $\true$ and $\false$.

Most programming languages have a built-in type $\Bool$, which is the result type of the equality operator.

\section{Integers Modulo}

For $m>0$, the set $\Z_m$ consists of the elements $\{0,\ldots,m-1\}$.

Most programming languages do not offer $\Z_m$ for every $m$.
Usually, they offer at most $\Z_{2^k}$ for $k=8$ (usally called \emph{byte}), $k=16$ (\emph{word}), $k=32$ (\emph{integer}), and/or $k=64$ (\emph{long}).

Note that, depending on the programming language, the built-in type $\Int$ may refer to one of those (usually for $k=32$) or to $\Z$.

If we need $\Z_m$ for a specific $m$, we usually work with $\Int$ and use the $\modop$ operation to ensure we remain inside $\Z_m$.
\footnote{Note that some programming languages implement $\divop$ and $\modop$ in unexpected ways for negative arguments.}

\section{Enumerations}

For fresh names $l_1,\ldots,l_n$, the set $\Enum\{l_1,\ldots,l_n\}$ has exactly $n$ elements, which are called $l_1,\ldots,l_n$.

The names $l_i$ must be fresh.
That means they may not have been defined previously.
This is similar to how the name of a new function or class must be fresh.
This is because defining an enumeration set introduces new values, namely the $l_i$.

Most programming languages allow defining enumeration types in some way.
For example, in SML:

\begin{lstlisting}
datatype answer = yes | no | maybe
\end{lstlisting}

Or in C:

\begin{lstlisting}
enum {yes, no, maybe} answer;
\end{lstlisting}


%char, string

\chapter{Number-Based Data-Structures}\label{sec:ad:numberds}
% \section{Countable Sets}

The sets $\N$, $\Z$, and $\Q$ are well-known from mathematics.

Working with $\Z$ (as opposed to $\Z_m$ for some $m$) is called \emph{arbitrary precision arithmetic}.
$\Z$ may or may not be the built-in type $\Int$---that depends on the programming language.
If not, $\Int$ is $\Z_m$ for some $m$---in those languages, there is usually a library that defines $\Z$.

A data structure for $\Q$ can be defined by using pairs of integers.

We usually do not use a special data structure for $\N$ and instead just use the positive values of $\Z$.
Alternatively, we can give a (very inefficient) definition of $\N$ as an inductive type as in Ex.~\ref{ex:ad:euclid2}.

\section{Uncountable Sets}

We cannot implement data structures for $\R$ and $\C$ because they are uncountable.

There are some approximate solutions to work with $\R$.
For example, we can simply represent a real number $r$ as a function $\N\to\{0,\ldots,9\}$ that provides the infinite decimal expansion of $r$.
Because we can only represent countably many functions as effective objects, not all real numbers can be represented like.
However, all practically useful ones can.
A major drawback of this representation is that we cannot give an algorithm for equality (because we would have to check that two functions are equal for infinitely many arguments), thus crippling the data structure.

For the $\C$, it is often sufficient to work with the countable set $\Q+\Q i$, which is the set of complex numbers whose real and imaginary parts are rational.



\chapter{List-Like Data Structures}\label{sec:ad:lists}
% exercise: implement stack and queue using linked list
 The specification and several data structures for mutable and immutable lists are already discussed in Sect.~\ref{sec:ad:listsort:spec}.

Here we only discuss some additional data structures for the set $A^*$.

\section{Stacks}\label{sec:ad:stack}

$Stack[A]$ is a data structure for the set $A^*$.

$Stack[A]$ is very similar to $List[A]$.
The difference is that $Stack[A]$ provides \emph{less} functionality.
While $List[A]$ is a general-purpose list, $Stack[A]$ is custom-fitted to one specific, very common use case.
By requiring fewer operations, they allow more optimized implementations.

Stacks can be mutable or (less commonly) immutable.
Here we will use the mutable variant.
The functions for mutable stacks are:

\begin{ctabular}{|l|l|l|}
\hline
function & returns & effect \\
\hline
$push(x\in A^*, a\in A)\in\Unit$ & nothing & prepend $a$ to $x$\\
$pop(x\in A^*)\in A^?$ & the first element of $x$ (if any) & remove the first element of $x$ \\
$top(x\in A^*)\in A^?$ & the first element of $x$ (if any) & none \\
\hline
\end{ctabular}

The intuition behind stacks is that they provide a LIFO store of data.
LIFO means last-in-first-out because every $pop$ returns the most recently pushed value.
This is exactly the behavior of a literal stack of items: We can put an item on top of a stack ($push$), remove an item from the stack ($pop$), or check what item is on top ($top$).
We cannot easily see or remove the other items.

Very often, the LIFO behavior is exactly what is needed.
For example, when we solve a maze, we can push every decision we make.
When we hit a dead end, we trace back our steps---for that, we have to pop the most recent decision, and so on.

\section{Queues}\label{sec:ad:queue}

Queues are very similar to stacks.
Everything about stacks also applies to queues except for the following.

The functions for mutable queues are:

\begin{ctabular}{|l|l|l|}
\hline
function & returns & effect \\
\hline
$enqueue(x\in A^*, a\in A)\in\Unit$ & nothing & append $x$ to $A$\\
$dequeue(x\in A^*)\in A^?$ & the first element of $x$ & remove the first element of $x$ \\
$empty(x\in A^*)\in\Bool$ & true if $x$ is empty & none \\
\hline
\end{ctabular}

The intuition behind queues is that they provide a FIFO store of data.
FIFO means first-in-first-out because every $dequeue$ returns the least recently enqueued value.
This is exactly the behavior of a literal queue of people: Every newcomer has to queue up at the end of the queue ($enqueue$), and every time a server is ready the first in line gets served ($dequeue$).
Newcomers cannot cut in line, and the server cannot easily see who else is waiting.

Very often, the FIFO behavior is exactly what is needed.
For example, when we have a list of tasks that need to be done.
Every time we create a new task, we enqueue it, and whenever we have time we dequeue the next task.

Queues are often used when components exchange messages or commands.
In that case, some components---called the producers---only call enqueue, and other components---called the consumers---only call dequeue.
For example, the producers can be different programs, $A$ is the type of print jobs, and the consumers are different printers.

More complex queue data structures may also for dequeueing based on priority (see also Sect.~\ref{sec:ad:heapqueue}).

\section{Buffers}\label{sec:ad:buffer}

Buffers are conceptually very similar to queues.
But $\mathit{Buffer}[A]$ is usually optimized for enqueueing and dequeueing many elements of $A$ at once.
Therefore, while stacks and queues can be implemented well using linked lists, buffers usually use arrays to be faster.

A typical $\mathit{Buffer}[A]$ consists of three components:
\begin{compactitem}
 \item an $Array[A]$ $b$
 \item two integers $begin$ and $end$ indicating the first and last valid entry in the array.
\end{compactitem}
Enqueueing writes to $b[end+1]$ and increments $end$.
Dequeueing reads from $b[begin]$ and increments $begin$.


A buffer overflow occurs when incrementing $begin$ 

For example, when a browser receives a web page, its network component loads the page into a $\mathit{Buffer}[\Char]$.
In parallel, its HTML parser component starts processing the partially received page.
That way the HTML page can be displayed partially already before it is fully loaded.

Buffers are almost always used automatically when a program is writing to a file.
In that case, a $\mathit{Buffer}[\Int]$ or $\mathit{Buffer}[\Char]$ is used that holds the data written to the file.
The write command does not actually write data to the file directly---it only enqueues it in the buffer.
That is advantageous because enqueueing to a buffer in memory is much faster than writing to the hard drive.
While the program is already moving on, the programming language libraries or the operating system work in the background to periodically dequeue and write all characters to the file.

When working with files, an important operation is \emph{flushing} the buffer.
This forces the immediate processing of all data in the buffer.
Flushing happens automatically at the latest when the program terminates.
However, occasionally manual flushing is necessary:
\begin{compactitem}
 \item When a program terminates with an error, buffers have to be flushed to avoid losing data.
 \item When a program writes log data to a file that the programmer wants to read immediately, it is important to flush regularly to make sure the programmer reads updated information.
\end{compactitem}

\section{Iterators}\label{sec:ad:iter}

\subsection{Specification}

$Iterator[A]$ is a data structure for the set $A^*$.

Iterators are usually mutable.
Their functionality is even more restricted than the one of stacks and queues:

\begin{ctabular}{|l|l|l|}
\hline
function & returns & effect \\
\hline
$getNext(x\in A^*)\in A$ & the first element of $x$ & remove the first element of $x$ \\
$hasNext(x\in A^*)\in\Bool$ & $\true$ if $x$ is not empty & none \\
\hline
\end{ctabular}

The typical way to use an iterator $i\in Iterator[A]$ is the following:
\begin{acode}
\awhile{hasNext(i)}{
  a := getNext(i)\\
  \text{do something with $a$ here}
}
\end{acode}
This is called \textbf{traversing} the iterator.
Afterwards the iterator is traversed and cannot be used again.

$Iterator[A]$ may look somewhat boring.
In order to understand the value of iterators, we have to make one definition:
A data structure $D[A]$ is called \textbf{iterable} if there is a function
 \[iterator(x\in D[A])\in Iterator[A]\]

Now the imoprtance of iterators follows from two facts:
\begin{compactitem}
 \item Many data structures $D$ are iterable (see Sect.~\ref{sec:ad:iter:create}).
 \item Many important operations for $D$ can be realized using only the functionality of iterators (see Sect.~\ref{sec:ad:iter:use}).
\end{compactitem}
Thus, iterators provide a sweet-spot in the trade-off between simplicity and expressivity---they are very simple,  but we can do a lot with them.

\begin{remark}[Simplicity vs. Expressivity]
The trade-offs between simplicity and expressivity comes up again and again in computer science.
The best data structures combine both properties, but usually they are mutually exclusive.

All the important data structures presented in Part~\ref{sec:ad:ds} have become important because they do well in this way.
\end{remark}

\subsection{Working with Iterable Data Structures}\label{sec:ad:iter:use}

Let us assume an iterable data structure $D[A]$.
Our goal is to define functions on $x\in D[A]$ that use only $iterator(x)$.
There are indeed many of those.
Some important ones are:
\begin{ctabular}{|l@{}l@{}l|l|}
\hline
function &&& returns \\
\hline
\multicolumn{4}{|c|}{below, let $X=iterator(x)$}\\
$length$&$(x\in D[A])$&$\in \N$ & numbers of elements in $X$ \\
$contains$&$(x\in D[A],\; a\in A)$&$\in \Bool$ & $\true$ if $a$ occurs in $X$ \\
$index$&$(x\in D[A],\; a\in A)$&$\in \N^?$ & the position of the first occurrence of $a$ in $X$ (if any)\\
$find$&$(x\in D[A],\; p\in A\to\Bool)$&$\in A^?$ & the first element $a$ in $X$ (if any) such that $p(a)$ is $\true$ \\
$count$&$(x\in D[A],\; p\in A\to\Bool)$&$\in\N$ & the number of elements $a$ in $X$ for which $p(a)$ is $\true$ \\
$forall$&$(x\in D[A],\; p\in A\to\Bool)$&$\in \Bool$ & $\true$ if $p(a)$ is $\true$ for every element $a$ in $X$ \\
$exists$&$(x\in D[A],\; p\in A\to\Bool)$&$\in \Bool$ & $\true$ if $p(a)$ is $\true$ for some element $a$ in $X$ \\
$map$&$(x\in D[A],f\in A\to B)$&$\in Iterator[B]$ & an iterator for $[f(a_1),\ldots,f(a_n)]$ where $x=[a_1,\ldots,a_n]$ \\
$filter$&$(x\in D[A],p\in A\to \Bool)$&$\in Iterator[B]$ & like $x$ but skips elements that do not satisfy $p$ \\
$results$&$(x\in D[A],\;f\in A\to B)$&$\in List[B]$ & the list of results from applying $f$ to all $a$ in $X$ \\
$fold$&$(x\in D[A],\; b\in B, f\in A\times B\to B)$&$\in B$ & $f(a_1,f(a_2,\ldots,f(a_n,b))\ldots)$ with $X=[a_1,\ldots,a_n]$\\
\hline
\end{ctabular}
All of the above functions should not have a side-effect.
However, some of them take other functions as arguments.
It is usually a bad to do so, but it is technically possible that these functions have side-effects.
There is only one exception where we explicitly allow $f$ to have a side-effect:
\begin{ctabular}{|l|l|l|}
\hline
function & returns & effect \\
\hline
$foreach(x\in D[A],f\in A\to \Unit)\in \Unit$ & nothing & apply $f$ to all $a$ in $X$ \\
\hline
\end{ctabular}

The trick behind $map$ (and the difference to $results$) is that $x$ is not traversed right away.
Instead, we create a new iterator that, when traversed, applies $f$.
That way we ensure that $f$ is applied only as often as necessary.


\subsection{Making Data Structures Iterable}\label{sec:ad:iter:create}

We can give a data structure for iterators as an abstract class:
\begin{acode}
\aclassA{Iterator[A]}{}{}{
 \afun[\Bool]{hasNext}{}{}
 \acomment{precondition for $getNext$ is $hasNext==\true$}\\
 \afun[A]{getNext}{}{}
}
\end{acode}

Then we can define, e.g.,  $map$ as follows:
\begin{acode}
\aclass{Map[A,B]}{x:Iterator[A], f:A\to B}{Iterator[B]}{
  \afunI[\Bool]{hasNext}{}{x.hasNext}\\
  \afunI[B]{getNext}{}{f(x.getNext)}
}\\
\\
\afunI{map}{x:D[A], f:A\to B}{\anew{Map[A,B]}{iterator(x),f}}
\end{acode}

Many important data structures are naturally iterable, and that can be realized by implementing the abstract class.
That includes in particular all data structures for lists:
\begin{acode}
\aclass{ListIterator[A]}{l: List[A]}{Iterator[A]}{
  index := 0\\
  \afunI[\Bool]{hasNext}{}{index < length(l)} \\
  \afun[A]{getNext}{}{
    a := get(l, index) \\
    index := index + 1 \\
    a
  }
}\\
\\
\afunI[{Iterator[A]}]{iterator}{l:List[A]}{\anew{ListIterator}{l}}
\end{acode}

\section{Streams}

$Stream[A]$ is not a data structure for the set $A^*$.
Instead, it is a data structure for the set $A^\N$.

The set $A^\N$ contains functions $f:\N\to A$, which we can think of as infinite lists $[f(0),f(1),\ldots]$.
Because they are so similar to lists, they are usually treated together with lists, even though they do not realize the same set.

The set $A^\N$ is uncountable.
Therefore, not all possible streams are effective objects that can be represented in a physical machine.
However, for many practical purposes, it is fine to treat $Stream[A]$ as if it were the type of all possible streams.

$Stream[A]$ is usually implemented in the same way as $Iterator[A]$ with the understanding that $hasNext$ is always $\true$, i.e., the stream is never over.

Consequently, the functions on $Iterator[A]$ behave slightly differently when used for $Stream[A]$.
For exapmle:
\begin{compactitem}
 \item We cannot call $length$, $count$, $results$, $fold$, and $foreach$ on streams.
 \item We can call $contains$ on a stream. However, the function may run forever if the searched-for element is not in the stream.
 The same caveat applies to $index$, $find$, $forall$, and $exists$.
 \item We can call $map$ (because $map$ returns a new iterator without actually applying the map-function to all elements right away).
\end{compactitem}

\section{Heaps}\label{sec:ad:heaplists}

Heaps are formally defined in Sect.~\ref{sec:ad:heaps}.

$Heap[A,O]$ is not a data structure for the set $A^*$.
Instead, it is a data structure for the subset of $A^*$ containing only lists sorted according to $O$.
Therefore, heaps are very useful for sorting and prioritizing.
We discuss applications of heaps to lists in Sect.~\ref{sec:ad:heapqueue} and~\ref{sec:ad:heapsort}.

First we introduce some basic operations on heaps in Sect.~\ref{sec:ad:heapops}.

\subsection{Operations on Heaps}\label{sec:ad:heapops}

Because heaps are mostly used for efficiency, they are usually mutable.
The main operations on a heap are similar to those on a stack:

\begin{ctabular}{|l|l|l|}
\hline
function & returns & effect \\
\hline
$insert(x\in Heap[A,O], a\in A)\in\Unit$ & nothing & add $a$ to $x$ in any position\\
$extract(x\in Heap[A,O])\in A^?$ & the $O$-smallest element of $x$ (if any) & remove that element from $x$ \\
$find(x\in Heap[A,O])\in A^?$ & the $O$-smallest element of $x$ (if any) & none \\
\hline

\end{ctabular}

$insert$, $extract$, and $find$ for heaps correspond exactly to $push$, $pop$, and $top$ for stacks.
The crucial different is that $insert(x,a)$ does not prepend $a$ to $x$---instead, it is unspecified where and how $x$ is added.
$extract$ and $find$ do not return the most recently added element---instead, they return the smallest element with respect to $O$.

It is unspecified what exactly a heap looks like and where and how $insert$ actually performs the insertion.
That way heaps have a lot of freedom to organize the data in an efficient way.
That freedom is exploited to make the operations $extract$ and $find$ fast.
\medskip

Because $Heap[A,O]$ is underspecified, there are many different options how to implement heaps.
In practice, there are dozens of competing variants using different efficiency trade-offs.
A critical property is that all operations take only $O(\log n)$ where $n$ is the number of elements in the heap.

\subsection{Implementations}

The most important case of $Heap[A,O]$ are binary heaps $H$, i.e., binary trees over $A$ that are also heaps.
There is a wide variety of optimized implementations of heaps.


\subsubsection{Using Trees}

For a straightforward implementation, we use a tree.

Let $n$ be the number of nodes in $H$ and $h$ be the height of $H$.
All operations are such that $H$ remains almost-perfect: for every depth $d<h$ there are maximally many nodes, i.e., $2^d$ nodes.
At depth $h$, we have to allow for fewer than $2^h$ nodes because not every $n$ there is a perfect heap.
We use the convention that the nodes at level $h$ are as far to the left as possible.
That way, we always have $h\leq\log_2 n$, and all branches have length $h$ or $h-1$, i.e., $O(\log_2 n)$.

$find$ is trivial: We return the root of $H$. That takes $O(1)$.

$insert(H,x)$ inserts $x$ into one of the branches with minimal length.
If the heap is perfect, we extend it to a new level $h+1$ and insert $x$ all the way to the left.
Otherwise, we add it in the left-most free slot at level $h$.
The insertion occurs at the position that keeps the branch sorted.
Because it was sorted already, that requires $O(l)$ operations, where $l$ is the length of the branch, i.e., $O(\log_2 n)$.
It is easy to check that the resulting tree is again a heap.

$extract$ removes the root of $H$ and returns it.
That takes $O(1)$.
Additionally, we have to repair the heap property.
To do that, we take some leaf $l$ of $H$ and put it at the root.
Now we have an almost-perfect binary tree again, but it is not a heap yet: $l$ may be too big to be the root.
Therefore, we push $l$ down by iteratively swapping it with its smallest child until we have a heap.
Finding a leaf and pushing along some branch takes $O(\log_2 n)$.

\subsubsection{Using Arrays}

For efficiency, it is often preferable to store the nodes of the heap as an array.

This requires a bijection that translates between positions in the heap to positions in the array.
Let $h$ be the height of the heap.
Then a position in the heap is a list $p\in\{left,right\}^*$ of length up to $h$ that describes the path from the root to a node.
A position in the array is an integer $i\in\{0,\ldots,2^{h+1}-1\}$.

It is straightforward to give such a bijection.
For example, we can number the nodes of the heap in BFS order.
Then the heap-position $p$ corresponds to the array position $2^{length(p)}+j$ where $j$ is the number obtained by treating $p$ as a binary number (with $left$ and $right$ corresponding to $0$ and $1$).

\subsection{Priority Queues}\label{sec:ad:heapqueue}

A $PriorityQueue[A]$ behaves like a $Queue[A]$ except that dequeueing returns the element with the highest priority.

This is achieved by using a data structure for $Heap[A,O]$ where $O$ orders elements by decreasing priority.
Then $insert$ and $extract$ correspond to $enqueue$ and $dequeue$.

\subsection{Heapsort Algorithm}\label{sec:ad:heapsort}

Heapsort is a sorting algorithm that runs in $\Theta(n\log n)$.

If $\leq$ is the total order for sorting, a simple heapsort is given by
\begin{acode}
\afun[A^*]{heapsort}{x: A^*}{
  h := \anew{Heap[A,\geq]}{}\\
  \\
  left := x\\
  \afor{i}{0}{length(x)-1}{
    next := left.head \\
    insert(h,next)\\
    left := left.tail\\
  }\\
  res := Nil\\
  \afor{i}{0}{length(x)-1}{
    next := extract(h)\\
    res := prepend(next, res)
  }
}
\end{acode}

This uses two loops using $length(n)$ iterations each.
The first loop throws all elements of $x$ into the heap; the second loop pulls them out again and builds the list $res$ to be returned.
Because $extract$ always returns the greatest element, the result is automatically sorted.
Any other implementation of a priority queue yields a corresponding sorting algorithm.

If $n$ is the length of the list, each $insert$ and $extract$ operation takes at most $\Theta(\log n)$.
Thus, heapsort runs in $\Theta(n\log n)$.
\medskip

There are much more optimized implementations of heapsort than the above example, possibly using optimized implementations of heaps.
In particular, there are encodings of the heap structure in an array, which allow using heapsort as an in-place sorting algorithm.
With those optimizations, heapsort is among the fastest sorting algorithms (but still takes $\Theta(n\log n)$).


\chapter{Tree-Like Data Structures}\label{sec:ad:trees}
% exercise: proof Thm.~\ref{thm:ad:bintree}
 \section{Specification}

After lists, trees are the next most important data structure in computer science.
They can be seen as a generalization of lists where the elements are not arranged in a row, but branching is allowed.

\subsection{General Trees}

There are many equivalent definitions.
The easiest is by graphical example: A tree is something that looks like

\tikzstyle{node}=[circle,draw]
\begin{center}
\begin{tikzpicture}
\node[node] (0) at (0,0) {};
\node[node] (00) at (-2,-1) {};
\node[node] (01) at (0,-1) {};
\node[node] (02) at (2,-1) {};
\node[node] (000) at (-2,-2) {};
\node[node] (010) at (-1,-2) {};
\node[node] (011) at (1,-2) {};
\draw[arrow] (0) -- (00);
\draw[arrow] (0) -- (01);
\draw[arrow] (0) -- (02);
\draw[arrow] (00) -- (000);
\draw[arrow] (01) -- (010);
\draw[arrow] (01) -- (011);
\end{tikzpicture}
\end{center}

A more formal definition is this:

\begin{definition}[Tree]\label{def:ad:tree}
A \textbf{tree} is a connected directed graph in which
\begin{compactitem}
 \item there is exactly one node (called the \textbf{root}) with in-degree $0$,
 \item all other nodes have in-degree $1$.
\end{compactitem}
\end{definition}
Here we already used the more general concept of graphs, which we define formally in Sect.~\ref{sec:ad:graphs}.

Talking about the shape and parts of a tree can be confusing.
Therefore, we introduce some vocabulary that helps us:

\begin{definition}[Parts of a Tree]\label{def:ad:treeaux}
For every edge from $p$ to $c$, we call $p$ the \textbf{parent} of $c$ and $n$ a \textbf{child} of $p$.
Thus, the root has no parent; every non-root node has exactly one parent.
A node may have any number of children.
A node with $0$ children is called a \textbf{leaf}.
A node that is neither the root nor a child is called an \textbf{inner node}.

For every path from $a$ to $d$, we call $a$ an \textbf{ancestor} of $d$ and $d$ a \textbf{descendant} of $a$.
Thus, all nodes are descendants of the root
Every node is an ancestor/descendant of itself; a \textbf{proper} ancestor/descendant of $n$ is an ancestor/descendant that is not $n$.

The number of proper ancestors of $n$ is called the \textbf{depth} of $n$.
Thus, the root has depth $0$.

For a node $n$, the descendants of $n$ form a tree again, which has root $n$.
It is called the \textbf{subtree} at $n$.

A path from the root to a leaf is called a \textbf{branch}.
Thus, every leaf $l$ is part of exactly one branch, whose length is the depth of $l$.
The length of the longest branch(es) is called the \textbf{height} of the tree.
\end{definition}

Def.~\ref{def:ad:tree} only defines the abstract shape of trees.
But trees are only useful if we can store some data in each node.
For example, the following is a tree of integers:

\begin{center}
\begin{tikzpicture}
\node[node] (0) at (0,0) {5};
\node[node] (00) at (-2,-1) {3};
\node[node] (01) at (0,-1) {6};
\node[node] (02) at (2,-1) {1};
\node[node] (000) at (-2,-2) {0};
\node[node] (010) at (-1,-2) {6};
\node[node] (011) at (1,-2) {5};
\draw[arrow] (0) -- (00);
\draw[arrow] (0) -- (01);
\draw[arrow] (0) -- (02);
\draw[arrow] (00) -- (000);
\draw[arrow] (01) -- (010);
\draw[arrow] (01) -- (011);
\end{tikzpicture}
\end{center}

Once we store data in a tree, we have be a bit more careful: the order of children matters now.
For example, the above tree of integers is different from the tree of integers below even both are based on the same tree.

\begin{center}
\begin{tikzpicture}
\node[node] (0) at (0,0) {5};
\node[node] (00) at (-2,-1) {3};
\node[node] (01) at (0,-1) {6};
\node[node] (02) at (2,-1) {1};
\node[node] (000) at (-2,-2) {0};
\node[node] (010) at (-1,-2) {5};
\node[node] (011) at (1,-2) {6};
\draw[arrow] (0) -- (00);
\draw[arrow] (0) -- (01);
\draw[arrow] (0) -- (02);
\draw[arrow] (00) -- (000);
\draw[arrow] (01) -- (010);
\draw[arrow] (01) -- (011);
\end{tikzpicture}
\end{center}

Keeping track of the order makes the definition more complicated.
The following definition is one way possibility to define it formally:

\begin{definition}[Trees over a Set]\label{def:ad:labeledtree}
The set $Tree[A]$ contains the \textbf{trees over the set} $A$.
Such a tree over $A$ consists of
\begin{compactitem}
 \item a set $N$ (whose elements we call the nodes),
 \item a function $label:N\to A$ that maps nodes to elements of $A$ ($label(n)$ is called the label of $n$, it is the data stored in each node),
 \item a function $children:N\to N^*$ that maps every node to its list of children,
\end{compactitem}
such that $N$ and $children$ define a tree.
\end{definition}

\begin{remark}[Leaf-Labeled Trees]
$Tree[A]$ contains trees in which \emph{every} node stores data from $A$.
Occassionally, we are also interested in trees where only the \emph{leafs} are labeled.
And sometimes we need trees where inner nodes are labeled with elements of $A$ and leafs with elements of $B$.

We ignore those trees here.
But when working with someone else's tree data structures, it is important to check which nodes are labeled with what.
\end{remark}

\subsection{Binary Trees}

Binary trees are an important special case:

\begin{definition}[Binary Tree]\label{def:ad:bintree}
A \textbf{binary tree} is a tree in which all nodes have at most $2$ children.
If a node has $2$ children, they are called the \textbf{left} and \textbf{right} child.

Binary trees over a set are defined accordingly.

A binary tree is called \textbf{full} if all non-leaf nodes have exactly two children.
A full binary tree is called \textbf{complete} all all leafs have the same depth.
\end{definition}

For example, the following are, from left to right, a non-full, a full but not perfect, and a perfect binary tree of integers:
\begin{center}
\begin{tikzpicture}[scale=.7]
\node[node] (0) at (0,0) {5};
\node[node] (00) at (-2,-1) {3};
\node[node] (01) at (2,-1) {1};
\node[node] (010) at (1,-2) {6};
\draw[arrow] (0) -- (00);
\draw[arrow] (0) -- (01);
\draw[arrow] (01) -- (010);
\end{tikzpicture}
\tb\tb
\begin{tikzpicture}[scale=.7]
\node[node] (0) at (0,0) {5};
\node[node] (00) at (-2,-1) {3};
\node[node] (01) at (2,-1) {1};
\node[node] (010) at (1,-2) {6};
\node[node] (011) at (3,-2) {5};
\draw[arrow] (0) -- (00);
\draw[arrow] (0) -- (01);
\draw[arrow] (01) -- (010);
\draw[arrow] (01) -- (011);
\end{tikzpicture}
\tb\tb
\begin{tikzpicture}[scale=.7]
\node[node] (0) at (0,0) {5};
\node[node] (00) at (-2,-1) {3};
\node[node] (01) at (2,-1) {1};
\node[node] (000) at (-3,-2) {0};
\node[node] (001) at (-1,-2) {2};
\node[node] (010) at (1,-2) {6};
\node[node] (011) at (3,-2) {5};
\draw[arrow] (0) -- (00);
\draw[arrow] (0) -- (01);
\draw[arrow] (00) -- (000);
\draw[arrow] (00) -- (001);
\draw[arrow] (01) -- (010);
\draw[arrow] (01) -- (011);
\end{tikzpicture}
\end{center}


It is important to know the number of nodes in a binary tree:

\begin{theorem}\label{thm:ad:bintree}
A binary tree of height $h$ has at most $2^n$ nodes at depth $n$.
It has at most $2^{h+1}-1$ nodes in total.

If it is perfect, it has exactly $2^n$ nodes at depth $n$ and exactly $2^{h+1}-1$ nodes in total.
\end{theorem}
\begin{proof}
Exercise.
\end{proof}

In particular, the number of nodes grows exponentially with the depth.
Vice versa, we can organize $n$ nodes as a binary tree of height $\log_2 n$.
The latter property is often useful to obtain logarithmic implementations: if we organize $n$ elements in a (nearly) perfect binary tree, we can reach any element in $\log_2 n$ steps.

\subsection{Trees for Ordered Data}

\subsubsection{Binary Search Trees}

%\begin{definition}[Binary Search Trees]
%If $O$ is a total order on $A$, then $BST[A,O]$ is the subset of $Tree[A]$ containing only full binary trees in which ??? %needs optional left and optional right child
%\end{definition}



\subsubsection{Heaps}\label{sec:ad:heaps}

\begin{definition}[Heap]
If $O$ is a total order on $A$, then $Heap[A,O]$ is the subset of $Tree[A]$ containing only trees in which all branches are sorted with respect to $O$.
\end{definition}

The elements of $Heap[\Z,\leq]$ are also called \textbf{min-heaps}.
The elements of $Heap[\Z,\geq]$ are also called \textbf{max-heaps}.

The left tree below is a (binary) min-heap, the right one is neither a min-heap nor a max-heap:

\begin{center}
\begin{tikzpicture}
\node[node] (0) at (0,0) {5};
\node[node] (00) at (-2,-1) {12};
\node[node] (01) at (2,-1) {7};
\node[node] (010) at (1,-2) {12};
\node[node] (011) at (3,-2) {9};
\draw[arrow] (0) -- (00);
\draw[arrow] (0) -- (01);
\draw[arrow] (01) -- (010);
\draw[arrow] (01) -- (011);
\end{tikzpicture}
\tb\tb
\begin{tikzpicture}
\node[node] (0) at (0,0) {5};
\node[node] (00) at (-2,-1) {12};
\node[node] (01) at (2,-1) {3};
\node[node] (010) at (1,-2) {12};
\node[node] (011) at (3,-2) {4};
\draw[arrow] (0) -- (00);
\draw[arrow] (0) -- (01);
\draw[arrow] (01) -- (010);
\draw[arrow] (01) -- (011);
\end{tikzpicture}
\end{center}

In a heap, the every node is smaller than all its descendants.
The root is always the smallest element in the heap.
That makes heaps practical for sorting.
Applications are presented in Sect.~\ref{sec:ad:heaplists}.

\section{Data Structures}

Trees can be mutable or immutable.
However, trees are mostly used to store data.
Many algorithms work with a single mutable tree and insert data into it or delete data from it over time.

We consider two different data structures and use the following as an example tree
\begin{center}
\begin{tikzpicture}[scale=.7]
\node[node] (0) at (0,0) {5};
\node[node] (00) at (-2,-1) {3};
\node[node] (01) at (2,-1) {1};
\node[node] (010) at (1,-2) {6};
\draw[arrow] (0) -- (00);
\draw[arrow] (0) -- (01);
\draw[arrow] (01) -- (010);
\end{tikzpicture}
\end{center}

\subsection{Trees Using Lists}

The simplest data structure for trees uses lists:

\begin{acode}
\aclassI{Tree[A]}{data: A,\; children: List[Tree[A]]}{}{}
\end{acode}

The example tree is represented as
\[\anew{Tree[\Z]}{5,\; \big[\anew{Tree[\Z]}{3,Nil},\;\anew{Tree[\Z]}{1, [\anew{Tree[\Z]}{6,Nil}]}\big]}\].

\subsection{Trees Without Lists and With Null}

Some programmers or programming languages prefer a more awkward (but slightly less memory-intensive) data structure that does not use lists.

Here every node has two pointers: one to its first child and one to its next sibling:
\begin{acode}
\aclassI{Node[A]}{data: A,\; firstChild: Node[A], nextSibling: Node[A]}{}{}
\end{acode}
For leafs, the field $firstChild$ is $null$; for the last child of a node, the field $nextSibling$ is $null$.
It would be better not to use $null$. But programmers who use this data structure usually do not mind.

The example tree is represented as
\[\anew{Node[\Z]}{5,\; \anew{Tree[\Z]}{3,null, \;\anew{Tree[A]}{1, \anew{Tree[\Z]}{6,null,null}}, null}, null}\]

\section{Important Algorithms}

\subsection{Depth-First Search}

\subsection{Breadth-Frist Search}

\subsection{Min-Max Algorithm}

\section{Data Structures and Algotithms for Heaps}



\section{Search Trees}

Binary search trees and red-black trees are discussed in Sect.~\ref{sec:ad:sets}.



\chapter{Set-Like Data Structures}\label{sec:ad:sets}
% specification, list sets, binary search trees, red-black trees (BST with log n guarantee for insert/delete), hash sets

\chapter{Function-Like Data Structures}\label{sec:ad:functions}
% specification, list map, hash map

\chapter{Product-Like Data Structures}\label{sec:ad:products}
% specification, products, records, classes

\chapter{Union-Like Data Structures}\label{sec:ad:unions}
% disjoint union, Option/null, inductive types
 % exercise: 1567 -> 21

\chapter{Graph-Like Data Structures}\label{sec:ad:graphs}
% specification
% algorithms for graphs: search, minimum spanning tree, shortest path, maximum flow

\chapter{Algebraic Data Structures}\label{sec:ad:theories}
% based on one operation, based on two operations, based on one relation

\part{Important Families of Algorithms}\label{sec:ad:algo}

\chapter{Divide and Conquer}
% binary search vs. linear search
% Karatsuba: multiply polynomials by using upper/lower coefficients (prelim: multiply linear polynomials with 3 multiplications), variant: n-bit int multiplication
 
\chapter{Dynamic Programming}

\chapter{Greedy Algorithms}

% what else?
\chapter{Recursion}

\chapter{Backtracking}

\chapter{Randomization}

\chapter{Parallelization and Distribution}

\chapter{Protocols}

% one week of geometry


\part{Concrete Languages}

\chapter{Data Description Languages}
 
 \section{JSON}

 \section{XML}
 
 \section{UML}

\chapter{Programming Languages}
 

%\part{Conclusion}
%
%\chapter{Summary}
%  \footnote{This chapter may be outdated.}

\section{Type Theories}

\paragraph{Syntax}
The syntax of a typed language $L$ consists of signatures $\Sigma$, contexts $\Gamma$ for every signature, well-formed expressions in context, and subsitutions $\gamma$ for every pair of $\Sigma$-contexts that translate well-formed expressions.
In all cases, we can choose between a type-theoretic (judgmental) notation (in black) and a set-theoretic notation (in gray).

Languages differ (only) in what classes of well-formed expressions there are:
\begin{itemize}
	\item FOL: terms,
	\item STT, HOL: types and typed terms,
	\item LF: kinds, kinded type families and typed terms.
\end{itemize}

\begin{center}
\begin{tabular}{|ll|ll|}
\hline
\multicolumn{4}{|c|}{$\issig[L]{\Sigma}$ \tb {\color{gray}$\Sigma\in\Sig^L$}} \\[.2cm]
\hline
$\iscont[L]{\Sigma}{\Gamma}$                   & {\color{gray}$\Gamma\in\Con^L(\Sigma)$}    & 
$\issubs[L]{\Sigma}{\Gamma}{\Gamma'}{\gamma}$  & {\color{gray}$\gamma:\Gamma'\arr\Gamma'$} \\[.2cm]
\hline
$\oftype[L]{\Sigma}{\Gamma}{E_1}{E_2}$         & {\color{gray}$E_1\in\op{wf}^{L,\Sigma}(\Gamma,E_2)$}   &
$\oftype[L]{\Sigma}{\Gamma'}{\overline{\gamma}(E_1)}{\overline{\gamma}(E_2)}$
                             & {\color{gray}$\overline{\gamma}:\op{wf}^{L,\Sigma}(\Gamma,E_2)\arr \op{wf}^{L,\Sigma}(\Gamma',\overline{\gamma}(E_2))$} \\[.2cm]
\hline
\end{tabular}
\end{center}

There are two ways to translate the above table: into another signature along a signature morphism, or into the semantic domain along a model.

\paragraph{Translation into Other Signatures}
Here, the syntax is translated along a signature morphism $\sigma$ from $\Sigma$ to $\Sigma'$, contexts, substitutions, and well-formed expressions are translated to their counterparts.
The counterpart of the substitution-value-lemma is a lemma that states that substitutions $\ov{\gamma}(-)$ and signature morphisms $\ov{\sigma}(-)$ commute.

Languages differ (only) in how signature morphisms are represented, depending on what classes of well-formed expressions are present. In all cases, they can be written as list of pairs of a symbol and an expression.

\begin{center}
\begin{tabular}{|l|l|}
\hline
\multicolumn{2}{|c|}{$\ismorph{\Sigma}{\Sigma'}{\sigma}$}\\[.2cm]
\hline
$\iscont[L]{\Sigma'}{\ov{\sigma}(\Gamma)}$ & $\issubs[L]{\Sigma'}{\ov{\sigma}(\Gamma)}{\ov{\sigma}(\Gamma')}{\ov{\sigma}(\gamma)}$ \\[.2cm]
\hline
$\oftype[L]{\Sigma'}{\ov{\sigma}(\Gamma)}{\ov{\sigma}(E_1)}{\ov{\sigma}(E_2)}$  &
  $\ov{\sigma}\circ\ov{\gamma} = \ov{\ov{\sigma}(\gamma)}\circ\ov{\sigma}$\\[.2cm]
\hline
\end{tabular}
\end{center}

\paragraph{Translation into Models}
Here, signatures are interpreted by models $I$, contexts by an $I$-assignments $\alpha$, and the well-formed $\Gamma$-expressions $E$ as set-theoretical objects $\semm{E}{I,\alpha}$. Substitutions are interpreted as mappings between assignments, and the substitution-value lemma states that substitution $\ov{\gamma}(-)$ and interpretation $\sem{-}$ commute.

Languages differ (only) in what set-theoretical objects serve as the interpretation of well-formed expressions.
\begin{itemize}
	\item FOL: elements of some set for terms
	\item STT, HOL: sets for base types, elements of sets for terms,
	\item LF: classes for kinds, elements of classes (i.e., sets) for type families, elemens of sets for terms.
\end{itemize}

\begin{center}
\begin{tabular}{|l|l|}
\hline
\multicolumn{2}{|c|}{$I\in\Mod^L{\Sigma}$}\\[.2cm]
\hline
$\alpha\in\semm{\Gamma}{I}$ & $\semm{\gamma}{I}:\semm{\Gamma'}{I}\arr\semm{\Gamma}{I}$ \\[.2cm]
\hline
$\semcm{\Gamma}{E_1}{I,\alpha}\in\semcm{\Gamma}{E_2}{I,\alpha}$ & $\semcm{\Gamma}{\ov{\gamma}(E)}{I,\alpha'}=\semcm{\Gamma}{E}{I,\semm{\gamma}{I}(\alpha')}$ \\[.2cm]
\hline
\end{tabular}
\end{center}


\section{Logics}

Logics adds formulas and consequence to the above pictures.

\paragraph{Syntax}
The syntax of a logic $L$ adds formulas and proofs as classes of well-formed expressions. Sometimes these are new classes, sometimes they are special cases of existing classes (e.g., terms). In the former case, all definitions and results have to be extended with the new cases. In particular, new judgments for well-formed formulas and well-formed proofs must be added. In the latter case, these are special cases of existing definitions and results.
\begin{itemize}
	\item FOL standalone: Both formulas and proofs are new. Therefore, all definitions and results must be extended for them. $F:\FORM$ and $p:\PROOF\;F$ are just notations for well-formed formulas and proofs, which happen to use a colon but have no connection to typing.
	\item HOL in STT: $\FORM$ can be defined as a type and formulas as terms of that type. Therefore, all properties of formulas are inherited from STT; in particular, well-formedness of formulas is a special case of typing. For proofs, definitions and results must be extended.
	\item FOL or HOL in LF: $\FORM$ and $\PROOF\;F$ can be defined as types and formulas and proof as terms of these types. Therefore, no new definitions or results are necessary; in particular, well-formedness of terms and proofs (possibly using assumptions) are special cases of typing.
\end{itemize}

Once the judgments for formulas and proofs are in place, we obtain the notions of theories, sentences, and proof-theoretical consequence. Usually, these judgments are most interesting for the case of empty contexts (but the general is needed to make the involved induction go through).

The type-theoretical method builds on explicit proofs. Thus, theories lists $\Delta=a_1:\PROOF\;F_1,\ldots,a_n:\PROOF\;F_n$ of named assumptions. The model-theoretical method considers proofs to be of secondary importance and writes theories as set $\Theta=\{F_1,\ldots,F_n\}$ of sentences.

\begin{center}
\begin{tabular}{|ll|}
\hline
$\issig[L]{\Sigma}$                      & {\color{gray}$\Sigma\in\Sig^L$} \\[.2cm]
%$\iscont[L]{\Sigma}{\Delta}$             & {\color{gray}$(\Sigma,\Theta)\in\Th^L$} \\[.2cm]
\hline
%$\oftype[L]{\Sigma}{\Gamma}{F}{\FORM}$   &  \\
$\oftype[L]{\Sigma}{\cdot}{F}{\FORM}$    & {\color{gray}$F\in\Sen^L(\Sigma)$} \\[.2cm]
\hline
%$\oftype[L]{\Sigma}{\Gamma;\Delta}{p}{\PROOF\;F}$ & \\
\multicolumn{2}{|c|}{$\oftype[L]{\Sigma}{\cdot;\Delta}{p}{\PROOF\;F}$} 
%& {\color{gray}$F\in\Thm^L(\Sigma,\Theta)$}
\\[.2cm]
\hline
\end{tabular}
\end{center}

\paragraph{Translation into Other Signatures}
Signature morphisms translate formulas to formulas and proofs to proofs. Depending on whether formulas and proofs are special cases of terms, this may or may not require no definitions.

Once the translation of formulas is in place, we obtain the translation of theories by translating the axioms component-wise.

\begin{center}
\begin{tabular}{|ll|}
\hline
\multicolumn{2}{|c|}{$\ismorph[L]{\Sigma}{\Sigma'}{\sigma}$}\\[.2cm]
\hline
$\isform[L]{\Sigma'}{\cdot}{\ov{\sigma}(F)}$ & {\color{gray}$\Sen^L(\sigma):\Sen^L(\Sigma)\arr\Sen^L(\Sigma')$} \\[.2cm]
\hline
\multicolumn{2}{|c|}{$\oftype[L]{\Sigma'}{\cdot;\ov{\sigma}(\Delta)}{\ov{\sigma}(p)}{\PROOF\;\ov{\sigma}(F)}$}
% & {\color{gray}$\Sen^L(\sigma):\Thm^L(\Sigma,\Theta)\arr\Thm^L(\Sigma',\ov{\sigma}(\Theta))$}
\\[.2cm]
\hline
\end{tabular}
\end{center}

\paragraph{Translation into Models}
Models interpret formulas as truth values. Logics may differ in the interpretation of $\FORM$, the set of truth values, which at least contains one designated truth value $1$. Usually, the of truth values is $\{0,1\}$. Models do not interpret proofs but soundness requires them to respect proofs in the sense that provable formulas must be interpreted as $1$.

Once the interpretation of formulas is in place, we obtain the interpretation of theories $(\Sigma,\Theta)$ as the collection $\Sigma$-models satisfying all axioms in $\Theta$.

\begin{center}
\begin{tabular}{|c|}
\hline
$I\in\Mod^L(\Sigma)$\\[.2cm]
\hline
$\moda{I}{\Sigma}{F}$ \tb iff \tb $\semm{F}{I}=1$ \\[.2cm]
\hline
if $I\in\Mod^L(\Sigma,\Theta)$, then $\moda{I}{\Sigma}{F}$ \\[.2cm]
\hline
\end{tabular}
\end{center}

\paragraph{Consequence}
Proof-theoretical consequence is defined in terms of well-formed proofs. Model-theoretical consequence is defined in terms of models. In both cases, we obtain definitions of well-formed theory morphisms as consequence-preserving signature morphisms.

\begin{center}
\begin{tabular}{|c|c|}
\hline
$\dera[L]{\Theta}{\Sigma}{F}$ \tb iff\tb ex. $p$ s.t. $\oftype[L]{\Sigma}{;\Delta}{p}{\PROOF\;F}$
  & {\color{gray} $\moda[L]{\Theta}{\Sigma}{F}$ \tb iff \tb f.a. $I\in\Mod^L(\Sigma,\Theta)$ holds $\moda[L]{I}{\Sigma}{F}$} \\ [.2cm]
\hline
\multicolumn{2}{|c|}{$\ismorph[L]{\Sigma}{\Sigma'}{\sigma}$ theory morphism from $(\Sigma,\Theta)$ to $(\Sigma',\Theta')$ \tb iff \tb for all $F_i\in\Theta$} \\[.2cm]
$\dera[L]{\Theta}{\Sigma}{F}$ & {\color{gray}$\moda[L]{\Theta}{\Sigma}{F}$} \\[.2cm]
\hline
\end{tabular}
\end{center}

\paragraph{Soundness and Completeness}
Soundness and completeness are the two implications between the two different possibilities to define consequence. While soundness is indispensable, completeness may sometimes be sacrificed to avoid using undecidable sets of rules or unusual models.

FOL is essentially the most complicated logic for which completeness holds.


\part{Appendix}

\appendix

\chapter{Mathematical Preliminaries}\label{sec:math}

\section{Power Sets}\label{sec:math:powerset}

\input{\currfiledir powerset}

\section{Relations and Functions}\label{sec:math:relfun}

\input{\currfiledir relfun}

\section{Binary Relations on a Set}\label{sec:math:binrel}

\input{\currfiledir binrel}

\section{Binary Functions on a Set}\label{sec:math:binop}

\input{\currfiledir binop}

\section{The Integer Numbers}\label{sec:math:int}

\input{\currfiledir integers}

\section{Size of Sets}\label{sec:math:setsize}

\input{\currfiledir setsize}

\section{Important Sets and Functions}\label{sec:math:sets}

\input{\currfiledir sets}



\tocentryBib

\input{\currfilebase.bblp}
%\bibliographystyle{alpha}
%\bibliography{../../../../../frabe/Program_Data/Latex/bib/rabe,../../../../../frabe/Program_Data/Latex/bib/historical}
%\bibliography{../../../../../homedir/Program_Data/Latex/bib/rabe,../../../../../homedir/Program_Data/Latex/bib/historical}

\end{document}