\section{Void}

The set $\Void$ contains no elements.

Not surprisingly, it is rarely used.
However, it is nice to have when dealing with operations that do not return.
For example, we say that throwing an exception or terminating the program returns an element of $\Void$.

Most programs do not need the type void.
And most programming language either do not have it or only have it under the hood.

\section{Unit}

The set $\Unit$ contains exactly one element, which we write $()$.

It is rarely used because if we know that $x\in\Unit$, we already know the value of $x$.
Thus, having a value of type $\Unit$ gives us no information.

However, $\Unit$ is nice to have when dealing with operations that do return, but do not return a value.
In that case, we say that the operation returns type $\Unit$.

For example, assignments, loops, and print statements return $\Unit$.
Many methods of mutable data structures also return $\Unit$.
For example, using $\Unit$, we can specify $insert$ for a mutable list from Sect.~\ref{sec:ad:list:spec} as $insert(x\in A^*,a\in A,n\in\N)\in\Unit$.

Functional programming languages usually have a built-in type $\Unit$.
That way, in a functional programming langauge, every operation has a return type.

\section{Booleans}

The set $\Bool$ contains exactly two elements, which we call $\true$ and $\false$.

Most programming languages have a built-in type $\Bool$, which is the result type of the equality operator.

\section{Integers Modulo}

For $m>0$, the set $\Z_m$ consists of the elements $\{0,\ldots,m-1\}$.

Most programming languages do not offer $\Z_m$ for every $m$.
Usually, they offer at most $\Z_{2^k}$ for $k=8$ (usally called \emph{byte}), $k=16$ (\emph{word}), $k=32$ (\emph{integer}), and/or $k=64$ (\emph{long}).

Note that, depending on the programming language, the built-in type $\Int$ may refer to one of those (usually for $k=32$) or to $\Z$.

If we need $\Z_m$ for a specific $m$, we usually work with $\Int$ and use the $\modop$ operation to ensure we remain inside $\Z_m$.
\footnote{Note that some programming languages implement $\divop$ and $\modop$ in unexpected ways for negative arguments.}

\section{Enumerations}

For fresh names $l_1,\ldots,l_n$, the set $\Enum\{l_1,\ldots,l_n\}$ has exactly $n$ elements, which are called $l_1,\ldots,l_n$.

The names $l_i$ must be fresh.
That means they may not have been defined previously.
This is similar to how the name of a new function or class must be fresh.
This is because defining an enumeration set introduces new values, namely the $l_i$.

Most programming languages allow defining enumeration types in some way.
For example, in SML:

\begin{lstlisting}
datatype answer = yes | no | maybe
\end{lstlisting}

Or in C:

\begin{lstlisting}
enum {yes, no, maybe} answer;
\end{lstlisting}
