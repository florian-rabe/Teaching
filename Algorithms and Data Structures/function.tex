\section{Specification}

Functions implement the set $A\to B$, also written $B^A$.
Two fundamentally different ways of treating functions must be distinguished.

\emph{Algorithmic functions} are defined by algorithms that transform the input of type $A$ into the output of type $B$.
They are always immutable, and the only operation on them is function application:
\begin{ctabular}{|l|l|}
\hline
function & returns \\
\hline
$apply[A,B](f\in A\to B,a\in A)\in B$ & $f(a)$\\
\hline
\end{ctabular}

\emph{Tabular functions} are finite sets of pairs $(a,b)\in A\times B$ indicating that $a$ should be mapped to $b$.
They are usually partial functions (unless $A$ is finite and the function contains a pair for every $a\in A$).
Apart from possibly being partial, they are a special case of algorithmic functions: We can define $f(a)$ by searching for the pair $(a,b)$ in $f$ and returning $b$.

Tabular functions can be mutable or immutable.
We only give the mutable case, which is the most useful in practice.
The main operations are
\begin{ctabular}{|l|l|l|}
\hline
function & returns & effect \\
\hline
$isDefined[A,B](f\in A\to B, a\in A)\in \B$ & true if $f$ is defined for $a$ & none \\
$apply[A,B](f\in A\to B,a\in A)\in B^?$ & $f(a)$ if defined & none\\
$keys[A,B](f\in A\to B)\in List[A]$ & values for which $f$ is defined & none \\
\hline
$update[A,B](f\in A\to B, a\in A, b\in B)\in \Unit$ & nothing & change $f$ to also map $a$ to $b$ \\
\hline
\end{ctabular}

\section{Data Structures for Algorithmic Functions}

\subsection{As a Primitive Type}

All functional programming languages include a primitive feature to form the function type $A\to B$.
In fact, the existence of this type is the defining characteristic of being a functional language.
Individual functions that map every $x:A$ to $t(x)$ are built using $\lambda$-abstraction $\alam[A]{x}{t(x)}$.

Most untyped languages also provide an untyped $\lambda$-abstraction as $\alam{x}{t(x)}$.

\subsection{As a Class}

In non-functional object-oriented languages, we can define the type $A\to B$ as a class:

\begin{acode}
\acomment{$A\to B$}\\
\aclassA{Function[A,B]}{}{}{
 \afun[B]{apply}{x:A}{}
}\\
\tb\acomment{$\alam[A]{x}{t(x)}$}\\
\anew{Function[A,B]}{}\ablock{
 \afun[B]{apply}{x:A}{t(x)}
}
\end{acode}

Thus, object-oriented languages are in some sense also functional, except that the syntax for $\lambda$-abstractions is very awkward.

\section{Data Structures for Tabular Functions}

Data structures for tabular functions $TableFun[A,B]$ can be easily realized by using data structures for $Set[A\times B]$.
If the programming language does not offer product types or in order to optimize better, the corresponding data structures for sets can be modified to handle functions.
We give some examples.

\subsection{List Functions}

\paragraph{Design}
It is straightforward to define the data structure $ListFun[A,B]$ of list-backed functions in analogy to $ListSet[A]$.
It represents the function that maps $a_i$ to $b_i$ as the list $[(a_1,b_1),\ldots,(a_n,b_m)]$.

\paragraph{Complexity}
The complexity properties are essentially the same as for $ListSet[A]$.
$isDefined$, $apply$, and $update$ are linear in the size of the function because we have to find the corresponding pair in the list.

\subsection{Hash Tables}

\paragraph{Design}
For a hash function $hash:A\to \Z_m$, the data structure $HashTable[A,B]=Array[ListFun[A,B]](m)$ holds functions backed by a hash set.

For a hash table $h:HashTable[A,B]$, each $h[i]$ holds the list of pairs $(a,b)$ for which $hash(a)=i$.

\paragraph{Complexity}
The complexity properties are essentially the same as for $HashSet[A]$.

\subsection{Binary Search Trees}

\paragraph{Design}
Binary search trees can be easily generalized to represent a function $f$.
The basic idea is to make a binary search tree for the set of keys.
Each node in the tree then stores not only the key $a$ but also the corresponding value $f(a)$.

\paragraph{Complexity}
The complexity properties are essentially is the same as when representing sets.

% includes list
