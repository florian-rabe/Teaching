The specification and several data structures for mutable and immutable lists are already discussed in Sect.~\ref{sec:ad:listsort:spec}.

Here we only discuss some additional data structures for the set $A^*$.

\section{Stacks}\label{sec:ad:stack}

$Stack[A]$ is a data structure for the set $A^*$.

$Stack[A]$ is very similar to $List[A]$.
The difference is that $Stack[A]$ provides \emph{less} functionality.
While $List[A]$ is a general-purpose list, $Stack[A]$ is custom-fitted to one specific, very common use case.
By requiring fewer operations, they allow more optimized implementations.

Stacks can be mutable or (less commonly) immutable.
Here we will use the mutable variant.
The functions for mutable stacks are:

\begin{ctabular}{|l|l|l|}
\hline
function & returns & effect \\
\hline
$push(x\in A^*, a\in A)\in\Unit$ & nothing & prepend $a$ to $x$\\
$pop(x\in A^*)\in A^?$ & the first element of $x$ (if any) & remove the first element of $x$ \\
$top(x\in A^*)\in A^?$ & the first element of $x$ (if any) & none \\
\hline
\end{ctabular}

The intuition behind stacks is that they provide a LIFO store of data.
LIFO means last-in-first-out because every $pop$ returns the most recently pushed value.
This is exactly the behavior of a literal stack of items: We can put an item on top of a stack ($push$), remove an item from the stack ($pop$), or check what item is on top ($top$).
We cannot easily see or remove the other items.

Very often, the LIFO behavior is exactly what is needed.
For example, when we solve a maze, we can push every decision we make.
When we hit a dead end, we trace back our steps---for that, we have to pop the most recent decision, and so on.

\section{Queues}\label{sec:ad:queue}

Queues are very similar to stacks.
Everything about stacks also applies to queues except for the following.

The functions for mutable queues are:

\begin{ctabular}{|l|l|l|}
\hline
function & returns & effect \\
\hline
$enqueue(x\in A^*, a\in A)\in\Unit$ & nothing & append $x$ to $A$\\
$dequeue(x\in A^*)\in A^?$ & the first element of $x$ & remove the first element of $x$ \\
$empty(x\in A^*)\in\Bool$ & true if $x$ is empty & none \\
\hline
\end{ctabular}

The intuition behind queues is that they provide a FIFO store of data.
FIFO means first-in-first-out because every $dequeue$ returns the least recently enqueued value.
This is exactly the behavior of a literal queue of people: Every newcomer has to queue up at the end of the queue ($enqueue$), and every time a server is ready the first in line gets served ($dequeue$).
Newcomers cannot cut in line, and the server cannot easily see who else is waiting.

Very often, the FIFO behavior is exactly what is needed.
For example, when we have a list of tasks that need to be done.
Every time we create a new task, we enqueue it, and whenever we have time we dequeue the next task.

Queues are often used when components exchange messages or commands.
In that case, some components---called the producers---only call enqueue, and other components---called the consumers---only call dequeue.
For example, the producers can be different programs, $A$ is the type of print jobs, and the consumers are different printers.

More complex queue data structures may also for dequeueing based on priority (see also Sect.~\ref{sec:ad:heapqueue}).

\section{Buffers}\label{sec:ad:buffer}

Buffers are conceptually very similar to queues.
But $\mathit{Buffer}[A]$ is usually optimized for enqueueing and dequeueing many elements of $A$ at once.
Therefore, while stacks and queues can be implemented well using linked lists, buffers usually use arrays to be faster.

A typical $\mathit{Buffer}[A]$ consists of three components:
\begin{compactitem}
 \item an $Array[A]$ $b$
 \item two integers $begin$ and $end$ indicating the first and last valid entry in the array.
\end{compactitem}
Enqueueing writes to $b[end+1]$ and increments $end$.
Dequeueing reads from $b[begin]$ and increments $begin$.


A buffer overflow occurs when incrementing $begin$ 

For example, when a browser receives a web page, its network component loads the page into a $\mathit{Buffer}[\Char]$.
In parallel, its HTML parser component starts processing the partially received page.
That way the HTML page can be displayed partially already before it is fully loaded.

Buffers are almost always used automatically when a program is writing to a file.
In that case, a $\mathit{Buffer}[\Int]$ or $\mathit{Buffer}[\Char]$ is used that holds the data written to the file.
The write command does not actually write data to the file directly---it only enqueues it in the buffer.
That is advantageous because enqueueing to a buffer in memory is much faster than writing to the hard drive.
While the program is already moving on, the programming language libraries or the operating system work in the background to periodically dequeue and write all characters to the file.

When working with files, an important operation is \emph{flushing} the buffer.
This forces the immediate processing of all data in the buffer.
Flushing happens automatically at the latest when the program terminates.
However, occasionally manual flushing is necessary:
\begin{compactitem}
 \item When a program terminates with an error, buffers have to be flushed to avoid losing data.
 \item When a program writes log data to a file that the programmer wants to read immediately, it is important to flush regularly to make sure the programmer reads updated information.
\end{compactitem}

\section{Iterators}\label{sec:ad:iter}

\subsection{Specification}

$Iterator[A]$ is a data structure for the set $A^*$.

Iterators are usually mutable.
Their functionality is even more restricted than the one of stacks and queues:

\begin{ctabular}{|l|l|l|}
\hline
function & returns & effect \\
\hline
$getNext(x\in A^*)\in A$ & the first element of $x$ & remove the first element of $x$ \\
$hasNext(x\in A^*)\in\Bool$ & $\true$ if $x$ is not empty & none \\
\hline
\end{ctabular}

The typical way to use an iterator $i\in Iterator[A]$ is the following:
\begin{acode}
\awhile{hasNext(i)}{
  a := getNext(i)\\
  \text{do something with $a$ here}
}
\end{acode}
This is called \textbf{traversing} the iterator.
Afterwards the iterator is traversed and cannot be used again.

$Iterator[A]$ may look somewhat boring.
In order to understand the value of iterators, we have to make one definition:
A data structure $D[A]$ is called \textbf{iterable} if there is a function
 \[iterator(x\in D[A])\in Iterator[A]\]

Now the imoprtance of iterators follows from two facts:
\begin{compactitem}
 \item Many data structures $D$ are iterable (see Sect.~\ref{sec:ad:iter:create}).
 \item Many important operations for $D$ can be realized using only the functionality of iterators (see Sect.~\ref{sec:ad:iter:use}).
\end{compactitem}
Thus, iterators provide a sweet-spot in the trade-off between simplicity and expressivity---they are very simple,  but we can do a lot with them.

\begin{remark}[Simplicity vs. Expressivity]
The trade-offs between simplicity and expressivity comes up again and again in computer science.
The best data structures combine both properties, but usually they are mutually exclusive.

All the important data structures presented in Part~\ref{sec:ad:ds} have become important because they do well in this way.
\end{remark}

\subsection{Working with Iterable Data Structures}\label{sec:ad:iter:use}

Let us assume an iterable data structure $D[A]$.
Our goal is to define functions on $x\in D[A]$ that use only $iterator(x)$.
There are indeed many of those.
Some important ones are:
\begin{ctabular}{|l@{}l@{}l|l|}
\hline
function &&& returns \\
\hline
\multicolumn{4}{|c|}{below, let $X=iterator(x)$}\\
$length$&$(x\in D[A])$&$\in \N$ & numbers of elements in $X$ \\
$contains$&$(x\in D[A],\; a\in A)$&$\in \Bool$ & $\true$ if $a$ occurs in $X$ \\
$index$&$(x\in D[A],\; a\in A)$&$\in \N^?$ & the position of the first occurrence of $a$ in $X$ (if any)\\
$find$&$(x\in D[A],\; p\in A\to\Bool)$&$\in A^?$ & the first element $a$ in $X$ (if any) such that $p(a)$ is $\true$ \\
$count$&$(x\in D[A],\; p\in A\to\Bool)$&$\in\N$ & the number of elements $a$ in $X$ for which $p(a)$ is $\true$ \\
$forall$&$(x\in D[A],\; p\in A\to\Bool)$&$\in \Bool$ & $\true$ if $p(a)$ is $\true$ for every element $a$ in $X$ \\
$exists$&$(x\in D[A],\; p\in A\to\Bool)$&$\in \Bool$ & $\true$ if $p(a)$ is $\true$ for some element $a$ in $X$ \\
$map$&$(x\in D[A],f\in A\to B)$&$\in Iterator[B]$ & an iterator for $[f(a_1),\ldots,f(a_n)]$ where $x=[a_1,\ldots,a_n]$ \\
$filter$&$(x\in D[A],p\in A\to \Bool)$&$\in Iterator[B]$ & like $x$ but skips elements that do not satisfy $p$ \\
$results$&$(x\in D[A],\;f\in A\to B)$&$\in List[B]$ & the list of results from applying $f$ to all $a$ in $X$ \\
$fold$&$(x\in D[A],\; b\in B, f\in A\times B\to B)$&$\in B$ & $f(a_1,f(a_2,\ldots,f(a_n,b))\ldots)$ with $X=[a_1,\ldots,a_n]$\\
\hline
\end{ctabular}
All of the above functions should not have a side-effect.
However, some of them take other functions as arguments.
It is usually a bad to do so, but it is technically possible that these functions have side-effects.
There is only one exception where we explicitly allow $f$ to have a side-effect:
\begin{ctabular}{|l|l|l|}
\hline
function & returns & effect \\
\hline
$foreach(x\in D[A],f\in A\to \Unit)\in \Unit$ & nothing & apply $f$ to all $a$ in $X$ \\
\hline
\end{ctabular}

The trick behind $map$ (and the difference to $results$) is that $x$ is not traversed right away.
Instead, we create a new iterator that, when traversed, applies $f$.
That way we ensure that $f$ is applied only as often as necessary.


\subsection{Making Data Structures Iterable}\label{sec:ad:iter:create}

We can give a data structure for iterators as an abstract class:
\begin{acode}
\aclassA{Iterator[A]}{}{}{
 \afun[\Bool]{hasNext}{}{}
 \acomment{precondition for $getNext$ is $hasNext==\true$}\\
 \afun[A]{getNext}{}{}
}
\end{acode}

Then we can define, e.g.,  $map$ as follows:
\begin{acode}
\aclass{Map[A,B]}{x:Iterator[A], f:A\to B}{Iterator[B]}{
  \afunI[\Bool]{hasNext}{}{x.hasNext}\\
  \afunI[B]{getNext}{}{f(x.getNext)}
}\\
\\
\afunI{map}{x:D[A], f:A\to B}{\anew{Map[A,B]}{iterator(x),f}}
\end{acode}

Many important data structures are naturally iterable, and that can be realized by implementing the abstract class.
That includes in particular all data structures for lists:
\begin{acode}
\aclass{ListIterator[A]}{l: List[A]}{Iterator[A]}{
  index := 0\\
  \afunI[\Bool]{hasNext}{}{index < length(l)} \\
  \afun[A]{getNext}{}{
    a := get(l, index) \\
    index := index + 1 \\
    a
  }
}\\
\\
\afunI[{Iterator[A]}]{iterator}{l:List[A]}{\anew{ListIterator}{l}}
\end{acode}

\section{Streams}

$Stream[A]$ is not a data structure for the set $A^*$.
Instead, it is a data structure for the set $A^\N$.

The set $A^\N$ contains functions $f:\N\to A$, which we can think of as infinite lists $[f(0),f(1),\ldots]$.
Because they are so similar to lists, they are usually treated together with lists, even though they do not realize the same set.

The set $A^\N$ is uncountable.
Therefore, not all possible streams are effective objects that can be represented in a physical machine.
However, for many practical purposes, it is fine to treat $Stream[A]$ as if it were the type of all possible streams.

$Stream[A]$ is usually implemented in the same way as $Iterator[A]$ with the understanding that $hasNext$ is always $\true$, i.e., the stream is never over.

Consequently, the functions on $Iterator[A]$ behave slightly differently when used for $Stream[A]$.
For exapmle:
\begin{compactitem}
 \item We cannot call $length$, $count$, $results$, $fold$, and $foreach$ on streams.
 \item We can call $contains$ on a stream. However, the function may run forever if the searched-for element is not in the stream.
 The same caveat applies to $index$, $find$, $forall$, and $exists$.
 \item We can call $map$ (because $map$ returns a new iterator without actually applying the map-function to all elements right away).
\end{compactitem}

\section{Heaps}\label{sec:ad:heaplists}

Heaps are formally defined in Sect.~\ref{sec:ad:heaps}.

$Heap[A,O]$ is not a data structure for the set $A^*$.
Instead, it is a data structure for the subset of $A^*$ containing only lists sorted according to $O$.
Therefore, heaps are very useful for sorting and prioritizing.
We discuss applications of heaps to lists in Sect.~\ref{sec:ad:heapqueue} and~\ref{sec:ad:heapsort}.

First we introduce some basic operations on heaps in Sect.~\ref{sec:ad:heapops}.

\subsection{Operations on Heaps}\label{sec:ad:heapops}

Because heaps are mostly used for efficiency, they are usually mutable.
The main operations on a heap are similar to those on a stack:

\begin{ctabular}{|l|l|l|}
\hline
function & returns & effect \\
\hline
$insert(x\in Heap[A,O], a\in A)\in\Unit$ & nothing & add $a$ to $x$ in any position\\
$extract(x\in Heap[A,O])\in A^?$ & the $O$-smallest element of $x$ (if any) & remove that element from $x$ \\
$find(x\in Heap[A,O])\in A^?$ & the $O$-smallest element of $x$ (if any) & none \\
\hline

\end{ctabular}

$insert$, $extract$, and $find$ for heaps correspond exactly to $push$, $pop$, and $top$ for stacks.
The crucial different is that $insert(x,a)$ does not prepend $a$ to $x$---instead, it is unspecified where and how $x$ is added.
$extract$ and $find$ do not return the most recently added element---instead, they return the smallest element with respect to $O$.

It is unspecified what exactly a heap looks like and where and how $insert$ actually performs the insertion.
That way heaps have a lot of freedom to organize the data in an efficient way.
That freedom is exploited to make the operations $extract$ and $find$ fast.
\medskip

Because $Heap[A,O]$ is underspecified, there are many different options how to implement heaps.
In practice, there are dozens of competing variants using different efficiency trade-offs.
A critical property is that all operations take only $O(\log n)$ where $n$ is the number of elements in the heap.

\subsection{Implementations}

The most important case of $Heap[A,O]$ are binary heaps $H$, i.e., binary trees over $A$ that are also heaps.
There is a wide variety of optimized implementations of heaps.


\subsubsection{Using Trees}

For a straightforward implementation, we use a tree.

Let $n$ be the number of nodes in $H$ and $h$ be the height of $H$.
All operations are such that $H$ remains almost-perfect: for every depth $d<h$ there are maximally many nodes, i.e., $2^d$ nodes.
At depth $h$, we have to allow for fewer than $2^h$ nodes because not every $n$ there is a perfect heap.
We use the convention that the nodes at level $h$ are as far to the left as possible.
That way, we always have $h\leq\log_2 n$, and all branches have length $h$ or $h-1$, i.e., $O(\log_2 n)$.

$find$ is trivial: We return the root of $H$. That takes $O(1)$.

$insert(H,x)$ inserts $x$ into one of the branches with minimal length.
If the heap is perfect, we extend it to a new level $h+1$ and insert $x$ all the way to the left.
Otherwise, we add it in the left-most free slot at level $h$.
The insertion occurs at the position that keeps the branch sorted.
Because it was sorted already, that requires $O(l)$ operations, where $l$ is the length of the branch, i.e., $O(\log_2 n)$.
It is easy to check that the resulting tree is again a heap.

$extract$ removes the root of $H$ and returns it.
That takes $O(1)$.
Additionally, we have to repair the heap property.
To do that, we take some leaf $l$ of $H$ and put it at the root.
Now we have an almost-perfect binary tree again, but it is not a heap yet: $l$ may be too big to be the root.
Therefore, we push $l$ down by iteratively swapping it with its smallest child until we have a heap.
Finding a leaf and pushing along some branch takes $O(\log_2 n)$.

\subsubsection{Using Arrays}

For efficiency, it is often preferable to store the nodes of the heap as an array.

This requires a bijection that translates between positions in the heap to positions in the array.
Let $h$ be the height of the heap.
Then a position in the heap is a list $p\in\{left,right\}^*$ of length up to $h$ that describes the path from the root to a node.
A position in the array is an integer $i\in\{0,\ldots,2^{h+1}-1\}$.

It is straightforward to give such a bijection.
For example, we can number the nodes of the heap in BFS order.
Then the heap-position $p$ corresponds to the array position $2^{length(p)}+j$ where $j$ is the number obtained by treating $p$ as a binary number (with $left$ and $right$ corresponding to $0$ and $1$).

\subsection{Priority Queues}\label{sec:ad:heapqueue}

A $PriorityQueue[A]$ behaves like a $Queue[A]$ except that dequeueing returns the element with the highest priority.

This is achieved by using a data structure for $Heap[A,O]$ where $O$ orders elements by decreasing priority.
Then $insert$ and $extract$ correspond to $enqueue$ and $dequeue$.

\subsection{Heapsort Algorithm}\label{sec:ad:heapsort}

Heapsort is a sorting algorithm that runs in $\Theta(n\log n)$.

If $\leq$ is the total order for sorting, a simple heapsort is given by
\begin{acode}
\afun[A^*]{heapsort}{x: A^*}{
  h := \anew{Heap[A,\geq]}{}\\
  \\
  left := x\\
  \afor{i}{0}{length(x)-1}{
    next := left.head \\
    insert(h,next)\\
    left := left.tail\\
  }\\
  res := Nil\\
  \afor{i}{0}{length(x)-1}{
    next := extract(h)\\
    res := prepend(next, res)
  }
}
\end{acode}

This uses two loops using $length(n)$ iterations each.
The first loop throws all elements of $x$ into the heap; the second loop pulls them out again and builds the list $res$ to be returned.
Because $extract$ always returns the greatest element, the result is automatically sorted.
Any other implementation of a priority queue yields a corresponding sorting algorithm.

If $n$ is the length of the list, each $insert$ and $extract$ operation takes at most $\Theta(\log n)$.
Thus, heapsort runs in $\Theta(n\log n)$.
\medskip

There are much more optimized implementations of heapsort than the above example, possibly using optimized implementations of heaps.
In particular, there are encodings of the heap structure in an array, which allow using heapsort as an in-place sorting algorithm.
With those optimizations, heapsort is among the fastest sorting algorithms (but still takes $\Theta(n\log n)$).
