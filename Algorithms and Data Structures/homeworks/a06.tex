\documentclass[a4paper]{article}

\usepackage[course={Algorithms and Data Structures},number=6,date=2017-03-14,duedate=2017-03-23,duetime=11:00]{../../myhomeworks}

\newcounter{chapter} % needed for dependencies of mylecturenotes
\usepackage[root=../..]{../../mylecturenotes}
\usepackage{../../macros/algorithm}

\begin{document}

\header

As before, you are not allowed to use any existing implementations in your programming language's libraries.

Exception: You may use libraries for concepts we have learned previously, e.g., for linked lists.

\begin{problem}{Stacks}{8}
In any object-oriented programming language, implement the data structure of mutable stacks.
Make sure that all stack operations take constant time.

This should be a class with a mutable field for an immutable linked list.
We say that the stack is \emph{backed} by an immutable list.

Write a test program that
\begin{compactitem}
 \item creates a new stack of integers
 \item pushes some values onto the stack
 \item pops all values and prints them
\end{compactitem}

Depending on your programming language, this might look as follows:
\begin{acode}
\aclass{Stack[A]}{}{}{
 \akey{private} elements := Nil[A] \acomment{the immutable linked list backing the stack, initially empty}\\\\
 \afun[\Unit]{push}{x:A}{\ldots}\\
 \afun[{Option[A]}]{pop}{}{\ldots}\\
 \afun[{Option[A]}]{top}{}{\ldots}\\
}
\\
test := \anew{Stack[\Int]}{}\\
test.push(4)\\
test.push(5)\\
test.push(2)\\
\awhile{test.top().isDefined}{
  \aprint{test.pop()}
}
\acomment{prints 2,5,4}
\end{acode}

\end{problem}

\begin{problem}{Queues}{6}
Implement the data structure of mutable queues.
Make sure that all queue operations take constant time.

For example, this could be a class like in the first problem but backed by a doubly-linked list.

Write a test program that
\begin{compactitem}
 \item creates a new queue of integers
 \item enqueues some values in the queue
 \item dequeues all values and prints them
\end{compactitem}
\end{problem}


\begin{problem}{Queues backed by Two Singly-Linked Lists}{6}
Doubly-linked lists can be tricky to implement.
As an alternative, give an implementation of $Queue[A]$ that is backed by \emph{two} immutable singly-linked lists.

This can be done in such a way that all queue operations take constant time with the exception that $dequeue$ sometimes takes linear time.
\end{problem}

\begin{problem}{Buffers}{8}
Implement a fixed-size circular buffer as follows: $\mathit{Buffer}[A]$ consists of
\begin{compactitem}
 \item an $Array[A]$ named $elements$ of some fixed size $n$
 \item an integer $0\leq begin <n$ indicating the first valid entry in $elements$
 \item an integer $0\leq size\leq n$ indicating the number of valid entries in $elements$
\end{compactitem}
Enqueueing writes to $elements[(begin+size)\modop n]$ and increments $size$.
Dequeueing reads from $elements[begin]$, increments $begin$ modulo $n$, and decrements $size$.\footnote{I fixed a small mistake in this explanation.}

Because addition and incrementing are computed modulo $n$, the buffer becomes circular: when reaching index $n$ (which is outside the buffer), we start from $0$ again.

Your implementation should detect \emph{buffer underflow} (trying to dequeue from an empty buffer) and \emph{buffer overflow} (trying to enqueue into a full buffer) and raise appropriate exceptions.

Write a test program that
\begin{compactitem}
 \item creates a new buffer of integers
 \item enqueues some values in it
 \item dequeues all values and prints them
\end{compactitem}
\end{problem}
\end{document}
