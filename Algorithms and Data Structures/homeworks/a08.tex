\documentclass[a4paper]{article}

\usepackage[course={Algorithms and Data Structures},number=8,date=2017-03-28,duedate=2017-04-06,duetime=11:00]{../../myhomeworks}

\newcounter{chapter} % needed for dependencies of mylecturenotes
\usepackage[root=../..]{../../mylecturenotes}
\usepackage{../../macros/algorithm}

\begin{document}

\header


\begin{problem}{List Sets}{8}
Implement an abstract class for iterable sets.
The iterator should return the elements in the order of insertion.

Extend it to a concrete class for mutable list-backed sets.

Write a test program that
\begin{compactitem}
\item creates a set of integers,
\item inserts some elements,
\item prints all elements using the iterator.
\end{compactitem}

Depending on your programming language, this might look as follows:
\begin{acode}
\aclassA{Set[A]}{}{Iterable[A]}{
 \afunI[\Bool]{contains}{x:A}{}\\
 \afunI[\Unit]{insert}{x:A}{}\\
 \afunI[\Unit]{delete}{x:A}{}\\
 \afunI[{Iterator[A]}]{iterator}{}{}
}\\
\\
\aclass{ListSet[A]}{}{Set[A]}{
 \akey{private} elements := Nil[A] \acomment{the immutable linked list backing the set, initially empty}\\\\
 \afun[\Bool]{contains}{x:A}{\ldots}\\
 \afun[\Unit]{insert}{x:A}{\ldots}\\
 \afun[\Unit]{delete}{x:A}{\ldots}\\
 \afun[{Iterator[A]}]{iterator}{}{\ldots}
}
\\\\
test := \anew{ListSet[\Int]}{}\\
test.insert(4)\\
test.insert(2)\\
test.insert(4)\\
foreach(test.iterator, \alam{x}{\aprint{x}})
\acomment{prints 2,4 or 4,2}
\end{acode}
\end{problem}

\begin{problem}{Binary Search Trees}{8}
Implement an iterable data structure for $BST[\Int,\leq]$.
The iterator should return the elements in $\leq$-order.

Write a test program that
\begin{compactitem}
\item creates a set of integers,
\item inserts some elements,
\item prints all elements using the iterator.
\end{compactitem}

If you combine it with the previous problem, this might look as follows:
\begin{acode}
\aclass{IntBST}{}{Set[\Int]}{
 \acomment{the tree backing the set, initially a single node}\\
 \akey{private} elements : Tree[Option[\Int]] = \anew{Tree[\Int]}{None,Nil}\\\\
 \afun[\Bool]{contains}{x:A}{\ldots}\\
 \afun[\Unit]{insert}{x:A}{\ldots}\\
 \afun[\Unit]{delete}{x:A}{\ldots}\\
 \afun[{Iterator[A]}]{iterator}{}{\ldots}
}
\end{acode}

Make sure that
\begin{compactitem}
 \item the implementations of $insert$ and $delete$ preserve the necessary BST property,
 \item the implementations of $insert$, $delete$, and $contains$ are logarithmic in the best case.
\end{compactitem}
\end{problem}

\begin{problem}{Hash Sets}{8}
Implement an abstract class for $HashSet[A]$ sets for an arbitrary hash function $hash:A\to \Z_m$.
The iterator may return elements in any order.
Extend it to a concrete class $LastDigitHashSet$ that hashes integers based on their last digit.

Write a test program that
\begin{compactitem}
\item creates a $LastDigitHashSet$ of integers,
\item inserts some elements,
\item prints all elements using the iterator.
\end{compactitem}

Depending on your programming language, this may look as follows:
\begin{acode}
\aclassA{HashSet[A]}{m:\Int}{Set[A]}{
 \acomment{the array backing the hash set, initially all buckets empty}\\
 \akey{private} bucket : Array[ListSet[A]] = \anew{Array[ListSet[A]]}{m}\\
 \afor{x}{0}{m-1}{bucket[i]:=\anew{ListSet[A]}{}}\\\\
 \acomment{the abstract hash function that subclasses must provide}\\
 \acomment{postcondition: return value between $0$ and $m-1$}\\
 \afunI[\Int]{hash}{x:A}{}\\
 \\
 \acomment{the functions implemented in this class}\\
 \afun[\Bool]{contains}{x:A}{\ldots}\\
 \afun[\Unit]{insert}{x:A}{\ldots}\\
 \afun[\Unit]{delete}{x:A}{\ldots}\\
 \afun[{Iterator[A]}]{iterator}{}{\ldots}
}
\\\\
\aclass{LastDigitHashSet}{}{HashSet[\Int](10)}{
 \afunI[\Int]{hash}{x:A}{x \modop 10}\\
}
\end{acode}

Make sure that your implementation exploits the hashing, i.e., $contains(x)$, $insert(x)$, and $delete(x)$ should only access $bucket[hash(x)]$.
\end{problem}

\end{document}
