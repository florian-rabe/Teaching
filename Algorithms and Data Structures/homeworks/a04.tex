\documentclass[a4paper]{article}

\usepackage[course={Algorithms and Data Structures},number=4,date=2017-02-28,duedate=2017-03-09,duetime=11:00,solutions]{../../myhomeworks}

\newcounter{chapter} % needed for dependencies of mylecturenotes
\usepackage[root=../..]{../../mylecturenotes}
\usepackage{../../macros/algorithm}

\begin{document}

\header

\begin{problem}{Correctness}{2+2+2}
Consider the following algorithm for reverting a list:
\begin{acode}
\afun[{List[A]}]{revertFun[A]}{x:List[A]}{
  rest := x\\
  rev := []\\
  \awhile{rest \neq []}{
    rev := cons(rest.head, rev)\\
    rest := rest.tail
  }\\
  \areturn{rev} \footnotemark
}
\end{acode}
\footnotetext{This line was accidentally missing in the original formulation. But the intention of the algorithm was clear from the context.}

Its specification is
\begin{compactitem}
 \item precondition: nothing
 \item postcondition: $revertFun(x)==revert(x)$
\end{compactitem}

Prove correctness by doing the following:
\begin{compactenum}
 \item Give a loop invariant $F(x,rest,rev)$ for the while-loop.
 \item Argue informally why it implies partial correctness. 
 \item Give a termination ordering $T(x,rest,rev)$ for the while-loop.
\end{compactenum}

\begin{solution}
\begin{compactenum}
 \item $x==revert(rev)+rest$
 \item When terminating, $rest==[]$. Plugging that into the loop invariant yields $x==reverse(rev)$ and therefore $reverse(x)=rev$.
 Because $rev$ is the value that is returned, we have $rev==revertFun(x)$.
 Combining the two yields the postcondition.
 \item $length(rest)$
\end{compactenum}
\end{solution}
\end{problem}

\begin{problem}{Lists}{6+6}
Implement a data structure for polymorphic lists twice: once using an inductive data type in a functional language, and once using classes/structs or comparable primitives for (mutable or immutable) singly-linked lists in an imperative or object-oriented language.

Specifically, the implementation should have
\begin{compactitem}
 \item the data type definition itself (2 points)
 \item polymorphic functions for $concat$ and $reverse$ (2 points each)
\end{compactitem}

Remarks:
\begin{compactitem}
\item Most programming languages have libraries for lists.
Naturally you are not allowed to use those libraries.
\item If you use a multi-paradigm programming language that supports both data type and classes, both implementations can use the same programming language.
Otherwise, use two different languages.
\item Because you have to be polymorphic, you have to use a typed programming language that supports polymorphism.
That can be tricky, and the lecture notes contain some examples (Sect.~2.4) to get you started.
If you are completely stuck, drop the polymorphism and implement it only for lists of integers.
That costs half the points.
\end{compactitem}
\end{problem}


\begin{problem}{Sorted List}{2+2+(2+2+2)}
In any programming language and possibly extending one of your implementations from above, implement the following:
\begin{enumerate}
 \item An abstract data structure $Ord[A]$ for orders. You may reuse/adapt the examples given in Sect.~2.4.
 \item A function $sorted[A](x: List[A], ord: Ord[A]): \Bool$ that checks if a list is sorted.\footnote{Checking whether a list is sorted works for any order. A \emph{total} order would be needed to ensure any list \emph{can} be sorted.}
 \item $3$ concrete instances of $Ord[A]$:
  \begin{itemize}
   \item $IntSmaller:Ord[\Int]$ for the order  $\leq$
   \item $Divisible:Ord[\Int]$ for the order $|$
   \item $Lexicographic:Ord[\String]$ for the order in which words are listed in a dictionary
  \end{itemize}
\end{enumerate}
\end{problem}

\end{document}
