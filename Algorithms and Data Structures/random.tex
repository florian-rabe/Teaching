A randomized algorithm is an algorithm that systematically use randomness as part of their implementation.
Typically, the randomness makes the result (or at least the run time) unpredictable.

Randomized algorithms present two theoretical problems:
\begin{compactenum}
\item Because the result is not determined by the input, they are technically not algorithms at all.
\item They may not always return a correct result.
\end{compactenum}

Often randomized algorithm are used when it would be too expensive to find a correct solution.
Ideally, but not necessarily, if an incorrect result is returned, it is at least approximately correct with respect to some measure.

For example, we could give a randomized variant of quicksort that chooses the pivot element randomly.
That algorithm has a high probability of running in $\Theta(n\log n)$.

Randomized algorithms are often used in problems in statistical analysis or scientific simulation where correct solutions are infeasible anyway.
Often a correct algorithm is run on a random sample of the input, from which the overall result is estimated.