\section{Specification}

$A^?$ is a set containing all the elements of $A$ and one additional element $\bot$.

$A^?$ is used to represent an optional value of $A$.
The element $\bot$ is used to represent an undefined/absent value.

Options are usually immutable.
The main operations on $A^?$ are
\begin{ctabular}{|l|l|l|}
\hline
function & returns & effect \\
\hline
$getOrElse(x\in A^?, default: A)\in A$ & get the optional value or a default value & none\\
$get(x\in A^?)\in A$ & get the optional value & error if absent \\
$map(x\in A^?, f\in A\to B)\in B^?$ & apply $f$ to the optional value & none\\
\hline
\end{ctabular}

\section{Data Structures}

\subsection{Using Inductive Types}

In functional programming languages, a data structure for optional values can be realized as an inductive type:

\begin{acode}
\adata{Option[A]}{Some(value:A),None}
\end{acode}

Such a definition (except for possibly using different names) is usually part of the standard library of the language.

\subsection{Using Pointers}

In languages that use pointers, we can represent $Option[A]$ as the type $Pointer[A]$ (written $A*$ in C) of pointers to elements of $A$.
In that case, the $null$ pointer represents $\bot$.

Object-oriented languages do not necessarily expose pointers to the programmers (e.g., C++ does, but Java does not).
However, even then they use pointers internally, and any class-type provides the value $null$.
In this situation it is impossible to represent the set $A$ correctly as a class---any class for $A$ is automatically a data structure for $A^?$.
This often causes ambiguous specifications and subtle errors.
Therefore, it is good practice to never use $null$ even when possible.
