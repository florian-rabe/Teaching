There are three important data structures for the Cartesian product $A\times B$.

\section{Positional Products}

The simplest data structure directly implements the Cartesian product.
There are multiple ways to do that.

\subsection{Primitive Products}

Functional programming languages usually include product types as a primitive feature.
This includes the types $A*B$ in SML or $(A,B)$ in Scala.

\subsection{Products as an Inductive Data Type}

Products can also easily be defined as an inductive data type:
\begin{acode}
\adata{Product[A,B]}{{{Pair(A,B)}}}\\
\\
\afun[A]{ProjectLeft[A,B]}{p: Product[A,B]}{\amatch{p}{\acase{{Pair(a,b)}}{a}}}\\
\\
\afun[B]{ProjectRight[A,B]}{p: Product[A,B]}{\amatch{p}{\acase{{Pair(a,b)}}{b}}}\\
\end{acode}

\section{Labeled Products: Structures/Records}

Very often, we need the Cartesian product of more than two types.
Moreover, it can become tedious to keep track of the individual components.
For example, if we work with the product $\Z\times \Z\times \String\times \N$, code like $p.3$ to access the third component can become hard to read.

Labeled products introduce a name for every component and use those names instead of numbers to access the components.

\subsection{Specification}

Let $l_1,\ldots,l_n$ be some fresh unique names.

The set $P=\{l_1:A_1,\ldots,l_n:A_n\}$ is isomorphic to the set $A_1\times \ldots\times A_n$.

Its elements are $\{l_1=a_1,\ldots,l_n=a_n\}$ for $a_i\in A_i$ with the convention that the components can be given in any order.

Given $u\in P$, we write $u.{l_i}$ for the value of the component $l_i$ in $u$.

\subsection{Data Structures}

Data structures for labeled products are part of almost every programming language.
However, they occur with very different names and notations.

For example, they are called
\begin{compactitem}
 \item records in SML,
 \item structs in C,
 \item classes in object-oriented languages,
 \item objects in Javascript,
 \item dictionaries in Python.
\end{compactitem}

In untyped languages like Javascript and Python, of course the type $\{l_1:A_1,\ldots,l_n:A_n\}$ does not exist.
But the values $\{l_1=a_1,\ldots,l_n=a_n\}$ and $u.{l_i}$ work as in typed languages.

In object-oriented languages, there is often no special data structure for labeled products.
Instead, labeled products are simply a special case of classes as in
\begin{acode}
\acomment{$\{l_1:A_1,\ldots,l_n:A_n\}$}\\
\aclassA{P}{}{}{
 \aval{l_1}{A_1}{}\\
 \vdots\\
 \aval{l_n}{A_n}{}
}\\
\\
\tb\acomment{$\{l_1=a_1,\ldots,l_n=a_n\}$}\\
\anew{P}{}\ablock{
 \aval{l_1}{}{a_1}\\
 \vdots\\
 \aval{l_n}{}{a_n}
}
\end{acode}


\section{Recursive Products: Classes}

Often we want products to be recursive in the sense that the fields of the product refer to the whole product.
This is a typical application of classes.

We have seen examples already, e.g., in the data structures for lists.
For example, we can understand $List[A]$ as the product of $A$ (for the head) and $List[A]$ (for the tail).

In the C-family of languages, recursive products are usually realized as structs $S$ that contain pointers to elements of type $S$.
For example, we might use
\begin{lstlisting}
struct IntList {
  int head;
  IntList* tail;
}
\end{lstlisting}
