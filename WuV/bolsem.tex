This chapter introduces different kinds of semantics.
The emphasis will be on relative semantics by compositional translations.

We use BOL as a running example and give four different semantics of BOL --- using the four other aspects:
\begin{center}
\begin{tabular}{llll}
Section & Aspect & kind of semantic language & semantic language\\
\hline
\ref{sec:bolsem:ded} & deduction & logic & SFOL \\
\ref{sec:bolsem:conc} & concretization & database language & SQL \\
\ref{sec:bolsem:comp} & computation & programming language & Scala \\
\ref{sec:bolsem:narr} & narration & natural language & English \\
\end{tabular}
\end{center}

%Note that we could also give an ontological semantics of BOL, e.g., by using OWL as the semantic language.
%BOL and OWL are already so similar that the translation would be rather straightforward.
%Therefore, we omit it.

\section{Kinds of Semantics}

To define a language system, we need to define the well-formedness predicates.
Moreover, we need to define the semantics of the well-formed vocabularies and expressions.
One way to do that is by translating the syntax into another language and then use existing definitions of well-formedness there.

But we also need to be able to get off the ground, i.e., to define a semantics from scratch when we do not have another language available.
This is usually done by giving an inference system for the well-formedness predicates and other judgments such as truth.

We can also do both: if we have a semantics via an inference system and another one via translation, we can show that the latter respects the former.
That leads to the concepts of soundness (everything well-formed is translated to something well-formed) and its dual completeness.

\subsection{Absolute Semantics}

Absolute semantics is very hard to define in general.
A typical approach is to single out one property of the syntax and then use an inference system to define when that property holds.

For example, for BOL we could use the property whether a formula can be proved from the axiom.
The inference system then is the proof calculus that define the provable formulas.

\subsection{Relative Semantics}

Relative semantics uses an additional component as a reference point for semantics.

\paragraph{By Translation}
Semantics by translation uses a second language as the target of a translation.
Both the vocabularies and the expressions are translated, namely to vocabularies and expressions of the target language.
Usually, the target language already has a semantics, and the semantics of the original language is obtained by composing the two.

\begin{figure}[hbt]
\begin{center}
\begin{tabular}{l|l}
Aspect & Example\\\hline
%Ontological & OWL $\to$ FOL \\
Deductive & FOL $\to$ set theory\\
Computational & C++ $\to$ assembly \\
Concrete & SPARQL $\to$ SQL \\
Narrative & English $\to$ German \\
\end{tabular}
\caption{Examples of Semantics by Translation}\label{fig:trans}
\end{center}
\end{figure}

The correspondence for the syntax between context-free grammars and inductive data types can be extended to the semantics.
Now we have a correspondence between case-based function definitions and inductive functions.

\begin{definition}
A \textbf{semantics by translation} consists of the following parts:
\begin{compactitem}
 \item syntax: a language system $l$
 \item semantic language: a language system $L$ (from a different or the same aspect as $l$)
 \item semantic prefix: a vocabulary $P$ in $L$ that is prefixed to the translation of all vocabularies of $l$
 \item vocabulary translation: a function that translates every $l$-vocabulary $T$ to an $L$-vocabulary $P,\sem{T}$
\end{compactitem}
\end{definition}

Critically, the semantic language (which is itself a language system and can thus have a semantics itself) must be a language whose semantics we already know.
Therefore, it is often important to give multiple equivalent semantics --- choosing a different semantics for different audiences, who might be familiar with different languages.

The role of the semantic prefix $P$ is to define once and for all the $L$-material that we need in general to interpret $l$-vocabularies (in the case of BOL: ontologies).
It occurs at the beginning of all interpretations of ontologies.
In particular, it is equal to the interpretation of empty vocabulary.

\paragraph{By Interpretation}
Semantics by interpretation uses a situation relative to which expressions are evaluated.
The role of the situation is to supply the meaning of all identifiers declared in the vocabulary.
Given a situation, inductive functions map all expressions to their semantics.
Usually, the result of interpretation is an extremely simple object, whose semantics is self-evident.

\begin{figure}[hbt]
\begin{tabular}{l|ll}
Aspect & Situation & Provides\\\hline
%Ontological & ABox & individuals and their properties \\
Deductive & model & interpretations of the function/predicate/... symbols\\
Computational & environment & e.g., standard input/output \\
Concrete & database & set of objects for each type \\
Narrative & semantic dictionary & meanings of the words \\
\end{tabular}
\caption{Examples of Situational Semantics by Interpretation}\label{fig:sit}
\end{figure}

Situational semantics can be seen as a special case of semantics by translation as follows:
\begin{itemize}
\item The target language is an implicit assumed background language such as mathematics or the computer hardware.
Once this is made explicit, situations can be described as vocabularies of the background language.
\item The semantic prefix describes what a situation is in general, i.e., the possible situations for the empty vocabulary.
\item The semantics of a vocabulary extends the semantic prefix with the possible ways to interpret the identifiers.
\item The semantics of an expression $E$ is the function mapping the situation $S$ to the situational semantics of $E$ under $S$. This can be expressed as an expression of the target language.
\end{itemize}

\section{Compositionality}

\paragraph{Introduction}
There are some general principles shared by all translations:
\begin{compactitem}
 \item Every $l$-declaration is translated to an $L$-declaration for the same name, and ontologies are translated declaration-wise.
 \item For every non-terminal $N$ of $l$, there is one inductive function $\sem{-}_N$ mapping complex $l$-expressions derived from $N$ to $L$-expressions.
 \item The base cases of references to declared $l$-identifiers are translated to themselves, i.e., to the identifiers of the same name declared in $L$.
 \item The other cases are compositional: every case for a complex $l$-expression recurses only into the semantics of the direct subexpressions.
\end{compactitem}

The notion of \emph{compositionality} captures these properties.
An interpretation function is compositional if the interpretation of any kind of expression $E(e_1,\ldots,e_n)$ with subexpressions $e_i$ only depends on $E$ and the interpretation of the $e_i$, i.e., \[\sem{E(e_1,\ldots,e_n)}=\sem{E}(\sem{e_1},\ldots,\sem{e_n})\] for some semantic operation $\sem{E}$.
Compositionality is also called the substitution property or the homomorphism property.
See also Def.~\ref{ex:compositional}.

More rigorously, we define a compositional translation as follows:
\begin{definition}[Compositional Semantics]
Consider a semantics for syntax grammar $l$ and interpretation function $\sem{-}$.

$\sem{-}$ is compositional if it is defined as follows:
\begin{compactitem}
 \item a family of functions $\sem{-}_N$, one for every non-terminal $N$ of $l$
 \item for every expressions $E$ derived from $N$, we put $\sem{E}=\sem{E}_N$
 \item each $\sem{-}_N$ is defined by induction on the productions for $N$
 \item for each production $N\bbc *(N_1,\ldots,N_r)$ and all expressions $e_i$ derived from $N_i$
   \[\sem{*(e_1,\ldots,e_r)}_N=\sem{*}(\sem{e_1}_{N_1},\ldots,\sem{e_r}_{N_r})\]
   for some $L$-expression $\sem{*}$
\end{compactitem}

Without loss of generality, we can assume that every production is of the form $N\bbc *(N_1,\ldots,N_r)$ where the $N_i$ are all the non-terminals on the right-hand side and $*$ is a stand-in for all the terminal symbols.
\end{definition}

\paragraph{Compositional Translations of Contexts}
We can extend every compositional translation to contexts, substitutions, and expressions in contexts:

\begin{definition}
Given a translation $\sem{-}$ as above, for a non-terminal $N$, we define $\sem{N}$ as the non-terminal from which the translations of $N$-expressions are derived.

Then we define:
\[\sem{x_1:N_1,\ldots,x_n:N_n}:=x_1:\sem{N_1},\ldots,x_n:\sem{N_n}\]
\[\sem{x_1:=w_1,\ldots,x_n:=w_n}:=x_1:=\sem{w_1},\ldots,x_n:=\sem{w_n}\]
\[\sem{x}:=x\]
\end{definition}

The requirement of compositionality is critical for two reasons:
\begin{compactitem}
\item A non-compositional translation could translate $l$-expressions derived from the same non-terminal $N$ to $L$-expressions derived from different non-terminals. Then we would not be able to define $\sem{N}$.
\item The definition $\sem{x}:=x$ adds a case to the case distinction in the compositional translation function.
Without compositionality, this would not make sense.
\end{compactitem}

\begin{theorem}[Type Preservation]
For a compositional translation as above, we have
  \[\Gamma\vdash_l w:N \tb\mimplies\tb \sem{\Gamma}\vdash_L \sem{w}:\sem{N}\]
\end{theorem}

\paragraph{Substitution Theorem}
The main value of compositionality is the following:
\begin{theorem}[Substitution Theorem]
Consider a compositional semantics.

For every context $\Gamma=x_1:N_1,\ldots,x_r:N_r$, every syntax expression $\Gamma \vdash_l E:N$,
and every substitution $\vdash_l \gamma:\Gamma$
\[\sem{E[\gamma]}=\sem{E}[\sem{\gamma}]\]
\end{theorem}

Formulated without substitutions, this means that for every syntax expression $E(e_1,\ldots,e_r)$ derived from $N$, where the $e_i$ are subexpression derived from non-terminal $N_i$, we have
\[\sem{E(e_1,\ldots,e_n)}_N=\sem{E}(\sem{e_1}_{N_1},\ldots,\sem{e_n}_{N_r})\]

Simply put, a semantics is compositional iff it is defined by mutually inductive translation functions with only compositional cases.
The latter is very easy to check by inspecting the shape of the finitely many cases of the definition.
The former is a powerful property because it applies to any of the infinitely many expressions of the syntax.

\paragraph{Non-Compositional Semantics}

It is highly desirable but not always possible to give a compositional translation.
Sometimes a feature of the syntactic language cannot be directly interpreted in the semantic language.
In that case, it may still be possible to give a non-compositional translation.

\begin{example}[Non-Compositional Translation via Sub-Induction]
A simple example of non-compositionality is the translation of natural numbers based on zero, one, and addition (i.e., $N\bbc 0\bnfalt 1\bnfalt N+N$) into natural numbers based on zero and successor (i.e., $N\bbc 0\bnfalt\cn{succ}(N)$):
It is straightforward to translate zero and one compositionally:
\[\sem{0}=0 \tb\sem{1}=\cn{succ}(0)\]
Now we would like to translate \[\sem{m+n}=\sem{+}(\sem{m},\sem{n}),\] but there is no way to define $\sem{+}$ in terms of zero and successor.
Instead, we need subcases:
\[\sem{m+n}=\cas{\sem{m}\mifc n=0 \\ \cn{succ}(\sem{m})\mifc n=1 \\ \sem{(m+n_1)+n_2}\mifc n=n_1+n_2}\]
This corresponds to the usual definition of addition, i.e., $\sem{+}$, by induction.
\end{example}

Other common examples of non-compositional translations are
\begin{compactitem}
 \item several important logical theorems such as
  \begin{compactitem}
   \item cut elimination, which is the translation from sequent calculus with cut to sequent calculus without cut,
   \item the deduction theorem, which is the translation from natural deduction to Hilbert calculus,
  \end{compactitem}
 \item almost anything done by an optimizing compiler, e.g., loop unrolling or function inlining,
 \item query optimization done by a database, e.g., turning a WHERE of a JOIN into a JOIN of WHEREs,
 \item almost all translations between natural languages, e.g., when words are ambiguous and a different translation must be chosen for the same word based on the context (The introduction of richer intermediate structures like ASTs and functions as values into the translation can recover some compositionality here).
\end{compactitem}

Typical sources of non-compositionality in formal language translations are:
\begin{compactitem}
 \item A case in the translation function requires subcases which inspect the $e_i$ and treat them differently.
 \item A case in the translation function requires subcases which translate an expression differently based on the context in which it occurs.
 \item The translation function requires nested inductions, i.e., a case in the translation function (which is already inductive) requires a sub-induction on one of the sub-expressions.
 \item The semantic prefix is not fixed but depends on the translated object, i.e, the top-level case of the translation scans through the entire argument $X$ to collect all occurrences of a particular feature and then custom-builds the semantic prefix of $\sem{X}$.
\end{compactitem}
See also Ex.~\ref{ex:noncompositional}.

Such non-compositional translations are undesirable for multiple reasons:
\begin{compactitem}
 \item The implementation is more complicated and error-prone.
 \item Reasoning about the translation is more difficult.
 \item The custom semantic prefix can be large.
\end{compactitem}

But most importantly, non-compositional translations are less robust.
Firstly, if we add a production to the syntax, a compositional translation is easy to extend: just add a case to the translation.
But a non-compositional translation may additionally require a new subcase wherever subcases/subinductions are used.
Moreover, if a custom semantic prefix is used, its definition may have to be amended, at least it must be rechecked.

Secondly, in practice there are two sources of complex expressions: the ones already mentioned in the language, and the ones used later for other reasons.
For example, some complex expressions occur already \emph{statically} in the definition of a vocabulary $V$.
But others might be appear \emph{dynamically} later, e.g., when talking about $V$, proving properties of $V$, or running queries on $V$.
Thus, the definition of $V$ and the use of complex expressions are decoupled: $V$ is defined statically once and for all, and complex expressions relative to $V$ can be created and used dynamically.
But if a custom semantic prefix is used, only the static occurrences inside $V$ can be considered for building the prefix.
Thus, it is not possible to translate expressions dynamically unless the semantic prefix is extended all the time while $V$ is used.

\section{Deductive Semantics}\label{sec:bolsem:ded}

We fix one language that we have already understood and define an interpretation function that maps all complex expression of BOL to the semantic language.
For simple ontology languages like BOL, ALC, OWL, etc., it is common to use first-order logic (FOL) as the deductive semantic language.
More specifically, we use SFOL, the typed variant of FOL:

\subsection{A Basic Semantic Language: SFOL}\label{sec:wuv:tfol}
  %\section{Overview}
%
%Various methods have been developed to represent and perform inferences.
%We structure our presentation by how each method relates to computation, the aspect most whose integration with inference has drawn the most attention.
%In general, the ubiquity of underspecified function symbols and quantified variables means that logical expressions usually do not normalize to unique values.
%At best, computations like $y:=f(x)$ can be represented as open-ended conjectures where different options for $y$ are produced, each together with a proof of the respective equality.
%Therefore, inference systems usually sacrifice computation or at least its efficiency.
%
%\emph{Proof assistants} sit at the extreme end of this spectrum.
%They employ strong logics and high-level declarations to provide a convenient way to formalize domain knowledge and reason about it.
%The reasoning is usually interactive in order to represent inferences that are too difficult to be fully automated.
%Most proof assistants integrate at least some of the other methods to overcome this weakness.
%
%Further along the spectrum, \emph{automated theorem provers} use simpler logics than interactive proof assistants.
%They are fully automatic and much faster, but can handle much fewer theorems, and typically do not check their proofs.
%\emph{Satisfiability checkers} continue this progression by aiming at decidable automation support, whereas theorem proving is usually an semi-decidable search problem.
%That limits them to propositional logic or specific theories of more expressive logics (usually of first-order logic) that are complete, i.e., where every formula can be proved or disproved.
%In the special cases, where satisfiability checkers are applicable, they come close to verified computation systems.
%
%Orthogonal to the above triplet, there are several methods for realizing Turing-complete computation naturally inside a logic.
%Here imperative and object-oriented computation are usually avoided in favor of other programming paradigms that are easier to reason about.
%\emph{Rewriting} aims at optimizing the $f(x)\rewrites y$ progression, allowing users to mark specific transformations as rewrite steps.
%\emph{Terminating recursion} is the method of adding recursive functions to a logic in order to make it a pure functional programming language.
%Finally, \emph{logic programming} restricts attention to theorems of a special form, for which proof search is simple and predictable so that users can represent computations by supplying axioms that guide the proof search.

We give typed first-order logic (SFOL) as a language system.

\begin{definition}
Fig.~\ref{fig:sfol} gives the context-free grammar.
The vocabulary symbol is $Thy$. The expression symbols are $Y$, $T$, and $F$.
\end{definition}

\begin{figure}[hbt]
\begin{commgrammar}
\gcomment{Vocabularies: theories}\\
\gprod{Thy}{\rep{D}}{}\\
\gcomment{Declarations}\\
\gprod{D}{\kw{type}\; \ID}{type declaration}\\
\galtprod{\kw{fun}\; \ID:\rep{Y}\to Y}{function symbol declaration}\\
\galtprod{\kw{pred}\; \ID:\rep{Y}\to \kw{form}}{predicate symbol declaration}\\
\galtprod{\kw{axiom}\;F}{axiom}\\
\gcomment{type expressions}\\
\gprod{Y}{\ID}{atomic type} \\
\gcomment{term expressions}\\
\gprod{T}{\ID(\rep{T})}{function symbol applied to arguments} \\
\galtprod{\ID}{term variables} \\
\gcomment{formula expressions}\\
\gprod{F}{\ID(\rep{T})}{predicate symbol applied to arguments} \\
\galtprod{T\doteq_Y T}{equality of terms at a type} \\
\galtprod{\top}{truth} \\
\galtprod{\bot}{falsity} \\
\galtprod{F\wedge F}{conjunction} \\
\galtprod{F\vee F}{disjunction} \\
\galtprod{F\impl F}{implication} \\
\galtprod{\neg F}{negation} \\
\galtprod{\forall \ID:Y.F}{universal quantification at a type} \\
\galtprod{\exists \ID:Y.F}{existential quantification at a type} \\
\gcomment{Identifiers}\\
\gprod{\ID}{\text{alphanumeric string}}{}\\
\end{commgrammar}
\caption{Syntax of SFOL}\label{fig:sfol}
\end{figure}


\subsection{Semantics}

\begin{definition}[Deductive Semantics of BOL]\label{def:bolsem:sfol}
The semantic prefix is the SFOL-theory containing
\begin{compactitem}
 \item a type $\iota$ (for individuals),
 \item additional types and constants corresponding to base types and values of BOL.
\end{compactitem}

Every BOL-ontology $O$ is interpreted as the SFOL-theory $P,\sem{O}$, where $\sem{O}$ is defined in Fig.~\ref{fig:bolsem:sfol}.
\end{definition}

As foreshadowed above, we can observe some general principles:
Every BOL-declaration is translated to an SFOL-declaration for the same name, and ontologies are translated declaration-wise.
For every kind of complex BOL-expression, there is one inductive function mapping BOL-expressions to SFOL-expressions.
The base cases of references to declared BOL-identifiers are translated to themselves, i.e., to the identifiers of the same name declared in the SFOL-theory.
The other cases are compositional: every case for a complex BOL-expression recurses only into the semantics of the direct subexpressions.

\begin{figure}[tbh]\centering
\begin{tabular}{l|l}
BOL Syntax $X$ & Semantics $\sem{X}$ in SFOL\\
\hline
\hline
ontology & SFOL theory \\
$D_1,\ldots,D_n$ & $\sem{D_1},\ldots,\sem{D_n}$ \\
\hline
BOL declaration & FOL declaration \\
\kw{individual}\,$i$ & nullary function symbol $i:\iota$ \\
\kw{concept}\,$i$  & unary predicate symbol $i\sq\iota$ \\
\kw{relation}\,$i$ & binary predicate symbol $i\sq\iota\times \iota$ \\
\kw{property}\,$i:T$ & binary predicate symbol $i\sq\iota\times T$ \\
$I\; \texttt{is-a}\; C$ & axiom $\sem{C}(\sem{I})$\\
$I_1\; R\; I_2$ & axiom $\sem{R}(\sem{I_1},\sem{I_2})$\\
$I\; P\; V$ & axiom $\sem{P}(\sem{I},\sem{V})$\\
$F$ & axiom $\sem{F}$\\
\hline
Formula & Formula without free variables\\
$C_1 \Equiv C_2$ & $\forall x:\iota.\sem{C_1}(x)\Leftrightarrow \sem{C_2}(x)$\\
$C_1 \sqsubseteq C_2$ & $\forall x:\iota.\sem{C_1}(x)\impl \sem{C_2}(x)$\\
$I\; \texttt{is-a}\; C$ & $\sem{C}(\sem{I})$\\
$I_1\; R\; I_2$ & $\sem{R}(\sem{I_1},\sem{I_2})$\\
$I\; P\; V$ & $\sem{P}(\sem{I},\sem{V})$\\
\hline
Individual & Terms of type $\iota$ \\
$i$ & $i$ \\
\hline
Concept & Formula with free variable $x:\iota$\\
$i$ & $i(x)$\\
$\top$ & $\true$\\
$\bot$ & $\false$\\
$C_1 \sqcup C_2$ & $\sem{C_1}(x)\vee\sem{C_2}(x)$\\
$C_1 \sqcap C_2$ & $\sem{C_1}(x)\wedge\sem{C_2}(x)$\\
$\forall R.C$    & $\forall y:\iota.\sem{R}(x,y)\impl \sem{C}(y)$\\
$\exists R.C$    & $\exists y:\iota.\sem{R}(x,y)\wedge \sem{C}(y)$\\
$\dom\, R$ & $\exists y:\iota.\sem{R}(x,y)$\\
$\rng\, R$ & $\exists y:\iota.\sem{R}(y,x)$\\
$\dom\, P$ & $\exists y:T.\sem{P}(x,y)$  \tb($T$ is type of $P$)\\
\hline
Relation & Formula with free variables $x:\iota,y:\iota$\\
$i$ & $i(x,y)$\\
$R_1 \cup R_2$ & $\sem{R_1}(x,y)\vee \sem{R_2}(x,y)$\\
$R_1 \cap R_2$ & $\sem{R_1}(x,y)\wedge \sem{R_2}(x,y)$\\
$R_1 ; R_2$ & $\exists m:\iota.\sem{R_1}(x,m)\wedge \sem{R_2}(m,y)$\\
$R^{-1}$          & $\sem{R}(y,x)$\\
$R^*$          & (tricky, omitted)\\
$\Delta_C$     & $x\doteq y\wedge \sem{C}(x)$\\
\hline
Property of type $T$ & Formula with free variables $x:\iota,y:T$\\
$i$ & $i(x,y)$\\
\end{tabular}
\caption{Interpretation Function for BOL into SFOL}\label{fig:bolsem:sfol}
\end{figure}

\clearpage

The consequence closure of SFOL, using the usual semantics of SFOL, induces the desired consequence closure for BOL:
\begin{definition}[Consequence Closure]
We say that a BOL-statement $F$ is a consequence of an ontology $O$ if $\sem{F}$ is an SFOL-theorem of $P,\sem{O}$.
%A BOL-ontology $O$ is consistent if $P,\sem{O}$ is consistent as an SFOL-theory.
\end{definition}

\begin{example}
We interpret the example ontology from Ex.~\ref{ex:bol}.
Excluding the semantic prefix, it results in
\[\mathll{
\FR:\iota,\; \WuV: \iota,\;\\
\person\sq\iota,\;\male\sq\iota,\;\instr\sq\iota,\;\crs\sq\iota,\;\\
\tch\sq \iota\times\iota,\;\hc\sq\iota\times \float \\
\instr(\FR)\wedge\male(\FR),\;\crs(\WuV),\;\tch(\FR,\WuV),\;\hc(\WuV,7.5)\\
\forall x:\iota.\male(x)\impl\person(x),\\
\forall x:\iota.\instr(x)\impl\person(x),\\
\forall x:\iota.(\exists y:\iota.\tch(x,y))\impl\instr(x),\\
\forall x:\iota.(\exists y:\iota.\tch(y,x))\impl\crs(x),\\
\forall x:\iota.(\exists y:\iota.\tch(y,x))\Darr\crs(x),\\
\forall x:\iota.\crs(x)\impl\exists y:\iota.\tch(y,x)\wedge\instr(y)
}\]
\end{example}

\begin{example}[Compositionality]\label{ex:compositional}
The interpretation of BOL is compositional.

For example, consider the case of composition of relations:
 \[\sem{R_1 ; R_2}= \exists m:\iota.\sem{R_1}(x,m)\wedge \sem{R_2}(m,y)\]
Here we have an expression $E(e_1,\ldots,e_n)$ with $n=2$ and $E$ is the $;$-operator mapping $(e_1,e_2)\mapsto e_1;e_2$, i.e., $R_1$ and $R_2$ are the direct subexpressions of $R_1;R_2$.
The semantics is a relatively complicated FOL-formula, but it only depends on $\sem{R_1}$ and $\sem{R_2}$ --- everything else is fixed.
We have $\sem{;}=(p_1,p_2)\mapsto \exists m:\iota.p_1(x,m)\wedge p_2(m,y)$, i.e., the interpretation of the $;$-operator is the function that maps two predicates $p_1,p_2$ to the formula $\exists m:\iota.p_1(x,m)\wedge p_2(m,y)$.
Then we have \[\sem{R_1;R_2}=\sem{;}(\sem{R_1},\sem{R_2}).\]
\end{example}

\begin{example}[Non-Compositional Translation via Custom Semantic Prefix]\label{ex:noncompositional}
In Fig.~\ref{fig:bolsem:sfol}, we omitted the case for the transitive closure.
That was because it is not possible to translate it compositionally into FOL.
We can only do it non-compositionally with a custom semantic prefix:

We define the FOL-interpretation of an ontology $O$ by $\sem{O}=P_O,\sem{O}$, where $P_O$ is a custom semantic prefix.
$P_O$ is different for every ontology $O$ and is defined as follows:

\begin{compactenum}
 \item We scan through $O$ and collect all occurrences of $R^*$ for any (not necessarily atomic) relation $R$.
 \item $P_O$ contains the following declarations for each $R$:
  \begin{compactitem}
  \item A binary predicate symbol $C_R\sq i\times i$. Note that $R$ may be a complex expression; so we have to generate a fresh name $C_R$ here.
  \item The axiom $\forall x:\iota,y:\iota.\,R(x,y)\impl C_R(x,y)$, i.e., $C_R$ extends $R$.
  \item The axiom $\forall x:\iota,y:\iota,z:\iota.\,C_R(x,y)\wedge C_R(y,z)\impl C_R(x,z)$, i.e., $C_R$ is transitive.
  \end{compactitem}
 \item We add the case $\sem{R^*}=C_R(x,y)$ to the interpretation function.
\end{compactenum}

Intuitively, every occurrence of the $^*$-operator is removed from the language and replaced with a fresh name that is axiomatized to have the needed properties.
All of these axioms are added to the semantic prefix.
\end{example}

\section{Concretized Semantics}\label{sec:bolsem:conc}

We give an alternative semantics using a semantic language for concrete data.
Specifically we use the database language SQL.

\subsection{An SQL-Inspired Basic Database Language}\label{sec:wuv:bdl}
  We give an SQL-like database language as a language system.

\begin{definition}
Fig.~\ref{fig:bdl} gives the context-free grammar.
The vocabulary symbol is $S$. The expression symbols are $T$, $R$, $V$, and $F$.
\end{definition}

\begin{figure}[hbt]
\begin{commgrammar}
\gcomment{Vocabularies: Schemas}\\
\gprod{S}{\rep{D}}{}\\
\gcomment{Declarations}\\
\gprod{D}{\kw{CREATE TABLE}\; \ID\,\{\rep{CT}\}}{table}\\
\galtprod{\kw{INSERT}\,R\,\kw{INTO} \,\ID}{row in a table}\\
\gprod{CT}{\ID:Y}{column type}\\
\gcomment{Table expressions (i.\,e., table-valued queries)}\\
\gprod{T}{\ID}{atomic tables} \\
\galtprod{\kw{JOIN}\,\rep{T}}{join of tables}\\
\galtprod{\kw{UNION}\,\rep{T}}{union of tables}\\
\galtprod{\kw{INTER}\,\rep{T}}{intersection of tables}\\
\galtprod{\kw{SELECT}\, \rep{\ID}\,\kw{FROM}\, T\,\kw{WHERE}\, F}{subtable}\\
\galtprod{T\;\kw{AGGREGATE}\,\rep{CA}}{aggregation}\\
\gprod{CA}{\kw{COUNT}(\ID)}{count values in a column}\\
\galtprod{\kw{MAX}(\ID)}{maximum value in a column}\\
\galtprod{\ldots}{minimum, sum, etc.}\\
\gcomment{Row expressions (i.e\,., single row--valued queries)}\\
\gprod{R}{\kw{VALUES}\,(\rep{CD})}{explicit row} \\
\galtprod{T}{table, but may only contain one row} \\
\gprod{CD}{\ID=V}{column definition}\\
\gcomment{Cell expressions (i.\,e., single column, single row--valued queries)}\\
\gprod{V}{R}{row, but may only contain one column} \\
\galtprod{\ID(\rep{V})}{built-in function symbol applied to values} \\
\gcomment{Formulas}\\
\gprod{F}{V}{boolean value}\\
\galtprod{V=V}{equality of values}\\
\galtprod{R\,\kw{IN}\,T}{containment of rows in tables}\\
\galtprod{\ldots}{boolean operators}\\
\gcomment{Identifiers}\\
\gprod{\ID}{\text{alphanumeric string}}{}\\
\end{commgrammar}
\caption{Syntax of BDL}\label{fig:bdl}
\end{figure}


\subsection{Semantics}

Even though this is a very different knowledge aspect, the general principles of the semantics are the same:
Every BOL-declaration is translated to an SQL declaration, and ontologies are translated declaration-wise.
For every kind of complex expression, there is one inductive function mapping BOL-expressions to SQL-expressions.

In SQL, we can nicely see the difference between declarations and expressions: the former are translated to side effect-ful statements, the latter to side effect-free queries.

\begin{definition}[Concretized Semantic of BOL]\label{def:bolsem:sql}
The \textbf{semantic prefix} consists of the following SQL-statements
\begin{compactitem}
 \item a type $ID$ of identifiers (if not already supported anyway by the underlying database)
 \item declarations of all base types and values of BOL (if not already supported anyway by the underlying database)
 \item TABLE individuals (id: ID, name: string), where the id field is unique and automatically generated when inserting values
\end{compactitem}

Every BOL-ontology $O$ is interpreted as the sequence $P,\sem{O}$ of SQL statements, where $\sem{O}$ is defined in Fig.~\ref{fig:bolsem:sql}.
\end{definition}

\begin{figure}\centering
\begin{tabular}{l|l}
BOL Syntax $X$ & Semantics $\sem{X}$ in SQL\\
\hline
\hline
ontology & SQL schema (list of statements)\\
$D_1,\ldots,D_n$ & $\sem{D_1},\ldots,\sem{D_n}$ \\
\hline
BOL declaration ($C$, $R$, $P$ atomic) & SQL statement \\
\kw{individual}\,$i$ & INSERT VALUES (name="$i$") INTO individuals \\
\kw{concept}\,$i$  & TABLE $i$ (id: ID)\\
\kw{relation}\,$i$ & TABLE $i$ (subject: ID, object: ID) \\
\kw{property}\,$i:Y$ & TABLE $i$ (subject: ID, value: $Y$) \\
$I\; \texttt{is-a}\; C$ & INSERT VALUES (id=$\sem{I}$) INTO $C$\\
$I_1\; R\; I_2$ & INSERT VALUES (subject=$\sem{I_1}$, object=$\sem{I_2}$) INTO R\\
$I\; P\; V$ & INSERT VALUES (subject=$\sem{I}$, value=$V$) INTO P\\
$F$ & consistency check, consequence closure (omitted)\\
\hline
Formula & Query that returns empty result iff formula is true \\
$C_1 \Equiv C_2$ & $(\sem{C_1}\sm\sem{C_2})$ UNION $(\sem{C_2}\sm\sem{C_1})$\\
$C_1 \sqsubseteq C_2$ & $\sem{C_1}\sm\sem{C_2}$\\
$I\; \texttt{is-a}\; C$ & $\sem{I}$ IN $\sem{C}$\\
$I_1\; R\; I_2$ & ($\sem{I_1}$, $\sem{I_2}$) IN $R$ \\
$I\; P\; V$ & ($\sem{I}$, $V$) IN $P$ \\
\hline
Individual & an identifier from the table individuals \\
$i$ & SELECT id FROM individuals WHERE name="$i$" \\
\hline
Concept & SQL table expression for $(id: ID)$\\
$i$ & SELECT * FROM $i$\\
$\top$ & individuals\\
$\bot$ & individuals WHERE false\\
$C_1 \sqcup C_2$ & $\sem{C_1}$ UNION $\sem{C_2}$\\
$C_1 \sqcap C_2$ & $\sem{C_1}$ INTERSECT $\sem{C_2}$\\
$\forall R.C$    & individuals $\sm$ (SELECT subject FROM $\sem{R}$ WHERE object NOT IN $\sem{C})$ \\
$\exists R.C$    & SELECT DISTINCT subject FROM $\sem{R}$, $\sem{C}$ WHERE object=id\\
$\dom\, R$ & SELECT DISTINCT subject FROM $\sem{R}$\\
$\rng\, R$ & SELECT DISTINCT object FROM $\sem{R}$\\
$\dom\, P$ & SELECT DISTINCT subject FROM $\sem{P}$\\
\hline
Relation & SQL table expression for (subject: ID, object: ID)\\
$i$ & SELECT * FROM $i$\\
$R_1 \cup R_2$ & $\sem{R_1}$ UNION $\sem{R_2}$\\
$R_1 \cap R_2$ & $\sem{R_1}$ INTERSECT $\sem{R_2}$\\
$R_1 ; R_2$ & SELECT DISTINCT l.subject, r.object FROM $\sem{R_1}$ AS l, $\sem{R_2}$ AS r \\
            & \tb\tb WHERE l.object = r.subject\\
$R^{-1}$          & SELECT object AS subject, subject AS object FROM $\sem{R}$\\
$R^*$          & (tricky, omitted)\\
$\Delta_C$     & SELECT id AS subject, id AS object FROM $\sem{C}$\\
\hline
Property of type $Y$ & SQL table expression for (subject: ID, value: Y)\\
$i$ & SELECT * FROM $i$\\
\end{tabular}
\medskip

Using the abbreviations: $S\sm T$ = SELECT id FROM $S$ WHERE id NOT IN $T$\\
$S,T$ = JOIN $S,T$\\
\caption{Interpretation Function for BOL into SQL}\label{fig:bolsem:sql}
\end{figure}
\clearpage

\begin{remark}[Limitations]
Our interpretation of BOL in SQL is restricted to assertions using only atomic expressions.
For example, in the case for $I$ \texttt{is-a} $C$, we assume that $I$ and $C$ are names.
Thus, we have already created an individual for $I$ and a table for $C$, and we can thus insert the former into the latter.
The general case would be more complicated but is much less important in practice.
But other expressions very quickly become more difficult.

The interpretation of formulas into SQL is less obvious because SQL is not a logic and therefore does not define a consequence closure.
Thus, we can only use axioms for consistency checks in SQL.
But that requires first carrying out an explicit consequence closure that adds all implied assertions to the database.
\end{remark}

\begin{example}
We interpret the example ontology from Ex.~\ref{ex:bol}.
Excluding the semantic prefix, the entity declarations and assertions result in the following
\begin{lstlisting}
INSERT VALUES (name="FlorianRabe") INTO individuals
INSERT VALUES (name="WuV") INTO individuals
TABLE person (id: ID)
TABLE male (id: ID)
TABLE instructor (id: ID)
TABLE course (id: ID)
TABLE teach (subject: ID, object: ID)
TABLE creditValue (subject: ID, value: float)
INSERT VALUES (id=2) INTO course
INSERT VALUES (subject=1, object=2) INTO teach
INSERT VALUES (subject=1, value=7.5) INTO creditValue
\end{lstlisting}

Here we assume that inserting into the table individuals has automatically assigned the ids $1$ and $2$ to our two individuals.

The concept assertion about $\FR$ using $\sqcap$ cannot be handled by this semantics.
Therefore, we skip that assertion.
The two missing assertions
\begin{lstlisting}
INSERT VALUES (id=1) INTO instructor
INSERT VALUES (id=1) INTO male
\end{lstlisting}
must then be provided by performing the consequence closure.

Moreover, the axioms result in the following consistency checks, i.e., queries that should be empty:
\begin{lstlisting}
SELECT * FROM male $\sm$ SELECT * FROM person
SELECT * FROM instructor $\sm$ SELECT * FROM person
SELECT * FROM (SELECT DISTINCT subject FROM teach) $\sm$ SELECT * FROM instructor
SELECT * FROM (SELECT DISTINCT object FROM teach) $\sm$ SELECT * FROM course
(SELECT * FROM (SELECT DISTINCT subject FROM creditValue) $\sm$ SELECT * FROM course)
   UNION (SELECT * FROM course $\sm$ SELECT DISTINCT subject FROM creditValue)
SELECT * FROM course $\sm$
  (SELECT DISTINCT subject
   FROM (SELECT object AS subject, subject AS object FROM teach), instructor
   WHERE object=id)
\end{lstlisting}
Some of these checks will only succeed after performing the consequence closure.
In particular, the table \verb|person| misses the entry $1$ for the individual $\FR$ because the assertion {\FR\;\texttt{is-a}\;\person} is only present as a consequence
\end{example}

\section{Computational Semantics}\label{sec:bolsem:comp}

We give an alternative semantics using computation, i.e., by using a programming language as the semantic language.
Specifically, we use the programming language Scala.

\subsection{A Scala-Inspired Basic Programming Language}\label{sec:wuv:bpl}
  We give a simple programming language as a language system.

\begin{definition}
Fig.~\ref{fig:bpl} gives the context-free grammar.
The vocabulary symbol is $P$. The expression symbols are $Y$, $V$, and $F$.
\end{definition}

\begin{figure}[hbt]
\begin{commgrammar}
\gcomment{Vocabularies: Programs}\\
\gprod{P}{\rep{D}}{}\\
\gcomment{Declarations}\\
\gprod{D}{\kw{class}\; \ID[\rep{\ID}]\,\kw{extends}\,\rep{\ID}\{\rep{d}\}}{class definition}\\
\galtprod{\kw{object}\; \ID\,\kw{extends}\,\rep{\ID}\{\rep{d}\}}{object definition}\\
\gprod{d}{\kw{val}\;\ID:Y\opt{=T}}{immutable field in a class/object, possibly abstract}\\
\galtprod{\kw{var}\;\ID:Y=T}{mutable field in a class, with initial value}\\
\gcomment{type expressions}\\
\gprod{Y}{\ID}{atomic type (class)} \\
\galtprod{\ID}{built-in type (booleans, int, etc.)} \\
\galtprod{Y(\rep{Y})}{type operator applied to type arguments} \\
\gcomment{term expressions}\\
\gprod{T}{\ID}{atomic value (class, value, variable)} \\
\galtprod{\ID}{built-in value (boolean operators, etc.)} \\
\galtprod{T\doteq_Y T}{equality of terms at a type} \\
\galtprod{T:Y}{instance check} \\
\galtprod{T(\rep{T})}{function applied to values} \\
\galtprod{\kw{new}\,\ID\,\{\rep{d}\}}{new instance of class} \\
\galtprod{T.\ID}{field access in an object} \\
\galtprod{T.\ID=T}{assignment to a mutable field in an object}\\
\galtprod{\{\rep{T}\}}{sequencing (;-operator)}\\
\galtprod{d}{local declaration}\\
\galtprod{\ID=T}{assignment to a local variable}\\
\galtprod{\kw{if}\,(T)\, T\,\kw{else}\, T}{if-then-else}\\
\galtprod{\kw{while}\,(T)\, T}{while-loop}\\
\gcomment{Formulas}\\
\gprod{F}{T}{terms of boolean type} \\
\gcomment{Identifiers}\\
\gprod{\ID}{\text{alphanumeric string}}{}\\
\end{commgrammar}
\caption{Syntax of BPL}\label{fig:bpl}
\end{figure}


\subsection{Semantics}

Again, the general principles are the same:
Every BOL-declaration is translated to a Scala-declaration, and ontologies are translated declaration-wise to Scala-programs.
For every kind of complex expression, there is one inductive function mapping BOL-expressions to Scala-objects.

\begin{definition}[Computational Semantic of BOL]\label{def:bolsem:scala}
The \textbf{semantic prefix} consists of the following Scala statements
\begin{compactitem}
 \item classes for all BOL-base types and values for them (if not already present in Scala)
 \item classes for individuals and hash sets of objects:
\begin{lstlisting}
import scala.collection.mutable.HashSet
val individuals = new HashSet[String]
\end{lstlisting}
\end{compactitem}

Every BOL-ontology $O$ is interpreted as the Scala program $P,\sem{O}$, where $\sem{O}$ is defined in Fig.~\ref{fig:bolsem:scala}.
\end{definition}

\newcommand{\rA}{\Rightarrow}

\begin{figure}\centering
\begin{tabular}{l|l}
BOL Syntax $X$ & Semantics $\sem{X}$ in Scala\\
\hline
\hline
ontology & Scala program \\
$D_1,\ldots,D_n$ & \kw{object} $Program\,\{P,\,\sem{D_1},\ldots,\sem{D_n}\}$ \\
\hline
BOL declaration ($C$, $R$, $P$ atomic) & Scala declaration \\
\kw{individual}\,$i$ & val $i$ = "$i$"; individuals += $i$ \\
\kw{concept}\,$i$  & val $i$ = new HashSet[String]\\
\kw{relation}\,$i$ & val $i$ = new HashSet[(String,String)] \\
\kw{property}\,$i:T$ & val $i$ = new HashSet[(String,T)] \\
$I\; \texttt{is-a}\; C$ & $\sem{C}$ += $\sem{I}$\\
$I_1\; R\; I_2$ & $\sem{R}$ += $(\sem{I_1},\sem{I_2})$\\
$I\; P\; V$ & $\sem{P}$ += $(\sem{I},\sem{V})$\\
$F$ & assertions, consequence closure (omitted)\\
\hline
Formula & Program that evaluates the formula to a Boolean \\
$C_1 \Equiv C_2$ & \{val $c1$ = $\sem{C_1}$; val $c2$ = $\sem{C_2}$; \\
                 & \tb $c_1$.forall(x $\rA$ $c_2$.contains(x)) \&\& $c_2$.forall(x $\rA$ $c_1$.contains(x))\}\\
$C_1 \sqsubseteq C_2$ & \{val $c1$ = $\sem{C_1}$; val $c2$ = $\sem{C_2}$; $c_1$.forall(x $\rA$ $c_2$.contains(x))\}\\
$I\; \texttt{is-a}\; C$ & $\sem{C}$.contains($\sem{I}$)\\
$I_1\; R\; I_2$ & $\sem{R}$.contains($(\sem{I_1},\sem{I_2})$)\\
$I\; P\; V$ & $\sem{P}$.contains($(\sem{I},\sem{V})$)\\
\hline
Individual & String object\\
$i$ & $i$ \\
\hline
Concept & HashSet[String] object\\
$i$ & $i$\\
$\top$ & individuals\\
$\bot$ & HashSet[String].empty()\\
$C_1 \sqcup C_2$ & $\sem{C_1}$.union($\sem{C_2}$)\\
$C_1 \sqcap C_2$ & $\sem{C_1}$.inter($\sem{C_2}$)\\
$\forall R.C$    & \{val c = $\sem{C}$; val r = $\sem{R}$; val e = individuals.clone; \\
                 & \tb r.foreach(x $\rA$ if (!c.contains(x.\_2)) e -= x.\_1); e\} \\
$\exists R.C$    & \{val c = $\sem{C}$; val r = $\sem{R}$; val e = new HashSet[String]; \\
                 & \tb r.foreach(x$ \rA$ if (c.contains(x.\_2)) e += x.\_1); e\}\\
$\dom\, R$ & \{val c = new HashSet[String]; $\sem{R}$.foreach(x $\rA$ c += x.\_1); c\}\\
$\rng\, R$ & \{val c = new HashSet[String]; $\sem{R}$.foreach(x $\rA$ c += x.\_2); c\}\\
$\dom\, P$ & \{val c = new HashSet[String]; $\sem{P}$.foreach(x $\rA$ c += x.\_1); c\}\\
\hline
Relation & HashSet[(String,String)] object\\
$i$ & $i$\\
$R_1 \cup R_2$ & $\sem{R_1}$.union($\sem{R_2}$)\\
$R_1 \cap R_2$ & $\sem{R_1}$.inter($\sem{R_2}$)\\
$R_1 ; R_2$ &  \{val r1 = $\sem{R_1}$; val r2 = $\sem{R_2}$; val e = new HashSet[(String,String)]; \\
            & \tb r1.foreach(x $\rA$ r2.foreach(y $\rA$ if (x.\_2 == y.\_1) e += (x.\_1,y.\_2))); e\}\\
$R^{-1}$    & \{val r = new HashSet[(String,String)]; $\sem{R}$.foreach(x $\rA$ r += (x.\_2,x.\_1)); r\}\\
$R^*$          & \{val r = $\sem{R}$; var e = $\sem{\Delta_\top}$; \kw{while} (!equal(comp(e,r), e)) \{e = comp(e,r)\}; e\}\\
$\Delta_C$     & \{val r = new HashSet[(String,String)]; $\sem{C}$.foreach(x $\rA$ r += (x,x)); r\}\\
\hline
Property of type $T$ & HashSet[(String,T)] object\\
$i$ & $i$\\
\end{tabular}
\caption{Interpretation Function for BOL into Scala (assuming functions equal and comp on relations)}\label{fig:bolsem:scala}
\end{figure}

\begin{remark}[Scala Syntax]
In Scala, val $x=e$ evaluates $e$ and stores the result in $x$.
$\{d_1; \ldots; d_n\}$ is evaluated by executing all $d_i$ in order and returning the result of $d_n$.

$(A,B)$ is the product type $A\times B$ with pairing operator $(x,y)$ and projection functions $\_1$ and $\_2$. $x\rA F(x)$ is $\lambda x.F(x)$.

The class HashSet is part of the standard library and offers function += and -= to add/remove elements, contains to test elementhood, and forall, foreach to quantifiy/iterate over elements.

Types of variables are inferred if omitted.
\end{remark}

\begin{remark}[Limitations]
Our interpretation of BOL in Scala has similar problems as the one in SQL.
We restrict entities in assertions to be atomic.
And we assume that all assertions implied by the consequence closure have already been obtained and added to the ontology.
\end{remark}

\begin{example}
We interpret the example ontology from Ex.~\ref{ex:bol}.
Excluding the semantic prefix, the entity declarations and assertions result in the following
\begin{lstlisting}
individuals += "FlorianRabe"
individuals += "WuV"
val person = new HashSet[String]
val male = new HashSet[String]
val person = new HashSet[String]
val course = new HashSet[String]
val teach = new new HashSet[(String,String)]
val creditValue = new HashSet[(String,float)]
course += WuV
teach += ("FlorianRabe", WuV)
creditValue += (WuV, 7.5)
\end{lstlisting}

Like for SQL, the two statements
\begin{lstlisting}
instructor += "FlorianRabe"
male += "FlorianRabe"
\end{lstlisting}
must be obtained by consequence closure because we cannot handle the $\sqcap$ assertion.
Note that we could easily compute the hash set \verb|instructor.diff(male)| and add to it.
But that would not add anything to the two constituent sets.

If we thing of the axioms as consistency checks, we can translate them to assertions, i.e., Boolean expressions that must be true.
We only give some examples:
\begin{lstlisting}
{val c1 = male; val c2 = person; c1.forall(x $\rA$ c2.contains(x))}

{
  val c1 = course;
  val c2 = {
    val c = instructor;
    val r = {
      val r = new HashSet[(String,String)];
      teach.foreach(x $\rA$ r += (x._2,x._1));
      r
    }
    val e = new HashSet[String];
    r.foreach(x $\rA$ if (c.contains(x._2)) e += x._1);
    e
  };
  c1.forall(x $\rA$ c2.contains(x))
}
\end{lstlisting}

%instructor $\sqsubseteq$ person
%dom teach $\sqsubseteq$ instructor
%rng teach $\sqsubseteq$ course
%dom hasCredits $\Equiv$ course
%course $\sqsubseteq$ $\exists$ teach$^{-1}$ instructor

\end{example}

\section{Narrative Semantics}\label{sec:bolsem:narr}

We give an alternative semantics using narration, i.e., by using a natural language as the semantic language.
Specifically, we use the natural language English.

Again, the general principles are the same:
Every BOL-declaration is translated to an English sentence, and ontologies are translated declaration-wise to English texts.
For every kind of complex expression, there is one inductive function mapping BOL-expressions to English phrases.

\begin{definition}[Narrative Semantic of BOL]\label{def:bolsem:eng}
The \textbf{semantic prefix} consists of English statements explaining
\begin{compactitem}
 \item the base types of BOL (if they are not universally known),
 \item that we rely on a lexicon to correctly form plurals (indicated by -s) and verb forms (indicated by -s, -ing, -ed).
\end{compactitem}

Every BOL-ontology $O$ is interpreted as the English text $P,\sem{O}$, where $\sem{O}$ is defined in Fig.~\ref{fig:bolsem:eng}.
\end{definition}

%It recursively goes through the BOL expressions and generates strings.
%It uses a lexicon function $lex$ (for lexicon) that maps BOL identifiers into (lemmata $\hat=$ base forms of) English words of the appropriate syntactic category for the base cases.

% @MK: The lexicon corresponds to the model of a logical semantics, and I've defered its explanation to do it jointly later.

\begin{figure}\centering
\begin{tabular}{l|l}
BOL Syntax $X$ & Semantics $\sem{X}$ in English\\
\hline\hline
ontology & English text \\
$D_1,\ldots,D_n$ & $\sem{D_1},\ldots,\sem{D_n}$ \\
\hline
BOL declaration & dictionary entry or true sentence\\
\kw{individual}\,$i$ & $i$ is a proper noun.\\
\kw{concept}\,$i$  & $i$ is a common noun.\\
\kw{relation}\,$i$ & $i$ is a transitive verb. \\
\kw{property}\,$i:T$ & $i$ is a common noun for a property that can take $T$-values. \\
$I\; \texttt{is-a}\; C$ & $\sem{I}$ $\sem{C}$.\\
$I_1\; R\; I_2$ & $\sem{I_1}$ $\sem{R}$ $\sem{I_2}$.\\
$I\; P\; V$ & $\sem{I}$ has $\sem{P}$ $\sem{V}$.\\ 
$F$ & $\sem{F}$.\\
\hline
Formula & sentence \\
$C_1 \Equiv C_2$ & $\sem{C_1}$ is the same as $\sem{C_2}$.\\
$C_1 \sqsubseteq C_2$ & everything that $\sem{C_1}$s also $\sem{C_2}$s.\\
$I\; \texttt{is-a}\; C$ & $\sem{I}$ $\sem{C}$.\\
$I_1\; R\; I_2$ & $\sem{I_1}$ $\sem{R}$s $\sem{I_2}$.\\
$I\; P\; V$ & $\sem{I}$ has $\sem{P}$ $\sem{V}$.\\ 
\hline
Individual & noun phrase (to be used as a subject or object)\\
$i$ & $i$ \\
\hline
Concept & intransitive verb phrase (to be plugged after a subject) \\
$i$ & is a $i$\\
$\top$ & is anyone\\
$\bot$ & is no one\\
$C_1 \sqcup C_2$ & $\sem{C_1}$ or $\sem{C_2}$\\
$C_1 \sqcap C_2$ & $\sem{C_1}$ and $\sem{C_2}$\\
%$C_1 \sqcap C_2$ & is a $\sem{C_1}$ that $\sem{C_2}$s & $C_1$ simple\\
$\forall R.C$    & $\sem{R}$s only things that $\sem{C}$ \\
$\exists R.C$    & $\sem{R}$s something that $\sem{C}$\\
$\dom\, R$ & $\sem{R}$s something\\
$\rng\, R$ & is $\sem{R}$ed by something\\
$\dom\, P$ & has some $\sem{P}$\\
\hline
Relation & transitive verb phrase (to be plugged between subject and object)\\
$i$ & $i$\\
$R_1 \cup R_2$ & $\sem{R_1}$s or $\sem{R_2}$s\\
$R_1 \cap R_2$ & $\sem{R_1}$s and $\sem{R_2}$s\\
$R_1 ; R_2$ & $\sem{R_1}$s something that $\sem{R_2}$s\\
$R^{-1}$    & is $\sem{R}$ed by\\
$R^*$       & $\sem{R}$s something that $\sem{R}$s something and so on that $\sem{R}$s\\
$\Delta_C$  & $\sem{C}$s and is the same as\\
\hline
Property of type $T$ & property phrase\\
$i$ & $i$\\
\end{tabular}
\caption{Interpretation Function for BOL into English (intransitive VP version)}\label{fig:bolsem:eng}
\end{figure}

Natural language defines a consequence closure by appealing to consequence in natural language.
That is well-defined as long as we express ourselves precisely enough.
\begin{definition}[Consequence Closure]
We say that a BOL-statement $F$ is a consequence of an ontology $O$ if $\sem{F}$ is a consequence of $P,\sem{O}$.
\end{definition}

\begin{example}
We interpret the example ontology from Ex.~\ref{ex:bol}.
Excluding the semantic prefix and the lexicon lookup, it results in the following text:
\medskip

FlorianRabe is a proper noun.\\
WuV is a proper noun.\\
person is a common noun.\\
male is a common noun.\\
instructor is a common noun.\\
course is a common noun.\\
teach is a transitive verb.\\
creditValue is a common noun for a property that can take $\float$-values.\\
FlorianRabe is a instructor and is a male.\\
WuV is a course.\\
FlorianRabe teachs WuV.\\
WuV has creditValue 7.5.\\
everything that is a male also is a person.\\
everything that is a instructor also is a person.\\
everything that teachs is a instructor.\\
everything that is teached by something is a instructor.\\
has some creditValue is the same as is a course.\\
everything that is a course also is teached by something that is a instructor.
\medskip

This English is very clunky of course.
Multiple tweaks would be needed to get the grammar right:
\begin{compactitem}
 \item It is "teaches" and "taught" instead of "teachs" and "teached",
 \item It is "an instructor" instead of "a instructor",
 \item Sentences start with upper case letters.
 \item Proper nouns often have different names in the ontology than in reality, e.g., it should be "Florian Rabe" and "credit value" instead of "FlorianRabe" and "creditValue".
\end{compactitem}
Moreover, the language could be polished in many places.
For example, "is a instructor and is a male" could become "is a instructor and a male" with a relatively easy special case treatment, or it could become "is a male instructor" with a more complex semantics that interprets some concepts via nouns and some via adjectives.
\end{example}

%\begin{example}
%  Consider an ontology with declarations $\kw{individual}\; P$, $\kw{individual}\; M$, $\kw{relation}\; l$, $\kw{relation}\; h$ and a lexicon function $P\mapsto\text{Peter}, M\mapsto\text{Mary}, l\mapsto\text{love}$, then  $\sem{P\; l\; M}$ = ``Peter loves Mary''. 
%\end{example}
%
%\begin{example}
%  If we extend the ontology above with $\kw{concept} d$ and $\kw{concept} c$ and the lexicon with $lex(d)=\text{is a dog}$ and $lex(c)=\text{is a cat}$, then $M\; \mathtt{is-a}\; c$ translates to ``Mary is a cat'', and  $P\; \mathtt{is-a}\; c\sqcap d$ translates to ``Peter is a cat or is a dog''.
%\end{example}

\begin{remark}[Variants of English]
We are relatively open as to what kind of English we want to use as the semantic language.
The simplest choice would be to use plain English as you could find in a novel or newspaper article.
But for many applications (e.g., formal ontologies in the STEM fields), we would rather use STEM English, i.e., English interspersed with formulas, diagrams, and epistemic cues like ``Definition'', ``Theorem'', ``Proof'', and even $\Box$.
For this kind of English, \latex is a good target format.
We can even use special \latex dialects like \sTeX~\cite{stex} where we can capture more of the semantic properties.
\end{remark}

\begin{remark}[Better Language Generation]
While the target languages in the other translations are formal languages engineered for regularity and simplicity (in terms of language primitives), natural languages have evolved in practical human communication.
As a consequence, the translation in Def.~\ref{def:bolsem:eng} results in English that is clumsy at best and non-grammatical in general.
We can think of the result as \textbf{BOL-pidgin} English.

Let us have a look at some of the problems that appear in both translations:
\begin{itemize}
\item We need a lexicon to obtain inflection information and the translation tries to remedy that by appending ``s'' in various places.
This works in some cases but not in others.
%\item We have introduced some pattern matching and conditional rules in \Cref{fig:bolsem:eng} to alleviate the greatest awkwardnesses, but much more would be needed, which would lead to things like ``Peter hass Mary'' or ``Peter has as childs Mary''. 
\item there are many linguistic devices that serve an important role in natural language, but which we are not targeting.
An example is plural objects for aggregation.
Say we have $P \mathtt{is-a} C$, $M \mathtt{is-a} C$, this would translate to ``P is a $\sem{C}$, M is a $\sem{C}$'' in BOL-pidgin, whereas in natural English we would aggregate this to ``P and M are $\sem{C}$s''.
\end{itemize}

A way out is to utilize special systems for dealing with the surface structure of natural language.
An example of this is the Grammatical Framework~ (GF, \cite{gf}): it allows specifying a rich formal language of \textbf{abstract syntax trees} for natural language (ASTs) together with language-specific \textbf{linearizations}, which amount to recursive functions that translate ASTs to language-specific strings.
GF comes with  a large resource library that provides a comprehensive, language-independent AST specification and linearizations for over 35 languages.
We will not pursue this here, but there is a special course ``Logic-based Natural Language Semantics'' at FAU in the Winter Semesters that covers these and related topics.
One of the major issues that need to be addressed there and here is the notion of compositionality, which is central to all processing and semantics. We will address it next, and come back to it time and again later. 
\end{remark}


%\section{Exercise 2}
%
%Implement the syntax and semantics of BOL.
%
%You can choose the programming language to use. We will use Scala in our examples.
%
%You can choose which semantics to implement.
%The ones that translate directly to Scala or to English are easier because it does not require implementing the syntax of the target language as well.
%The ones that translate to FOL or to SQL require an implementation of the syntax of the respective target language.
%You have to implement that as well or use a library for it.
%
%We recommend not focusing on implementing the syntax and semantics in their entirety.
%It is more instructive to save time by choosing a sublanguage of BOL (by omitting some productions) and to use the time to implement a second semantics.