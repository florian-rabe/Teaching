\section{Ontological Knowledge}

\begin{frame}\frametitle{Components of an Ontology}
8 main declarations
\begin{itemize}
 \item \textbf{individual} --- concrete objects that exist in the real world, e.g., "Florian Rabe" or "WuV"
 \item \textbf{concept} --- abstract groups of individuals, e.g., "instructor" or "course"
 \item \textbf{relation} --- binary relations between two individuals, e.g., "teaches"
 \item \textbf{properties} --- binary relations between an individuals and a concrete value (a number, a date, etc.), e.g., "has-credits"
 \item \textbf{concept assertions} --- the statement that a particular individual is an instance of a particular concept
 \item \textbf{relation assertions} --- the statement that a particular relation holds about two individuals
 \item \textbf{property assertions} --- the statement that a particular individual has a particular value for a particular property
 \item \textbf{axioms} --- statements about relations between concepts, e.g., "instructor" $\sqsubseteq$ "person"
\end{itemize}
\end{frame}

\begin{frame}\frametitle{Divisions of an Ontology}
\begin{blockitems}{Abstract vs. concrete}
 \item TBox: concepts, relations, properties, axioms
  \lec{everything that does not use individuals}
 \item ABox: individuals and assertions
\end{blockitems}

\begin{blockitems}{Named vs. unnamed}
 \item Signature: individuals, concepts, relations, properties \lec{together called entities or resources}
 \item Theory: assertions, axioms
\end{blockitems}
\end{frame}

\begin{frame}\frametitle{Comparison of Terminology}
\begin{center}
\tiny
\begin{tabular}{l|llll|l}
 Here       & OWL      & Description logics & ER model & UML & semantics via logics\\
\hline
 individual & instance & individual & entity & object, instance & constant\\
 concept    & class    & concept &  entity-type & class & unary predicate\\
 relation   & object property & role & role & association & binary predicate \\
 property   & data property   & (not common) & attribute & field of base type & binary predicate\\
\end{tabular}
\medskip

\begin{tabular}{l|ll}
 domain & individual & concept \\
\hline
type theory, logic & constant, term & type \\
set theory  & element & set \\
database    & row & table \\
philosophy\footnote{as in \url{https://plato.stanford.edu/entries/object/}} & object & property \\
grammar & proper noun & common noun \\
\end{tabular}
\end{center}
\end{frame}

\begin{frame}\frametitle{Ontologies as Sets of Triples}
General idea:
\begin{itemize}
\item Turn everything into a relation/property assertion
\item Represent ontologies as sets of subject-predicate-object triples
\item Obtain efficient representation of ontologies using RDF and RDFS
\end{itemize}

\begin{center}
\begin{tabular}{l|lll}
Assertion & \multicolumn{3}{c}{Triple} \\
          & Subject & Predicate & Object \\
\hline
entities           & \multicolumn{3}{|l}{recover from what's mentioned in assertions} \\
concept assertion  & "Florian Rabe" & \texttt{is-a} & "instructor" \\
relation assertion & "Florian Rabe" & "teaches" & "WuV" \\
property assertion & "WuV" & "has credits" & 7.5 \\
axiom              & \multicolumn{3}{|l}{only some special cases work, e.g.,}\\
\tb subconcept axiom & "instructor" & \texttt{subClassOf} & "person"\\
\end{tabular}
\end{center}
\end{frame}

\begin{frame}\frametitle{Special Entities}
RDF and RDFS define special entities for use in ontologies:
\begin{itemize}
 \item "rdfs:Resource": concept of which all individuals are an instance and thus of which every concept is a subconcept
 \item "rdf:type": relates an entity to its type:
  \begin{itemize}
   \item an individual to its concept (corresponding to \texttt{is-a} above)
   \item other entities to their special type (see below)
  \end{itemize}
 \item "rdfs:Class": special class for the type of classes
 \item "rdf:Property": special class for the type of properties
 \item "rdfs:subClassOf": a special relation that relates a subconcept to a superconcept
% \item "rdfs:subPropertyOf": a special relation that relates a relation to one that it implies
 \item "rdfs:domain": a special relation that relates a relation to the concepts of its subjects
 \item "rdfs:range": a special relation that relates a relation/property to the concept/type of its objects
\end{itemize}

Goal/effect: capture as many parts as possible as RDF triples.
\end{frame}

\begin{frame}\frametitle{Declarations as Triples using Special Entities}
\begin{center}
\begin{tabular}{l|lll}
Assertion & \multicolumn{3}{c}{Triple} \\
          & Subject & Predicate & Object \\
\hline
individual & individual & "rdf:type" & "rdfs:Resource" \\
concept  & concept & "rdf:type" & "rdf:Class" \\
relation & relation & "rdf:type" & "rdf:Property" \\
property & property & "rdf:type" & "rdf:Property" \\
concept assertion  & individual & "rdf:type" & concept \\
relation assertion & individual & relation & individual \\
property assertion & individual & property & value \\
\hline
\multicolumn{4}{l}{for special forms of axioms}\\
$c\sqsubseteq d$ & $c$ & "rdfs:subClassOf" & $d$ \\
%$r\sqsubseteq s$ & $r$ & "rdfs:subPropertyOf" & s \\
$\dom\,r\Equiv c$ & $r$ & "rdfs:domain" & $c$ \\
$\rng\, r\Equiv c$ & $r$ & "rdfs:range" & $c$ \\
\end{tabular}
\end{center}
\end{frame}

\begin{frame}\frametitle{An Example Ontology Language}
see syntax of BOL in the lecture notes
\end{frame}

\begin{frame}\frametitle{A Real-Life Ontology Language}
See online resources for OWL.

Some specialties:
\begin{itemize}
\item Slightly different names than in BOL
\item No strict distinction between individuals, concepts, relations - just resources
\item Some special axioms, e.g., to make relations transitive
\item Multiple sublanguages with varying expressivity/implementability: Lite, DL, Full
\end{itemize}

BOL vs. OWL:
\begin{itemize}
\item BOL is simpler, more systematically structured \glec{good for teaching, prototypes}
\item OWL is the standard \glec{the one to use for better or worse}
\end{itemize}
\end{frame}

\begin{frame}\frametitle{Exercise 1}
As a team, build an ontology for a university.

Using git, OWL, and WebProtege are good ways to start.
\bigskip

{\small (In WebProtege, set "suffix" to "user supplied name" in "New Entity Settings". Otherwise, it'll get messy when you share your ontology.)}
\end{frame}

\section{Ontology Morphisms}

\begin{frame}\frametitle{Idea}
Intuition of morphism $m$
\begin{itemize}
\item connects two ontologies, written $m:V\to W$
\item maps $V$-symbols to $W$-expressions
\item extends homomorphically to ma $V$-expressions to $W$-expressions
 \glec{replace every symbol with its assignment}
 \glec{like substitutions for contexts}
\end{itemize}

Purpose
\begin{itemize}
\item extend $V$ with entirely new declarations \\
  special case of $W=V,E$, and identity morphism $V\to W$
\item extend the vocabulary with definitions \\
   special case $m:V,E\to V$, and $m$ maps new symbols to definitions
\item ontology evolution: $V$ is old ontology, $W$ new, $m$ interprets $V$ in $W$
\item transfer legacy content from old to new ontology
\end{itemize}
\end{frame}

\begin{frame}\frametitle{Example: The Common Sense Ontology}
Situation:
\begin{itemize}
\item society uses one ontology for common sense knowledge
\item changes over time
\end{itemize}

Special aspects:
\begin{itemize}
\item unwritten
\item not actually fully agreed upon
\item sometimes subject to political debate
\item no formal ontology language good enough to capture practical nuances
\item many society members not comfortable with formal languages
\end{itemize}

\lec{but still always exists implicitly}

Idea: see political proposals as ontology evolution
\end{frame}

\begin{frame}[fragile]\frametitle{Example: Ontology Change}
Assume $V$ is BOL vocabulary containing
\begin{itemize}
\item concepts $\cn{man}$, $\cn{woman}$
\item axioms $\cn{man}\sqcup\cn{woman}\equiv \top$ and $\cn{man}\sqcap\cn{woman}\equiv \bot$
\end{itemize}
\glec{simplified cis-normative world view}

and $W$ contains
\begin{itemize}
\item concepts $\cn{sexmale}$, $\cn{sexfemale}$, $\cn{cis}$, $\cn{trans}$
\item appropriate axioms
\end{itemize}
\glec{one possible proposal to accommodate transgender people}

Now ontology evolution from $V$ to $W$
\begin{itemize}
\item legacy content = any law, policy etc. referring to $V$-concepts
\item morphism $\cn{gendermatters}$
 \begin{itemize}
 \item $\cn{man}\mapsto (\cn{sexmale}\sqcap\cn{cis})\sqcup (\cn{sexfemale}\sqcap\cn{trans})$
 \item $\cn{woman}\mapsto (\cn{sexfemale}\sqcap\cn{cis})\sqcup (\cn{sexmale}\sqcap\cn{trans})$
 \end{itemize}
\item alternative morphism $\cn{sexmatters}$
 \begin{itemize}
 \item $\cn{man}\mapsto \cn{sexmale}$
 \item $\cn{woman}\mapsto \cn{sexfemale}$
 \end{itemize}
\end{itemize}
\end{frame}

\begin{frame}\frametitle{Side Note: Knowledge Representation is Apolitical}
Knowledge representation makes no judgment about which ontologies or morphisms are fair, moral, politically correct, etc.

It can only make judgments about whether an ontology (morphism) is practical, e.g., based on
\begin{itemize}
\item well-formedness and consistency of an ontology
\item decidability, efficiency of querying
\item simplicity, e.g., measured by
\begin{itemize}
\item number of declarations or the size of expressions
\item number of axioms about each symbol
\end{itemize}
\item existence and simplicity of morphisms
\end{itemize}

Languages must allow for expressing whichever knowledge or opinion the user has.
Users have to judge if an ontology or morphisms is the right one.
\end{frame}

\begin{frame}\frametitle{BOL Morphisms}
Syntax: Extend grammar with vocabulary morphisms
\begin{commgrammar}
\gprod{M}{\rep{A}: O\to O}{morphisms}\\
\gprod{A}{\ID\mapsto I}{individual assignment}\\
\galtprod{\ID\mapsto C}{concept assignment}\\
\galtprod{\ID\mapsto R}{relation assignment}\\
\galtprod{\ID \mapsto P}{property assignment}
\end{commgrammar}

Well-formedness for $M:O\to O'$:
\begin{itemize}
\item one assignment $\ID\mapsto E$ for each declaration $\ID$ of $O$
\item $E$ must be an $O'$-expression of the right type
 \begin{itemize}
 \item individual symbols to individual expressions
 \item concept symbols to concept expressions
 \item relation symbols to relation expressions
 \item property symbols of type $V$ to property expressions of type $V$
 \item what about assertions and axioms? \glec{see below}
 \end{itemize}
\end{itemize}
\end{frame}

\begin{frame}\frametitle{Homomorphic Extension}
Definition:
\begin{itemize}
\item Given morphism $m:O\to O'$,
\item define mapping from $O$-expressions $E$ to $O'$-expressions $m(E)$ by
\item replacing every $O$-symbol $s$ in $E$ \\ with the expression $s\mapsto E$ provided by $m$.
\end{itemize}

This is why morphisms must contain exactly one assignment for every $O$-symbol.
\end{frame}

\begin{frame}\frametitle{BOL Morphisms: What about Axioms?}
Definition:
\begin{itemize}
\item A morphism $m:O\to O'$ is well-formed if
\item for every axiom/assertion $F$ in $O$,
\item we have that $m(F)$ is a theorem of $O'$.
\end{itemize}

Theorem:
\begin{itemize}
\item if $\vdash_O E:E'$ then $\vdash_{O'} m(E):m(E')$
\item if $\vdash_O F$ then $\vdash_{O'} m(F)$
\end{itemize}
 \lec{morphisms preserve truth}

Mapping axioms works best if
\begin{itemize}
\item every axiom/assertion has a name
\item new expression kind for proofs \glec{given by derivations of some absolute deductive semantics}
 \glec{axioms = proof symbols = atomic proofs}
\item morphisms contain assignments $a\mapsto P$ of axiom $a$ to proof $P$
\end{itemize}
\end{frame}

\begin{frame}\frametitle{Exercise ???}
Extend your implementation of BOL with vocabulary morphisms.
This should include a function that computes the homomorphic extension.

Apply it to a test morphism.
\end{frame}

\begin{frame}\frametitle{Prevalence of Morphisms}
Deduction
\begin{itemize}
\item algebraic hierarchy, e.g., $Monoid\to Group$
\item theory $\to$ model, e.g., $Group\to Integer$
\end{itemize}
Computation
\begin{itemize}
\item class extension
\item interface implementation
\item type class instances
\item functor
\item API adapters
\end{itemize}
Concrete data
\begin{itemize}
\item between tables: database views
\item between schemas: database migration
\end{itemize}

General: module systems for building large vocabularies
\end{frame}

\section{Typed Ontologies and Database Schmemas}

\begin{frame}\frametitle{Motivation}
\begin{blockitems}{Main ideas}
\item Ontology abstractly describes concepts and relations
\item Tool maintains concrete data set
\item Focus on efficiently
  \begin{itemize}
  \item identifying (i.e., assign names)
  \item representing
  \item processing
  \item querying
  \end{itemize}
  large sets of concrete data
\end{blockitems}

\begin{blockitems}{Recall: TBox-ABox distinction}
  \item TBox: general parts, abstract, fixed
   \lec{main challenge: correct modeling of domain}
  \item ABox: concrete individuals and assertions about them, growing
   \lec{main challenge: aggregate them all}
\end{blockitems}
\end{frame}

\begin{frame}\frametitle{Concrete Data}
\begin{blockitems}{Concrete is}
\item Base values: integers, strings, booleans, etc.
\item Collections: sets, multisets, lists (always finite)
\item Aggregations: tuples, records (always finite)
\item User-defined concrete data: enumerations, inductive types
\item Advanced objects: finite maps, graphs, etc.
\end{blockitems}

\begin{blockitems}{Concrete is not}
\item Free symbols to be interpreted by a model
 \lec{exception: foreign function interfaces}
\item Variables (free or bound)
 \lec{$\lambda$-abstraction, quantification}
\item Symbolic expressions
 \lec{formulas, algorithms}
 Exceptions:
  \begin{itemize}
  \item expressions of inductive type
  \item application of built-in functions
  \item queries that return concrete data
  \end{itemize}
\end{blockitems}
\end{frame}

\begin{frame}\frametitle{Breakout question}
What is the difference between
\begin{itemize}
\item an OWL ontology
\item an SQL database
\end{itemize}
\end{frame}

\begin{frame}\frametitle{Two Approaches}
\begin{blockitems}{Based on \emph{untyped} (Curry-typed) ontology languages}
\item Representation based on \emph{knowledge graph}
\item Ontology written in BOL-like language
\item Data maintained as \emph{set of triples}
  \glec{tool = triple store}
\item Typical language/tool design
 \begin{itemize}
 \item ontology and query language \emph{separate}
  \glec{e.g., OWL, SPARQL}
 \item triple store and query engine integrated
  \glec{e.g., Virtuoso tool}
 \end{itemize}
\end{blockitems}

\begin{blockitems}{Based on \emph{typed} (Church-typed) ontology languages}
\item Representation based on \emph{abstract data types}
\item Ontology written as database schema
\item Data maintained as \emph{tables}
  \glec{tool = (relational) database}
\item Typical language/tool design
 \begin{itemize}
 \item ontology and query language \emph{integrated}
  \glec{e.g., SQL}
 \item table store and query engine integrated
  \glec{e.g., SQLite tool}
 \end{itemize}
\end{blockitems}
\end{frame}

\begin{frame}\frametitle{Evolution of Approaches}
\begin{blockitems}{Our usage is non-standard}
 \item Common
  \begin{itemize}
  \item ontologies = untyped approach, OWL, triples,  SPARQL
  \item databases = typed approach, tables, SQL
  \end{itemize}
 \item Our understanding: two approaches evolved from same idea
	\begin{itemize}
	\item triple store = untyped database
	\item SQL schema = typed ontology
	\end{itemize}
\end{blockitems}

\begin{blockitems}{Evolution}
\item Typed-untyped distinction minor technical difference
\item Optimization of respective advantages causes speciation
\item Today segregation into different
 \begin{itemize}
 \item jargons
 \item languages, tools
 \item communities, conferences
 \item courses
 \end{itemize}
\end{blockitems}
\end{frame}

\begin{frame}\frametitle{Curry-typed concrete data}
\begin{blockitems}{Central data structure = knowledge graph}
\item nodes = individuals $i$
 \begin{itemize}
 \item identifier
 \item sets of concepts of $i$
 \item key-value sets of properties of $i$
 \end{itemize}
\item edges = relation assertions
 \begin{itemize}
 \item from subject to object
 \item labeled with name of relation
 \end{itemize}
\end{blockitems}

\begin{blockitems}{Processing strengths}
\item store: as triple set
\item edit: Protege-style or graph-based
\item visualize: as graph
  \glec{different colors for concepts, relations}
\item query: match, traverse graph structure
\item untyped data simplifies integration, migration
\end{blockitems}
\end{frame}

\begin{frame}\frametitle{Church-typed concrete data}
\begin{blockitems}{Central data structure = relational database}
\item tables = abstract data type
\item rows = objects of that type
\item columns = fields of ADT
\item cells = values of fields
\end{blockitems}

\begin{blockitems}{Processing strengths}
\item store: as CSV text files, or similar
\item edit: SQL commands or table editors
\item visualize: as table view
\item query: relational algebra
\item typed data simplifies selecting, sorting, aggregating
\end{blockitems}
\end{frame}

\begin{frame}\frametitle{Identifiers}
\begin{blockitems}{Curry-Typed Knowledge graph}
\item concept, relation, property names given in TBox
\item individual names attached to nodes
\end{blockitems}

\begin{blockitems}{Church-Typed Database}
\item table, column names given in schema
\item row identified by distinguished column (= key) \\
options
 \begin{itemize}
 \item preexistent characteristic column
 \item added upon insertion
  \begin{itemize}
  \item UUID string
  \item incremental integers
  \item concatenation of characteristic list of columns
  \end{itemize} 
 \end{itemize}
\item column/row identifiers formed by qualifying with table name
\end{blockitems}
\end{frame}

\begin{frame}\frametitle{Axioms}
\begin{blockitems}{Curry-Typed Knowledge Graph}
\item traditionally very expressive axioms
\item yields inferred assertions
\item triple store must do consequence closure to return correct query results
\item not all axioms supported by every triple store
\end{blockitems}

\begin{blockitems}{Church-Typed Database}
\item typically no axioms
\item instead consistency constraints, triggers
\item allows limited support for axioms without calling it that way
\item stronger need for users to program the consequence closure manually
\end{blockitems}
\end{frame}

\begin{frame}\frametitle{Open/Closed World}
\begin{itemize}
\item Question: is the data complete?
 \begin{itemize}
 \item closed world: yes
 \item open world: not necessarily
 \end{itemize}
\item Dimensions of openness
 \begin{itemize}
  \item existence of individual objects
  \item assertions about them
 \end{itemize}
\item Sources of openness
  \begin{itemize}
  \item more exists but has not yet been added
  \item more could be created later
  \end{itemize}
\item Orthogonal to typed/untyped distinction, but in practice
 \begin{itemize}
 \item knowledge graphs use open world
 \item databases use closed world
 \end{itemize}
\end{itemize}
\lec{Open world is natural state, closing adds knowledge}
\end{frame}

\begin{frame}\frametitle{Closing the World}
\begin{blockitems}{Derivable consequences}
 \item induction: prove universal property by proving for each object
 \item negation by failure: atomic property false if not provable
 \item term-generation constraint: only nameable objects exist
\end{blockitems}

\begin{blockitems}{Enabled operations}
 \item universal set: all objects
 \item complement of concept/type
 \item defaults: assume default value for property if not otherwise asserted
\end{blockitems}

\begin{blockitems}{Monotonicity problem}
 \item monotone operation: bigger world = more results
 \item examples: union, intersection, $\exists R.C$, join, IN conditions
 \item counter-examples: complement, $\forall R.C$, NOT IN conditions
\end{blockitems}
\lec{technically, non-monotone operations in open world dubious}
\end{frame}

\begin{frame}\frametitle{Summary}
\begin{tabular}{l|ll}
  & semantic web & relational databases \\
\hline
ontology aspect & TBox of ontology & SQL schema \\
conceptual model & knowledge graph & set of tables \\
concrete data aspect & ABox of ontology & SQL database \\
concrete data storage & set of triples & set of rows of the tables \\
concrete data formats & RDF & CSV \\
concrete data tool & triple store & database implementation \\
typing & soft/Curry & hard/Church\\
query language & SPARQL & SQL SELECT query \\
openness of world & tends to be open & tends to be closed \\
\end{tabular}
\end{frame}


\begin{frame}\frametitle{Exercise 2}
Extend your ontology with an ABox and axioms.
Export it in RDF format to a triple store like Virtuoso and run a concrete query in SPARQL.

Export it in OWL format to a reasoner like FaCT++ and run a deductive query.
Potentially, do this by installing a plugin for a reasoner in your ontology IDE.
\end{frame}