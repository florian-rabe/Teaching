\section{Ontological Knowledge}

\begin{frame}\frametitle{Components of an Ontology}
8 main declarations
\begin{itemize}
 \item \textbf{individual} --- concrete objects that exist in the real world, e.g., "Florian Rabe" or "WuV"
 \item \textbf{concept} --- abstract groups of individuals, e.g., "instructor" or "course"
 \item \textbf{relation} --- binary relations between two individuals, e.g., "teaches"
 \item \textbf{properties} --- binary relations between an individuals and a concrete value (a number, a date, etc.), e.g., "has-credits"
 \item \textbf{concept assertions} --- the statement that a particular individual is an instance of a particular concept
 \item \textbf{relation assertions} --- the statement that a particular relation holds about two individuals
 \item \textbf{property assertions} --- the statement that a particular individual has a particular value for a particular property
 \item \textbf{axioms} --- statements about relations between concepts, e.g., "instructor" $\sqsubseteq$ "person"
\end{itemize}
\end{frame}

\begin{frame}\frametitle{Divisions of an Ontology}
\begin{blockitems}{Abstract vs. concrete}
 \item TBox: concepts, relations, properties, axioms
  \lec{everything that does not use individuals}
 \item ABox: individuals and assertions
\end{blockitems}

\begin{blockitems}{Named vs. unnamed}
 \item Signature: individuals, concepts, relations, properties \lec{together called entities or resources}
 \item Theory: assertions, axioms
\end{blockitems}
\end{frame}

\begin{frame}\frametitle{Comparison of Terminology}
\begin{center}
\tiny
\begin{tabular}{l|llll|l}
 Here       & OWL      & Description logics & ER model & UML & semantics via logics\\
\hline
 individual & instance & individual & entity & object, instance & constant\\
 concept    & class    & concept &  entity-type & class & unary predicate\\
 relation   & object property & role & role & association & binary predicate \\
 property   & data property   & (not common) & attribute & field of base type & binary predicate\\
\end{tabular}
\medskip

\begin{tabular}{l|ll}
 domain & individual & concept \\
\hline
type theory, logic & constant, term & type \\
set theory  & element & set \\
database    & row & table \\
philosophy\footnote{as in \url{https://plato.stanford.edu/entries/object/}} & object & property \\
grammar & proper noun & common noun \\
\end{tabular}
\end{center}
\end{frame}

\begin{frame}\frametitle{Ontologies as Sets of Triples}
General idea:
\begin{itemize}
\item Turn everything into a relation/property assertion
\item Represent ontologies as sets of subject-predicate-object triples
\item Obtain efficient representation of ontologies using RDF and RDFS
\end{itemize}

\begin{center}
\begin{tabular}{l|lll}
Assertion & \multicolumn{3}{c}{Triple} \\
          & Subject & Predicate & Object \\
\hline
entities           & \multicolumn{3}{|l}{recover from what's mentioned in assertions} \\
concept assertion  & "Florian Rabe" & \texttt{is-a} & "instructor" \\
relation assertion & "Florian Rabe" & "teaches" & "WuV" \\
property assertion & "WuV" & "has credits" & 7.5 \\
axiom              & \multicolumn{3}{|l}{only some special cases work, e.g.,}\\
\tb subconcept axiom & "instructor" & \texttt{subClassOf} & "person"\\
\end{tabular}
\end{center}
\end{frame}

\begin{frame}\frametitle{Special Entities}
RDF and RDFS define special entities for use in ontologies:
\begin{itemize}
 \item "rdfs:Resource": concept of which all individuals are an instance and thus of which every concept is a subconcept
 \item "rdf:type": relates an entity to its type:
  \begin{itemize}
   \item an individual to its concept (corresponding to \texttt{is-a} above)
   \item other entities to their special type (see below)
  \end{itemize}
 \item "rdfs:Class": special class for the type of classes
 \item "rdf:Property": special class for the type of properties
 \item "rdfs:subClassOf": a special relation that relates a subconcept to a superconcept
% \item "rdfs:subPropertyOf": a special relation that relates a relation to one that it implies
 \item "rdfs:domain": a special relation that relates a relation to the concepts of its subjects
 \item "rdfs:range": a special relation that relates a relation/property to the concept/type of its objects
\end{itemize}

Goal/effect: capture as many parts as possible as RDF triples.
\end{frame}

\begin{frame}\frametitle{Declarations as Triples using Special Entities}
\begin{center}
\begin{tabular}{l|lll}
Assertion & \multicolumn{3}{c}{Triple} \\
          & Subject & Predicate & Object \\
\hline
individual & individual & "rdf:type" & "rdfs:Resource" \\
concept  & concept & "rdf:type" & "rdf:Class" \\
relation & relation & "rdf:type" & "rdf:Property" \\
property & property & "rdf:type" & "rdf:Property" \\
concept assertion  & individual & "rdf:type" & concept \\
relation assertion & individual & relation & individual \\
property assertion & individual & property & value \\
\hline
\multicolumn{4}{l}{for special forms of axioms}\\
$c\sqsubseteq d$ & $c$ & "rdfs:subClassOf" & $d$ \\
%$r\sqsubseteq s$ & $r$ & "rdfs:subPropertyOf" & s \\
$\dom\,r\Equiv c$ & $r$ & "rdfs:domain" & $c$ \\
$\rng\, r\Equiv c$ & $r$ & "rdfs:range" & $c$ \\
\end{tabular}
\end{center}
\end{frame}

\begin{frame}\frametitle{An Example Ontology Language}
see syntax of BOL in the lecture notes
\end{frame}

\begin{frame}\frametitle{A Real-Life Ontology Language}
See online resources for OWL.

Some specialties:
\begin{itemize}
\item Slightly different names than in BOL
\item No strict distinction between individuals, concepts, relations - just resources
\item Some special axioms, e.g., to make relations transitive
\item Multiple sublanguages with varying expressivity/implementability: Lite, DL, Full
\end{itemize}

BOL vs. OWL:
\begin{itemize}
\item BOL is simpler, more systematically structured \glec{good for teaching, prototypes}
\item OWL is the standard \glec{the one to use for better or worse}
\end{itemize}
\end{frame}

\begin{frame}\frametitle{Exercise 1}
As a team, build an ontology for a university.

Using git, OWL, and WebProtege are good ways to start.
\bigskip

{\small (In WebProtege, set "suffix" to "user supplied name" in "New Entity Settings". Otherwise, it'll get messy when you share your ontology.)}
\end{frame}

\begin{frame}\frametitle{Example: The Common Sense Ontology}
\begin{blockitems}{Situation}
\item society uses one ontology for common sense knowledge
\item changes over time
\end{blockitems}
\lec{content relative to ontology: laws, regulations, etc.}

\begin{blockitems}{Special aspects}
\item unwritten
\item not actually fully agreed upon
\item sometimes subject to political debate
\item no formal ontology language good enough to capture practical nuances
\item many society members not comfortable with formal languages
\end{blockitems}
\lec{but still always exists implicitly}

Idea: see political proposals as ontology evolution
\end{frame}

\section{Ontology Morphisms}

\begin{frame}\frametitle{Idea}
\begin{blockitems}{Intuition of morphism $m$}
\item connects two ontologies, written $m:V\to W$
\item maps $V$-symbols to $W$-expressions
\item extends homomorphically to ma $V$-expressions to $W$-expressions
 \glec{replace every symbol with its assignment}
 \glec{like substitutions for contexts}
\end{blockitems}

\begin{blockitems}{Purpose}
\item extend $V$ with entirely new declarations \\
  special case of $W=V,E$, and identity morphism $V\to W$
\item extend the vocabulary with definitions \\
   special case $m:V,E\to V$, and $m$ maps new symbols to definitions
\item ontology evolution: $V$ is old ontology, $W$ new, $m$ interprets $V$ in $W$
\item transfer legacy content from old to new ontology
\end{blockitems}
\end{frame}

\begin{frame}[fragile]\frametitle{Exercise 2}
We write ontologies for sex and gender.

Write two ontologies for
\begin{itemize}
\item cis-normative world view with just men and women
\item trans-inclusive world view that accommodates sex and gender
\end{itemize}
and relate them with ontology morphisms.
\end{frame}

\begin{frame}\frametitle{Side Note: Knowledge Representation is Apolitical}
Knowledge representation makes no judgment about which ontologies or morphisms are fair, moral, politically correct, etc.

\begin{blockitems}{It can only judge practicality, e.g.,}
\item well-formedness and consistency
\item decidability, efficiency of querying
\item simplicity, e.g., measured by
\begin{itemize}
\item number of declarations or the size of expressions
\item number of axioms about each symbol
\end{itemize}
\item existence and simplicity of morphisms
\end{blockitems}

Languages must allow for expressing whichever knowledge/opinion the user has.

Only users can judge if an ontology is correct.
\end{frame}

\begin{frame}\frametitle{BOL Morphisms Formally}
Syntax: Extend grammar with vocabulary morphisms
\begin{commgrammar}
\gprod{M}{\rep{A}: O\to O}{morphisms}\\
\gprod{A}{i\mapsto I}{individual assignment}\\
\galtprod{c\mapsto C}{concept assignment}\\
\galtprod{r\mapsto R}{relation assignment}\\
\galtprod{p\mapsto P}{property assignment}
\end{commgrammar}

Well-formedness for $M:O\to O'$:
\begin{itemize}
\item one assignment $\ID\mapsto E$ for each declaration $\ID$ of $O$
\item $E$ must be an $O'$-expression of the right kind
 \begin{itemize}
 \item individual symbols to individual expressions
 \item concept symbols to concept expressions
 \item relation symbols to relation expressions
 \item property symbols of type $V$ to property expressions of type $V$
 \item what about assertions and axioms? \glec{see below}
 \end{itemize}
\end{itemize}
\end{frame}

\begin{frame}\frametitle{Homomorphic Extension}
Given morphism $m:O\to O'$, define
\begin{itemize}
\item mapping $\ov{m}$ from $O$-expressions $E$ to $O'$-expressions $\ov{m}(E)$ by
\item replacing every $O$-symbol $s$ in $E$ \\ with the expression $s\mapsto E$ provided by $m$.
\end{itemize}
\lec{Notation: $m(E)$ instead of $\ov{m}(E)$}
\bigskip

Well-defined mapping because morphisms must contain exactly one assignment for every $O$-symbol.
\end{frame}


\begin{frame}\frametitle{BOL Morphisms: What about Axioms?}
A morphism $m:O\to O'$ is well-formed if
\begin{itemize}
\item for every axiom/assertion $F$ in $O$,
\item we have that $m(F)$ is a theorem of $O'$.
\end{itemize}
\bigskip

Theorem: Morphisms preserve truth
\begin{itemize}
\item if $\vdash_O E:E'$ then $\vdash_{O'} m(E):m(E')$
\item if $\vdash_O F$ then $\vdash_{O'} m(F)$
\end{itemize}
\bigskip

Mapping axioms works best if
\begin{itemize}
\item every axiom/assertion has a name
\item new expression kind for proofs \glec{given by derivations of some absolute deductive semantics}
 \glec{axioms = proof symbols = atomic proofs}
\item morphisms contain assignments $a\mapsto P$ of axiom $a$ to proof $P$
\end{itemize}
\end{frame}

\begin{frame}[fragile]\frametitle{Example/Exercise 2}
\begin{blockitems}{Assume $CisNormative$ is BOL vocabulary containing}
\item concepts $\cn{man}$, $\cn{woman}$
\item axioms $\cn{man}\sqcup\cn{woman}\Equiv \top$ and $\cn{man}\sqcap\cn{woman}\Equiv \bot$
\end{blockitems}
\glec{simplified cis-normative world view}

\begin{blockitems}{and $TransFriendly$ contains \hfill\small one way to accommodate transgender people}
\item concepts $\cn{sexmale}$, $\cn{sexfemale}$, $\cn{cis}$, $\cn{trans}$
\item appropriate axioms
\end{blockitems}

\begin{blockitems}{Now have ontology morphism $CisNormative \to TransFriendly$}
\item morphism $\cn{gendermatters}$
 \begin{itemize}
 \item $\cn{man}\mapsto (\cn{sexmale}\sqcap\cn{cis})\sqcup (\cn{sexfemale}\sqcap\cn{trans})$
 \item $\cn{woman}\mapsto (\cn{sexfemale}\sqcap\cn{cis})\sqcup (\cn{sexmale}\sqcap\cn{trans})$
 \end{itemize}
\item alternative morphism $\cn{sexmatters}$
 \begin{itemize}
 \item $\cn{man}\mapsto \cn{sexmale}$
 \item $\cn{woman}\mapsto \cn{sexfemale}$
 \end{itemize}
\end{blockitems}
\end{frame}

\begin{frame}\frametitle{Prevalence of Morphisms}
Deduction
\begin{itemize}
\item algebraic hierarchy, e.g., $Monoid\to Group$
\item theory $\to$ model, e.g., $Group\to Integer$
\end{itemize}
Computation
\begin{itemize}
\item class extension
\item interface implementation
\item type class instances
\item functor
\item API adapters
\end{itemize}
Concrete data
\begin{itemize}
\item between tables: database views
\item between schemas: database migration
\end{itemize}

General: module systems for building large vocabularies
\end{frame}
