\section{Ontological Knowledge}

\begin{frame}\frametitle{Components of an Ontology}
8 main declarations
\begin{itemize}
 \item \textbf{individual} --- concrete objects that exist in the real world, e.g., "Florian Rabe" or "WuV"
 \item \textbf{concept} --- abstract groups of individuals, e.g., "instructor" or "course"
 \item \textbf{relation} --- binary relations between two individuals, e.g., "teaches"
 \item \textbf{properties} --- binary relations between an individuals and a concrete value (a number, a date, etc.), e.g., "has-credits"
 \item \textbf{concept assertions} --- the statement that a particular individual is an instance of a particular concept
 \item \textbf{relation assertions} --- the statement that a particular relation holds about two individuals
 \item \textbf{property assertions} --- the statement that a particular individual has a particular value for a particular property
 \item \textbf{axioms} --- statements about relations between concepts, e.g., "instructor" $\sqsubseteq$ "person"
\end{itemize}
\end{frame}

\begin{frame}\frametitle{Divisions of an Ontology}
\begin{blockitems}{Abstract vs. concrete}
 \item TBox: concepts, relations, properties, axioms
  \lec{everything that does not use individuals}
 \item ABox: individuals and assertions
\end{blockitems}

\begin{blockitems}{Named vs. unnamed}
 \item Signature: individuals, concepts, relations, properties \lec{together called entities or resources}
 \item Theory: assertions, axioms
\end{blockitems}
\end{frame}

\begin{frame}\frametitle{Comparison of Terminology}
\begin{center}
\tiny
\begin{tabular}{l|llll|l}
 Here       & OWL      & Description logics & ER model & UML & semantics via logics\\
\hline
 individual & instance & individual & entity & object, instance & constant\\
 concept    & class    & concept &  entity-type & class & unary predicate\\
 relation   & object property & role & role & association & binary predicate \\
 property   & data property   & (not common) & attribute & field of base type & binary predicate\\
\end{tabular}
\medskip

\begin{tabular}{l|ll}
 domain & individual & concept \\
\hline
type theory, logic & constant, term & type \\
set theory  & element & set \\
database    & row & table \\
philosophy\footnote{as in \url{https://plato.stanford.edu/entries/object/}} & object & property \\
grammar & proper noun & common noun \\
\end{tabular}
\end{center}
\end{frame}

\begin{frame}\frametitle{Ontologies as Sets of Triples}
General idea:
\begin{itemize}
\item Turn everything into a relation/property assertion
\item Represent ontologies as sets of subject-predicate-object triples
\item Obtain efficient representation of ontologies using RDF and RDFS
\end{itemize}

\begin{center}
\begin{tabular}{l|lll}
Assertion & \multicolumn{3}{c}{Triple} \\
          & Subject & Predicate & Object \\
\hline
entities           & \multicolumn{3}{|l}{recover from what's mentioned in assertions} \\
concept assertion  & "Florian Rabe" & \texttt{is-a} & "instructor" \\
relation assertion & "Florian Rabe" & "teaches" & "WuV" \\
property assertion & "WuV" & "has credits" & 7.5 \\
axiom              & \multicolumn{3}{|l}{only some special cases work, e.g.,}\\
\tb subconcept axiom & "instructor" & \texttt{subClassOf} & "person"\\
\end{tabular}
\end{center}
\end{frame}

\begin{frame}\frametitle{Special Entities}
RDF and RDFS define special entities for use in ontologies:
\begin{itemize}
 \item "rdfs:Resource": concept of which all individuals are an instance and thus of which every concept is a subconcept
 \item "rdf:type": relates an entity to its type:
  \begin{itemize}
   \item an individual to its concept (corresponding to \texttt{is-a} above)
   \item other entities to their special type (see below)
  \end{itemize}
 \item "rdfs:Class": special class for the type of classes
 \item "rdf:Property": special class for the type of properties
 \item "rdfs:subClassOf": a special relation that relates a subconcept to a superconcept
% \item "rdfs:subPropertyOf": a special relation that relates a relation to one that it implies
 \item "rdfs:domain": a special relation that relates a relation to the concepts of its subjects
 \item "rdfs:range": a special relation that relates a relation/property to the concept/type of its objects
\end{itemize}

Goal/effect: capture as many parts as possible as RDF triples.
\end{frame}

\begin{frame}\frametitle{Declarations as Triples using Special Entities}
\begin{center}
\begin{tabular}{l|lll}
Assertion & \multicolumn{3}{c}{Triple} \\
          & Subject & Predicate & Object \\
\hline
individual & individual & "rdf:type" & "rdfs:Resource" \\
concept  & concept & "rdf:type" & "rdf:Class" \\
relation & relation & "rdf:type" & "rdf:Property" \\
property & property & "rdf:type" & "rdf:Property" \\
concept assertion  & individual & "rdf:type" & concept \\
relation assertion & individual & relation & individual \\
property assertion & individual & property & value \\
\hline
\multicolumn{4}{l}{for special forms of axioms}\\
$c\sqsubseteq d$ & $c$ & "rdfs:subClassOf" & $d$ \\
%$r\sqsubseteq s$ & $r$ & "rdfs:subPropertyOf" & s \\
$\dom\,r\Equiv c$ & $r$ & "rdfs:domain" & $c$ \\
$\rng\, r\Equiv c$ & $r$ & "rdfs:range" & $c$ \\
\end{tabular}
\end{center}
\end{frame}

\begin{frame}\frametitle{An Example Ontology Language}
The structural part declares/introduces names:
%see syntax of BOL in the lecture notes
\begin{commgrammar}
\gcomment{Vocabularies: Ontologies}\\
\gprod{O}{\rep{D}}{}\\
\gcomment{Declarations}\\
\gprod{D}{\kw{individual}\; i}{atomic individual}\\
\galtprod{\kw{concept}\; c}{atomic concept}\\
\galtprod{\kw{relation}\; r}{atomic relation}\\
\galtprod{\kw{property}\; p: T}{atomic property}\\
\galtprod{\kw{axiom}\; F}{axiom}\\
\gcomment{Identifiers}\\
\gprod{i,c,r,p}{\text{alphanumeric string}}{}\\
\end{commgrammar}
\end{frame}

\begin{frame}\frametitle{An Example Ontology Language}
Expressions use the declared names to build complex entities
%see syntax of BOL in the lecture notes
\begin{commgrammar}
\gcomment{Individual expressions}\\
\gprod{I}{i}{atomic individuals}\\
\gcomment{Concept expressions}\\
\gprod{C}{c}{atomic concepts}\\
\galtprod{\top}{universal concept}\\
\galtprod{\bot}{empty concept}\\
\galtprod{C \sqcup C}{union of concepts}\\
\galtprod{C \sqcap C}{intersection of concepts}\\
\galtprod{\forall R.C}{universal relativization}\\
\galtprod{\exists R.C}{existential relativization}\\
\galtprod{\dom R}{domain of a relation}\\
\galtprod{\rng R}{range of a relation}\\
\galtprod{\dom P}{domain of a property}\\
\end{commgrammar}
\end{frame}

\begin{frame}\frametitle{An Example Ontology Language}
Typically one kind of expression for every kind of declaration:
%see syntax of BOL in the lecture notes
\begin{commgrammar}
\gcomment{Relation expressions}\\
\gprod{R}{r}{atomic relations}\\
\galtprod{R \cup R}{union of relations}\\
\galtprod{R \cap R}{intersection of relations}\\
\galtprod{R ; R}{composition of relations}\\
\galtprod{R^*}{transitive closure of a relation}\\
\galtprod{R^{-1}}{dual relation}\\
\galtprod{\Delta_C}{identity relation of a concept}\\
\gcomment{Property expressions}\\
\gprod{P}{p}{atomic properties}\\
\end{commgrammar}
\end{frame}

\begin{frame}\frametitle{An Example Ontology Language}
Built-in types and values may be part of the expressions
%see syntax of BOL in the lecture notes
\begin{commgrammar}
\gcomment{Basic types and values}\\
\gprod{T}{\itg \bnfalt \float \bnfalt \bool \bnfalt \strg}{types}\\
\gprod{V}{\text{(omitted)}}{values}
\end{commgrammar}
\end{frame}

\begin{frame}\frametitle{A Real-Life Ontology Language}
See online resources for OWL.

Some specialties:
\begin{itemize}
\item Slightly different names than in BOL
\item No strict distinction between individuals, concepts, relations - just resources
\item Some special axioms, e.g., to make relations transitive
\item Multiple sublanguages with varying expressivity/implementability: Lite, DL, Full
\end{itemize}

BOL vs. OWL:
\begin{itemize}
\item BOL is simpler, more systematically structured \glec{good for teaching, prototypes}
\item OWL is the standard \glec{the one to use for better or worse}
\end{itemize}
\end{frame}

\begin{frame}\frametitle{Exercise 1}
As a team, build an ontology for a university.

Using git, OWL, and WebProtege are good ways to start.
\bigskip

{\small (In WebProtege, set "suffix" to "user supplied name" in "New Entity Settings". Otherwise, it'll get messy when you share your ontology.)}
\end{frame}

\begin{frame}\frametitle{Example: The Common Sense Ontology}
\begin{blockitems}{Situation}
\item society uses one ontology for common sense knowledge
\item changes over time
\end{blockitems}
\lec{content relative to ontology: laws, regulations, etc.}

\begin{blockitems}{Special aspects}
\item unwritten
\item not actually fully agreed upon
\item sometimes subject to political debate
\item no formal ontology language good enough to capture practical nuances
\item many society members not comfortable with formal languages
\end{blockitems}
\lec{but still always exists implicitly}

Idea: see political proposals as ontology evolution
\end{frame}

