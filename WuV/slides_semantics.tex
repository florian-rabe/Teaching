\section{Kinds of Semantics}

\begin{frame}\frametitle{Recall}
Recall:

\begin{center}
\begin{tabular}{l|l}
Syntax & Data \\
\hline
Semantics & Knowledge
\end{tabular}
\end{center}

Representing
\begin{itemize}
\item syntax = formal language
\begin{itemize}
\item grammar  \glec{context-free part}
\item type system \glec{context-sensitive well-formedness}
\end{itemize}
\item data = words in the syntax
\begin{itemize}
\item set of vocabularies
\item set of typed expressions for each vocabulary
\end{itemize}
\item semantics = \alert{???}
\item knowledge = emergent property of having well-formed words with semantics
\end{itemize}
\end{frame}

\begin{frame}\frametitle{Semantics as Querying}
\begin{blockitems}{General Idea}
 \item Semantics answers questions about the syntax
 \item Based on the intended meaning of the syntax
 \item Specifies the meaning by giving the answers
\end{blockitems}

\begin{blockitems}{Special Cases}
 \item Deductive semantics: answers the question whether a formula is a theorem
 \item Computational semantics: answers the question what the result of a program is
 \item Narrative semantics: allows answering any question if we understand natural language
 \item Concrete data semantics: answers the question what objects with certain properties exist
\end{blockitems}
\end{frame}


\begin{frame}\frametitle{Semantics as Imperfect Modeling}
\begin{itemize}
 \item Actual meaning of real-world difficult to model
 \begin{itemize}
 \item practical argument: any practically interesting system has too many rules
  \glec{cf. physics, e.g., three-body problem already chaotic}
 \item theoretical argument: no language can fully model itself
  \glec{cf. G\"odel's incompleteness theorems}
 \end{itemize}
 \item Practical semantics is approximation of ideal semantics
 \item Use practical purpose as guide for defining semantics
  \begin{itemize}
   \item a set of questions that can be asked \glec{e.g., judgments or queries}
   \item a definition of what the answers are
  \end{itemize}
  \lec{restrict questions according to practical needs}
\end{itemize}
\end{frame}

\begin{frame}\frametitle{Aspects of Semantics}
\begin{itemize}
\item Documentation: answers given as text
\begin{itemize}
\item pro: easy to read for humans, critical to build intuitions
\item con: often ambiguous, contradictory, or incomplete
\end{itemize}
\item Specification: correct answers defined by rule system \glec{also called calculus or inference system}
\begin{itemize}
\item pro: good stepping stone between the other two levels
\item con: accomplishes the pros of neither of them
\end{itemize}
\item Implementation: answers computed by algorithm
\begin{itemize}
\item pro: easy to automate, critical for efficiency and scale
\item con: essentially impossible to understand or analyze
\end{itemize}
\item Unit testing: set of query/answer pairs
\begin{itemize}
\item pro: easy to write, automate
\item con: does not cover the whole semantics
\end{itemize}
\end{itemize}

Rule system is sweet spot to connect human- and machine-friendly definitions.
\end{frame}

\begin{frame}\frametitle{Relative Semantics by Translation}
Components:
\begin{itemize}
\item Two syntaxes
\begin{itemize}
\item object-language $l$ \glec{e.g., BOL}
\item meta-language $L$ \glec{e.g., SFOL, Scala, SQL, English}
\end{itemize}
\item Semantics of $L$ assumed fixed
 \glec{captures what we already know}
\item Semantics of $l$ by translation into $L$
\end{itemize}
\lec{semantics of $l$ \emph{relative} to to existing semantics of $L$}

Problem: just kicking the can?
\end{frame}

\begin{frame}\frametitle{Discussion of Semantics by Translation}
\begin{blockitems}{Advantages}
\item a few meta-languages yield semantics for many languages
\item easy to develop new languages
\item good connection between syntax and semantics via compositionality, substitution theorem
\end{blockitems}

\begin{blockitems}{Disadvantages}
\item does not solve the problem once and for all
\item impractical without implementation of semantics of meta-language
\item meta-languages typically much more expressive than needed for object-languages
\item translations can be difficult, error-prone
\end{blockitems}

Also needed: absolute semantics
\end{frame}

\begin{frame}\frametitle{Absolute vs. Relative Semantics}
Absolute = self-contained, no use of meta-language $L$

\begin{blockitems}{Get off the ground}
 \item semantics for a few important meta-languages
  \glec{e.g., FOL, assembly language, set theory}
 \item relative semantics for all other languages, e.g.,
  \begin{itemize}
  \item model theory: logic $\to$ set theory
  \item compilation: Scala $\to$ JVM $\to$ assembly
  \end{itemize}
\end{blockitems}

\begin{blockitems}{Redundant semantics}
 \item common to give
 \begin{itemize}
  \item relative and absolute semantics for same syntax
  \item multiple relative semantics
   \glec{translations to different aspects}
  \item sometimes even maybe multiple absolute ones
 \end{itemize}
 \item Allows understanding syntax from multiple perspectives
 \item Allows cross-checking \glec{show equivalence of two semantics}
\end{blockitems}
\end{frame}


\begin{frame}\frametitle{Example: Recall Syntax of Arithmetic Language}
Syntax: represented as formal grammar

\begin{commgrammar}
\gcomment{Numbers}\\
\gprod{N}{0\bnfalt 1}{literals}\\
\galtprod{N+N}{sum}\\
\galtprod{N*N}{product}\\
\gcomment{Formulas}\\
\gprod{F}{N\doteq N}{equality}\\
\galtprod{N\leq N}{ordering by size}\\
\end{commgrammar}

Implementation as inductive data type
\end{frame}

\begin{frame}\frametitle{Example: Absolute Semantics}
\begin{blockitems}{Represented as judgments defined by sets of rules}
\item unclear what judgments to use
\item here: computation $\der N \rewrites N$ and truth $\der F$
\end{blockitems}

For numbers $n$: Rules to normalize numbers into values
\[\rul{}{\der N+0\rewrites N} \tb \rul{}{\der N*0\rewrites 0}\tb \rul{}{\der N*1\rewrites N}\]
\[\rul{}{\der N*(R+S)\rewrites N*R+N*S}\]
and their commutative variants as well as
\[\rul{}{\der L+(M+N)\rewrites (L+M)+N}\]


For formulas $f$: rules to determine true formulas
\[\rul{}{\der N\doteq N} \tb \rul{}{\der 0\leq N} \tb \rul{\der L\leq M}{\der L+N\leq M+N}\]
\end{frame}

\begin{frame}\frametitle{Example: Absolute Semantics (2)}
Checking if an absolute semantics works as intended is hard.

Here: number rules allow
\begin{enumerate}
\item eliminating all cases where arguments of $*$ are $0$, $1$, or $+$; thus, no more $*$
\item eliminating all cases where arguments of $+$ are $0$
\item shift brackets of nested $+$ to the left
\item left: $0$ or $(\ldots(1+1)\ldots+1)$ --- isomorphic to natural numbers
\end{enumerate}
formula rules allow
\begin{enumerate}
\item concluding equality if identical normal forms
\item reducing $M+1\leq N+1$ to $M\leq N$, repeat until $0\leq N$
\end{enumerate}
\end{frame}


\begin{frame}\frametitle{Example: Relative Semantics}
Semantics: represented as translation into known language
\medskip

Problem: Need to choose a known language first\\
Here: unary numbers represented as strings

Built-in data (strings and booleans):
\begin{commgrammar}
%\gcomment{Strings}\\
\gprod{S}{""}{empty}\\
\galtprod{(\texttt{Unicode)}}{character sequence}\\
%\gcomment{Booleans}\\
\gprod{B}{\cn{true}}{truth}\\
\galtprod{\cn{false}}{falsity}\\
\end{commgrammar}

Built-in operations to work on the data:
\begin{itemize}
\item concatenation of strings $S\bbc \cn{conc}(S,S)$
\item replacing all occurrences of $c$ in $S_1$ with $S_2$ $S\bbc \cn{replace}(S_1,c,S_2)$
\item equality test: $B\bbc S_1==S_2$
\item prefix test: $B\bbc \cn{startsWith}(S_1,S_2)$
\end{itemize}
\end{frame}

\begin{frame}\frametitle{Example: Relative Semantics}
\begin{blockitems}{Represented as function from syntax to semantics}
\item mutually recursive, inductive functions for each non-terminal symbol
\item compositional: recursive call on immediate subterms of argument
\end{blockitems}

For numbers $n$: semantics $\sem{n}$ is a string
\begin{itemize}
\item $\sem{0}=""$
\item $\sem{1}="|"$
\item $\sem{m+n}=\cn{conc}(\sem{m},\sem{n})$
\item $\sem{m*n}=\cn{replace}(\sem{m},"|",\sem{n})$
\end{itemize}
\medskip

For formulas $f$: semantics $\sem{f}$ is a boolean
\begin{itemize}
\item $\sem{m\doteq n}=\sem{m}==\sem{n}$
\item $\sem{m\leq n}=\cn{startsWith}(\sem{n},\sem{m})$
\end{itemize}
\end{frame}

\begin{frame}\frametitle{Example: Equivalence of Semantics}
\begin{blockitems}{For formulas}
\item if $\der F$, then $\sem{F}=true$ 
 \glec{usually called \emph{soundness}}
\item if $\sem{F}=true$, then $\der F$
 \glec{usually called \emph{completeness}}
\end{blockitems}

\begin{blockitems}{For numbers}
\item $\der N\rewrites 0$ iff $\sem{N}=""$
\item $\der N\rewrites (\ldots(1+1)\ldots+1)$ iff $\sem{N}="|\ldots|"$
\end{blockitems}
\end{frame}


\section{Relative Semantics for BOL}

\begin{frame}\frametitle{Semantics of BOL}
\begin{center}
\begin{tabular}{lll}
Aspect & kind of semantic language & semantic language\\
\hline 
deduction & logic & SFOL \\
concretization & database language & SQL \\
computation & programming language & Scala \\
narration & natural language & English \\
\end{tabular}
\end{center}
\end{frame}

\begin{frame}\frametitle{Narrative Semantics of BOL in English}
We discussed earlier
\begin{itemize}
\item the rough design of how natural languages can be seen as a formal systems
  \begin{itemize}
  \item grammar book = syntax
  \item sentences = formulas
  \item dictionary + common sense statements = standard library
  \item domain-specific dictionary + sentences = vocabulary
  \end{itemize}
\item the translation from BOL to it
\item the non-compositional aspects of natural language
\end{itemize}
see details in the lecture notes
\end{frame}

\begin{frame}\frametitle{Deductive Semantics of BOL in SFOL}
We discuss
\begin{itemize}
\item the grammar of SFOL
\item context-sensitive languages with variable binding (of which SFOL is an example)
\item an implementation of SFOL in Scala
\item the translation from BOL to SFOL
\item compositionality of the translation
\item the issue of
 \begin{itemize}
 \item non-compositionality
 \item the need for a semantic prefix
 \end{itemize}
\end{itemize}
\end{frame}

\begin{frame}\frametitle{Deductive Semantics of BOL in SFOL: Translation}
\begin{tabular}{l|l}
BOL Syntax $X$ & Semantics $\sem{X}$ in SFOL\\
\hline
\hline
ontology & SFOL theory \\
$D_1,\ldots,D_n$ & $\kw{type}\,\iota,\,\sem{D_1},\ldots,\sem{D_n}$ \\
\hline
BOL declaration & SFOL declaration \\
\kw{individual}\,$i$ & nullary function symbol $i:\iota$ \\
\kw{concept}\,$c$  & unary predicate symbol $c\sq\iota$ \\
\kw{relation}\,$r$ & binary predicate symbol $r\sq\iota\times \iota$ \\
\kw{property}\,$p:T$ & binary predicate symbol $p\sq\iota\times T$ \\
\kw{axiom}\, $F$ & axiom $\sem{F}$\\
\hline
Formula & Formula without free variables\\
$C_1 \Equiv C_2$ & $\forall x:\iota.\sem{C_1}(x)\Leftrightarrow \sem{C_2}(x)$\\
$C_1 \sqsubseteq C_2$ & $\forall x:\iota.\sem{C_1}(x)\impl \sem{C_2}(x)$\\
$I\; \texttt{is-a}\; C$ & $\sem{C}(\sem{I})$\\
$I_1\; R\; I_2$ & $\sem{R}(\sem{I_1},\sem{I_2})$\\
$I\; P\; V$ & $\sem{P}(\sem{I},\sem{V})$\\
\hline
Individual & Terms of type $\iota$ \\
$i$ & $i$ \\
\hline
\end{tabular}
\end{frame}

\begin{frame}\frametitle{Deductive Semantics of BOL in SFOL: Translation (2)}
\begin{tabular}{l|l}
BOL Syntax $X$ & Semantics $\sem{X}$ in SFOL\\
\hline
\hline
Concept & Formula taking term $x:\iota$\\
$c$ & $c(x)$\\
$\top$ & $\true$\\
$\bot$ & $\false$\\
$C_1 \sqcup C_2$ & $\sem{C_1}(x)\vee\sem{C_2}(x)$\\
$C_1 \sqcap C_2$ & $\sem{C_1}(x)\wedge\sem{C_2}(x)$\\
$\forall R.C$    & $\forall y:\iota.\sem{R}(x,y)\impl \sem{C}(y)$\\
$\exists R.C$    & $\exists y:\iota.\sem{R}(x,y)\wedge \sem{C}(y)$\\
$\dom\, R$ & $\exists y:\iota.\sem{R}(x,y)$\\
$\rng\, R$ & $\exists y:\iota.\sem{R}(y,x)$\\
$\dom\, P$ & $\exists y:T.\sem{P}(x,y)$  \tb($T$ is type of $P$)\\
\hline
\end{tabular}
\end{frame}

\begin{frame}\frametitle{Deductive Semantics of BOL in SFOL: Translation (3)}
\begin{tabular}{l|l}
BOL Syntax $X$ & Semantics $\sem{X}$ in SFOL\\
\hline
\hline
Relation & Formula taking terms $x:\iota,y:\iota$\\
$r$ & $r(x,y)$\\
$R_1 \cup R_2$ & $\sem{R_1}(x,y)\vee \sem{R_2}(x,y)$\\
$R_1 \cap R_2$ & $\sem{R_1}(x,y)\wedge \sem{R_2}(x,y)$\\
$R_1 ; R_2$ & $\exists m:\iota.\sem{R_1}(x,m)\wedge \sem{R_2}(m,y)$\\
$R^{-1}$          & $\sem{R}(y,x)$\\
$R^*$          & (tricky, omitted)\\
$\Delta_C$     & $x\doteq y\wedge \sem{C}(x)$\\
\hline
Property of type $T$ & Formula taking terms $x:\iota,y:T$\\
$p$ & $p(x,y)$\\
\hline
\end{tabular}
\end{frame}


%\begin{frame}\frametitle{Example: SFOL Morphisms}
%Morphisms can be defined in the same way for every language.
%
%In an SFOL morphism $m:V\to W$, we assign as follows:
%\begin{itemize}
%\item type symbols $\kw{type}\, y$ to type expressions $Y$
%\item function symbols $\kw{fun}\, f:Y_1 \ldots Y_n \to Y$ to term expressions of type $m(Y)$ with input types $m(Y_1)\ldots m(Y_n)$
%\item predicate symbols $\kw{pred}\, p\sq Y_1 \ldots Y_n$ to formula expressions with input types $m(Y_1)\ldots m(Y_n)$
%\item axioms stating $F$ to proofs of $m(F)$
%\end{itemize}
%
%One subtle issue:
%\begin{itemize}
%\item Problem: SFOL has function/predicate symbols with inputs but no term/formula expressions with inputs
%\item Solution: context represents inputs:
%\begin{itemize}
%\item $f\mapsto [\Gamma] T$ where $\Gamma\vdash_{W} T:m(Y)$
%\item $p\mapsto [\Gamma] F$ where $\Gamma\vdash_{W} F:Formula$
%\end{itemize}
%where $\Gamma=x_1:m(Y_1),\;\ldots\;x_n:m(Y_n)$
%\end{itemize}
%\end{frame}

%\begin{frame}\frametitle{Concrete Data Semantics of BOL in SQL}
%We discuss
%\begin{itemize}
%\item the grammar of an SQL-like language
%\item the translation from BOL to it
%\item the limitations of the translation
% \begin{itemize}
% \item partiality, e.g., for concept assertions where the concept is not atomic
% \item lack of formal axioms in SQL
% \end{itemize}
%\end{itemize}
%see details in the lecture notes
%\end{frame}
%
%\begin{frame}\frametitle{Computational Semantics of BOL in Scala}
%We discuss
%\begin{itemize}
%\item the grammar of a Scala-like language
%\item the translation from BOL to it
%\item the limitations of the translation
%\end{itemize}
%see details in the lecture notes
%\end{frame}

\section{Relative Semantics by Translation}

\begin{frame}\frametitle{General Definition}
A semantics by translation consists of
\begin{itemize}
 \item syntax: a formal system $l$
 \item semantic language: a formal system $L$
  \glec{different or same aspect as $l$}
 \item semantic prefix: a vocabulary $P$ in $L$
  \glec{formalizes fundamentals that are needed to represent $l$-objects}
 \item interpretation: translates every $l$-vocabulary $T$ to an $L$-vocabulary $P,\sem{T}$
\end{itemize}
\end{frame}

\begin{frame}\frametitle{Common Principles}
Properties shared by all semantics by translation
\lec{not part of formal definition, but best practices}
\begin{itemize}
 \item $l$-declaration translated to $L$-declaration for the same name
 \item vocabularies translated declaration-wise
 \item one inductive function for every kind of complex $l$-expression
  \begin{itemize}
   \item individuals, concepts, relations, properties, formulas
   \item maps $l$-expressions to $L$-expressions
  \end{itemize}
 \item atomic cases (base cases): $l$-identifier translated to $L$-identifier of the same name
  \glec{or something very similar}
 \item complex cases (step cases): compositional
\end{itemize}
\end{frame}

\begin{frame}\frametitle{Compositionality}
Case for operator $*$ in translation function compositional iff \\
interpretation of $*(e_1,\ldots,e_n)$ only depends on on the interpretation of the $e_i$

\[\sem{*(e_1,\ldots,e_n)}=\sem{*}(\sem{e_1},\ldots,\sem{e_n})\]
for some function $\sem{*}$
\bigskip

Example: $;$-operator of BOL in translation to FOL
\begin{itemize}
 \item translation: $\sem{R_1 ; R_2}= \exists m:\iota.\sem{R_1}(x,m)\wedge \sem{R_2}(m,y)$
 \item special case of the above via
  \begin{itemize}
  \item $*=;$
  \item $n=2$
  \item $\sem{;}=(p_1,p_2)\mapsto \exists m:\iota.p_1(x,m)\wedge p_2(m,y)$
  \end{itemize}
 \item Indeed, we have $\sem{R_1;R_2}=\sem{;}(\sem{R_1},\sem{R_2})$
\end{itemize}
\end{frame}


\begin{frame}\frametitle{Compositionality (2)}
Translation compositional iff
\begin{itemize}
\item one translation function for each non-terminal
 \glec{all written $\sem{-}$}
\item each defined by one induction on syntax
 \glec{i.e., one case for production}
 \glec{mutually recursive}
\item all cases compositional
\end{itemize}
\bigskip

Substitution theorem: a compositional translation satisfies
\[\sem{E(e_1,\ldots,e_n)}=\sem{E}(\sem{e_1},\ldots,\sem{e_n})\]
for
\begin{itemize}
\item every expression $E(N_1,\ldots,N_n)$ with non-terminals $N_i$
\item some function $\sem{E}$ that only depends on $E$
\end{itemize}
\end{frame}


\begin{frame}\frametitle{Compositionality (3)}
\[\sem{E(e_1,\ldots,e_n)}=\sem{E}(\sem{e_1},\ldots,\sem{e_n})\]
for every expression $E(N_1,\ldots,N_n)$ with non-terminals $N_i$
\bigskip

Now think of
\begin{itemize}
\item variable $x_i$ of type $N_i$ instead of non-terminal $N_i$
\item $E(x_1,\ldots,x_n)$ as expression with free variables $x_i$ of type $N_i$
\item expressions $e$ derived from $N$ as expressions of type $N$
\item $E(e_1,\ldots,e_n)$ as result of substituting $e_i$ for $x_i$
\item $\sem{E}(x_1,\ldots,x_n)$ as (semantic) expression with free variables $x_i$
\end{itemize}

Then both sides of equations act on $E(x_1,\ldots,x_n)$:
\begin{itemize}
\item left side yields $\sem{E(e_1,\ldots,e_n)}$ by
\begin{itemize}
\item first substitution $e_i$ for $x_i$
\item then semantics $\sem{-}$ of the whole
\end{itemize}
\item right side yields $\sem{E}(\sem{e_1},\ldots,\sem{e_n})$ by
\begin{itemize}
\item first semantics $\sem{-}$ of all parts
\item then substitution $\sem{e_i}$ for $x_i$
\end{itemize}
\end{itemize}
\lec{semantics commutes with substitution}
\end{frame}

\begin{frame}\frametitle{Non-Compositionality}
\begin{blockitems}{Examples}
 \item deduction: cut elimination, translation from natural deduction to Hilbert calculus
 \item computation: optimizing compiler, e.g., loop unrolling
 \item concretization: query optimization, e.g., turning a WHERE of a join into a join of WHEREs,
 \item narration: ambiguous words are translated based on context
\end{blockitems}

\begin{blockitems}{Typical sources}
 \item subcases in a case of translation function
  \begin{itemize}
  \item based on inspecting the arguments, e.g., subinduction
  \item based on context
  \end{itemize}
 \item custom-built semantic prefix
\end{blockitems}
\end{frame}

\begin{frame}\frametitle{Translation vs. Embedding}
\begin{blockitems}{Translation}
\item as above, $l$ and $L$ are at the same level
\item $l$-declarations represented as $L$-declarations
\glec{also called shallow embedding}
\end{blockitems}

\begin{blockitems}{Embedding}
\item $L$ is used as meta-language to represent $l$
 \glec{e.g., $L$ is programming language to implement $l$}
\item $l$-declarations represented as $L$-objects using an inductive type
\glec{also called deep embedding}
\end{blockitems}
\end{frame}

\begin{frame}\frametitle{Exercise 8}
Implement the translation from BOL to SFOL.
Translate your university ontology.

Formulate a BOL-conjecture and prove it by sending the translated version to an SFOL theorem prover.
\end{frame}

\section{Denotational Semantics}

\begin{frame}\frametitle{Relative Semantics: Denotation vs. Translation}
\begin{blockitems}{Translation semantics}
\item vocabularies mapped to vocabularies, expressions to expressions
\item symbols in input vocabulary yield symbols in output vocabulary
\item examples:
 \begin{itemize}
 \item BOL to SFOL
 \item compiling a programming language to another language
 \end{itemize}
\end{blockitems}

\begin{blockitems}{Denotational semantics}
\item symbols in vocabulary given concrete value
 \glec{value/meaning/denotation/interpretation}
\item model: maps every symbols to its concrete value
\item expressions interpreted relative to fixed situation
\item examples
 \begin{itemize}
 \item interpreting a program \glec{situation = input+run-time environment}
 \item interpreting logical formulas in a model \glec{situation = model}
 \end{itemize}
\end{blockitems}
\end{frame}

\begin{frame}\frametitle{Interpreted vs. Uninterpreted Symbols}
\begin{blockitems}{Interpreted = symbols with fixed semantics}
\item base types and their operations \glec{integer, etc.}
\item concrete data types \glec{enumerations, inductive types}
\end{blockitems}
\glec{semantics fixed by language}

\begin{blockitems}{Uninterpreted = semantics open to interpretation}
\item e.g, $a:\TYPE,\;f:a\to a$, etc.
\item axioms/definitions constrain/specify possible interpretations
\end{blockitems}
\glec{semantics constrained by vocabulary, fixed by situation}

\begin{blockitems}{Relative to situation}
\item semantics of all symbols fixed
\item semantics of every expression can be determined
 \glec{often but not necessarily computable}
\end{blockitems}
\end{frame}

\begin{frame}\frametitle{Example}
\begin{blockitems}{Vocabulary: rules of the world}
\item type $a$
\item operation $f:a\to a$
\item relations $r:a\to \prop$
\item axioms $F$ about $f,r$
\end{blockitems}

\begin{blockitems}{Situation: one concrete world $S$}
\item specific set $a^S$
\item specific function $f^S$ from $a^S$ to $a^S$
\item specific subset $r^S$ of $a^S$
\item proof that $F$ holds about $f^S$, $r^S$
\end{blockitems}

\begin{center}
\begin{tabular}{l|ll}
Aspect & Abstract Vocabulary & Situation \\
\hline
ontology & TBox & initial situation through ABox \\
data & schema & database \\
deduction & theory & model \\
computation & program & environment \\
narration & dictionary & domain-specific definitions \\
\end{tabular}
\end{center}
\end{frame}

\begin{frame}\frametitle{Terminology}
``vocabulary'' and ``situation'' are not standard names.
They are introduced here to unify the different kinds of languages.

The standard names vary by knowledge aspect:
\begin{center}
\begin{tabular}{l|lll}
Aspect & Vocabulary & Situation & Vocabulary+Situation\\
\hline
ontology & TBox & ABox & ontology \\
data & schema & database & SQL dump\\
deduction & theory & model & concrete theory\\
computation & program & environment & execution\\
narration & dictionary & technical jargon & \\
\end{tabular}
\end{center}
\end{frame}

\begin{frame}\frametitle{Approaches to Uninterpreted Symbols}
\begin{blockitems}{Usually only one extreme}
\item Computation: typically mostly interpreted symbols, situation only provides input/environment
 \glec{programs can be run directly}
\item Deduction: typically only uninterpreted symbols
 \glec{focus on studying the possible models}
\end{blockitems}

\begin{blockitems}{Some attempts at combining}
\item reasoning about programs \glec{e.g., functions with pre-/postconditions}
\item logics with built-in base types \glec{e.g., SMT solving}
\end{blockitems}
\end{frame}

\begin{frame}\frametitle{Initial Semantics}
\begin{blockitems}{Some situations can be captured in vocabulary}
\item ABox part of ontology 
\item schema and table entries part of SQL syntax
\item trickier when vocabulary symbols represent abstract sets/functions
 \glec{usually meta-language needed to define concrete semantics}
\end{blockitems}

\begin{blockitems}{Any vocabulary induces default situation}
\item inhabitants of a type are exactly the terms of that type
\item functions map exactly as given by axioms
\item every situation must be an extension
\item called initial situation, or initial semantics
\item examples:
\begin{itemize}
\item ABox, database tables if part of vocabulary
\item Herbrand model of logical theory
\end{itemize}
\end{blockitems}
\end{frame}

\section{Absolute Semantics for BOL}

\begin{frame}\frametitle{Judgments}
Goal: Answer the question whether a formula is a theorem.

Use deduction judgment: \[\Gamma\vdash^{BOL}_V F\]
for formula $F$

Notation:
\begin{itemize}
\item We drop the superscript $^{BOL}$ whenever clear.
\item We drop the subscript $_V$ whenever clear.
\item We drop the context $\Gamma$ if it is empty.
\end{itemize}
\end{frame}

\begin{frame}\frametitle{Lookup Rules}
The main rules that need to access the vocabulary:
\[\rul{f\minn V}{\vdash_V f}\]
\glec{for assertions or axioms f}
\medskip

Assumptions in the context are looked up accordingly:
\[\rul{x:f\minn \Gamma}{\Gamma\vdash f}\]
\end{frame}

\begin{frame}\frametitle{Rules for Subsumption and Equality}
Subsumption is an order with respect to equality:
\[\rul{}{\vdash c\sqsubseteq c}\]

\[\rul{\vdash c\sqsubseteq d \tb \vdash d\sqsubseteq e}{\vdash c\sqsubseteq e}\]

\[\rul{\vdash c\sqsubseteq d \tb \vdash d\sqsubseteq c}{\vdash c\Equiv d}\]

Equal concepts can be substituted for each other:
\[\rul{\vdash c\Equiv d\tb x:C\vdash f(x):\prop \tb \vdash f(c)}{\vdash f(d)}\]

\glec{This completely defines equality.}
\end{frame}

\begin{frame}\frametitle{Rules relating Instancehood and Subsumption}
\[\rul{\vdash i\isa c \tb \vdash c\sqsubseteq d}{\vdash i\isa d}\]
Read:
\begin{itemize}
\item if
 \begin{itemize}
 \item $i\isa c$
 \item $c\sqsubseteq d$
 \end{itemize}
\item then $i\isa d$
\end{itemize}

\[\rul{x:I,\,x\isa c\vdash x\isa d}{\vdash c\sqsubseteq d}\]
Read:
\begin{itemize}
\item if
 \begin{itemize}
 \item assuming an individual $x$ and $x\isa c$, then $x\isa d$
 \end{itemize}
\item then $c\sqsubseteq d$
\end{itemize}
\end{frame}

\begin{frame}\frametitle{Induction}
Consider from before
\[\rul{x:I,\,x\isa c\vdash x\isa d}{\vdash c\sqsubseteq d}\]

Question: Do we allow proving the hypothesis by checking for each individual $x$?
 \lec{induction}
\begin{itemize}
\item<2-> Open world: no
\item<3-> Closed world: yes
 \[\rul{\Gamma[x=i]\vdash f[x=i] \;\text{ for every individual } i}{\Gamma, x:I\vdash f(x)}\]
 \glec{effectively applicable if only finitely many individuals}
\end{itemize}
\end{frame}

\begin{frame}\frametitle{Rules for Union and Intersection of Concepts}
Union as the least upper bound:
\[\rul{}{\vdash c\sqsubseteq c\sqcup d} \tb\tb \rul{}{\vdash d\sqsubseteq c\sqcup d }\]
\[\rul{\vdash c\sqsubseteq h \tb \vdash d\sqsubseteq h}{\vdash c\sqcup d \sqsubseteq h}\]
\medskip

Dually, intersection as the greatest lower bound:
\[\rul{}{\vdash c\sqcap d\sqsubseteq c} \tb\tb \rul{}{\vdash c\sqcap d\sqsubseteq d}\]
\[\rul{\vdash h\sqsubseteq c \tb \vdash h\sqsubseteq d}{\vdash h \sqsubseteq c\sqcap d}\]
\end{frame}

\begin{frame}\frametitle{Rules for Existential and Universal}
Easy rules:
\begin{itemize}
\item Existential
\[\rul{\vdash i\,r\,j \tb \vdash j\isa c}{\vdash i \isa \exists r.c }\]
\item Universal
\[\rul{\vdash i \isa \forall r.c \tb \vdash i\,r\,j}{\vdash j\isa c}\]
\end{itemize}

Other directions are trickier:

\begin{itemize}
\item Existential
\[\rul{\vdash i \isa \exists r.c \tb j:I,\;i\,r\,j,\;j\isa c\vdash f}{\vdash f}\]
\item Universal
\[\rul{j:I,\; i\,r\,j\vdash j\isa c}{\vdash i\isa\forall r.c}\]
\end{itemize}
\end{frame}

\begin{frame}\frametitle{Selected Rules for Relations}
Inverse:
\[\rul{\vdash i \,r\,j}{\vdash j\,r^{-1}\,i}\]

Composition:
\[\rul{\vdash i \,r\,j \tb \vdash j\,s\,k}{\vdash i\,(r;s)\,k}\]

Transitive closure:
\[\rul{}{\vdash i \,r^*\,i} \tb \rul{\vdash i\,r\,j \tb \vdash j\,r^* k}{\vdash i\,r^* k}\]

Identity at concept $c$:
\[\rul{\vdash i \isa c}{\vdash i\,\Delta_c i}\]
\end{frame}

%\begin{frame}\frametitle{Exercise 10}
%Implement an absolute semantics for BOL for the judgment $\Gamma\vdash F$.
%
%This is too difficult in general. But it becomes doable if we choose a sufficiently simple fragment of BOL, e.g., only
%\begin{itemize}
%\item formulas: $I\mathtt{is-a} C$, $C\sqsubseteq C$
%\item concepts: identifiers, $C\sqcup C$, $\exists R.C$
%\item relations: identifiers
%\end{itemize}
%\end{frame}
%
%\begin{frame}\frametitle{Exercise 11}
%Implement an absolute concrete semantics for BOL.
%\end{frame}

%\begin{frame}\frametitle{Problems}
%Next
%\begin{itemize}
%\item four kinds of absolute semantics
% \lec{one per aspect}
%\item Each motivated by one kind of querying
%\item Each defines the aspect
% \glec{e.g., a logic is a language with deductive semantics}
%\end{itemize}
%
%Relation to previous slides
%\begin{itemize}
% \item before: querying via relative semantics
% \item just the special case where target language has corresponding absolute semantics
%  \glec{e.g., deductive querying possible given deductive semantics}
%  \glec{no matter if relative or absolute}
% \item conceptually, absolute semantics comes first, but easier to understand after querying
% \item discussed problems apply to absolute semantics accordingly
%\end{itemize}
%\end{frame}

