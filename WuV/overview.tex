\paragraph{Structure}

The subsequent \emph{parts} of this course follow the Tetrapod model with one part per aspect.
Each of these will describe the concepts, languages, and tools of the respective aspect as well as their relation to other aspects.

The aspects of the Tetrapod are typically handled in individual courses, which describe highly specialized languages and tools in depth.
On the contrary, the overall goal of this course will be seeing all of them as different approaches to semantics and knowledge representation.
The course will focus on universal principles and their commonalities and differences as well as their advantages and disadvantages.

The subsequent \emph{chapters} of this first part will be dedicated to aspect-independent material.
These will not necessarily be taught in the order in which they appear in these notes.
Instead, some of them will be discussed in connection to how they are relevant in individual aspects.

\paragraph{Exercises and Running Example}

Typical practical projects, e.g., the ones that a strong CS graduate might be put in charge of, involve heterogeneous data and knowledge that must be managed using a variety of optimized aspect-specific languages and tools.
Interoperability between these is often a major source of inefficiency and bugs.

The exercises accompanying the course will mimic this situation: they will be designed around a single large project that requires choosing and integrating methods, languages, and tools from all aspects.

Concretely, this project will be the development of a univis-like system for a university.
It will involve heterogeneous data such as course and program descriptions, legal texts, websites, grade tables, and transcript generation code.

Over the course of the semester students will implement a completely functional system applying the lessons of the course.
This is very unusual and often impossible for other courses: as any university course must teach many different things from a wide area, it is rarely possible to find a project that requires many and only lessons from a single course.
Here KRP is special because its material pervades all aspects of system development.

