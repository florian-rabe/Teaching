\section{Natural Language}

\begin{frame}\frametitle{Languages}
We give an overview of narrative languages as one corner of the Tetrapod.
\end{frame}

\begin{frame}\frametitle{Formal Languages for Natural Language}
\begin{blockitems}{Formal Languages}
\item Controlled natural languages: GF, \ldots,
\item Word processors: Microsoft Word, Libre Office Write, \ldots
\item Web-oriented languages: Markdown, HTML, \ldots
\end{blockitems}

\begin{blockitems}{Features}
\item Unrestricted or barely restricted natural language
\item Usually: visual presentation, e.g., fonts, colors
\item Possibly: formal structure, e.g.,
 \begin{itemize}
 \item sections
 \item lists, enumerations
 \item cross-references
 \end{itemize}
\item Rarely: explicit representation of ontological knowledge, e.g., definitions, references 
\end{blockitems}
\end{frame}

\begin{frame}\frametitle{Semantics by Translation}
\begin{blockitems}{Relative Semantics}
\item define semantics of a language $l$ relative to a language $L$ whose semantics is already fixed
\item by translation of $l$ into $L$
\end{blockitems}

\begin{blockitems}{Compositional Translation}
\item translation is given as context-free traversal of checked syntax tree
\item one function per non-terminal $N$
\item one case per production $N ::= R$
\item one recursive call per non-terminal in $R$
\end{blockitems}
\end{frame}

\begin{frame}\frametitle{A Narrative Semantics of BOL}
We give a semantics of BOL by translation into English according to Section 6.6 of the notes.
\end{frame}

\section{LaTeX}

\begin{frame}\frametitle{The sTeX Dialect of LaTeX}
LaTeX is a language for natural language documents.

sTeX enriches LaTeX to allow describing the ontology in addition to the narrative document.
In particular: definitions and references to them
\end{frame}

\begin{frame}\frametitle{Exercise 5}
Implement a semantics by translation for BOL that translates BOL to sTeX.

Concretely, your translation should take an ontology and a name N.
And it should return a file N.tex containing
\begin{itemize}
 \item for every named declaration: one English sentence with an sTeX declaration
 \item for every axiom: one English sentence in which all identifier references carry a hyperlink to the corresponding sTeX declaration
\end{itemize}

Run this on your university ontology to produce university.tex and university.pdf.
Write a second sTeX document containing a mission statement for the university in such a way that every mention of a concept carries a hyperlink to the respective declaration in university.pdf.
\end{frame}
