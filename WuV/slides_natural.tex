\section{Natural Language}

\begin{frame}\frametitle{Natural Language}
We give an overview of natural languages as narrative languages as one corner of the Tetrapod.
\end{frame}

\begin{frame}\frametitle{Semantics by Translation}
\begin{blockitems}{Relative Semantics}
\item define semantics of a language $l$ relative to a language $L$ whose semantics is already fixed
\item by translation of $l$ into $L$
\end{blockitems}

\begin{blockitems}{Compositional Translation}
\item translation is given as context-free traversal of checked syntax tree
\item one function per non-terminal $N$
\item one case per production $N ::= R$
\item one recursive call per non-terminal in $R$
\end{blockitems}
\end{frame}

\begin{frame}\frametitle{A Narrative Semantics of BOL}
We give a semantics of BOL by translation into English according to Section 6.6 of the notes.
\end{frame}

\section{LaTeX}

\begin{frame}\frametitle{The sTeX Dialect of LaTeX}
LaTeX is a language for natural language documents.

sTeX enriches LaTeX to allow describing the ontology in addition to the narrative document.
\end{frame}

\begin{frame}\frametitle{Exercise 5}
Implement a semantics by translation for BOL that translates BOL to sTeX.

Concretely, your translation should take an ontology and a name N.
And it should return a file N.tex containing
\begin{itemize}
 \item for every named declaration: one English sentence with an sTeX declaration
 \item for every axiom: one English sentence in which all identifier references carry a hyperlink to the corresponding sTeX declaration
\end{itemize}
\end{frame}
