\documentclass{book}

\usepackage{graphicx}
\usepackage{xkeyval}
\usepackage{multirow}
%\usepackage{bm} %% bold face math symbols
\usepackage{listings}
\lstset{mathescape}
\usepackage{../macros/mytikz}
%\usepackage{multicol}
\usepackage{stmaryrd}%\newcommand{\contra}{\lightning}
%\usepackage{rotating} \newcommand{\sw}[1]{\begin{sideways}#1\end{sideways}}
\usepackage[show]{ed}
%\usepackage{../macros/algorithm}
%\usepackage{ded}

\usepackage{../mylecturenotes}
\usepackage{../macros}
\usepackage{macros}
\usepackage{cleveref}

\title{Lectures Notes on Knowledge Representation and Processing}
\author{Florian Rabe and Michael Kohlhase}
\date{2020}

\begin{document}
\maketitle

\bigskip

These notes were originally prepared for our CS course at University Erlangen-Nuremberg (FAU) in Summer 2020.
They are directed at 3rd semester CS undergraduates and master students but should be intelligible even for earlier students and could be interesting also for PhD students and for students from adjacent majors.
The course is recommended both as a first course in the specialization area Artificial Intelligence as well as a one-off overview on on knowledge representation.

The course was developed in Summer 2020 from scratch and materials were built along the way.
It integrated current directions and recent results in research on knowledge representation pulling together materials in an entirely new and original way.

\tableofcontents

\newpage

%%%%%%%%%%%%%%%%%%%%%%%%%%%%%%%%%%%%%%%%%%%%%
\chapter{Meta-Remarks}
  \begin{center}
\textbf{Important stuff that you should read carefully!}
\end{center}

\paragraph{State of these notes}
I constantly work on my lecture notes.
Therefore, keep in mind that:
\begin{compactitem}
\item I am developing these notes in parallel with the lecture---they can grow or change throughout the semester.
\item These notes are neither a subset nor a superset of the material discussed in the lecture.
\item Unless mentioned otherwise, all material in these notes is exam-relevant (in addition to all material discussed in the lectures).
\end{compactitem}
\medskip

\paragraph{Collaboration on these notes}
I am writing these notes using LaTeX and storing them in a git repository on GitHub at \url{https://github.com/florian-rabe/Teaching}.
Familiarity with LaTeX as well as Git and GitHub is not part of this lecture. But it is essential skill for you.
Ask in the lecture if you have difficulty figuring it out on your own.
\medskip

As an experiment in teaching, I am inviting all of you to collaborate on these lecture notes with me.
\medskip

By forking and by submitting pull requests for this repository, you can suggest changes to these notes.
For example, you are encouraged to:
\begin{compactitem}
\item Fix typos and other errors.
\item Add examples and diagrams that I develop on the board during lectures.
\item Add solutions for the homeworks if I did not provide any (of course, I will only integrate solutions after the deadline).
\item Add additional examples, exercises, or explanations that you came up or found in other sources.
 If you use material from other sources (e.g., by copying an diagram from some website), make sure that you have the license to use it and that you acknowledge sources appropriately!
\end{compactitem}
The TAs and I will review and approve or reject the changes.
If you make substantial contributions, I will list you as a contributor (i.e., something you can put in your CV).
\medskip

Any improvement you make will not only help your fellow students, it will also increase your own understanding of the material.
Therefore, I can give you up to $10\%$ bonus credit for such contributions.
(Make sure your git commits carry a user name that I can connect to you.)
Because this is an experiment, I will have to figure out the details along the way.

\paragraph{Other Advice}
I maintain a list of useful advice for students at \url{https://svn.kwarc.info/repos/frabe/Teaching/general/advice_for_students.pdf}.
It is mostly targeted at older students who work in individual projects with me (e.g., students who work on their BSc thesis).
But much of it is useful for you already now or will become useful soon.
So have a look.

\paragraph{Skipped Chapters}
These notes were originally prepared for my 2nd semester CS course at Jacobs University in Spring 2017.
In that course, the following chapters were skipped or treated only very superficially: \ref{sec:ad:finiteds}, \ref{sec:ad:numbers}, \ref{sec:ad:option}, \ref{sec:ad:functions}, \ref{sec:ad:unions}, \ref{sec:ad:parallel}, \ref{sec:ad:prot}, \ref{sec:ad:random}, \ref{sec:ad:quantum}.
In the other chapters, almost all material was covered; only a few subsections were skipped.

\chapter{Fundamental Concepts}\label{sec:wuv:concepts}
      % correctness, dependability, quality, safety, security, privacy
    % Avizienis et al - Fundamental Concepts of Dependability (2001)


\chapter{Overview of This Course}
  \paragraph{Structure}

The subsequent \emph{parts} of this course follow the Tetrapod model with one part per aspect.
Each of these will describe the concepts, languages, and tools of the respective aspect as well as their relation to other aspects.

The aspects of the Tetrapod are typically handled in individual courses, which describe highly specialized languages and tools in depth.
On the contrary, the overall goal of this course will be seeing all of them as different approaches to semantics and knowledge representation.
The course will focus on universal principles and their commonalities and differences as well as their advantages and disadvantages.

The subsequent \emph{chapters} of this first part will be dedicated to aspect-independent material.
These will not necessarily be taught in the order in which they appear in these notes.
Instead, some of them will be discussed in connection to how they are relevant in individual aspects.

\paragraph{Exercises and Running Example}

Typical practical projects, e.g., the ones that a strong CS graduate might be put in charge of, involve heterogeneous data and knowledge that must be managed using a variety of optimized aspect-specific languages and tools.
Interoperability between these is often a major source of inefficiency and bugs.

The exercises accompanying the course will mimic this situation: they will be designed around a single large project that requires choosing and integrating methods, languages, and tools from all aspects.

Concretely, this project will be the development of a univis-like system for a university.
It will involve heterogeneous data such as course and program descriptions, legal texts, websites, grade tables, and transcript generation code.

Over the course of the semester students will implement a completely functional system applying the lessons of the course.
This is very unusual and often impossible for other courses: as any university course must teach many different things from a wide area, it is rarely possible to find a project that requires many and only lessons from a single course.
Here KRP is special because its material pervades all aspects of system development.



\chapter{Representing Syntax and Semantics}\label{sec:wuv:syntax}
%\footnote{This chapter is grouped with the general parts to whom it belongs conceptually. But in the lecture it was treated later.}
  \section{Context-Free Grammars}

\section{Inductive Data Types}

\section{Semantics as a Recursive Function}

\section{Context-Sensitive Syntax}



\chapter{Encoding Data}\label{sec:wuv:codecs}
%\footnote{This chapter is grouped with the general parts to whom it belongs conceptually. But in the lecture it was treated after the parts on ontological and concretized knowledge, which motivate it.}
  \section{Data Representation Languages}

\section{Typed Data}

\section{Encoding Typed Data in Untyped Representation Languages}

\chapter{Ontologies}
 \section{General Principles}

\paragraph{Motivation}
An ontology is an abstract representation of the main concepts in some domain.
Here \emph{domain} refers to any area of the real world such as mathematics, biology, diseases and medications, human relationships, etc.
Many examples can be found at \url{https://bioportal.bioontology.org/}, including the Gene ontology one of the biggest.

Contrary to the other four aspects, ontological knowledge representations do not aim at capturing the entire semantics of the domain objects.
Instead, they focus on defining unique identifiers for the those objects and describing some of their properties and relations to each other.

We use the word \textbf{ontologization} to refer to the process of organizing the knowledge of a domain in ontologies.

Ontologies are most valuable when they are \emph{standardized} (either sanctioned through a formal body or a quasi-standard because everyone uses it).
A standard ontology allows everybody in the domain to use the identifiers defined by the ontology in a way that avoids misunderstandings.
Thus, in the simplest form, an ontology can be seen as a dictionary defining the technical terms of a domain.
For example, the Gene ontology defines identifier \texttt{GO:0000001} to have the formal name "mitochondrion inheritance" and the informal definition "The distribution of mitochondria, including the mitochondrial genome, into daughter cells after mitosis or meiosis, mediated by interactions between mitochondria and the cytoskeleton.".

\paragraph{Ontology Languages}
An ontology is written in \textbf{ontology language}.
Common ontology languages are
\begin{compactitem}
 \item description logics such as ALC,
 \item the W3C ontology language OWL, which is the standard ontology languages of the semantic web,
 \item the entity-relationship model, which focuses on modeling rather than formal syntax,
 \item modeling languages like UML, which is the main ontology language used in software engineering.
\end{compactitem}

Ontology languages are not committed to a particular domain --- in the Tetrapod model, they correspond to programming languages and logics, which are similarly uncommitted.
Instead, an ontology language is a formal language that standardizes the syntax of how ontologies can be written as well as their semantics.

\paragraph{Ontologies}
The details of the syntax vary between ontology languages.
But as a general rule, every \textbf{ontology} declares
\begin{compactitem}
 \item \textbf{individual} --- concrete objects that exist in the real world, e.g., "Florian Rabe" or "WuV"
 \item \textbf{concept} --- abstract groups of individuals, e.g., "instructor" or "course"
 \item \textbf{relation} --- binary relations between two individuals, e.g., "teaches"
 \item \textbf{properties} --- binary relations between an individuals and a concrete value (a number, a date, etc.), e.g., "has-credits"
 \item \textbf{concept assertions} --- the statement that a particular individual is an instance of a particular concept
 \item \textbf{relation assertions} --- the statement that a particular relation holds about two individuals
 \item \textbf{property assertions} --- the statement that a particular individual has a particular value for a particular property
 \item \textbf{axioms} --- statements about relations between concepts, typically in the form subconcept of statements like "instructor" $\sqsubseteq$ "person"
\end{compactitem}

All assertions can be understood and spoken as subject-predicate-object \textbf{triples} as follows:
\begin{center}
\begin{tabular}{l|lll}
Assertion & \multicolumn{3}{c}{Triple} \\
          & Subject & Predicate & Object \\
\hline
concept assertion  & "Florian Rabe" & \texttt{is-a} & "instructor" \\
relation assertion & "Florian Rabe" & "teaches" & "WuV" \\
property assertion & "WuV" & "has credits" & 7.5 \\
\end{tabular}
\end{center}
This uses a special relation \texttt{is-a} between individuals and concepts.
Some languages group \texttt{is-a} with the other binary relations between individuals for simplicity although it is technically a little different.

The possible values of properties must be fixed by the ontology language.
Typically, it includes at least standard types such as integers, floating point numbers, and strings.
But arbitrary extensions are possible such as dates, RGB-colors, lists, etc.
In advanced languages, it is possible that the ontology even introduces its own basic types and values.

Ontologies are often divided into two parts:
\begin{compactitem}
 \item The \textbf{abstract} part contains everything that holds in general independent of which individuals: concepts, relations, properties, and axioms.
 It describes the general rules how the worlds works without committing to a particular set of inhabitants of the world.
 This part is commonly called the \textbf{TBox} (T for terminological).
 \item The \textbf{concrete} part contains everything that depends on the choice of individuals: individuals and assertions.
 It populates the world with inhabitants.
 This part is commonly called the \textbf{ABox} (A for assertional).
\end{compactitem}

A separate division into two parts is the following:
\begin{compactitem}
 \item The \textbf{signature} part contains everything that introduces a \textbf{named entity}: individuals, concepts, relations, and properties.
 \item The \textbf{theory} part contains everything that describes which statements about the named entities are true: assertions and axioms.
\end{compactitem}


\paragraph{Synonyms}
Because these principles pervade all formal languages, many competing synonyms are used in different domains.
Common synonyms are:
\begin{center}
\begin{tabular}{l|llll|l}
 Here       & OWL      & Description logics & ER model & UML & semantics via logics\\
\hline
 individual & instance & individual & entity & object, instance & constant\\
 concept    & class    & concept &  entity-type & class & unary predicate\\
 relation   & object property & role & role & association & binary predicate \\
 property   & data property   & (not common) & attribute & field of base type & binary predicate\\
\end{tabular}
\end{center}

In particular, the individual-concept relation occurs everywhere and is known under many names:
\begin{center}
\begin{tabular}{l|ll}
 domain & individual & concept \\
\hline
type theory, logic & constant, term & type \\
set theory  & element & set \\
database    & row & table \\
philosophy\footnote{as in \url{https://plato.stanford.edu/entries/object/}} & object & property \\
grammar & proper noun & common noun \\
\end{tabular}
\end{center}

%%%%%%%%%%%%%%%%%%%%%%%%%%%%%%%%%%%%%%%%%%%%%
\section{A Basic Ontology Language}

\begin{figure}[hbt]
\begin{commgrammar}
\gcomment{Ontologies}\\
\gprod{O}{\rep{D}}{}\\
\gcomment{Declarations}\\
\gprod{D}{\kw{individual}\; \ID}{atomic individual}\\
\galtprod{\kw{concept}\; \ID}{atomic concept}\\
\galtprod{\kw{relation}\; \ID}{atomic relation}\\
\galtprod{\kw{property}\; \ID: T}{atomic property}\\
\galtprod{I\; \texttt{is-a}\; C}{concept assertion}\\
\galtprod{I\; R\; I}{relation assertion}\\
\galtprod{I\; P\; V}{property assertion}\\
\galtprod{F}{other axioms}\\
\gcomment{Formulas}\\
\gprod{F}{C \Equiv C}{concept equality}\\
\galtprod{C \sqsubseteq C}{concept subsumption}\\
\gcomment{Individual expressions}\\
\gprod{I}{\ID}{atomic individuals}\\
\gcomment{Concept expressions}\\
\gprod{C}{\ID}{atomic concepts}\\
\galtprod{C \sqcup C}{union of concepts}\\
\galtprod{C \sqcap C}{intersection of concepts}\\
\galtprod{\forall R.C}{universal relativization}\\
\galtprod{\exists R.C}{existential relativization}\\
\galtprod{\dom R}{domain of a relation}\\
\galtprod{\rng R}{range of a relation}\\
\gcomment{Relation expressions}\\
\gprod{R}{\ID}{atomic relations}\\
\galtprod{R \cup R}{union of relations}\\
\galtprod{R \cap R}{intersection of relations}\\
\galtprod{R ; R}{composition of relations}\\
\galtprod{R^*}{transitive closure of a relation}\\
\galtprod{R^{-1}}{dual relation}\\
\galtprod{\Delta_C}{identity relation of a concept}\\
\gcomment{Property expressions}\\
\gprod{P}{\ID}{atomic properties}\\
\gcomment{Identifiers}\\
\gprod{\ID}{\text{alphanumeric string}}{}\\
\gcomment{Basic types and values}\\
\gprod{T}{\itg \alt \float \alt \bool \alt \strg}{types}\\
\gprod{T}{\text{(omitted)}}{values}
\end{commgrammar}
\caption{Grammar of BOL}\label{fig:bol}
\end{figure}

\clearpage

We could study practical ontology languages like ALC or OWL now.
But those feature a lot of other details that can block the view onto the essential parts.
Therefore, we first define a basic ontology language ourselves in order to have full control over the details.

\subsection{Syntax}

\begin{definition}[Syntax of BOL]
A BOL-ontology is given by the grammar in Fig.~\ref{fig:bol}.
It is well-formed if
\begin{compactitem}
 \item no identifier is declared twice,
 \item every property assertion assigns a value of the type required by the property declaration,
 \item every reference to an atomic individual/concept/relation/property is declared as such.
\end{compactitem}
\end{definition}

The above grammar exhibits some general structure that we find throughout formal KR languages.
In particular, an ontology consists of \textbf{named declarations} of four different kinds of entities as well as some assertions and axioms about them.
Each entity declaration clarifies which kind it is (in our case by starting with a keyword) and introduces a new entity identifier.
For each kind, there are complex expressions.
These are anonymous and built inductively; their base cases are references to the corresponding identifiers.
Sometimes (in our case: individuals and properties), the references are the only expressions of the kind.
Sometimes (in our case: concepts and relations), there can be many productions for complex expressions.
The complex expressions are used to build axioms; in our case, these are the three kinds of assertions and other formulas.

\subsection{Deductive Semantics}

We give a semantics of BOL as an example of a semantics by translation.
We fix one language that we have already understood and define an interpretation function that maps all complex expression of the syntax into the semantic language.
Specifically, we give a deductive/logical semantics, i.e., the semantic language is a logic.

For simple ontology languages like BOL, ALC, OWL, etc., it is common to use first-order logic (FOL) as the semantic language.
More specifically, we use SFOL, the typed variant of FOL with
\begin{definition}[Logical Semantics of BOL]
The \textbf{semantic prefix} $P$ is the FOL-theory containing
\begin{compactitem}
 \item a type $\iota$ (for individuals),
 \item additional types and constants corresponding to base types and values of BOL.
\end{compactitem}

Then every BOL-ontology $O$ is interpreted as the FOL-theory $P,\sem{O}$, where $\sem{O}$ is defined in Fig.~\ref{fig:bolsem}.
\end{definition}

Like with the syntax, we can observe some general principles.
Every BOL-declaration is translated to a FOL declaration for the same name, and ontologies are translated declaration-wise.
For every kind of complex expression, there is one inductive function mapping BOL-expressions to FOL-expressions.
The base cases of references to declared identifiers are translated to themselves, i.e., to the identifiers of the same name declared in the FOL theory.
The other cases are compositional: every case for a complex expression recurses only into the semantics of the direct subexpressions.

The role of the semantic prefix $P$ is to define once and for all the FOL material that we need in general to interpret ontologies.
It occurs at the beginning of all interpretations of ontologies.
In particular, it is equal to the interpretation of empty ontology.

\begin{figure}\centering
\begin{tabular}{l|l}
BOL Syntax $X$ & Semantics $\sem{X}$ in FOL\\
\hline
\hline
ontology & FOL theory \\
$D_1,\ldots,D_n$ & $\sem{D_1},\ldots,\sem{D_n}$ \\
\hline
BOL declaration & FOL declaration \\
\kw{individual}\,$i$ & nullary function symbol $i:\iota$ \\
\kw{concept}\,$i$  & unary predicate symbol $i\sq\iota$ \\
\kw{relation}\,$i$ & binary predicate symbol $i\sq\iota\times \iota$ \\
\kw{property}\,$i:T$ & binary predicate symbol $i\sq\iota\times T$ \\
$I\; \texttt{is-a}\; C$ & axiom $\sem{C}(\sem{I})$\\
$I_1\; R\; I_2$ & axiom $\sem{R}(\sem{I_1},\sem{I_2})$\\
$I\; P\; V$ & axiom $\sem{P}(\sem{I},\sem{V})$\\
$F$ & axiom $\sem{F}$\\
\hline
Formula & Formula without free variables\\
$C_1 \Equiv C_2$ & $\forall x:\iota.\sem{C_1}(x)\Leftrightarrow \sem{C_2}(x)$\\
$C_1 \sqsubseteq C_2$ & $\forall x:\iota.\sem{C_1}(x)\impl \sem{C_2}(x)$\\
\hline
Individual & Terms of type $\iota$ \\
$i$ & $i$ \\
\hline
Concept & Formula with free variable $x:\iota$\\
$i$ & $i(x)$\\
$C_1 \sqcup C_2$ & $\sem{C_1}(x)\vee\sem{C_2}(x)$\\
$C_1 \sqcap C_2$ & $\sem{C_1}(x)\wedge\sem{C_2}(x)$\\
$\forall R.C$    & $\forall y:\iota.\sem{R}(x,y)\impl \sem{C}(y)$\\
$\exists R.C$    & $\exists y:\iota.\sem{R}(x,y)\wedge \sem{C}(y)$\\
$\dom\, R$ & $\exists y:\iota.\sem{R}(x,y)$\\
$\rng\, R$ & $\exists y:\iota.\sem{R}(y,x)$\\
\hline
Relation & Formula with free variables $x:\iota,y:\iota$\\
$i$ & $i(x,y)$\\
$R_1 \cup R_2$ & $\sem{R_1}(x,y)\vee \sem{R_2}(x,y)$\\
$R_1 \cap R_2$ & $\sem{R_1}(x,y)\wedge \sem{R_2}(x,y)$\\
$R_1 ; R_2$ & $\exists m:\iota.\sem{R_1}(x,m)\wedge \sem{R_2}(m,y)$\\
$R^{-1}$          & $\sem{R}(y,x)$\\
$R^*$          & (tricky, omitted)\\
$\Delta_C$     & $x\doteq y\wedge \sem{C}(x)$\\
\hline
Property of type $T$ & Formula with free variables $x:\iota,y:T$\\
$i$ & $i(x,y)$\\
\end{tabular}
\caption{Interpretation Function for BOL into FOL}\label{fig:bolsem}
\end{figure}

\subsection{Concretized Semantics}

We give an alternative semantics using a semantic language for concrete data.
Specifically we focus on the SQL database language.

Even though this is a very different knowledge aspect, the general principles of the semantics are the same:
Every BOL-declaration is translated to an SQL declaration, and ontologies are translated declaration-wise.
For every kind of complex expression, there is one inductive function mapping BOL-expressions to SQL-expressions.

In SQL, we can nicely see the difference between declarations and expressions: the former are translated to side effect-ful statements, the latter to side effect-free queries.

\begin{definition}[Concretized Semantic of BOL]
The \textbf{semantic prefix} consists of the following SQL statements
\begin{compactitem}
 \item a type $ID$ of identifiers (if not already supported anyway by the underlying database)
 \item declarations of all base types and values of BOL (if not already supported anyway by the underlying database)
 \item CREATE TABLE individuals (id ID, name string), where the id field is unique and automatically generated when inserting values
\end{compactitem}

Every BOL-ontology $O$ is interpreted as a sequence $P,\sem{O}$ of SQL statements, where $\sem{O}$ is defined in Fig.~\ref{fig:bolsemsql}.
\end{definition}

\begin{figure}\centering
\begin{tabular}{l|l}
BOL Syntax $X$ & Semantics $\sem{X}$ in SQL\\
\hline
\hline
ontology & SQL statements \\
$D_1,\ldots,D_n$ & $\sem{D_1},\ldots,\sem{D_n}$ \\
\hline
BOL declaration ($I$, $C$, $R$ atomic) & SQL statement \\
\kw{individual}\,$i$ & INSERT INTO individuals (name) VALUES ($i$) \\
\kw{concept}\,$i$  & CREATE TABLE $i$ (id ID)\\
\kw{relation}\,$i$ & CREATE TABLE $i$ (subject ID, object ID) \\
\kw{property}\,$i:T$ & CREATE TABLE $i$ (subject ID, object $T$) \\
$I\; \texttt{is-a}\; C$ & INSERT INTO $C$ VALUES ($\sem{I}$)\\
$I_1\; R\; I_2$ & INSERT INTO $R$ (subject, object) VALUES ($\sem{I_1}$, $\sem{I_2}$)\\
$I\; P\; V$ & INSERT INTO $P$ (subject, object) VALUES ($\sem{I}$, $V$)\\
$F$ & consistency check, consequence closure (omitted)\\
\hline
Formula & Query that returns empty result iff formula is true \\
$C_1 \Equiv C_2$ & $(\sem{C_1}\sm\sem{C_2})$ UNION $(\sem{C_2}\sm\sem{C_1})$\\
$C_1 \sqsubseteq C_2$ & $\sem{C_1}\sm\sem{C_2}$\\
\hline
Individual & an identifier from the table individuals \\
$i$ & SELECT id FROM individuals WHERE name=$i$ \\
\hline
Concept & SQL query for one-column table\\
$i$ & SELECT * FROM $i$\\
$C_1 \sqcup C_2$ & $\sem{C_1}$ UNION $\sem{C_2}$\\
$C_1 \sqcap C_2$ & $\sem{C_1}$ INTERSECT $\sem{C_2}$\\
$\forall R.C$    & individuals $\sm$ (SELECT subject FROM $\sem{R}$ WHERE object NOT IN $\sem{C})$ \\
$\exists R.C$    & SELECT DISTINCT subject FROM $\sem{R}$, $\sem{C}$ WHERE object=id\\
$\dom\, R$ & SELECT DISTINCT subject FROM $\sem{R}$\\
$\rng\, R$ & SELECT DISTINCT object FROM $\sem{R}$\\
\hline
Relation & SQL query for two-column table\\
$i$ & SELECT * FROM $i$\\
$R_1 \cup R_2$ & $\sem{R_1}$ UNION $\sem{R_2}$\\
$R_1 \cap R_2$ & $\sem{R_1}$ INTERSECT $\sem{R_2}$\\
$R_1 ; R_2$ & SELECT DISTINCT l.subject, r.object FROM $\sem{R_1}$ AS l, $\sem{R_2}$ AS r \\
            & \tb\tb WHERE l.object = r.subject\\
$R^{-1}$          & SELECT object, subject FROM $\sem{R}$\\
$R^*$          & (tricky, omitted)\\
$\Delta_C$     & SELECT id AS subject, id AS object FROM $\sem{C}$\\
\hline
Property of type $T$ & SQL query for two-column table\\
$i$ & SELECT * FROM $i$\\
\end{tabular}
\medskip

Using the abbreviation: $S\sm T$ = SELECT id FROM $S$ WHERE id NOT IN $T$\\
\caption{Interpretation Function for BOL into SQL}\label{fig:bolsemsql}
\end{figure}

%\footnotetext{This case is a mini homework.}

\paragraph{Complex Expressions in Assertions}
Our interpretation of BOL in SQL is restricted assertions using only atomic expressions.
For example, in the case for $I$ \texttt{is-a} $C$, we assume that $I$ and $C$ are names.
Thus, we have already created an individual for $I$ and a table for $C$, and we can thus insert the former into the latter.

The general case would be more complicated but is much less important in practice.
For example, the concept assertion "Florian Rabe" \texttt{is-a} "instructor" $\sqcap$ "male" could be handled by adding the individual to both tables.
But other expressions very quickly become more difficult.

\paragraph{Translation of Formulas}
The interpretation of formulas into SQL is less obvious because SQL is not a logic.
We have to consider two subtleties.

Firstly, a formula may express a consistency condition that must not violated by the ontology.
For example, ontologies may contain contradictory assertions or violations of uniqueness constraints such as a person should only have one father or fathers should be male.
In FOL, this amounts to $\sem{O}$ being an inconsistent theory, which can be reasoned about as usual.
In SQL, we can mimic it by checking the consistency of the database.
(Consistency is undecidable for FOL in general but often decidable for ontology languages.)

Secondly, a formula may express a closure operation that must be mimicked by adding implied assertions to the ontology.
For example, if there is a subconcept axiom "instructor" $\sqsubseteq$ "person" and a concept assertion "Florian Rabe" is-a "instructor", we have to add the implied concept assertion "Florian Rabe" is-a "person".
In FOL, this happens automatically by the calculus of FOL, which adds all theorems of a theory.
In SQL, we have to mimic it manually. (Consequence is undecidable for FOL in general but often decidable for ontology languages.)

\subsection{Narrative Semantics}

We will now look at still another semantics for BOL: the \textbf{narrative semantics} is given by translation into a natural language --- here English.

Note that we are relatively open of what kind of English we want to use as a target language.
The simplest choice would be to use plain English, as you could find in a novel or newspaper article.
But for many applications (e.g. STEM ontologies), we would rather use STEM English, i.e. English interspersed with formulas, diagrams, and epistemic cues like ``Definition'', ``Theorem'', ``Proof'', and even $\Box$.
For this kind of English, {\LaTeX} is a good target format.
Finally, we can use flexiformal formats like \sTeX~\cite{Kohlhase:ulsmf08,URL:sTeX:github}, where we can capture more of the semantic properties.

In \Cref{fig:bolsem:narrative1} we present a translation into plain English in the style of the other aspect semantics translations above.
It recursively goes over the BOL expressions and generates strings, using a lexicon function $lex$ (for lexicon) that maps BOL identifiers into (lemmata $\hat=$ base forms of) English words of the appropriate syntactic category for the base cases.

\begin{figure}\centering
\begin{tabular}{l|l}
BOL Syntax $X$ & Narrative Semantics $\sem{X}$ in English\\
\hline\hline
ontology & English document \\
$D_1,\ldots,D_n$ & $\sem{D_1},\ldots,\sem{D_n}$ \\
\hline
BOL declaration & dictionary entry or true sentence\\
\kw{individual}\,$i$ & $lex(i)$ is a proper noun\\
\kw{concept}\,$i$  & $lex(i)$ is a common noun\\
\kw{relation}\,$i$ & $lex(i)$ is a transitive verb \\
\kw{property}\,$i:T$ & $lex(i)$ is a common noun for a property that can take $T$-values \\
%$I\; \texttt{is-a}\; \forall R.C$ & all those that $\sem{I}$ $\sem{R}$s are $\sem{C}$\\
%$I\; \texttt{is-a}\; \exists R. C$ & $\sem{I}$ $\sem{R}$s a $\sem{C}$\\
$I\; \texttt{is-a}\; C$ & $\sem{I}$ $\sem{C}$\\
$I_1\; R\; I_2$ & $\sem{I_1}$ $\sem{R}$ $\sem{I_2}$\\
$I\; P\; V$ & $\sem{I}$ has $\sem{P}$ $\sem{V}$\\ 
$F$ & $\sem{F}$\\
\hline
Formula & Sentence \\
$C_1 \Equiv C_2$ & $\sem{C_1}$ing is the same as $\sem{C_2}$ing\\
%$c \Equiv C$ & $\sem{C_2}$ abbreviates $\sem{C}$ & $c$ a constant\\
$C_1 \sqsubseteq C_2$ & everything that $\sem{C_1}$s also $\sem{C_2}$s\\
\hline
Individual & noun phrase (to be used as a subject or object)\\
$i$ & $lex(i)$ \\
\hline
Concept & intransitive verb phrase (to be plugged after a subject) \\
$i$ & is a $lex(i)$\\
$C_1 \sqcup C_2$ & $\sem{C_1}$ or $\sem{C_2}$\\
$C_1 \sqcap C_2$ & $\sem{C_1}$ and $\sem{C_2}$\\
%$C_1 \sqcap C_2$ & is a $\sem{C_1}$ that $\sem{C_2}$s & $C_1$ simple\\
$\forall R.C$    & $\sem{R}$s only things that $\sem{C}$ \\
$\exists R.C$    & $\sem{R}$s something that $\sem{C}$s\\
$\dom\, R$ & $\sem{R}$s something\\
$\rng\, R$ & is $\sem{R}$ed by something\\
\hline
Relation & transitive verb phrase (to be plugged between subject and object)\\
$i$ & $lex(i)$\\
$R_1 \cup R_2$ & $\sem{R_1}$s or $\sem{R_2}$s\\
$R_1 \cap R_2$ & $\sem{R_1}$s and $\sem{R_2}$s\\
$R_1 ; R_2$ & $\sem{R_1}$s something that $\sem{R_2}$s\\
$R^{-1}$    & is $\sem{R}$ed by\\
$R^*$       & $\sem{R}$s something that $\sem{R}$s something and so on that $\sem{R}$s\\
$\Delta_C$  & $\sem{C}$s and is the same as\\
\hline
Property of type $T$ & property phrase\\
$i$ & $lex(i)$\\
\end{tabular}
\caption{Interpretation Function for BOL into English (intransitive VP version)}\label{fig:bolsem:narrative1}
\end{figure}

\begin{example}
  Consider an ontology with declarations $\kw{individual}\; P$, $\kw{individual}\; M$, $\kw{relation}\; l$, $\kw{relation}\; h$ and a lexicon function $P\mapsto\text{Peter}, M\mapsto\text{Mary}, l\mapsto\text{love}$, then  $\sem{P\; l\; M}$ = ``Peter loves Mary''. 
\end{example}

\begin{example}
  If we extend the ontology above with $\kw{concept} d$ and $\kw{concept} c$ and the lexicon with $lex(d)=\text{is a dog}$ and $lex(c)=\text{is a cat}$, then $M\; \mathtt{is-a}\; c$ translates to ``Mary is a cat'', and  $P\; \mathtt{is-a}\; c\sqcap d$ translates to ``Peter is a cat or is a dog''.
\end{example}

But while the target languages in the other translations are formal languages engineered for regularity (see also the discussion in \Cref{sec:compositionality}) and simplicity (in terms of language primitives), while natural languages have evolved in practical human communication.
As a consequence, the translation in \Cref{fig:bolsem:narrative} results in English that is clumsy at best and non-grammatical in general.
We can think of this as \textbf{BOL-pidgin} English.

Let us have a look at some of the problems that appear in both translations:
\begin{itemize}
\item our lexicon does not have any inflection information and the translation tries to remedy that by appending ``s'' in various places. This works for the example above but not for relations that verbalize to ``has a'' or ``has as child''.
\item We have introduced some pattern matching and conditional rules in \Cref{fig:bolsem:narrative} to alleviate the greatest awkwardnesses, but much more would be needed, which would lead to things like ``Peter hass Mary'' or ``Peter has as childs Mary''. 
\item there are many linguistic devices that serve an important role in natural language, but which we are not tageting. An example is plural objects for aggregation. Say we have $P \mathtt{is-a} C$, $P \mathtt{is-a} C$, this would translate to ``Peter is a $\sem{C}$, Mary is a $\sem{C}$'' in BOL-pidgin, whereas in natural English would aggregate this to ``Peter and Mary are $\sem{C}$s''.
\end{itemize}

A way out here is to utilize special systems for dealing with the surface structure of natural language.
An example of this is the GF system (Grammatical Framework~\cite{Ranta:gfpmg11,GF:url}) which allows to specify a rich formal language \textbf{abstract syntax trees} (ASTs) together with language-specific \textbf{linearizations}, which amount to recursive functions that translate ASTs to language-specific strings.
GF comes with  a large resource library~\cite{GFResourceGrammar:url} that provides a comprehensive, language-independent AST specification and  linearlizations for over 35 languages.
We will not pursue this here, there is a special course ``Logic-based Natural Languge Semantics'' at FAU in the Winter Semesters that covers these and related topics.   

\subsection{Compositionality}\label{sec:compositionality}

\paragraph{Definition}
An interpretation function is compositional if the interpretation of any kind of expression $E(e_1,\ldots,e_n)$ with direct subexpressions $e_i$ only depends on $E$ and the interpretation of the $e_i$, i.e., \[\sem{E(e_1,\ldots,e_n)}=\sem{E}(\sem{e_1},\ldots,\sem{e_n})\] for some semantic operation $\sem{E}$.

The interpretations of BOL in FOL and SQL are compositional.
For example, consider the case of composition of relations:
 \[\sem{R_1 ; R_2}= \exists m:\iota.\sem{R_1}(x,m)\wedge \sem{R_2}(m,y)\]
We have $n=2$ and $E$ is the $;$-operator mapping $(e_1,e_2)\mapsto e_1;e_2$, i.e., $R_1$ and $R_2$ are the direct subexpressions of $R_1;R_2$.
The semantics is a relatively complicated FOL-formula, but it only depends on $\sem{R_1}$ and $\sem{R_2}$ --- everything else is fixed.
We have $\sem{;}=(p_1,p_2)\mapsto \exists m:\iota.p_1(x,m)\wedge p_2(m,y)$, i.e., the interpretation of the $;$-operator is the function that maps two predicates $p_1,p_2$ to the formula $\exists m:\iota.p_1(x,m)\wedge p_2(m,y)$.
Then we have \[\sem{R_1;R_2}=\sem{;}(\sem{R_1},\sem{R_2}).\]


It is highly desirable but not always possible to give a compositional translation.
Sometimes a feature of the syntactic language cannot be directly interpreted in the semantic language.
In that case, it may still be possible to give a non-compositional translation.

\begin{example}[Non-Compositional Translation via Sub-Induction]
A simple example of non-compositionality is the translation of natural numbers based on zero, one, and addition (i.e., $N\bbc 0\bnfalt 1\bnfalt N+N$) into natural numbers based on zero and successor (i.e., $N\bbc 0\bnfalt\cn{succ}(N)$):
It is straightforward to translate zero and one compositionally:
\[\sem{0}=0 \tb\sem{1}=\cn{succ}(0)\]
Now we would like to translate \[\sem{m+n}=\sem{+}(\sem{m},\sem{n}),\] but there is no way to define $\sem{+}$ in terms of zero and successor.
Instead, we need subcases:
\[\sem{m+n}=\cas{\sem{m}\mifc n=0 \\ \cn{succ}(\sem{m})\mifc n=1 \\ \sem{(m+n_1)+n_2}\mifc n=n_1+n_2}\]
This corresponds to the usually definition of addition, i.e., $\sem{+}$, by induction.
\end{example}

Other common examples of non-compositional translations are
\begin{compactitem}
 \item several important logical theorems such as
  \begin{compactitem}
   \item cut elimination, which is he translation from sequent calculus with cut to sequent calculus without cut,
   \item the deduction theorem, which is the translation from natural deduction to Hilbert calculus,
  \end{compactitem}
 \item almost anything done by an optimizing compiler, e.g., loop unrolling or function inlining,
 \item query optimization done by a database, e.g., turning a WHERE of a join into a join of WHEREs,
 \item almost all translations between natural languages, e.g., when words are ambiguous and a different translation must be chosen for the same word based on the context.
\end{compactitem}

Typical sources of non-compositionality in formal language translations are:
\begin{compactitem}
 \item A case in the translation function requires subcases which inspect the $e_i$ and treat them differently.
 \item A case in the translation function requires subcases which translate an expression differently based on the context in which it occurs.
 \item The translation function requires nested inductions, i.e., a case in the translation function (which is already inductive) requires a sub-induction on one of the sub-expressions.
 \item The semantic prefix is not fixed but depends on the translated object, i.e, the top-level case of the translation scans through the entire argument $X$ to collect all occurrences of a particular feature and then custom-builds the semantic prefix of $\sem{X}$.
\end{compactitem}

In Fig.~\ref{fig:bolsem}, we omitted the case for the transitive closure.
That was because it is not possible to translate it compositionally into FOL.
We can only do it non-compositionally with a custom semantic prefix:

\begin{example}[Non-Compositional Translation via Custom Semantic Prefix]
We define the FOL-interpretation of an ontology $O$ by $\sem{O}=P_O,\sem{O}$, where $P_O$ is a custom semantic prefix.
$P_O$ is different for every ontology $O$ and is defined as follows:

\begin{compactenum}
 \item We scan through $O$ and collect all occurrences of $R^*$ for any (not necessarily atomic) relation $R$.
 \item $P_O$ contains the following declarations for each $R$:
  \begin{compactitem}
  \item A binary predicate symbol $C_R\sq i\times i$. Note that $R$ may be a complex expression; so we have to generate a fresh name $C_R$ here.
  \item The axiom $\forall x:\iota,y:\iota:R(x,y)\impl C_R(x,y)$, i.e., $C_R$ extends $R$.
  \item The axiom $\forall x:\iota,y:\iota,z:\iota.C_R(x,y)\wedge C_R(y,z)\impl C_R(x,x)$, i.e., $C_R$ is transitive.
  \end{compactitem}
 \item We add the case $\sem{R^*}=C_R(x,y)$ to the interpretation function.
\end{compactenum}

Intuitively, every occurrence of the $^*$-operator is removed from the language and replaced with a fresh name that is axiomatized to have the needed properties.
All of these axioms are added to the semantic prefix.
\end{example}

Such non-compositional translations are undesirable for multiple reasons:
\begin{compactitem}
 \item The implementation is more complicated and error-prone.
 \item Reasoning about the translation is more difficult.
 \item The custom semantic prefix can be large.
\end{compactitem}

But most importantly, non-compositional translations are less robust.
Firstly, if we add a production to the syntax, a compositional translation is easy to extend: just add a case to the translation.
But a non-compositional translation may additionally require a new subcase wherever subcases/subinductions are used.
Moreover, if a custom semantic prefix is used, its definition may have to be amended, at least it must be rechecked.

Secondly, in practice there are two sources of complex expressions: the ones already mentioned in the language, and the ones used later for other reasons.
For example, in BOL some complex expressions occur already \emph{statically} in the definition of an ontology $O$.
But others might be appear \emph{dynamically} later, e.g., when talking about $O$, proving properties of $O$, or running queries on $O$.
Thus, the definition of $O$ and the use of complex expressions are decoupled: $O$ is defined statically once and for all, and complex expressions relative to $O$ can be created and used dynamically.
But if a custom semantic prefix is used, only the static occurrences inside $O$ can be considered for building the prefix.
Thus, it is not possible to translate the dynamic occurrences of the transitive closure unless the semantic prefix is extended all the time as $O$ is used.

%%%%%%%%%%%%%%%%%%%%%%%%%%%%%%%%%%%%%%%%%%%%%
\section{Representing Ontologies as Triples}

It is common to represent an entire ontology as a set of subject-predicate-object triples.
That makes handling ontologies very simple and efficient.
This is the preferred representation of the semantic web.

However, while, e.g., relation assertions are naturally triples, not all declarations are, and some tricks may be necessary.

\paragraph{Inferring the Entity Declarations}
The entity declarations are not naturally triples.
But we can usually infer them from the assertions as follows: any identifier that occurs in a position where an entity of a certain kind is expected is assumed to be declared as an entity for that kind.

For example, the individuals are what occurs as the subject of a concept, relation, or property assertion or as the object of a relation assertion.
It is conceivable that there are individuals that occur in none of these.
But that is unusual because they would be disconnected from everything in the ontology.

If we give TBox and ABox together, this inference approach usually works well.
But if we only give a TBox, this would often not allow inferring all entities.
The only place where they could occur in the TBox is in the axioms, and it is quite possible to have concept, relation, and property declarations that are not used in the axioms.
In fact, it is not unusual not to have any axioms.

\paragraph{Special Predicates}
To turn declarations into triples, we can use reflection, i.e., the process of talking about our language constructs as if they were data.

Reflection requires introducing some built-in entities that represent the features of the language.
In the semantic web area, this is performed using the following entities:
\begin{compactitem}
 \item "rdfs:Resource": a built-in concept of which all individuals are an instance and thus of which every concept is a subconcept
 \item "rdf:type": a special predicate that relates an entity to its type:
  \begin{compactitem}
   \item an individual to its concept (corresponding to \texttt{is-a} above)
   \item other entities to their special type (see below)
  \end{compactitem}
 \item "rdfs:Class": a special class to be used as the type of classes
 \item "rdf:Property": a special class to be used as the type of properties
 \item "rdfs:subClassOf": a special relation that relates a subconcept to a superconcept
% \item "rdfs:subPropertyOf": a special relation that relates a relation to one that it implies
 \item "rdfs:domain": a special relation that relates a relation to the concepts of its subjects
 \item "rdfs:range": a special relation that relates a relation/property to the concept/type of its objects
\end{compactitem}
Here "rdf" and "rdfs" refer to the RDF (Resource Description Framework) and RDFS (RDF Schema) namespaces, which correspond to W3C standards defining those special entities.

Thus, we can represent many and in particular the most important entity declarations as triples:
\begin{center}
\begin{tabular}{l|lll}
Assertion & \multicolumn{3}{c}{Triple} \\
          & Subject & Predicate & Object \\
\hline
individual & individual & "rdf:type" & "rdfs:Resource" \\
concept  & concept & "rdf:type" & "rdf:Class" \\
relation & relation & "rdf:type" & "rdf:Property" \\
property & property & "rdf:type" & "rdf:Property" \\
concept assertion  & individual & "rdf:type" & concept \\
relation assertion & individual & relation & individual \\
property assertion & individual & property & value \\
\hline
\multicolumn{4}{l}{for special forms of axioms}\\
$c\sqsubseteq d$ & $c$ & "rdfs:subClassOf" & $d$ \\
%$r\sqsubseteq s$ & $r$ & "rdfs:subPropertyOf" & s \\
$\dom\,r\Equiv c$ & $r$ & "rdfs:domain" & $c$ \\
$\rng\, r\Equiv c$ & $r$ & "rdfs:range" & $c$ \\
\end{tabular}
\end{center}

This is subject to the restriction that only atomic concepts and relations can be handled.
For example, only concept assertions can be handled that make an individual an instance of an \emph{atomic} concept.
This is particularly severe for axioms, where complex expressions occur most commonly in practice.
Here, the special relations allow capturing the most common axioms as triples.

\paragraph{Problems}
Reflection is subtle and can easily lead to inconsistencies.
We can see this in how the approach of RDF(S) special entities breaks the semantics via FOL.

For example, it treats classes both as concepts (when they occur as the object of a concept assertion) and as individuals (when they occur as subject or object of a "rdfs:subClassOf" relation assertion).
Similarly, "rdfs:Class" is used both as an individual and as a class.
In fact, the standard prescribes that "rdfs:Class" is an instance of itself.

In practice, this is handled pragmatically by using ontologies that make sense.
A formal way to disentangle this is to assume that there are two variants of "rdfs:Class", one as an individual and one as a class.
The translation must then translate "rdfs:Class" differently depending on how it is used.

It would be better if RDFS were described in a way that is consistent under the implicitly intended FOL semantics.
But the more pragmatic approach has the advantage of being more flexible.
For example, being able to treat every class, relation, or property also as an individual makes it easy to annotate metadata to them.
Metadata is a set of properties such as "rdfs:seeAlso" or "owl:versionInfo", whose subjects can be any entity.


\paragraph{Subject-Centered Representations}
When giving a set of triples, there are usually a lot of triples with the same subject.
For example, we could use a simple concrete syntax with one triple per line and whitespace separating subject, predicate, and object:
\begin{lstlisting}
"Florian Rabe" is-a "Instructor"
"Florian Rabe" is-a "male"
"Florian Rabe" "teaches" "WuV"
"Florian Rabe" "teaches" "KRMT"
"Florian Rabe" "age" 40
"Florian Rabe" "office" "11.137"
\end{lstlisting}

It is more human-friendly to group these triples in such a way that the subject only has to be listed once.
For example, we could use a concrete syntax like this, where the subject occurs first and then predicate-object pairs occur on indented lines:
\begin{lstlisting}
"Florian Rabe"
  is-a "Instructor"
  is-a "male"
  "teaches" "WuV"
  "teaches" "KRMT"
  "age" 40
  "office" "11.137"
\end{lstlisting}

If the same predicate occurs with multiple values, we can group those as well.
For example, we could give the objects for the same predicates as a list following the predicate:
\begin{lstlisting}
"Florian Rabe"
  is-a "Instructor" "male"
  "teaches" "WuV" "KRMT"
  "age" 40
  "office" "11.137"
\end{lstlisting}

Concrete syntaxes based on the triple representation of ontologies will usually adopt some kind of structure like this.
The details may vary.

%%%%%%%%%%%%%%%%%%%%%%%%%%%%%%%%%%%%%%%%
\section{Type Systems for Ontologies}

\subsection{Concepts as Types}


\subsection{Record Types}

%%% Local Variables:
%%% mode: latex
%%% TeX-master: "WuV_notes"
%%% mode: visual-line
%%% fill-column: 5000
%%% End:

%  LocalWords:  ontologization sqsubseteq commgrammar gcomment gprod galtprod Equiv sqcup sqcap relativization rng itg strg fig:bolsem sq impl x,y y,x iota,y m,y doteq fig:bolsemsql medskip Kohlhase:ulsmf08 Compositionality sec:compositionality bnfalt bnfalt bnfalt succ mifc mifc mifc iota,z y,z x,x subinductions rdfs:Resource rdf:type rdfs:Class rdf:Property rdfs:subClassOf rdfs:subPropertyOf rdfs:domain rdfs:range rdf rdfs Subject-Centered childs Ranta:gfpmg11,GF:url linearizations

 \section{Writing Ontologies}\label{sec:onto:write}
   \subsection{The OWL Language}

\paragraph{Abstract Syntax and Semantics}
Due to their central in knowledge representation, a number of languages for ontology writing exist.
Most importantly, the syntax and semantics of OWL, including several sublanguages, are standardized by the W3C.

OWL includes a number of built-in special entities.
Most importantly, \enquote{owl:Thing} corresponds to \enquote{rdfs:Resource} as the concept of all individuals.

\paragraph{Concrete Syntax}
Several concrete syntaxes have been defined and are commonly used for OWL\@.
The OWL2 primer\footnote{\url{https://www.w3.org/TR/2012/REC-owl2-primer-20121211/}} systematically describes examples in five different concrete syntaxes.

APIs for OWL implement the abstract syntax along with good support for reading/writing ontologies in any of the concrete syntaxes.

\subsection{The Protege Tool}

A widely used tool for writing ontologies in OWL is Protege\footnote{\url{https://protege.stanford.edu/}}.

To get started with Protege without getting confused, we need to continue understand how its key terminology maps to other contexts.
\begin{center}
\begin{tabular}{l@{\qquad}ll}
\toprule
 Here       & Protege & Edited in WebProtege via \\
 \midrule
individual & individual & listed in \enquote{Individuals} tab\\
concept    & class   & listed in \enquote{Classes} tab  \\
relation   & object property & listed in \enquote{Properties} tab\\
property   & data property & listed in \enquote{Properties} tab\\
concept assertion & Type & detail area of the individual in \enquote{Individuals} tab \\
relation assertion & Relationship & detail area of the subject in \enquote{Individuals} tab \\
property assertion & Relationship & detail area of the subject in \enquote{Individuals} tab \\
\bottomrule
\end{tabular}
\end{center}

Protege's interface treats some parts of the ontology specially:
\begin{compactitem}
\item The \enquote{Classes} tab organizes concepts using a tree view based on the subconcept relationship.
 Superclasses of a class can also be edited directed by listing parents.
 \item The \enquote{Properties} tab organizes properties using a tree view based on the subproperty (i.\,e., implication, subset) relationship.
 \item Axioms describing the domain and range of a property can be given directly in its details view.
\end{compactitem}

Note that classes can be in relationships with other classes as well even though that was not considered in the course so far.

\subsection{Exercise 1}

The topic of Exercise 1 is to use Protege to write an OWL ontology for a university.

Protege is a graphical editor for the abstract syntax of OWL.
Familiarize yourself with the various concrete syntaxes of OWL by writing an ontology that uses every feature once, downloading it in all available concrete syntaxes, and comparing those.

The minimal goal of the exercise session is to get a Hello World example going, at which point the task transitions into homework.
There will be no homework submission, but you will use your ontology throughout the course.

You should make sure you understand and setup the process in a way that supports you when you revisit and change your ontology many times throughout the semester.

Other than that, the task is deliberately unconstrained to mimic the typical situation at the beginning of a big project, where it is unclear what the ultimate requirements will be.


\chapter{Semantics for BOL}\label{sec:bolsem}
 This chapter introduces different kinds of semantics.
The emphasis will be on relative semantics by compositional translations.

We use BOL as a running example and give four different semantics of BOL --- using the four other aspects:
\begin{center}
\begin{tabular}{llll}
Section & Aspect & kind of semantic language & semantic language\\
\hline
\ref{sec:bolsem:ded} & deduction & logic & SFOL \\
\ref{sec:bolsem:conc} & concretization & database language & SQL \\
\ref{sec:bolsem:comp} & computation & programming language & Scala \\
\ref{sec:bolsem:narr} & narration & natural language & English \\
\end{tabular}
\end{center}

%Note that we could also give an ontological semantics of BOL, e.g., by using OWL as the semantic language.
%BOL and OWL are already so similar that the translation would be rather straightforward.
%Therefore, we omit it.

\section{Kinds of Semantics}

To define a language system, we need to define the well-formedness predicates.
Moreover, we need to define the semantics of the well-formed vocabularies and expressions.
One way to do that is by translating the syntax into another language and then use existing definitions of well-formedness there.

But we also need to be able to get off the ground, i.e., to define a semantics from scratch when we do not have another language available.
This is usually done by giving an inference system for the well-formedness predicates and other judgments such as truth.

We can also do both: if we have a semantics via an inference system and another one via translation, we can show that the latter respects the former.
That leads to the concepts of soundness (everything well-formed is translated to something well-formed) and its dual completeness.

\subsection{Absolute Semantics: By an Inference System}

\subsection{Relative Semantics}

Relative semantics uses an additional component as a reference point for semantics.

\paragraph{By Translation}

Semantics by translation uses a second language as the target of a translation.
Both the vocabularies and the expressions are translated, namely to vocabularies and expressions of the target language.
Usually, the target language already has a semantics, and the semantics of the original language is obtained by composing the two.

\begin{figure}[hbt]
\begin{tabular}{l|l}
Aspect & Example\\\hline
%Ontological & OWL $\to$ FOL \\
Deductive & FOL $\to$ set theory\\
Computational & C++ $\to$ assembly \\
Concrete & SPARQL $\to$ SQL \\
Narrative & English $\to$ German \\
\end{tabular}
\caption{Examples of Semantics by Translation}\label{fig:trans}
\end{figure}

The correspondence for the syntax between context-free grammars and inductive data types can be extended to the semantics.
Now we have a correspondence between case-based function definitions and inductive functions.

\begin{definition}
A \textbf{semantics by translation} consists of the following parts:
\begin{compactitem}
 \item syntax: a formal language $l$
 \item semantic language: a formal language $L$ (from a different or the same aspect as $l$)
 \item semantic prefix: a vocabulary $P$ in $L$ that is prefixed to the translation of all vocabularies of $l$
 \item interpretation: a function that translates every $l$-vocabulary $T$ to an $L$-vocabulary $P,\sem{T}$
\end{compactitem}
\end{definition}

Critically, the semantic language (which is itself a formal language and can thus have a semantics itself) must be a language whose semantics we already know.
Therefore, it is often important to give multiple equivalent semantics --- choosing a different semantics for different audiences, who might be familiar with different languages.

The role of the semantic prefix $P$ is to define once and for all the $L$-material that we need in general to interpret $l$-theories (in our case: ontologies).
It occurs at the beginning of all interpretations of ontologies.
In particular, it is equal to the interpretation of empty ontology.

\paragraph{By Interpretation}

Semantics by interpretation uses a situation relative to which expressions are evaluated.
The role of the situation is to supply the meaning of all identifiers declared in the vocabulary.
Given a situations, inductive functions map all expressions to their semantics.
Usually, the result of interpretation is an extremely simple object, whose semantics is self-evident.

\begin{figure}[hbt]
\begin{tabular}{l|ll}
Aspect & Situation & Provides\\\hline
%Ontological & ABox & individuals and their properties \\
Deductive & model & interpretations of the function/predicate/... symbols\\
Computational & environment & e.g., standard input/output \\
Concrete & database & set of objects for each type \\
Narrative & semantic dictionary & meanings of the words \\
\end{tabular}
\caption{Examples of Situational Semantics by Interpretation}\label{fig:sit}
\end{figure}

Situational semantics can be seen as a special case of semantics by translation as follows:
\begin{itemize}
\item The target language is an implicit assumed background language such as mathematics or the computer hardware.
Once this is made explicit, situations can be described as vocabularies of the background language.
\item The semantic prefix describes what a situation is in general, i.e., the possible situations for the empty vocabulary.
\item The semantics of a vocabulary extends the semantic prefix with the possible ways to interpret the identifiers.
\item The semantics of an expression $E$ is the function mapping the situation $S$ to the situational semantics of $E$ under $S$. This can be expressed as an expression of the target language.
\end{itemize}

\section{Compositionality}

\paragraph{Introduction}
There are some general principles shared by all translations:
\begin{compactitem}
 \item Every $l$-declaration is translated to an $L$-declaration for the same name, and ontologies are translated declaration-wise.
 \item For every non-terminal $N$ of $l$, there is one inductive function $\sem{-}_N$ mapping complex $l$-expressions derived from $N$ to $L$-expressions.
 \item The base cases of references to declared $l$-identifiers are translated to themselves, i.e., to the identifiers of the same name declared in $L$.
 \item The other cases are compositional: every case for a complex $l$-expression recurses only into the semantics of the direct subexpressions.
\end{compactitem}

The notion of \emph{compositionality} captures these properties.
An interpretation function is compositional if the interpretation of any kind of expression $E(e_1,\ldots,e_n)$ with subexpressions $e_i$ only depends on $E$ and the interpretation of the $e_i$, i.e., \[\sem{E(e_1,\ldots,e_n)}=\sem{E}(\sem{e_1},\ldots,\sem{e_n})\] for some semantic operation $\sem{E}$.
Compositionality is also called the substitution property or the homomorphism property.
See also Def.~\ref{ex:compositional}.

More rigorously, we define a compositional translation as follows:
\begin{definition}[Compositional Semantics]
Consider a semantics for syntax grammar $l$ and interpretation function $\sem{-}$.

$\sem{-}$ is compositional if it is defined as follows:
\begin{compactitem}
 \item a family of functions $\sem{-}_N$, one for every non-terminal $N$ of $l$
 \item for every expressions $E$ derived from $N$, we put $\sem{E}=\sem{E}_N$
 \item each $\sem{-}_N$ is defined by induction on the productions for $N$
 \item for each production $N\bbc *(N_1,\ldots,N_r)$ and all expressions $e_i$ derived from $N_i$
   \[\sem{*(e_1,\ldots,e_r)}_N=\sem{*}(\sem{e_1}_{N_1},\ldots,\sem{e_r}_{N_r})\]
   for some $L$-expression $\sem{*}$
\end{compactitem}

Without loss of generality, we can assume that every production is of the form $N\bbc *(N_1,\ldots,N_r)$ where the $N_i$ are all the non-terminals on the right-hand side and $*$ is a stand-in for all the terminal symbols.
\end{definition}

\paragraph{Compositional Translations of Contexts}
We can extend every compositional translation to contexts, substitutions, and expressions in contexts:

\begin{definition}
Given a translation $\sem{-}$ as above, for a non-terminal $N$, we define $\sem{N}$ as the non-terminal from which the translations of $N$-expressions are derived.

Then we define:
\[\sem{x_1:N_1,\ldots,x_n:N_n}:=x_1:\sem{N_1},\ldots,x_n:\sem{N_n}\]
\[\sem{x_1:=w_1,\ldots,x_n:=w_n}:=x_1:=\sem{w_1},\ldots,x_n:=\sem{w_n}\]
\[\sem{x}:=x\]
\end{definition}

The requirement of compositionality is critical for two reasons:
\begin{compactitem}
\item A non-compositional translation could translate $l$-expressions derived from the same non-terminal $N$ to $L$-expressions derived from different non-terminals. Then we would not be able to define $\sem{N}$.
\item The definition $\sem{x}:=x$ adds a case to the case distinction in the compositional translation function.
Without compositionality, this would not make sense.
\end{compactitem}

\begin{theorem}[Type Preservation]
For a compositional translation as above, we have
  \[\Gamma\vdash_l w:N \tb\mimplies\tb \sem{\Gamma}\vdash_L \sem{w}:\sem{N}\]
\end{theorem}

\paragraph{Substitution Theorem}
The main value of compositionality is the following:
\begin{theorem}[Substitution Theorem]
Consider a compositional semantics.

For every context $\Gamma=x_1:N_1,\ldots,x_r:N_r$, every syntax expression $\Gamma \vdash_l E:N$,
and every substitution $\vdash_l \gamma:\Gamma$
\[\sem{E[\gamma]}=\sem{E}[\sem{\gamma}]\]
\end{theorem}

Formulated without substitutions, this means that for every syntax expression $E(e_1,\ldots,e_r)$ derived from $N$, where the $e_i$ are subexpression derived from non-terminal $N_i$, we have
\[\sem{E(e_1,\ldots,e_n)}_N=\sem{E}(\sem{e_1}_{N_1},\ldots,\sem{e_n}_{N_r})\]

Simply put, a semantics is compositional iff it is defined by mutually inductive translation functions with only compositional cases.
The latter is very easy to check by inspecting the shape of the finitely many cases of the definition.
The former is a powerful property because it applies to any of the infinitely many expressions of the syntax.

\subsection{Non-Compositional Semantics}

It is highly desirable but not always possible to give a compositional translation.
Sometimes a feature of the syntactic language cannot be directly interpreted in the semantic language.
In that case, it may still be possible to give a non-compositional translation.

\begin{example}[Non-Compositional Translation via Sub-Induction]
A simple example of non-compositionality is the translation of natural numbers based on zero, one, and addition (i.e., $N\bbc 0\bnfalt 1\bnfalt N+N$) into natural numbers based on zero and successor (i.e., $N\bbc 0\bnfalt\cn{succ}(N)$):
It is straightforward to translate zero and one compositionally:
\[\sem{0}=0 \tb\sem{1}=\cn{succ}(0)\]
Now we would like to translate \[\sem{m+n}=\sem{+}(\sem{m},\sem{n}),\] but there is no way to define $\sem{+}$ in terms of zero and successor.
Instead, we need subcases:
\[\sem{m+n}=\cas{\sem{m}\mifc n=0 \\ \cn{succ}(\sem{m})\mifc n=1 \\ \sem{(m+n_1)+n_2}\mifc n=n_1+n_2}\]
This corresponds to the usual definition of addition, i.e., $\sem{+}$, by induction.
\end{example}

Other common examples of non-compositional translations are
\begin{compactitem}
 \item several important logical theorems such as
  \begin{compactitem}
   \item cut elimination, which is the translation from sequent calculus with cut to sequent calculus without cut,
   \item the deduction theorem, which is the translation from natural deduction to Hilbert calculus,
  \end{compactitem}
 \item almost anything done by an optimizing compiler, e.g., loop unrolling or function inlining,
 \item query optimization done by a database, e.g., turning a WHERE of a JOIN into a JOIN of WHEREs,
 \item almost all translations between natural languages, e.g., when words are ambiguous and a different translation must be chosen for the same word based on the context (The introduction of richer intermediate structures like ASTs and functions as values into the translation can recover some compositionality here).
\end{compactitem}

Typical sources of non-compositionality in formal language translations are:
\begin{compactitem}
 \item A case in the translation function requires subcases which inspect the $e_i$ and treat them differently.
 \item A case in the translation function requires subcases which translate an expression differently based on the context in which it occurs.
 \item The translation function requires nested inductions, i.e., a case in the translation function (which is already inductive) requires a sub-induction on one of the sub-expressions.
 \item The semantic prefix is not fixed but depends on the translated object, i.e, the top-level case of the translation scans through the entire argument $X$ to collect all occurrences of a particular feature and then custom-builds the semantic prefix of $\sem{X}$.
\end{compactitem}
See also Ex.~\ref{ex:noncompositional}.

Such non-compositional translations are undesirable for multiple reasons:
\begin{compactitem}
 \item The implementation is more complicated and error-prone.
 \item Reasoning about the translation is more difficult.
 \item The custom semantic prefix can be large.
\end{compactitem}

But most importantly, non-compositional translations are less robust.
Firstly, if we add a production to the syntax, a compositional translation is easy to extend: just add a case to the translation.
But a non-compositional translation may additionally require a new subcase wherever subcases/subinductions are used.
Moreover, if a custom semantic prefix is used, its definition may have to be amended, at least it must be rechecked.

Secondly, in practice there are two sources of complex expressions: the ones already mentioned in the language, and the ones used later for other reasons.
For example, some complex expressions occur already \emph{statically} in the definition of a vocabulary $V$.
But others might be appear \emph{dynamically} later, e.g., when talking about $V$, proving properties of $V$, or running queries on $V$.
Thus, the definition of $V$ and the use of complex expressions are decoupled: $V$ is defined statically once and for all, and complex expressions relative to $V$ can be created and used dynamically.
But if a custom semantic prefix is used, only the static occurrences inside $V$ can be considered for building the prefix.
Thus, it is not possible to translate expressions dynamically unless the semantic prefix is extended all the time while $V$ is used.

\section{Deductive Semantics}\label{sec:bolsem:ded}

We fix one language that we have already understood and define an interpretation function that maps all complex expression of BOL to the semantic language.
For simple ontology languages like BOL, ALC, OWL, etc., it is common to use first-order logic (FOL) as the deductive semantic language.
More specifically, we use SFOL, the typed variant of FOL:

\subsection{A Basic Semantic Language: SFOL}\label{sec:wuv:tfol}
  %\section{Overview}
%
%Various methods have been developed to represent and perform inferences.
%We structure our presentation by how each method relates to computation, the aspect most whose integration with inference has drawn the most attention.
%In general, the ubiquity of underspecified function symbols and quantified variables means that logical expressions usually do not normalize to unique values.
%At best, computations like $y:=f(x)$ can be represented as open-ended conjectures where different options for $y$ are produced, each together with a proof of the respective equality.
%Therefore, inference systems usually sacrifice computation or at least its efficiency.
%
%\emph{Proof assistants} sit at the extreme end of this spectrum.
%They employ strong logics and high-level declarations to provide a convenient way to formalize domain knowledge and reason about it.
%The reasoning is usually interactive in order to represent inferences that are too difficult to be fully automated.
%Most proof assistants integrate at least some of the other methods to overcome this weakness.
%
%Further along the spectrum, \emph{automated theorem provers} use simpler logics than interactive proof assistants.
%They are fully automatic and much faster, but can handle much fewer theorems, and typically do not check their proofs.
%\emph{Satisfiability checkers} continue this progression by aiming at decidable automation support, whereas theorem proving is usually an semi-decidable search problem.
%That limits them to propositional logic or specific theories of more expressive logics (usually of first-order logic) that are complete, i.e., where every formula can be proved or disproved.
%In the special cases, where satisfiability checkers are applicable, they come close to verified computation systems.
%
%Orthogonal to the above triplet, there are several methods for realizing Turing-complete computation naturally inside a logic.
%Here imperative and object-oriented computation are usually avoided in favor of other programming paradigms that are easier to reason about.
%\emph{Rewriting} aims at optimizing the $f(x)\rewrites y$ progression, allowing users to mark specific transformations as rewrite steps.
%\emph{Terminating recursion} is the method of adding recursive functions to a logic in order to make it a pure functional programming language.
%Finally, \emph{logic programming} restricts attention to theorems of a special form, for which proof search is simple and predictable so that users can represent computations by supplying axioms that guide the proof search.

\section{Syntax}

We give typed first-order logic (SFOL) as a language system.

\begin{definition}
Fig.~\ref{fig:sfol} gives the context-free grammar.
The vocabulary symbol is $Thy$. The expression symbols are $Y$, $T$, and $F$.
\end{definition}

\begin{figure}[hbt]
\begin{commgrammar}
\gcomment{Vocabularies: theories}\\
\gprod{Thy}{\rep{D}}{}\\
\gcomment{Declarations}\\
\gprod{D}{\kw{type}\; \ID:\kw{type}}{type declaration}\\
\galtprod{\kw{fun}\; \ID:\rep{Y}\to Y}{function symbol declaration}\\
\galtprod{\kw{pred}\; \ID:\rep{Y}\to \kw{prop}}{predicate symbol declaration}\\
\galtprod{\kw{axiom}\;F}{axiom}\\
\gcomment{type expressions}\\
\gprod{Y}{\ID}{atomic type} \\
\gcomment{term expressions}\\
\gprod{T}{\ID(\rep{T})}{function symbol applied to arguments} \\
\galtprod{\ID}{term variables} \\
\gcomment{formulas expressions}\\
\gprod{F}{\ID(\rep{T})}{predicate symbol applied to arguments} \\
\galtprod{T\doteq_Y T}{equality of terms at a type} \\
\galtprod{\top}{truth} \\
\galtprod{\bot}{falsity} \\
\galtprod{F\wedge F}{conjunction} \\
\galtprod{F\vee F}{disjunction} \\
\galtprod{F\impl F}{implication} \\
\galtprod{\neg F}{negation} \\
\galtprod{\forall \ID:Y.F}{universal quantification at a type} \\
\galtprod{\exists \ID:Y.F}{existential quantification at a type} \\
\gcomment{Identifiers}\\
\gprod{\ID}{\text{alphanumeric string}}{}\\
\end{commgrammar}
\caption{Syntax of SFOL}\label{fig:sfol}
\end{figure}

\section{Semantics}

\subsection{Semantics}

\begin{definition}[Deductive Semantics of BOL]\label{def:bolsem:sfol}
The semantic prefix is the SFOL-theory containing
\begin{compactitem}
 \item a type $\iota$ (for individuals),
 \item additional types and constants corresponding to base types and values of BOL.
\end{compactitem}

Every BOL-ontology $O$ is interpreted as the SFOL-theory $P,\sem{O}$, where $\sem{O}$ is defined in Fig.~\ref{fig:bolsem:sfol}.
\end{definition}

As foreshadowed above, we can observe some general principles:
Every BOL-declaration is translated to an SFOL-declaration for the same name, and ontologies are translated declaration-wise.
For every kind of complex BOL-expression, there is one inductive function mapping BOL-expressions to SFOL-expressions.
The base cases of references to declared BOL-identifiers are translated to themselves, i.e., to the identifiers of the same name declared in the SFOL-theory.
The other cases are compositional: every case for a complex BOL-expression recurses only into the semantics of the direct subexpressions.

\begin{figure}[tbh]\centering
\begin{tabular}{l|l}
BOL Syntax $X$ & Semantics $\sem{X}$ in SFOL\\
\hline
\hline
ontology & SFOL theory \\
$D_1,\ldots,D_n$ & $\sem{D_1},\ldots,\sem{D_n}$ \\
\hline
BOL declaration & FOL declaration \\
\kw{individual}\,$i$ & nullary function symbol $i:\iota$ \\
\kw{concept}\,$i$  & unary predicate symbol $i\sq\iota$ \\
\kw{relation}\,$i$ & binary predicate symbol $i\sq\iota\times \iota$ \\
\kw{property}\,$i:T$ & binary predicate symbol $i\sq\iota\times T$ \\
$I\; \texttt{is-a}\; C$ & axiom $\sem{C}(\sem{I})$\\
$I_1\; R\; I_2$ & axiom $\sem{R}(\sem{I_1},\sem{I_2})$\\
$I\; P\; V$ & axiom $\sem{P}(\sem{I},\sem{V})$\\
$F$ & axiom $\sem{F}$\\
\hline
Formula & Formula without free variables\\
$C_1 \Equiv C_2$ & $\forall x:\iota.\sem{C_1}(x)\Leftrightarrow \sem{C_2}(x)$\\
$C_1 \sqsubseteq C_2$ & $\forall x:\iota.\sem{C_1}(x)\impl \sem{C_2}(x)$\\
$I\; \texttt{is-a}\; C$ & $\sem{C}(\sem{I})$\\
$I_1\; R\; I_2$ & $\sem{R}(\sem{I_1},\sem{I_2})$\\
$I\; P\; V$ & $\sem{P}(\sem{I},\sem{V})$\\
\hline
Individual & Terms of type $\iota$ \\
$i$ & $i$ \\
\hline
Concept & Formula with free variable $x:\iota$\\
$i$ & $i(x)$\\
$\top$ & $\true$\\
$\bot$ & $\false$\\
$C_1 \sqcup C_2$ & $\sem{C_1}(x)\vee\sem{C_2}(x)$\\
$C_1 \sqcap C_2$ & $\sem{C_1}(x)\wedge\sem{C_2}(x)$\\
$\forall R.C$    & $\forall y:\iota.\sem{R}(x,y)\impl \sem{C}(y)$\\
$\exists R.C$    & $\exists y:\iota.\sem{R}(x,y)\wedge \sem{C}(y)$\\
$\dom\, R$ & $\exists y:\iota.\sem{R}(x,y)$\\
$\rng\, R$ & $\exists y:\iota.\sem{R}(y,x)$\\
$\dom\, P$ & $\exists y:T.\sem{P}(x,y)$  \tb($T$ is type of $P$)\\
\hline
Relation & Formula with free variables $x:\iota,y:\iota$\\
$i$ & $i(x,y)$\\
$R_1 \cup R_2$ & $\sem{R_1}(x,y)\vee \sem{R_2}(x,y)$\\
$R_1 \cap R_2$ & $\sem{R_1}(x,y)\wedge \sem{R_2}(x,y)$\\
$R_1 ; R_2$ & $\exists m:\iota.\sem{R_1}(x,m)\wedge \sem{R_2}(m,y)$\\
$R^{-1}$          & $\sem{R}(y,x)$\\
$R^*$          & (tricky, omitted)\\
$\Delta_C$     & $x\doteq y\wedge \sem{C}(x)$\\
\hline
Property of type $T$ & Formula with free variables $x:\iota,y:T$\\
$i$ & $i(x,y)$\\
\end{tabular}
\caption{Interpretation Function for BOL into SFOL}\label{fig:bolsem:sfol}
\end{figure}

\clearpage

The consequence closure of SFOL, using the usual semantics of SFOL, induces the desired consequence closure for BOL:
\begin{definition}[Consequence Closure]
We say that a BOL-statement $F$ is a consequence of an ontology $O$ if $\sem{F}$ is an SFOL-theorem of $P,\sem{O}$.
%A BOL-ontology $O$ is consistent if $P,\sem{O}$ is consistent as an SFOL-theory.
\end{definition}

\begin{example}
We interpret the example ontology from Ex.~\ref{ex:bol}.
Excluding the semantic prefix, it results in
\[\mathll{
\FR:\iota,\; \WuV: \iota,\;\\
\person\sq\iota,\;\male\sq\iota,\;\instr\sq\iota,\;\crs\sq\iota,\;\\
\tch\sq \iota\times\iota,\;\hc\sq\iota\times \float \\
\instr(\FR)\wedge\male(\FR),\;\crs(\WuV),\;\tch(\FR,\WuV),\;\hc(\WuV,7.5)\\
\forall x:\iota.\male(x)\impl\person(x),\\
\forall x:\iota.\instr(x)\impl\person(x),\\
\forall x:\iota.(\exists y:\iota.\tch(x,y))\impl\instr(x),\\
\forall x:\iota.(\exists y:\iota.\tch(y,x))\impl\crs(x),\\
\forall x:\iota.(\exists y:\iota.\tch(y,x))\Darr\crs(x),\\
\forall x:\iota.\crs(x)\impl\exists y:\iota.\tch(y,x)\wedge\instr(y)
}\]
\end{example}

\begin{example}[Compositionality]\label{ex:compositional}
The interpretation of BOL is compositional.

For example, consider the case of composition of relations:
 \[\sem{R_1 ; R_2}= \exists m:\iota.\sem{R_1}(x,m)\wedge \sem{R_2}(m,y)\]
Here we have an expression $E(e_1,\ldots,e_n)$ with $n=2$ and $E$ is the $;$-operator mapping $(e_1,e_2)\mapsto e_1;e_2$, i.e., $R_1$ and $R_2$ are the direct subexpressions of $R_1;R_2$.
The semantics is a relatively complicated FOL-formula, but it only depends on $\sem{R_1}$ and $\sem{R_2}$ --- everything else is fixed.
We have $\sem{;}=(p_1,p_2)\mapsto \exists m:\iota.p_1(x,m)\wedge p_2(m,y)$, i.e., the interpretation of the $;$-operator is the function that maps two predicates $p_1,p_2$ to the formula $\exists m:\iota.p_1(x,m)\wedge p_2(m,y)$.
Then we have \[\sem{R_1;R_2}=\sem{;}(\sem{R_1},\sem{R_2}).\]
\end{example}

\begin{example}[Non-Compositional Translation via Custom Semantic Prefix]\label{ex:noncompositional}
In Fig.~\ref{fig:bolsem:sfol}, we omitted the case for the transitive closure.
That was because it is not possible to translate it compositionally into FOL.
We can only do it non-compositionally with a custom semantic prefix:

We define the FOL-interpretation of an ontology $O$ by $\sem{O}=P_O,\sem{O}$, where $P_O$ is a custom semantic prefix.
$P_O$ is different for every ontology $O$ and is defined as follows:

\begin{compactenum}
 \item We scan through $O$ and collect all occurrences of $R^*$ for any (not necessarily atomic) relation $R$.
 \item $P_O$ contains the following declarations for each $R$:
  \begin{compactitem}
  \item A binary predicate symbol $C_R\sq i\times i$. Note that $R$ may be a complex expression; so we have to generate a fresh name $C_R$ here.
  \item The axiom $\forall x:\iota,y:\iota.\,R(x,y)\impl C_R(x,y)$, i.e., $C_R$ extends $R$.
  \item The axiom $\forall x:\iota,y:\iota,z:\iota.\,C_R(x,y)\wedge C_R(y,z)\impl C_R(x,z)$, i.e., $C_R$ is transitive.
  \end{compactitem}
 \item We add the case $\sem{R^*}=C_R(x,y)$ to the interpretation function.
\end{compactenum}

Intuitively, every occurrence of the $^*$-operator is removed from the language and replaced with a fresh name that is axiomatized to have the needed properties.
All of these axioms are added to the semantic prefix.
\end{example}

\section{Concretized Semantics}\label{sec:bolsem:conc}

We give an alternative semantics using a semantic language for concrete data.
Specifically we use the database language SQL.

\subsection{An SQL-Inspired Basic Database Language}\label{sec:wuv:bdl}
  We give an SQL-like database language as a formal system.

\begin{definition}
Fig.~\ref{fig:bdl} gives the context-free grammar.
The vocabulary symbol is $S$. The expression symbols are $T$, $R$, $V$, and $F$.
\end{definition}

\begin{figure}[hbt]
\begin{commgrammar}
\gcomment{Vocabularies: Schemas}\\
\gprod{S}{\rep{D}}{}\\
\gcomment{Declarations}\\
\gprod{D}{\kw{TABLE}\; \ID\;TT}{table}\\
\galtprod{\kw{INSERT}\,R\,\kw{INTO} \,\ID}{row in a table}\\
\gcomment{table types}\\
\gprod{TT}{\rep{CT}}{list of column types}\\
\gprod{CT}{\ID:Y}{column type}\\
\gcomment{table expressions (i.e., table-valued queries)}\\
\gprod{T}{\ID}{atomic tables} \\
\galtprod{\kw{JOIN}\,\rep{T}}{join of tables}\\
\galtprod{\kw{UNION}\,\rep{T}}{union of tables}\\
\galtprod{\kw{INTER}\,\rep{T}}{intersection of tables}\\
\galtprod{\kw{SELECT} \opt{\kw{DISTINCT}}\, CL\,\kw{FROM}\, T}{selection of columns}\\
\galtprod{T\,\kw{WHERE}\, F}{selection of rows}\\
\galtprod{T\,\kw{AS}\,\ID}{prefix for column names}\\
\galtprod{T\;\kw{AGGREGATE}\,\rep{CA}}{aggregation}\\
\gprod{CA}{\kw{COUNT}(\ID)}{count values in a column}\\
\galtprod{\kw{MAX}(\ID)}{maximum value in a column}\\
\galtprod{\ldots}{minimum, sum, etc.}\\
\gprod{CL}{* \bnfalt \brep{\ID \bnfalt \ID\,\kw{AS}\,\ID}}{list of column names}\\
\gcomment{row expressions (i.e., single row--valued queries)}\\
\gprod{R}{\kw{VALUES}\,(\rep{CD})}{explicit row} \\
\galtprod{T}{table, but may only contain one row} \\
\gprod{CD}{\ID=V}{column definition}\\
\gcomment{cell expressions (i.e., single column, single row--valued queries)}\\
\gprod{V}{R}{row, but may only contain one column} \\
\galtprod{\ID(\rep{V})}{built-in function symbol applied to values} \\
\galtprod{(base\,values)}{as for BOL} \\
\gcomment{base types}\\
\gprod{Y}{(base\,types)}{as for BOL} \\
\gcomment{formulas}\\
\gprod{F}{V}{boolean value}\\
\galtprod{V=V}{equality of values}\\
\galtprod{R\,\kw{IN}\,T}{containment of rows in tables}\\
\galtprod{R\,\kw{NOT IN}\,T}{opposite of containment}\\
\galtprod{\ldots}{boolean operators}\\
\gcomment{Identifiers}\\
\gprod{\ID}{\text{alphanumeric string}}{}\\
\end{commgrammar}
\caption{Syntax of SQL}\label{fig:bdl}
\end{figure}


\subsection{Semantics}

Even though this is a very different knowledge aspect, the general principles of the semantics are the same:
Every BOL-declaration is translated to an SQL declaration, and ontologies are translated declaration-wise.
For every kind of complex expression, there is one inductive function mapping BOL-expressions to SQL-expressions.

In SQL, we can nicely see the difference between declarations and expressions: the former are translated to side effect-ful statements, the latter to side effect-free queries.

\begin{definition}[Concretized Semantic of BOL]\label{def:bolsem:sql}
The \textbf{semantic prefix} consists of the following SQL-statements
\begin{compactitem}
 \item a type $ID$ of identifiers (if not already supported anyway by the underlying database)
 \item declarations of all base types and values of BOL (if not already supported anyway by the underlying database)
 \item TABLE individuals (id: ID, name: string), where the id field is unique and automatically generated when inserting values
\end{compactitem}

Every BOL-ontology $O$ is interpreted as the sequence $P,\sem{O}$ of SQL statements, where $\sem{O}$ is defined in Fig.~\ref{fig:bolsem:sql}.
\end{definition}

\begin{figure}\centering
\begin{tabular}{l|l}
BOL Syntax $X$ & Semantics $\sem{X}$ in SQL\\
\hline
\hline
ontology & SQL schema (list of statements)\\
$D_1,\ldots,D_n$ & $\sem{D_1},\ldots,\sem{D_n}$ \\
\hline
BOL declaration ($C$, $R$, $P$ atomic) & SQL statement \\
\kw{individual}\,$i$ & INSERT VALUES (name="$i$") INTO individuals \\
\kw{concept}\,$i$  & TABLE $i$ (id: ID)\\
\kw{relation}\,$i$ & TABLE $i$ (subject: ID, object: ID) \\
\kw{property}\,$i:Y$ & TABLE $i$ (subject: ID, value: $Y$) \\
$I\; \texttt{is-a}\; C$ & INSERT VALUES (id=$\sem{I}$) INTO $C$\\
$I_1\; R\; I_2$ & INSERT VALUES (subject=$\sem{I_1}$, object=$\sem{I_2}$) INTO R\\
$I\; P\; V$ & INSERT VALUES (subject=$\sem{I}$, value=$V$) INTO P\\
$F$ & consistency check, consequence closure (omitted)\\
\hline
Formula & Query that returns empty result iff formula is true \\
$C_1 \Equiv C_2$ & $(\sem{C_1}\sm\sem{C_2})$ UNION $(\sem{C_2}\sm\sem{C_1})$\\
$C_1 \sqsubseteq C_2$ & $\sem{C_1}\sm\sem{C_2}$\\
$I\; \texttt{is-a}\; C$ & $\sem{I}$ IN $\sem{C}$\\
$I_1\; R\; I_2$ & ($\sem{I_1}$, $\sem{I_2}$) IN $R$ \\
$I\; P\; V$ & ($\sem{I}$, $V$) IN $P$ \\
\hline
Individual & an identifier from the table individuals \\
$i$ & SELECT id FROM individuals WHERE name="$i$" \\
\hline
Concept & SQL table expression for $(id: ID)$\\
$i$ & SELECT * FROM $i$\\
$\top$ & individuals\\
$\bot$ & individuals WHERE false\\
$C_1 \sqcup C_2$ & $\sem{C_1}$ UNION $\sem{C_2}$\\
$C_1 \sqcap C_2$ & $\sem{C_1}$ INTERSECT $\sem{C_2}$\\
$\forall R.C$    & individuals $\sm$ (SELECT subject FROM $\sem{R}$ WHERE object NOT IN $\sem{C})$ \\
$\exists R.C$    & SELECT DISTINCT subject FROM $\sem{R}$, $\sem{C}$ WHERE object=id\\
$\dom\, R$ & SELECT DISTINCT subject FROM $\sem{R}$\\
$\rng\, R$ & SELECT DISTINCT object FROM $\sem{R}$\\
$\dom\, P$ & SELECT DISTINCT subject FROM $\sem{P}$\\
\hline
Relation & SQL table expression for (subject: ID, object: ID)\\
$i$ & SELECT * FROM $i$\\
$R_1 \cup R_2$ & $\sem{R_1}$ UNION $\sem{R_2}$\\
$R_1 \cap R_2$ & $\sem{R_1}$ INTERSECT $\sem{R_2}$\\
$R_1 ; R_2$ & SELECT DISTINCT l.subject, r.object FROM $\sem{R_1}$ AS l, $\sem{R_2}$ AS r \\
            & \tb\tb WHERE l.object = r.subject\\
$R^{-1}$          & SELECT object AS subject, subject AS object FROM $\sem{R}$\\
$R^*$          & (tricky, omitted)\\
$\Delta_C$     & SELECT id AS subject, id AS object FROM $\sem{C}$\\
\hline
Property of type $Y$ & SQL table expression for (subject: ID, value: Y)\\
$i$ & SELECT * FROM $i$\\
\end{tabular}
\medskip

Using the abbreviations: $S\sm T$ = SELECT id FROM $S$ WHERE id NOT IN $T$\\
$S,T$ = JOIN $S,T$\\
\caption{Interpretation Function for BOL into SQL}\label{fig:bolsem:sql}
\end{figure}
\clearpage

\begin{remark}[Limitations]
Our interpretation of BOL in SQL is restricted to assertions using only atomic expressions.
For example, in the case for $I$ \texttt{is-a} $C$, we assume that $I$ and $C$ are names.
Thus, we have already created an individual for $I$ and a table for $C$, and we can thus insert the former into the latter.
The general case would be more complicated but is much less important in practice.
But other expressions very quickly become more difficult.

The interpretation of formulas into SQL is less obvious because SQL is not a logic and therefore does not define a consequence closure.
Thus, we can only use axioms for consistency checks in SQL.
But that requires first carrying out an explicit consequence closure that adds all implied assertions to the database.
\end{remark}

\begin{example}
We interpret the example ontology from Ex.~\ref{ex:bol}.
Excluding the semantic prefix, the entity declarations and assertions result in the following
\begin{lstlisting}
INSERT VALUES (name="FlorianRabe") INTO individuals
INSERT VALUES (name="WuV") INTO individuals
TABLE person (id: ID)
TABLE male (id: ID)
TABLE instructor (id: ID)
TABLE course (id: ID)
TABLE teach (subject: ID, object: ID)
TABLE creditValue (subject: ID, value: float)
INSERT VALUES (id=2) INTO course
INSERT VALUES (subject=1, object=2) INTO teach
INSERT VALUES (subject=1, value=7.5) INTO creditValue
\end{lstlisting}

Here we assume that inserting into the table individuals has automatically assigned the ids $1$ and $2$ to our two individuals.

The concept assertion about $\FR$ using $\sqcap$ cannot be handled by this semantics.
Therefore, we skip that assertion.
The two missing assertions
\begin{lstlisting}
INSERT VALUES (id=1) INTO instructor
INSERT VALUES (id=1) INTO male
\end{lstlisting}
must then be provided by performing the consequence closure.

Moreover, the axioms result in the following consistency checks, i.e., queries that should be empty:
\begin{lstlisting}
SELECT * FROM male $\sm$ SELECT * FROM person
SELECT * FROM instructor $\sm$ SELECT * FROM person
SELECT * FROM (SELECT DISTINCT subject FROM teach) $\sm$ SELECT * FROM instructor
SELECT * FROM (SELECT DISTINCT object FROM teach) $\sm$ SELECT * FROM course
(SELECT * FROM (SELECT DISTINCT subject FROM creditValue) $\sm$ SELECT * FROM course)
   UNION (SELECT * FROM course $\sm$ SELECT DISTINCT subject FROM creditValue)
SELECT * FROM course $\sm$
  (SELECT DISTINCT subject
   FROM (SELECT object AS subject, subject AS object FROM teach), instructor
   WHERE object=id)
\end{lstlisting}
Some of these checks will only succeed after performing the consequence closure.
In particular, the table \verb|person| misses the entry $1$ for the individual $\FR$ because the assertion {\FR\;\texttt{is-a}\;\person} is only present as a consequence
\end{example}

\section{Computational Semantics}\label{sec:bolsem:comp}

We give an alternative semantics using computation, i.e., by using a programming language as the semantic language.
Specifically, we use the programming language Scala.

\subsection{A Scala-Inspired Basic Programming Language}\label{sec:wuv:bpl}
  We give a simple programming language as a language system.

\begin{definition}
Fig.~\ref{fig:bpl} gives the context-free grammar.
The vocabulary symbol is $P$. The expression symbols are $Y$, $V$, and $F$.
\end{definition}

\begin{figure}[hbt]
\begin{commgrammar}
\gcomment{Vocabularies: Programs}\\
\gprod{P}{\rep{D}}{}\\
\gcomment{Declarations}\\
\gprod{D}{\kw{class}\; \ID[\rep{X}]\,\kw{extends}\,\rep{\ID}\{\rep{d}\}}{class definition}\\
\galtprod{\kw{object}\; \ID\,\kw{extends}\,\rep{\ID}\{\rep{d}\}}{object definition}\\
\gprod{d}{\kw{val}\;\ID:Y\opt{=T}}{immutable field in a class/object, possibly abstract}\\
\galtprod{\kw{var}\;\ID:Y=T}{mutable field in a class, with initial value}\\
\gcomment{Type expressions}\\
\gprod{Y}{\ID[\rep{X}]}{atomic type (class) applied to type arguments} \\
\galtprod{\ID}{built-in type (booleans, int, etc.)} \\
\galtprod{X}{type variable} \\
\galtprod{Y=>Y}{function types} \\
\gcomment{Term expressions}\\
\gprod{T}{\ID}{atomic value (class, value, variable)} \\
\galtprod{\ID}{built-in value (boolean operators, etc.)} \\
\galtprod{T.\ID}{field access in an object} \\
\galtprod{T.\ID=T}{assignment to a mutable field in an object}\\
\galtprod{\ID=T}{assignment to a local variable}\\
\galtprod{T:Y}{instance check} \\
\galtprod{\kw{new}\,\ID\,\{\rep{d}\}}{new instance of class} \\
\galtprod{T\doteq_Y T}{equality of terms at a type} \\
\galtprod{(\ID:Y)=>T}{function} \\
\galtprod{T(\rep{T})}{function applied to values} \\
\galtprod{\{\rep{T}\}}{sequencing (;-operator)}\\
\galtprod{d}{local declaration}\\
\galtprod{\kw{if}\,(T)\, T\,\kw{else}\, T}{if-then-else}\\
\galtprod{\kw{while}\,(T)\, T}{while-loop}\\
\gcomment{Formula expressions}\\
\gprod{F}{T}{terms of boolean type} \\
\gcomment{Identifiers}\\
\gprod{\ID}{\text{alphanumeric string}}{}\\
\end{commgrammar}
\caption{Syntax of BPL}\label{fig:bpl}
\end{figure}


\subsection{Semantics}

Again, the general principles are the same:
Every BOL-declaration is translated to a Scala-declaration, and ontologies are translated declaration-wise to Scala-programs.
For every kind of complex expression, there is one inductive function mapping BOL-expressions to Scala-objects.

\begin{definition}[Computational Semantic of BOL]\label{def:bolsem:scala}
The \textbf{semantic prefix} consists of the following Scala statements
\begin{compactitem}
 \item classes for all BOL-base types and values for them (if not already present in Scala)
 \item classes for individuals and hash sets of objects:
\begin{lstlisting}
import scala.collection.mutable.HashSet
val individuals = new HashSet[String]
\end{lstlisting}
\end{compactitem}

Every BOL-ontology $O$ is interpreted as the Scala program $P,\sem{O}$, where $\sem{O}$ is defined in Fig.~\ref{fig:bolsem:scala}.
\end{definition}

\newcommand{\rA}{\Rightarrow}

\begin{figure}\centering
\begin{tabular}{l|l}
BOL Syntax $X$ & Semantics $\sem{X}$ in Scala\\
\hline
\hline
ontology & Scala program \\
$D_1,\ldots,D_n$ & $\sem{D_1},\ldots,\sem{D_n}$ \\
\hline
BOL declaration ($C$, $R$, $P$ atomic) & Scala declaration \\
\kw{individual}\,$i$ & val $i$ = "$i$"; individuals += $i$ \\
\kw{concept}\,$i$  & val $i$ = new HashSet[String]\\
\kw{relation}\,$i$ & val $i$ = new HashSet[(String,String)] \\
\kw{property}\,$i:T$ & val $i$ = new HashSet[(String,T)] \\
$I\; \texttt{is-a}\; C$ & $\sem{C}$ += $\sem{I}$\\
$I_1\; R\; I_2$ & $\sem{R}$ += $(\sem{I_1},\sem{I_2})$\\
$I\; P\; V$ & $\sem{P}$ += $(\sem{I},\sem{V})$\\
$F$ & assertions, consequence closure (omitted)\\
\hline
Formula & Program that evaluates the formula to a Boolean \\
$C_1 \Equiv C_2$ & \{val $c1$ = $\sem{C_1}$; val $c2$ = $\sem{C_2}$; \\
                 & \tb $c_1$.forall(x $\rA$ $c_2$.contains(x)) \&\& $c_2$.forall(x $\rA$ $c_1$.contains(x))\}\\
$C_1 \sqsubseteq C_2$ & \{val $c1$ = $\sem{C_1}$; val $c2$ = $\sem{C_2}$; $c_1$.forall(x $\rA$ $c_2$.contains(x))\}\\
$I\; \texttt{is-a}\; C$ & $\sem{C}$.contains($\sem{I}$)\\
$I_1\; R\; I_2$ & $\sem{R}$.contains($(\sem{I_1},\sem{I_2})$)\\
$I\; P\; V$ & $\sem{P}$.contains($(\sem{I},\sem{V})$)\\
\hline
Individual & String object\\
$i$ & $i$ \\
\hline
Concept & HashSet[String] object\\
$i$ & $i$\\
$\top$ & individuals\\
$\bot$ & HashSet[String].empty()\\
$C_1 \sqcup C_2$ & $\sem{C_1}$.union($\sem{C_2}$)\\
$C_1 \sqcap C_2$ & $\sem{C_1}$.inter($\sem{C_2}$)\\
$\forall R.C$    & \{val c = $\sem{C}$; val r = $\sem{R}$; val e = individuals.clone; \\
                 & \tb r.foreach(x $\rA$ if (!c.contains(x.\_2)) e -= x.\_1); e\} \\
$\exists R.C$    & \{val c = $\sem{C}$; val r = $\sem{R}$; val e = new HashSet[String]; \\
                 & \tb r.foreach(x$ \rA$ if (c.contains(x.\_2)) e += x.\_1); e\}\\
$\dom\, R$ & \{val c = new HashSet[String]; $\sem{R}$.foreach(x $\rA$ c += x.\_1); c\}\\
$\rng\, R$ & \{val c = new HashSet[String]; $\sem{R}$.foreach(x $\rA$ c += x.\_2); c\}\\
$\dom\, P$ & \{val c = new HashSet[String]; $\sem{P}$.foreach(x $\rA$ c += x.\_1); c\}\\
\hline
Relation & HashSet[(String,String)] object\\
$i$ & $i$\\
$R_1 \cup R_2$ & $\sem{R_1}$.union($\sem{R_2}$)\\
$R_1 \cap R_2$ & $\sem{R_1}$.inter($\sem{R_2}$)\\
$R_1 ; R_2$ &  \{val r1 = $\sem{R_1}$; val r2 = $\sem{R_2}$; val e = new HashSet[(String,String)]; \\
            & \tb r1.foreach(x $\rA$ r2.foreach(y $\rA$ if (x.\_2 == y.\_1) e += (x.\_1,y.\_2))); e\}\\
$R^{-1}$    & \{val r = new HashSet[(String,String)]; $\sem{R}$.foreach(x $\rA$ r += (x.\_2,x.\_1)); r\}\\
$R^*$          & (omitted)\\
$\Delta_C$     & \{val r = new HashSet[(String,String)]; $\sem{C}$.foreach(x $\rA$ r += (x,x)); r\}\\
\hline
Property of type $T$ & HashSet[(String,T)] object\\
$i$ & $i$\\
\end{tabular}
\caption{Interpretation Function for BOL into Scala}\label{fig:bolsem:scala}
\end{figure}

\begin{remark}[Scala Syntax]
In Scala, val $x=e$ evaluates $e$ and stores the result in $x$.
$\{d_1; \ldots; d_n\}$ is evaluated by executing all $d_i$ in order and returning the result of $d_n$.

$(A,B)$ is the product type $A\times B$ with pairing operator $(x,y)$ and projection functions $\_1$ and $\_2$. $x\rA F(x)$ is $\lambda x.F(x)$.

The class HashSet is part of the standard library and offers function += and -= to add/remove elements, contains to test elementhood, and forall, foreach to quantifiy/iterate over elements.

Types of variables are inferred if omitted.
\end{remark}

\begin{remark}[Limitations]
Our interpretation of BOL in Scala has similar problems as the one in SQL.
We restrict entities in assertions to be atomic.
And we assume that all assertions implied by the consequence closure have already been obtained and added to the ontology.
\end{remark}

\begin{example}
We interpret the example ontology from Ex.~\ref{ex:bol}.
Excluding the semantic prefix, the entity declarations and assertions result in the following
\begin{lstlisting}
individuals += "FlorianRabe"
individuals += "WuV"
val person = new HashSet[String]
val male = new HashSet[String]
val person = new HashSet[String]
val course = new HashSet[String]
val teach = new new HashSet[(String,String)]
val creditValue = new HashSet[(String,float)]
course += WuV
teach += ("FlorianRabe", WuV)
creditValue += (WuV, 7.5)
\end{lstlisting}

Like for SQL, the two statements
\begin{lstlisting}
instructor += "FlorianRabe"
male += "FlorianRabe"
\end{lstlisting}
must be obtained by consequence closure because we cannot handle the $\sqcap$ assertion.
Note that we could easily compute the hash set \verb|instructor.diff(male)| and add to it.
But that would not add anything to the two constituent sets.

If we thing of the axioms as consistency checks, we can translate them to assertions, i.e., Boolean expressions that must be true.
We only give some examples:
\begin{lstlisting}
{val c1 = male; val c2 = person; c1.forall(x $\rA$ c2.contains(x))}

{
  val c1 = course;
  val c2 = {
    val c = instructor;
    val r = {
      val r = new HashSet[(String,String)];
      teach.foreach(x $\rA$ r += (x._2,x._1));
      r
    }
    val e = new HashSet[String];
    r.foreach(x $\rA$ if (c.contains(x._2)) e += x._1);
    e
  };
  c1.forall(x $\rA$ c2.contains(x))
}
\end{lstlisting}

%instructor $\sqsubseteq$ person
%dom teach $\sqsubseteq$ instructor
%rng teach $\sqsubseteq$ course
%dom hasCredits $\Equiv$ course
%course $\sqsubseteq$ $\exists$ teach$^{-1}$ instructor

\end{example}

\section{Narrative Semantics}\label{sec:bolsem:narr}

We give an alternative semantics using narration, i.e., by using a natural language as the semantic language.
Specifically, we use the natural language English.

Again, the general principles are the same:
Every BOL-declaration is translated to an English sentence, and ontologies are translated declaration-wise to English texts.
For every kind of complex expression, there is one inductive function mapping BOL-expressions to English phrases.

\begin{definition}[Narrative Semantic of BOL]\label{def:bolsem:eng}
The \textbf{semantic prefix} consists of English statements explaining
\begin{compactitem}
 \item the base types of BOL (if they are not universally known),
 \item that we rely on a lexicon to correctly form plurals (indicated by -s) and verb forms (indicated by -s, -ing, -ed).
\end{compactitem}

Every BOL-ontology $O$ is interpreted as the English text $P,\sem{O}$, where $\sem{O}$ is defined in Fig.~\ref{fig:bolsem:eng}.
\end{definition}

%It recursively goes through the BOL expressions and generates strings.
%It uses a lexicon function $lex$ (for lexicon) that maps BOL identifiers into (lemmata $\hat=$ base forms of) English words of the appropriate syntactic category for the base cases.

% @MK: The lexicon corresponds to the model of a logical semantics, and I've defered its explanation to do it jointly later.

\begin{figure}\centering
\begin{tabular}{l|l}
BOL Syntax $X$ & Semantics $\sem{X}$ in English\\
\hline\hline
ontology & English text \\
$D_1,\ldots,D_n$ & $\sem{D_1},\ldots,\sem{D_n}$ \\
\hline
BOL declaration & dictionary entry or true sentence\\
\kw{individual}\,$i$ & $i$ is a proper noun.\\
\kw{concept}\,$i$  & $i$ is a common noun.\\
\kw{relation}\,$i$ & $i$ is a transitive verb. \\
\kw{property}\,$i:T$ & $i$ is a common noun for a property that can take $T$-values. \\
$I\; \texttt{is-a}\; C$ & $\sem{I}$ $\sem{C}$.\\
$I_1\; R\; I_2$ & $\sem{I_1}$ $\sem{R}$ $\sem{I_2}$.\\
$I\; P\; V$ & $\sem{I}$ has $\sem{P}$ $\sem{V}$.\\ 
$F$ & $\sem{F}$.\\
\hline
Formula & sentence \\
$C_1 \Equiv C_2$ & $\sem{C_1}$ is the same as $\sem{C_2}$.\\
$C_1 \sqsubseteq C_2$ & everything that $\sem{C_1}$s also $\sem{C_2}$s.\\
$I\; \texttt{is-a}\; C$ & $\sem{I}$ $\sem{C}$.\\
$I_1\; R\; I_2$ & $\sem{I_1}$ $\sem{R}$s $\sem{I_2}$.\\
$I\; P\; V$ & $\sem{I}$ has $\sem{P}$ $\sem{V}$.\\ 
\hline
Individual & noun phrase (to be used as a subject or object)\\
$i$ & $i$ \\
\hline
Concept & intransitive verb phrase (to be plugged after a subject) \\
$i$ & is a $i$\\
$\top$ & is anyone\\
$\bot$ & is no one\\
$C_1 \sqcup C_2$ & $\sem{C_1}$ or $\sem{C_2}$\\
$C_1 \sqcap C_2$ & $\sem{C_1}$ and $\sem{C_2}$\\
%$C_1 \sqcap C_2$ & is a $\sem{C_1}$ that $\sem{C_2}$s & $C_1$ simple\\
$\forall R.C$    & $\sem{R}$s only things that $\sem{C}$ \\
$\exists R.C$    & $\sem{R}$s something that $\sem{C}$\\
$\dom\, R$ & $\sem{R}$s something\\
$\rng\, R$ & is $\sem{R}$ed by something\\
$\dom\, P$ & has some $\sem{P}$\\
\hline
Relation & transitive verb phrase (to be plugged between subject and object)\\
$i$ & $i$\\
$R_1 \cup R_2$ & $\sem{R_1}$s or $\sem{R_2}$s\\
$R_1 \cap R_2$ & $\sem{R_1}$s and $\sem{R_2}$s\\
$R_1 ; R_2$ & $\sem{R_1}$s something that $\sem{R_2}$s\\
$R^{-1}$    & is $\sem{R}$ed by\\
$R^*$       & $\sem{R}$s something that $\sem{R}$s something and so on that $\sem{R}$s\\
$\Delta_C$  & $\sem{C}$s and is the same as\\
\hline
Property of type $T$ & property phrase\\
$i$ & $i$\\
\end{tabular}
\caption{Interpretation Function for BOL into English (intransitive VP version)}\label{fig:bolsem:eng}
\end{figure}

Natural language defines a consequence closure by appealing to consequence in natural language.
That is well-defined as long as we express ourselves precisely enough.
\begin{definition}[Consequence Closure]
We say that a BOL-statement $F$ is a consequence of an ontology $O$ if $\sem{F}$ is a consequence of $P,\sem{O}$.
\end{definition}

\begin{example}
We interpret the example ontology from Ex.~\ref{ex:bol}.
Excluding the semantic prefix and the lexicon lookup, it results in the following text:
\medskip

FlorianRabe is a proper noun.\\
WuV is a proper noun.\\
person is a common noun.\\
male is a common noun.\\
instructor is a common noun.\\
course is a common noun.\\
teach is a transitive verb.\\
creditValue is a common noun for a property that can take $\float$-values.\\
FlorianRabe is a instructor and is a male.\\
WuV is a course.\\
FlorianRabe teachs WuV.\\
WuV has creditValue 7.5.\\
everything that is a male also is a person.\\
everything that is a instructor also is a person.\\
everything that teachs is a instructor.\\
everything that is teached by something is a instructor.\\
has some creditValue is the same as is a course.\\
everything that is a course also is teached by something that is a instructor.
\medskip

This English is very clunky of course.
Multiple tweaks would be needed to get the grammar right:
\begin{compactitem}
 \item It is "teaches" and "taught" instead of "teachs" and "teached",
 \item It is "an instructor" instead of "a instructor",
 \item Sentences start with upper case letters.
 \item Proper nouns often have different names in the ontology than in reality, e.g., it should be "Florian Rabe" and "credit value" instead of "FlorianRabe" and "creditValue".
\end{compactitem}
Moreover, the language could be polished in many places.
For example, "is a instructor and is a male" could become "is a instructor and a male" with a relatively easy special case treatment, or it could become "is a male instructor" with a more complex semantics that interprets some concepts via nouns and some via adjectives.
\end{example}

%\begin{example}
%  Consider an ontology with declarations $\kw{individual}\; P$, $\kw{individual}\; M$, $\kw{relation}\; l$, $\kw{relation}\; h$ and a lexicon function $P\mapsto\text{Peter}, M\mapsto\text{Mary}, l\mapsto\text{love}$, then  $\sem{P\; l\; M}$ = ``Peter loves Mary''. 
%\end{example}
%
%\begin{example}
%  If we extend the ontology above with $\kw{concept} d$ and $\kw{concept} c$ and the lexicon with $lex(d)=\text{is a dog}$ and $lex(c)=\text{is a cat}$, then $M\; \mathtt{is-a}\; c$ translates to ``Mary is a cat'', and  $P\; \mathtt{is-a}\; c\sqcap d$ translates to ``Peter is a cat or is a dog''.
%\end{example}

\begin{remark}[Variants of English]
We are relatively open as to what kind of English we want to use as the semantic language.
The simplest choice would be to use plain English as you could find in a novel or newspaper article.
But for many applications (e.g., formal ontologies in the STEM fields), we would rather use STEM English, i.e., English interspersed with formulas, diagrams, and epistemic cues like ``Definition'', ``Theorem'', ``Proof'', and even $\Box$.
For this kind of English, \latex is a good target format.
We can even use special \latex dialects like \sTeX~\cite{stex} where we can capture more of the semantic properties.
\end{remark}

\begin{remark}[Better Language Generation]
While the target languages in the other translations are formal languages engineered for regularity and simplicity (in terms of language primitives), natural languages have evolved in practical human communication.
As a consequence, the translation in Def.~\ref{def:bolsem:eng} results in English that is clumsy at best and non-grammatical in general.
We can think of the result as \textbf{BOL-pidgin} English.

Let us have a look at some of the problems that appear in both translations:
\begin{itemize}
\item We need a lexicon to obtain inflection information and the translation tries to remedy that by appending ``s'' in various places.
This works in some cases but not in others.
%\item We have introduced some pattern matching and conditional rules in \Cref{fig:bolsem:eng} to alleviate the greatest awkwardnesses, but much more would be needed, which would lead to things like ``Peter hass Mary'' or ``Peter has as childs Mary''. 
\item there are many linguistic devices that serve an important role in natural language, but which we are not targeting.
An example is plural objects for aggregation.
Say we have $P \mathtt{is-a} C$, $M \mathtt{is-a} C$, this would translate to ``P is a $\sem{C}$, M is a $\sem{C}$'' in BOL-pidgin, whereas in natural English we would aggregate this to ``P and M are $\sem{C}$s''.
\end{itemize}

A way out is to utilize special systems for dealing with the surface structure of natural language.
An example of this is the Grammatical Framework~ (GF, \cite{gf}): it allows specifying a rich formal language of \textbf{abstract syntax trees} for natural language (ASTs) together with language-specific \textbf{linearizations}, which amount to recursive functions that translate ASTs to language-specific strings.
GF comes with  a large resource library that provides a comprehensive, language-independent AST specification and linearizations for over 35 languages.
We will not pursue this here, but there is a special course ``Logic-based Natural Language Semantics'' at FAU in the Winter Semesters that covers these and related topics.
One of the major issues that need to be addressed there and here is the notion of compositionality, which is central to all processing and semantics. We will address it next, and come back to it time and again later. 
\end{remark}


%\section{Exercise 2}
%
%Implement the syntax and semantics of BOL.
%
%You can choose the programming language to use. We will use Scala in our examples.
%
%You can choose which semantics to implement.
%The ones that translate directly to Scala or to English are easier because it does not require implementing the syntax of the target language as well.
%The ones that translate to FOL or to SQL require an implementation of the syntax of the respective target language.
%You have to implement that as well or use a library for it.
%
%We recommend not focusing on implementing the syntax and semantics in their entirety.
%It is more instructive to save time by choosing a sublanguage of BOL (by omitting some productions) and to use the time to implement a second semantics.

\chapter{Type Systems for Ontology Languages}\label{sec:onto:type}
   \section{Intrinsic vs. Extrinsic Typing}

\subsection{Overview}

We write $x:A$ to say that $x$ has type $A$.
There are two fundamentally different methods for introducing the types $A$, which are summarized in the following table:

\begin{center}
\begin{tabular}{l|ll}
& intrinsic & extrinsic \\
\hline
goes back to $\lambda$-calculus by & Church & Curry \\
general idea & objects carry their type with them & types are designated by the environment \\
typing is a & function from objects to types & relation between objects and types \\
objects have & unique type & any number of types \\
types often interpreted as & disjoint sets & unary predicates on a universal set \\
\hline
type inference for $x$ & uniquely infer $A$ from $x$ & try to find minimal $A$ such that $x:A$ \\
type checking & compare inferred and expected type & prove $x:A$ \\
subtyping $A<:B$ & mimicked by casting from $A$ to $B$ & defined by $x:A$ implies $x:B$ for all $x$ \\
typing decidable & yes unless too expressive & no unless expressivity restricted \\
typing errors are detected & usually statically (compile-time) & dynamically (run-time)\\
\hline
type of name introduced as & part of declaration & additional axiom \\
 \tb example               &  \kw{individual} "WuV":"course"  & \kw{individual} "Wuv", \; "WuV" \texttt{is-a} "course"\\
\hline
advantages & easy & flexible \\
           & unique type inference & allows subtyping \\
\hline
examples   & SFOL, SQL & OWL, Scala, English \\
           & most logics, functional PLs & ontology, OO, natural languages \\
           & many type theories & set theories
\end{tabular}
\end{center}

\begin{example}[Extrinsically Typed Ontology Language]
In BOL, the objects are the individuals, the types are the concepts, and \texttt{is-a} is the typing relation between them.
The typing is intrinsic:
\begin{compactitem}
 \item Individuals and their concept assertions are introduced in separate declarations.
 \item An individual may be an instance of any number of concepts.
 \item There is no primary concept that could be returned as the inferred type of an individual.
 \item Concepts are subject to subtyping $C\sqsubseteq C'$.
 \item Whether an individual is an instance of a concept, must be checked by reasoning about the \texttt{is-a} relation.
\end{compactitem}

Therefore, all semantics must interpret individuals as elements of a universal collection, and types as unary predicates on that.
Specifically, we have
\begin{center}
\begin{tabular}{l|lll}
semantics in  & universal collection & unary predicate & typing relation $i$ \texttt{is-a} $c$\\
\hline
FOL & type $\iota$  & predicate $c\sq\iota$ & $c(i)$ true\\
SQL & table Individuals & table containing ids & id of i in table $c$ \\
Scala & String & hash set of strings & $c$.contains($i$) \\
English & proper nouns & common nouns & "$i$ is a $c$" is true
\end{tabular}

We can also think of relations as objects.
However, BOL cannot express relation types at all, and there is no intrinsic typing.
Instead, the domain and range of a relation $r$ are given extrinsically via axioms about $\dom\,r$ and $\rng\,r$.
Like for individuals that allows flexibility as the same relations may have multiple types.
\end{center}
\end{example}

\begin{example}[Intrinsically Typed Ontology Language]
We could define TOL, a typed ontology language that arises as a variant of BOL.
The main differences would be
\begin{compactitem}
 \item Individuals are declared with a concept that serves as their type: \kw{individual} $i:C$.
 \item Concept assertions are dropped. They are now part of the individual declarations.
 \item Relations are declared with two concepts for their domain $D$ and range $R$: \kw{relation} $r<:D\times R$.
 \item Properties are declared with a concept for their domain $C$: \kw{property} $p<:C\times T$.
\end{compactitem}

TOL would make many ontologies more concise.
For example, we could simply write
\begin{lstlisting}
concept instructor
concept course
individual FlorianRabe : instructor
teach <: instrctor $\times$ course
\end{lstlisting}

However, we would lose flexibility.
If we want to add the concept "male", it would be difficult to make {\FR} have both types.
We might be able to remedy that by allowing intersections and declaring \lstinline|individual FlorianRabe: instructor $\sqcap$ male|.
But even then, we would have to commit to the type of each individual right away --- we cannot add different concept assertions for the same individual in different places, a common occurrence in building large ontologies.

Allowing $\sqcap$ would also introduce subtyping.
If we are careful in the design of TOL, that may still result in an elegant scalable language.
In particular, typing may remain decidable (depending on what other operations we allow).
But if we go too far, it may end up so complex that it would have been easier to go with extrinsic typing.

That is why we use intrinsic typing only in two related places in BOL:
\begin{compactitem}
 \item The base types and values use an intrinsic type system (whose details we omitted).
 \item The range of properties is given intrinsically by a base type.
\end{compactitem}
\end{example}

\begin{remark}[Subtyping]
Languages with subtyping usually have to use extrinsic type systems.
Typical sources of subtyping are
\begin{compactitem}
 \item explicit subtyping as in $\N<:\Z$
 \item comprehension/refinement as in $\{x:\N|x\neq 0\}$
 \item operations like union and intersection on types
 \item inheritance between classes, in which case subclass = subtype
 \item anonymous record types as in $\{x:\N,y:\Z\}<:\{x:\N\}$
\end{compactitem}
\end{remark}

\subsection{Combined Definition}

Neither intrinsic nor extrinsic typing is strictly better than the other.
The choice of type system is a very difficult trade-off when designing a language.

Many practical languages even combine both methods.
In that case, an intrinsic system is used for the most important high-level types and an extrinsic system is used to refine (some of) the high-level types:

\begin{definition}[Type System]
A \textbf{type system} consists of
\begin{compactitem}
 \item a collection, whose elements are called objects,
 \item a collection, whose elements are called intrinsic types,
 \item a function assigning to every object $x$ its intrinsic type $I$, in which case we write $x:I$,
 \item for some intrinsic types $I$
  \begin{compactitem}
   \item an intrinsic type $E_I$
   \item a relation $\in_I$ between objects with intrinsic types $I$ and $E_I$, called the extrinsic typing relation for $I$.
  \end{compactitem}
\end{compactitem}
\end{definition}

We can now recover the intuitions from above as special cases:
\begin{compactitem}
 \item A purely intrinsic type system is one in which $E_I$ and $\in_I$ are not given for any $I$.
 Thus, only objects and their intrinsic types remain.
 \item A purely extrinsic type system has two intrinsic types, namely $O$ (for objects) and $E_O$ (for types).
 $\in_O$ is the extrinsic typing relation between objects and types.
\end{compactitem}

\begin{example}
We can think of BOL as a combined type system.
The objects are all complex expressions.
The intrinsic types are the non-terminals $I$, $C$, $R$, $P$, and $F$, which separate the objects into the five kinds of individuals, concepts, relations, properties, and formulas.

An extrinsic typing relation exists only for $I$: we have $E_I=C$ and $\in_I$ is the \texttt{is-a} relation.
\end{example}

\begin{example}
In set theory, only a few intrinsic types are used for the high-level grouping of objects.
These include at least $set$ and $prop$.
Objects of these intrinsic types are called sets and propositions.
Some set theories also use an intrinsic type $class$.
Moreover, types like $set\to prop$ can be allowed as the types of unary predicates on sets.

Extrinsic typing is used only for the type $set$: we have $E_{set}=set$ and $\in_{set}$ is the usual elementhood relation between sets.
\end{example}

\section{Abstract Data Types}

\subsection{Motivation}

Recall the subject-centered representation of individuals described in Sect.~\ref{sec:onto:triple}.
Here we introduce an individual together with all assertions of which it is the subject as in
\begin{lstlisting}
individual "FlorianRabe"
  is-a "instructor" "male"
  "teach" "WuV" "KRMT"
  "age" 40
  "office" "11.137"
\end{lstlisting}

It is often desirable to use types to force the presence of such assertions.
We might wish require that every instructor teaches a list of things, and has an office.
Moreover, we can use types to specify the objects of the respective assertions: we can specify that only courses are taught and that the office is a string.
Rather than the relations with subjects "FlorianRabe" just happening to be around as well, the type system would now force their existence and the type of the object.
Forgetting to give such an assertion or giving it with the wrong object could be detected statically (i.e., without applying the semantics) and flagged as a typing error.

This leads to the idea of \textbf{subject-centered types}.
This could looks as follows:
\begin{lstlisting}
concept instructor
  teach course$^*$
  age: int
  office: string

individual "FlorianRabe": "instructor"
  is-a "male"
  "teach" "WuV" "KRMT"
  "age" 40
  "office" "11.137"
\end{lstlisting}
Now the type "instructor" forces the presence of a list of taught courses (The $^*$ is meant to indicate a list.), an integer for the age, and a string for the office.

We can now see that, in fact, every person should have an age, and not just every instructor.
Because every instructor is meant to be a person, we could try to capture this as well to avoid redundancy.
Moreover, every male is meant to be a person, too.

That leads to the idea of \textbf{modular types}.
This could look as follows:

\begin{lstlisting}
concept person
  age: int
  
concept male <: person

concept instructor <: person
  teach course$^*$
  office: string

individual "FlorianRabe": "instructor" $\sqcap$ "male"
  "teach" "WuV" "KRMT"
  "age" 40
  "office" "11.137"
\end{lstlisting}

Incidentally, that eliminates the need to independently declare relations and properties.
Instead, we can treat their occurrences inside the concept definitions as their declarations.

That has the added benefit that two relations/properties of the same name declared in different concepts can be distinguished and can have different types.
%For example, the relation "person.parent" could relate between a subject of type "person" and an object of type "person", and the relation

\subsection{Examples}

The general thrust of these ideas is to shift more and more information into an increasingly complex type system.
This is part of a trade-off: the more the type system can do,
\begin{compactitem}
 \item the more requirements can be expressed and violations thereof detected statically,
 \item the more complex the type system and its documentation and implementation become.
\end{compactitem}

Abstract data types have proved to be a particularly interesting trade-off on this expressivity-simplicity spectrum and are --- in one way or another --- part of many type systems
The following table gives an overview:
\begin{center}
\begin{tabular}{l|ll}
aspect & language & abstract data type \\
\hline
ontologization & UML & class \\
concretization & SQL & table schema \\
computation & Scala & class \\
deduction & various & theories, modules \\
narration & various & emergent feature
\end{tabular}
\end{center}

\subsection{Rigorous Definition}

There are many subtle design choices in defining abstract data types.
Therefore, they tend to look and behave a little differently in every type system that features them.
Here, we introduce a fairly general definition that subsumes many practical languages.

\begin{definition}[Abstract Data Type]
Consider an arbitrary type system.

An \textbf{abstract data type} (ADT) is
\begin{compactitem}
\item a \textbf{flat} type of the form
  \[\{c_1:T_1[=t_i],\ldots,c_n:T_n[=t_i]\}\]
  where the $c_i$ are distinct names, the $T_i$ are types, and the $t_i$ are optional and wherever given must have type $T_i$, or
\item a \textbf{mixin} type of the form $A_1*A_2$ for ADTs $A_i$.
\end{compactitem}

We say that a type system has \textbf{internal ADT} if all ADTs are types (and thus may in particular occur as the $T_i$ in a record type).
\end{definition}

The most important special case of an ADT are classes:

\begin{definition}[Class]
A class definition defines an ADT abbreviation of the form
\[a = a_1*\ldots*a_m*\{c_1:T_1,\ldots,c_n:T_n\}\]
where the $a_i$ are names of previously defined ADTs.

We call the $a_i$ the \textbf{superclasses} or \textbf{parent classes} and say that $a$ inherits from the $a_i$.
We call the $c_i$ the \textbf{fields} or \textbf{members} of $a$.
\end{definition}

In an OO-language, a class definition is more commonly written somehow like
\begin{lstlisting}
class $a$ extends $a_1$ with $\ldots$ with $a_m$ {
  $c_1$: $T_1$
  $\vdots$
  $c_n$: $T_n$
}
\end{lstlisting}
The details can vary, and special care must be taken in programming languages where initialization may have side effects.

Flat ADTs are the standard case, and all mixin ADTs can be simplified into flat ones.
This can be seen as a semantics in the sense that the language of flat and mixin ADT is translated to the language of flat ADTs.
\begin{definition}[Mixin Semantics]\label{def:mixinflat}
The \textbf{flattening} $\flt{A}$ of an ADT $A$ is defined as follows:
\begin{compactitem}
 \item if $A$ is flat: $\flt{A}=A$
 \item if $A$ is of the form $A_1*\ldots*A_n$:
 $\flt{A}$ arises by concatenating the fields of all $\flt{A_i}$ where duplicate field names are handled as follows:
  \begin{compactitem}
   \item if the same field (same name, types equal, definitions equal or both absent) occurs more than once, only the first occurrence is kept,
   \item if the fields $c:T_1[=t_i]$ and $c:T_2[=t_2]$ occur for inequal types $T_i$, $A$ is ill-formed,
   \item if the fields $c:T=t_1$ and $c:T=t_2$ occur for inequal objects $t_i$, $A$ is ill-formed,
   \item if the fields $c:T=t$ and $c:T$ occur, only the defined one is kept $(\ast)$.
  \end{compactitem}
\end{compactitem}
\end{definition}

\begin{remark}[Dependency Between Fields]
Our definition sweeps a very important but subtle detail under the rug: in a flat ADT with a field $c:T=t$, may $T$ and/or $t$ refer to fields declared later?
We sketch a few possible answers.

In the simplest case, we forbid such forward references.
Then ADTs are very well-behaved.
But we have a problem with the case $(\ast)$ in Def.~\ref{def:mixinflat}: if $c:T$ occurs before $c:T=t$, we cannot simply drop the former because intermediate fields may refer to $c$.
A straightforward solution would be to declare the ADT to be ill-formed.
But unfortunately, this case is very important in practice --- it occurs whenever $c:T$ is declared in an abstract class $c:T=t$ in a concrete class implementing it.

A more common solution is to allow the fields to be mutually recursive.
Consider a flat ADT with fields $\Gamma, c:T[=t], \Delta$ where $\Gamma$ and $\Delta$ are lists of fields.
Let $\Gamma'$ and $\Delta'$ arise by dropping all definitions.
Then we require that
\begin{compactitem}
 \item $T$ must be a well-formed type in context $\Gamma'$.
 Thus, the types may only refer to previous fields.
 \item $t$ must have type $T$ in context $\Gamma',c:T,\Delta'$.
 Thus, the definitions may be mutually recursive.
\end{compactitem}
This makes the case $(\ast)$ work.
But it comes at the price of recursion, which allows writing non-terminating fields (a feature in a programming language, but potentially undesirable in other settings).

Even so, the mutual-recursion solution is problematic in the presence of dependent types.
Here, dropping definitions is not always allowed:
$T$ might be well-formed in context $\Gamma$, but $\Gamma'$ might not even be a well-formed context at all.
Because OO-languages are usually not dependently-typed, this is not an issue in most settings.
\end{remark}

% \section{Concrete Data Types} = inductive types

\chapter{Querying for BOL}\label{sec:bolquery}
 \section{Overview}

Let us assume we have a semantics for our syntax.
We again write $l$ for the syntax, $L$ for the semantic, and $\sem{-}$ for the translation function.

We can now use the semantics to answer questions asked in the syntax.
Here we use the syntax to phrase a question and the semantics to determine the answer.

We call this \emph{querying}.
Contrary to standard practice, we will use that word in a very broad sense that covers all aspects.
It is more common to use the word only for concretized querying, where SQL has been developed, which has shaped many intuitions about querying.

Usually, querying requires the syntax to designate some non-terminals as \emph{propositional}.
A non-terminal is propositional if the semantics can make its words true.
Without a notion of propositions, it is impossible to define what questions and answers even are.

\begin{definition}[Propositions]
A context-free \textbf{syntax with propositions} is a context-free syntax with some designated non-terminal symbols.

A \textbf{semantics with theorems} is one that additionally defines some propositions to be theorems.
We write $\vdash F$ if $F$ is a theorem.
\end{definition}

That definition does not mean that any kind of logic is needed for querying.
Many languages use highly restricted notions of propositions that would not generally be considered as logic.
For example, languages might use equalities between objects or even equalities between certain objects as the only propositions.
The following table gives an overview:

\begin{center}
\begin{tabular}{l|l|l}
aspect & typical propositions & proposition operators\\
\hline
ontology language & assertions, concept equality/subsumption &\\
programming language & equality for some types &  boolean operators\\
database language & equality for base types &  boolean operators \\
logic & equality for all types & boolean operators, quantifiers\\
natural language & sentences & and, or, some, every, \ldots\\
\end{tabular}
\end{center}

Often the development of querying for a language leads to the discovery of omissions in the syntax: certain objects that are helpful to ask questions were omitted from the syntax because they were not needed to describe the data.
Then sometimes the syntax is extended with non-terminals or productions that seem like dead code: they are not needed for or not allowed in the official data.
The following table gives some examples:

\begin{center}
\begin{tabular}{l|l}
aspect & typical extensions\\
\hline
ontology language & conjunction of assertions \\
programming language & quantifiers\\
database language & membership in a table \\
logic & (already tries to capture all possible propositions)\\
natural language & (already captures all possible propositions) \\
\end{tabular}
\end{center}

\begin{example}[Propositions in BOL]
The obvious choice of propositions for BOL are the formulas.

In Rem.~\ref{fig:bol}, we mentioned that the BOL syntax from Fig.~\ref{rem:bol:ass} had some redundant parts that were grayed out.
Assertions are needed for writing ontologies only in such that they behave like axioms, i.e., they are automatically true.
But for querying BOL, we also need them to behave like formulas so that we can use them as questions, i.e., we must allow them to be true or false.

Moreover, it is common to also allow conjunctions.
Therefore, the BOL propositions are the conjunctions of formulas.
\end{example}

In the sequel, we will use each of the four kinds of semantics to 
% add evidence: proof/model, proofs+completeness proof, trace, denn-Satz
\begin{center}
\begin{tabular}{lllll}
Section & builds on Section & query & result\\
\hline
\ref{sec:bolquery:ded}  & \ref{sec:bolsem:ded}  & deduction & proposition & yes/no \\
\ref{sec:bolquery:conc} & \ref{sec:bolsem:conc} & concretization & proposition with free variables & true ground instances \\
\ref{sec:bolquery:comp} & \ref{sec:bolsem:comp} & computation & term & value \\
\ref{sec:bolquery:narr} & \ref{sec:bolsem:narr} & narration & question & answer \\
\end{tabular}
\end{center}


\begin{remark}[Meta-Level Questions]
Finally, any semantics admits a meta-level where additional questions can be asked.
Examples are asking for the consistency of a theory or the equivalence of two theories/programs/queries.
At the next-higher meta-level, we can ask about the completeness of a semantics or the equivalence of two semantics (of which completeness is a special case).
These meta-questions can usually not be expressed in the syntax, and we do not consider them a part of querying here.
But it is worth mentioning that the need to use yet another language (a meta-language) to ask these questions can be annoying, and some advancements in language design are about trying to integrate them into the syntax.
For example, reflection is the process of representing a language in itself so that the language can talk about itself.
That way meta-questions become regular questions.
\end{remark}


%%%%%%%%%%%%%%%%%%%%%%%%%%%%%%%%%%%%%%%%%%%
\section{Deductive Querying}\label{sec:bolquery:ded}

\subsection{Method}

We assume that $l$ is a syntax with propositions and that $L$ is a logic (and thus in particular has propositions) whose semantics has theorems.

It is not guaranteed $\sem{-}$ translates $l$-propositions to $L$-propositions.
If not, we assume there is some operation $\truelift$ in $L$ that we can use to lift the translations of $l$-propositions to $L$-propositions.

A \textbf{deductive query} consists of a proposition.
The answer to the query is yes or no.

Thus, the deductive semantics must determine which propositions are theorems, i.e., whether $\vdash_L\truelift(\sem{F})$.
This is usually done in one of two ways.

\textbf{Proof theory} uses a calculus for $L$ to derive true propositions.
Thus, we can say that an $l$-proposition $F$ is true if the calculus derives $\truelift(\sem{F})$.
Accordingly, if $L$ has a negation operator $\neg$, we can say that $F$ is false if the calculus derives $\truelift(\neg\sem{F})$.

\textbf{Model theory} uses a second deductive semantics, namely a translation from $L$ to an even stronger deductive language $M$, usually some form of set theory.
Then, we can say that $l$-propositions are true if the composition of the two translations yields a true $M$-proposition.

Either way, it is determined whether to answer an $l$-proposition with yes or no.

\subsection{Challenges}

\paragraph{Consistency}
The $L$-calculus might derive both $F$ and $\neg F$.
In that case $L$ is inconsistent and usually every formula.
We usually assume $L$ to be consistent even though we do not always prove that.

\paragraph{Decidability}
Deductive semantics is usually undecidable, i.e., there is no algorithm that takes in $F$
and always returns yes or no in finite time. 

Therefore, deductive querying is very difficult in general.
One has to run heuristics (theorem provers) to see if a proof of $F$ or $\neg F$ can be found.

A common compromise is to allow only a restricted set of propositions as queries for which decision procedures exist.
However, it can be tricky to find good restrictions, especially if the syntax allows for function symbols and equality.

For example, SFOL is undecidable.
But many fragments of SFOL are decidable, such as propositional logic and various fragments in between.

When giving a deductive semantics into SFOL, it is therefore important to check whether the image of $\sem{-}$ falls inside a decidable fragment.
This is typically the case for ontology languages.

\paragraph{Completeness}
Deductive semantics is usually incomplete, i.e., there are unanswered questions.
More precisely, the $L$-calculus typically derives $F$ for some propositions, $\neg F$ for some, but neither for some others.
The third kind of proposition cannot be answered by the semantics.

\begin{remark}
The work ``complete'' is used for two different things in logic.

Firstly, it can be a relation between two semantics, typically proof theory and a model theory.
That is the dominant meaning of the word as in, e.g., the completeness theorem for SFOL and G\"odel's incompleteness theorem.

Secondly, it can mean that a logic proves or disproves every proposition, i.e., there is no $F$ such that neither $F$ nor $\neg F$ are derivable.
That is the sense we use above.
This kind of completeness rarely holds, usually only in very restricted circumstances.
\end{remark}

Decidability and completeness are essentially the same problem.
Specifically, if completeness holds, we already obtain a decision procedure for the logic: to decide the truth of $F$, enumerate all proofs until a proof of $F$ or $\neg F$ is found.
Vice versa, if we have a sound decision procedure, running it on $F$ will prove either $F$ or $\neg F$.

\paragraph{Efficiency}
Independent of whether the semantics is complete/decidable, theorem proving is typically very expensive.

Therefore, in addition to identifying decidable fragments of a logic, it is desirable to identify \emph{efficiently} decidable fragments.
Typically, a semantics meant for efficient practical querying aims for polynomially decidable fragments.
This is the case for very simple ontology languages.
But it can quickly become exponential if the language of propositions becomes more expressive.

%%%%%%%%%%%%%%%%%%%%%%%%%%%%%%%%%%%%%%%%%%%
\section{Concretized Querying}\label{sec:bolquery:conc}
%  SQL, SPARQL

This was discussed on the slides.

%\subsection{Method}
%
%We make the same assumptions as for deductive querying.
%We further assume that the semantics is compositional.
%
%We define
%\begin{compactitem}
%\item a concretized query is an $l$-proposition $p$ in context $\Gamma$
%\item a \emph{single} result is a
% \begin{compactitem}
% \item a substitution $\vdash_l \gamma:\Gamma$
% \item such that $\vdash_L \truelift \sem{p[\gamma]}$
% \end{compactitem}
%\item the \emph{result set} is the set of all results
%\end{compactitem}
%
%\subsection{Challenges}
%
%\paragraph{Open World}
%
%\paragraph{Infinity of Results}
%
%\paragraph{Back-Translation of Results}
%

%%%%%%%%%%%%%%%%%%%%%%%%%%%%%%%%%%%%%%%%%%%
\section{Computational Querying}\label{sec:bolquery:comp}

This was discussed on the slides.

%\subsection{Method}
%
%\subsection{Challenges}
%
%\paragraph{Infinite Types}
%
%\paragraph{Termination}
%
%\paragraph{Confluence}

%%%%%%%%%%%%%%%%%%%%%%%%%%%%%%%%%%%%%%%%%%%
\section{Narrative Querying}\label{sec:bolquery:narr}

This was discussed on the slides.

%\subsection{Method}
%
%\subsection{Challenges}
%
%\paragraph{Natural Language Understanding}
%
%\paragraph{Implementing Common Sense}
%
%\paragraph{Consistency of Knowledge Base}

%\subsection{Equivalence of Semantics}

% open vs. closed world; BOL and FOL are open; SQL, comp., and Herbrand model close the world


%%% Local Variables:
%%% mode: latex
%%% TeX-master: "WuV_notes"
%%% End:


\chapter{Conclusion}\label{sec:wuv:conc}

%\part{Appendix}
%
%\appendix
%
%\chapter{Mathematical Preliminaries}\label{sec:math}
%
\section{Power Sets}\label{sec:math:powerset}

\input{\currfiledir powerset}

\section{Relations and Functions}\label{sec:math:relfun}

\input{\currfiledir relfun}

\section{Binary Relations on a Set}\label{sec:math:binrel}

\input{\currfiledir binrel}

\section{Binary Functions on a Set}\label{sec:math:binop}

\input{\currfiledir binop}

\section{The Integer Numbers}\label{sec:math:int}

\input{\currfiledir integers}

\section{Size of Sets}\label{sec:math:setsize}

\input{\currfiledir setsize}

\section{Important Sets and Functions}\label{sec:math:sets}

\input{\currfiledir sets}

%
%\tocentryBib
%
%\input{\currfilebase.bblp}
\bibliographystyle{alpha}
\bibliography{../macros/rabe,../macros/historical,../macros/pub_rabe,../macros/systems,kwarc}

\end{document}
%%% Local Variables:
%%% mode: latex
%%% TeX-master: t
%%% mode: visual-line
%%% fill-column: 5000
%%% End:

%  LocalWords:  generalremarks tocentryBib currfilebase.bblp
