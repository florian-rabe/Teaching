%\section{Overview}
%
%Various methods have been developed to represent and perform inferences.
%We structure our presentation by how each method relates to computation, the aspect most whose integration with inference has drawn the most attention.
%In general, the ubiquity of underspecified function symbols and quantified variables means that logical expressions usually do not normalize to unique values.
%At best, computations like $y:=f(x)$ can be represented as open-ended conjectures where different options for $y$ are produced, each together with a proof of the respective equality.
%Therefore, inference systems usually sacrifice computation or at least its efficiency.
%
%\emph{Proof assistants} sit at the extreme end of this spectrum.
%They employ strong logics and high-level declarations to provide a convenient way to formalize domain knowledge and reason about it.
%The reasoning is usually interactive in order to represent inferences that are too difficult to be fully automated.
%Most proof assistants integrate at least some of the other methods to overcome this weakness.
%
%Further along the spectrum, \emph{automated theorem provers} use simpler logics than interactive proof assistants.
%They are fully automatic and much faster, but can handle much fewer theorems, and typically do not check their proofs.
%\emph{Satisfiability checkers} continue this progression by aiming at decidable automation support, whereas theorem proving is usually an semi-decidable search problem.
%That limits them to propositional logic or specific theories of more expressive logics (usually of first-order logic) that are complete, i.e., where every formula can be proved or disproved.
%In the special cases, where satisfiability checkers are applicable, they come close to verified computation systems.
%
%Orthogonal to the above triplet, there are several methods for realizing Turing-complete computation naturally inside a logic.
%Here imperative and object-oriented computation are usually avoided in favor of other programming paradigms that are easier to reason about.
%\emph{Rewriting} aims at optimizing the $f(x)\rewrites y$ progression, allowing users to mark specific transformations as rewrite steps.
%\emph{Terminating recursion} is the method of adding recursive functions to a logic in order to make it a pure functional programming language.
%Finally, \emph{logic programming} restricts attention to theorems of a special form, for which proof search is simple and predictable so that users can represent computations by supplying axioms that guide the proof search.

\section{Syntax}

We give typed first-order logic (SFOL) as a language system.

\begin{definition}
Fig.~\ref{fig:sfol} gives the context-free grammar.
The vocabulary symbol is $Thy$. The expression symbols are $Y$, $T$, and $F$.
\end{definition}

\begin{figure}[hbt]
\begin{commgrammar}
\gcomment{Vocabularies: theories}\\
\gprod{Thy}{\rep{D}}{}\\
\gcomment{Declarations}\\
\gprod{D}{\kw{type}\; \ID:\kw{type}}{type declaration}\\
\galtprod{\kw{fun}\; \ID:\rep{Y}\to Y}{function symbol declaration}\\
\galtprod{\kw{pred}\; \ID:\rep{Y}\to \kw{prop}}{predicate symbol declaration}\\
\galtprod{\kw{axiom}\;F}{axiom}\\
\gcomment{type expressions}\\
\gprod{Y}{\ID}{atomic type} \\
\gcomment{term expressions}\\
\gprod{T}{\ID(\rep{T})}{function symbol applied to arguments} \\
\galtprod{\ID}{term variables} \\
\gcomment{formulas expressions}\\
\gprod{F}{\ID(\rep{T})}{predicate symbol applied to arguments} \\
\galtprod{T\doteq_Y T}{equality of terms at a type} \\
\galtprod{\top}{truth} \\
\galtprod{\bot}{falsity} \\
\galtprod{F\wedge F}{conjunction} \\
\galtprod{F\vee F}{disjunction} \\
\galtprod{F\impl F}{implication} \\
\galtprod{\neg F}{negation} \\
\galtprod{\forall \ID:Y.F}{universal quantification at a type} \\
\galtprod{\exists \ID:Y.F}{existential quantification at a type} \\
\gcomment{Identifiers}\\
\gprod{\ID}{\text{alphanumeric string}}{}\\
\end{commgrammar}
\caption{Syntax of SFOL}\label{fig:sfol}
\end{figure}

\section{Semantics}