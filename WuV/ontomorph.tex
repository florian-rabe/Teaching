Ontologies can be used to describe the state of the world.
Ontology \emph{morphisms} allow relating different worlds (or different versions of a world) to each other.

\begin{definition}[Ontology Morphism]
Given ontologies $V,W$, an ontology morphism $m:V\to W$ is an expression of the following grammar
\begin{commgrammar}
\gcomment{Morphisms}\\
\gprod{m}{\rep{A}}{}\\
\gcomment{Definitions}\\
\gprod{D}{i := I}{individual definition}\\
\galtprod{c := C}{concept definition}\\
\galtprod{r := R}{relation definition}\\
\galtprod{p := P}{property definition}\\
\end{commgrammar}
such that for all axioms $F$ in $V$, we have that $\ov{m}(F)$ is a theorem of $W$.

$m$ is well-formed if it contains exactly one definition of every $V$-identifier with a $W$-expression.
\end{definition}

\begin{definition}[Homomorphic Extension]
Given a morphism $m:V\to W$, we define its homomorphic extension $\ov{m}$ as the mapping from $V$-expressions to $W$-expressions that replaces
\begin{itemize}
\item every individual reference $i$ with $I$ where $i:=I$ is in $m$
\item every concept reference $c$ with $C$ where $c:=C$ is in $m$
\item every relation reference $r$ with $R$ where $r:=R$ is in $m$
\item every property reference $p$ with $P$ where $p:=P$ is in $m$
\end{itemize}
\end{definition}


%Consider the following test
%
%vocabulary in BOL by example
%
%domain vocabulary:
%Old =
%  concept man
%  concept woman
%  axiom a1: man union woman = top
%  axiom a2: man inter woman = empty
%
%codomain vocabulary (still to be worked out, supposed to represent transgender change):
%New =
%  concept cisman
%  concept transman
%  concept ciswoman
%  concept transwoman
%  axiom ...
%
%vocabulary morphism m maps every domain symbol to corresponding codomain expression
%actually: maps every domain declaration to a corresponding codomain expression
%
%in particular: axiom establishing F is mapped to a proof of m(F)
%  works best if we
%    * give names to the axioms
%    * extend the language with proof expressions
%
%Morphism m : V -> W
%induces compositional mapping of V-expression E to W-expressions m(E)
%by replacing every reference to a V-symbol s with the W-expression assigned to s by m, i.e., the expression e such that s:=e in s in m
%
%Exactly like for substitutions, but acting on the language symbols instead of the variables.
%
%gendermatters : Old -> New =
%  man :=     <some concept expression over New> , e.g., cisman union transman
%  woman := <some concept expression over New> , e.g., ciswoman union transwoman
%  a1 := <some proof of (cisman union transman) union (ciswoman union transwoman) = top>
%  a2 := <some proof of (cisman union transman) inter (ciswoman union transwoman) = empty>
%
%sexmatters : Old -> New =
%  man :=     cisman union transwoman
%  woman := ciswoman union transman
%
%sexandcismatters : Old -> New =
%  man := cisman
%  woman := ciswoman
%  a1 := <some proof of cisman union ciswoman = top>  // which may or may not exist
%  a1 := <some proof of cisman inter ciswoman = empty>