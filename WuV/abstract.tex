\section{Formal Systems}

%\section{Categories}

\subsection{Syntax}

\begin{definition}\label{def:css}
A \textbf{formal system} consists of
\begin{compactitem}
 \item a set $\Voc$ of vocabularies,
 \item for any vocabulary $V$, a set $\Exp_V$ of expressions
\end{compactitem}
%In case of $\wft{\Theta}$, we call $\Theta$ \textbf{well-formed}.
%In case of $\wff{\Theta}{E}$, we call $E$ a \textbf{well-formed} $\ExpSym$-expression over $\Theta$.

A formal system with morphisms
\begin{compactitem}
 \item additionally provides for any two vocabularies $V,W$, a set $\VocM(V,W)$ of vocabulary morphisms from $V$ to $W$
 \item for any vocabulary morphism $m\in\VocM(V,W)$, a mapping $\Exp_m:\Exp_V\to \Exp_W$
\end{compactitem}

A formal system with typing
\begin{compactitem}
\item additionally provides a relation $\der_V E:E'$ between expression $E,E'\in\Exp_V$.
\item such that for $m\in\VocM(V,W)$, we have that if $\der_V E:E'$, then $\der_W \Exp_m(E):\Exp_m(E')$.
\end{compactitem}

A formal system with propositions has typing and
\begin{compactitem}
\item additionally provides an expressions $\prop$, in which case
 \begin{itemize}
 \item expressions $\der_V F:\prop$ are called \emph{propositions} (or formulas)
 \item if $\der_V F_\prop$, then expressions $\der_V P:F$ are called \emph{proofs} of $F$
 \item if some proof $P$ of $F$ exists, we also write $\der_V F$
 \end{itemize}
\item and now type preservation guarantees for $m\in\VocM(V,W)$ we have that if $\der_V F$, then $\der_W \Exp_m(F)$.
\end{compactitem}

A formal system with equality has propositions and additionally provides a proposition $E\doteq_T E'$ for some expressions $T$ and all $E,E'\in\Exp_V(T)$.
\end{definition}

The vocabularies are usually lists of named declarations.
In that case, the vocabulary morphism from $V$ to $W$ are usually lists of assignments $c:=E$ where $c$ is the name of a declaration in $V$ and $E$ is a $W$-expression.
In that case, the mapping $\Exp_m$ usually arises by replacing every $V$-identifier with the $W$-expression provided by $m$.

\subsection{Translation}

\begin{definition}\label{def:css}
A \textbf{translation} from formal system $l$ to formal system $L$ consists of several mappings, all written $\sem{-}$:
\begin{compactitem}
 \item a vocabulary translation $\Voc_l\to \Voc_L$,
 \item for any $l$-vocabulary $V$, an expression translation $\Exp_V\to \Exp_{\sem{V}}$
\end{compactitem}
%In case of $\wft{\Theta}$, we call $\Theta$ \textbf{well-formed}.
%In case of $\wff{\Theta}{E}$, we call $E$ a \textbf{well-formed} $\ExpSym$-expression over $\Theta$.

If $l$ and $L$ have morphisms, a translation \textbf{maps morphisms} if it additionally provides
\begin{compactitem}
 \item for any two $l$-vocabularies $V,W$, a morphism translation $\VocM(V,W)\to \VocM(\sem{V},\sem{W})$
 \item such that for any vocabulary morphism $m\in\VocM(V,W)$ and expression $E\in\Exp_V$, we have $\sem{\Exp^l_m(E)}=\Exp^L_{\sem{m}}(\sem{E})$
\end{compactitem}

If $l$ and $L$ have typing, the translation \textbf{preserves typing} if $\der^l_V E:E'$ implies $\der^L_{\sem{V}}\sem{E}:\sem{E'}$.

If $l$ and $L$ have typing, the translation \textbf{preserves propositions} if $\sem{\prop^l}=\prop^L$.

If $l$ and $L$ have equality, the translation \textbf{preserves equality} if $\sem{E\doteq_T E'}=\sem{E}\doteq_{\sem{T}}\sem{E'}$.
\end{definition}

The vocabulary translation usually consists of a semantic prefix and a declaration-wise translation: A $V$-vocabulary $D_1,\ldots D_n$ is translated to $P,D'_1,\ldots,D'_n$ where the $D'_i$ are the translations of the the $D_i$.

If a translation preserves typing and propositions, it automatically preserves truth, i.e., if $\der^L_{\sem{V}}$ implies $\der^L_{\sem{V}}\sem{F}$.
But the requirement to preserve propositions is often too strong.
In practice, it is often better to use the following generalization:
\begin{definition}
If $l$ and $L$ have propositions, a \textbf{proposition-lifting} is an operation $\truelift:\sem{\prop^l}\to\prop^L$.

A translation with proposition-lifting \textbf{preserves truth} if $\der^l_V F$ implies $\der^L_{\sem{V}}\truelift\sem{F}$.
\end{definition}

\subsection{Interpretation}

\begin{definition}
An \textbf{interpretation} $I$ of a formal system $l$ consists of the following parts:
\begin{compactitem}
 \item for every $l$-vocabulary $V$ a set $\Sit_V$, whose elements are called situations,
 \item for every $l$-vocabulary $V$ and every situation $S\in \Sit_V$, a function $I_S$ mapping every $E\in \Exp_V$ to its interpretation $\sem{E}_S$.
\end{compactitem}
If there is only one interpretation, the value $I_S(E)$ is often written as $\semm{E}{S}$.

If $l$ has morphisms, the interpretation is called \textbf{institutional} if it additionally provides
\begin{compactitem}
 \item for any $l$-morphism $m:V\to W$, a function $\Sit_m\to \Sit_W\to \Sit_V$
 \item such that for any $E\in\Exp_V$ and $S\in \Sit_W$, we have that $\semm{\Exp_m(E)}{W} = \semm{E}{\Sit_m(S)}$.
\end{compactitem}

If $l$ has typing, the interpretation is called \textbf{standard} if $\der^l_V E:E'$ implies $\semm{E}{S}\in \semm{E'}{S}$, i.e., types are interpreted as sets and typing as set membership.

If $l$ has propositions, the interpretation is called \textbf{classical} if $\semm{\prop^l}{S}=\{0,1\}$, i.e., propositions are interpreted as truth values.

If $l$ has equality, the interpretation is called \textbf{canonical} if $\semm{E\doteq_T E'}{S}=1$ iff $\semm{E}{S}=\semm{E'}{S}$
\end{definition}

\section{Semantics}

Semantics can be defined absolutely or relatively.
In the latter case, both translations and interpretations can be used to assign relative semantics to a formal system.
Translation are relative to another formal system, interpretations to a situation.

\subsection{General Case}

\begin{definition}
Given a formal system with propositions, a \textbf{deductive semantics} consists of sets $\Thm_V\sq \Exp_V(\prop)$, called the \textbf{theorems}.
If $F\in\Thm_V$, we also write $\models_V F$.

If the formal system has typing, the truth judgment defines a deductive semantics by $\Thm_V=\{F\in\Exp_V(\prop)\;|\;\der_V F\}$.
In that case, we speak of \textbf{soundness} if $\der_V F$ implies $\models_V F$ and of \textbf{completeness} in the reverse case.
\end{definition}

\begin{definition}
Given a formal system with equality, a \textbf{computational semantics} consists of a binary relation $\Eval_V\sq\Exp_V\times\Exp_V$.
We also write $\Eval_V(E)=E'$ as $\der_V E\rewrites E'$.

If $\Eval_V$ is a function and $\der_V E\rewrites E'$ implies $\der E\doteq_T E'$, we call it a \textbf{normal form}.

If additionally $\der_V E\doteq E'$ iff $\Eval_V(E)=\Eval_V(E')$, we call it a \textbf{canonical form}.
\end{definition}

\begin{definition}
Given a formal system with typing and propositions, a \textbf{concrete semantics} consists of a function $\Inst$ that maps propositions $F$ in context $\Gamma$ to a set $\Inst_V(\Gamma,F)$.

If there is a deductive sematics, the concrete semantics and the deductive are compatible if every $\gamma\in \Inst_V(\Gamma,F)$ satisfies $\der_V \gamma:\Gamma$ and $\der_V F[\gamma]$.
\end{definition}

We speak of an \textbf{absolute} semantics if the above are provided by a system of inference rules for the typing and truth judgments.

\subsection{Relative Semantics by Translation}

Consider a formal system translation from $l$ to $L$ with proposition lifting.

\begin{definition}
Given a deductive semantics for $L$, we define a deductive semantics for $l$ by
\[\models^l_V F \tb\miff\tb \models^L_{\sem{V}}\truelift\sem{F}.\]
\end{definition}

\begin{definition}
Given a computational semantics for $L$, we define a computational semantics for $l$ by
\[\der^l_V E\rewrites E' \tb\miff\tb \der^L_{\sem{V}} \sem{E}\rewrites \sem{E'}.\]

This is a well-defined evaluation function only if $\Eval_L$ is and $\Eval^L_{\sem{V}}(\sem{E})$ is always in the image of $\sem{-}$.
\end{definition}

\begin{definition}
Given a concrete semantics for $L$, we define a concrete semantics for $l$ by
\[\gamma\in \Inst(\Gamma,F) \tb\miff\tb \sem{\gamma}\in\Inst(\sem{\Gamma},\truelift\sem{F}).\]
\end{definition}

\subsection{Relative Semantics by Interpretation}

Consider a classical interpretation $I$ of formal system $l$.
We can cast $I$ as a special case of a formal system translation with proposition-lifting under the following practically reasonable assumptions:
\begin{itemize}
\item There is (at least implicitly) a formal system $L$ for the background mathematical language.
\item The situations in $\Sit_V$ are always tuples containing one component for every $V$-symbol.
\end{itemize}

In that case we define a translation $\ov{I}:l\to L$ as follows:
\begin{itemize}
\item Every vocabulary $V$ is mapped to the $L$-vocabulary containing for every $V$-symbol $s$ a declaration of the same name representing the corresponding element in the tuple.
\item Every $V$-expression $E\in\Exp_V(T)$ is mapped to the function $\ov{I}(E)$ that maps $\Sit_V\ni S \mapsto I_S(E)$.
\item The proposition lifting is given by \[\truelift\, \ov{I}(F) = 1 \tb\miff \tb I_S(F)=1 \mforall S\in\Sit_V.\]
\end{itemize}

We can then use that translation to define semantics as above.

In particular, The proposition lifting becomes a lot more obvious if we write $\sem{}$ for $I$ and look at the induced deductive semantics: $\models F$ iff $\semm{F}{S}=1$ for all $S\in\Sit_V$.


%%%%%%%%%%%%%%%%%%%%%%%%%%%%%%%%%%%%
%% connect to querying
%\section{Kinds of Problems}
%
%\subsection{Problems as Intensional Descriptions}
%
%\begin{remark}[Intensional vs. Extensional]
%Consider a set $S$ of objects.
%An intensional description of $S$
%
%An extensional description of $S$
%\end{remark}
%
%\begin{remark}[Single vs. Multi-Variable Problem]
%We can think of 
%\end{remark}
%
%\begin{definition}
%A \textbf{problem} is a theory.
%%Problem = domain + intensional solution
%
%A \textbf{solution} is a model of the theory.
%% extensional
%\end{definition}
%
%\begin{center}
%\begin{tabular}{ll}
%Consistency (or satisfiability) & Does $P$ have a solution?\\
%Decision & Is $S$ a solution of $P$? \\
%Solving & Find a solution of $P$.\\
%Enumeration & List all solutions of $P$.\\
%\end{tabular}
%\end{center}
%
%\begin{example}
%A constraint satisfaction problem
%
%\end{example}
%
%\begin{example}
%A search problem
%%Search = problem + transition system
%\end{example}
%
%\begin{example}
%The satisfiability problem
%
%\end{example}
%
%\subsection{Families of Problems}

%Problem family = problem-valued function
%Algorithms: global consistency, global decision, global enumeration
%Finding one solution equivalent to finding all if family closed under exlcuding solutions
%Complexity classes: C and NC based on algorithms for global decision/enumeration
%
%Fixed domain problem families
%Galois connection between intensional descriptions and domain elements