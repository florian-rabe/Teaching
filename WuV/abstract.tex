\section{Formal Systems}

%\section{Categories}

\subsection{Syntax}

\begin{definition}\label{def:css}
A \textbf{formal system} consists of
\begin{compactitem}
 \item a set $\Voc$ of vocabularies,
 \item for any two vocabularies $V,W$, a set $\VocM(V,W)$ of vocabulary morphisms from $V$ to $W$
 \item for any vocabulary $V$, a set $\Exp_V$ of expressions
 \item for any vocabulary morphism $m\in\VocM(V,W)$, a mapping $\Exp_m:\Exp_V\to \Exp_W$
\end{compactitem}
%In case of $\wft{\Theta}$, we call $\Theta$ \textbf{well-formed}.
%In case of $\wff{\Theta}{E}$, we call $E$ a \textbf{well-formed} $\ExpSym$-expression over $\Theta$.

A formal system with typing
\begin{compactitem}
\item additionally provides a relation $\der_V E:E'$ between expression $E,E'\in\Exp_V$.
\item such that for $m\in\VocM(V,W)$, we have that if $\der_V E:E'$, then $\Exp_m(E):\Exp_m(E')$.
\end{compactitem}
\end{definition}

The vocabularies are usually lists of named declarations.
In that case, the vocabulary morphism from $V$ to $W$ are usually lists of assignments $c:=E$ where $c$ is the name of a declaration in $V$ and $E$ is a $W$-expression.
In that case, the mapping $\Exp_m$ usually arises by replacing every $V$-identifier with the $W$-expression provided by $m$.

\subsection{Translation}

\begin{definition}\label{def:css}
A \textbf{translation} from formal system $l$ to formal system $L$ consists of several mappings, all written $\sem{-}$:
\begin{compactitem}
 \item a vocabulary translation $\Voc_l\to \Voc_T$,
 \item for any two $l$-vocabularies $V,W$, a morphism translation $\VocM(V,W)\to \VocM(\sem{V},\sem{W})$
 \item for any $l$-vocabulary $V$, an expression translation $\Exp_V\to \Exp_{\sem{V}}$
 \item such that for any vocabulary morphism $m\in\VocM(V,W)$ and expression $E\in\Exp_V$, we have $\sem{\Exp^l_m(E)}=\Exp^L_{\sem{m}}(\sem{E})$
\end{compactitem}
%In case of $\wft{\Theta}$, we call $\Theta$ \textbf{well-formed}.
%In case of $\wff{\Theta}{E}$, we call $E$ a \textbf{well-formed} $\ExpSym$-expression over $\Theta$.

If $l$ and $L$ have typing, the translation \textbf{preserves typing} if $\der^l_V E:E'$ implies $\der^L_{\sem{V}}\sem(E):\sem(E')$.
\end{definition}

The vocabulary translation usually consists of a semantics prefix and a declaration-wise translation: A $V$-vocabulary $D_1,\ldots D_n$ is translated to $P,D'_1,\ldots,D'_n$ where the $D'_i$ are the translations of the the $D_i$.

\subsection{Interpretation}

\begin{definition}
An \textbf{interpretation} of a formal system $l$ consists of the following parts:
\begin{compactitem}
 \item for every $l$-vocabulary $V$ a set $Sit(V)$, whose elements are called situations,
 \item for every $l$-vocabulary $V$, every situation $S\in Sit(V)$, a function mapping every $E\in \Exp_V$ to its interpretation $\sem{E}_S$.
\end{compactitem}
\end{definition}

\section{Semantics}

Both translations and interpretations can be used to assign semantics to a formal system.
Both are relative either to another formal system or to a situation.

But we can also distinguish semantics by the kinds of questions they answer.

\subsection{Deductive}

\subsection{Computational}

\subsection{Concrete Data}

%%%%%%%%%%%%%%%%%%%%%%%%%%%%%%%%%%%
% connect to querying
\section{Kinds of Problems}

\subsection{Problems as Intensional Descriptions}

\begin{remark}[Intensional vs. Extensional]
Consider a set $S$ of objects.
An intensional description of $S$

An extensional description of $S$
\end{remark}

\begin{remark}[Single vs. Multi-Variable Problem]
We can think of 
\end{remark}

\begin{definition}
A \textbf{problem} is a theory.
%Problem = domain + intensional solution

A \textbf{solution} is a model of the theory.
% extensional
\end{definition}

\begin{center}
\begin{tabular}{ll}
Consistency (or satisfiability) & Does $P$ have a solution?\\
Decision & Is $S$ a solution of $P$? \\
Solving & Find a solution of $P$.\\
Enumeration & List all solutions of $P$.\\
\end{tabular}
\end{center}

\begin{example}
A constraint satisfaction problem

\end{example}

\begin{example}
A search problem
%Search = problem + transition system
\end{example}

\begin{example}
The satisfiability problem

\end{example}

\subsection{Families of Problems}

%Problem family = problem-valued function
%Algorithms: global consistency, global decision, global enumeration
%Finding one solution equivalent to finding all if family closed under exlcuding solutions
%Complexity classes: C and NC based on algorithms for global decision/enumeration
%
%Fixed domain problem families
%Galois connection between intensional descriptions and domain elements