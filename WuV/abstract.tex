\section{Categories}

\section{The Category of Vocabularies}

\subsection{Fully Abstract}

\subsection{With Grammar}

% named declarations

\subsection{By Inference System}

\section{Situations}

\section{Translations}


%%%%%%%%%%%%%%%%%%%%%%%%%%%%%%%%%%%
% connect to querying
\section{Problems}

\subsection{Problems as Intensional Descriptions}

\begin{remark}[Intensional vs. Extensional]
Consider a set $S$ of objects.
An intensional description of $S$

An extensional description of $S$
\end{remark}

\begin{remark}[Single vs. Multi-Variable Problem]
We can think of 
\end{remark}

\begin{definition}
A \textbf{problem} is a theory.
%Problem = domain + intensional solution

A \textbf{solution} is a model of the theory.
% extensional
\end{definition}

\begin{center}
\begin{tabular}{ll}
Consistency (or satisfiability) & Does $P$ have a solution?\\
Decision & Is $S$ a solution of $P$? \\
Solving & Find a solution of $P$.\\
Enumeration & List all solutions of $P$.\\
\end{tabular}
\end{center}

\begin{example}
A constraint satisfaction problem

\end{example}

\begin{example}
A search problem
%Search = problem + transition system
\end{example}

\begin{example}
The satisfiability problem

\end{example}

\subsection{Families of Problems}

%Problem family = problem-valued function
%Algorithms: global consistency, global decision, global enumeration
%Finding one solution equivalent to finding all if family closed under exlcuding solutions
%Complexity classes: C and NC based on algorithms for global decision/enumeration
%
%Fixed domain problem families
%Galois connection between intensional descriptions and domain elements