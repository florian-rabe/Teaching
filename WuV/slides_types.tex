%\begin{frame}\frametitle{Typing}
%Trivial intrinsic typing (Church) $\vdash e:^{int} E$
%\begin{itemize}
%\item $E$ is a non-terminal
%\item $e$ an expression derived from $E$
%\end{itemize}
%\medskip
%
%Refined by extrinsic typing (Curry) $\vdash e :^{ext} E$
%\begin{itemize}
%\item $e$ is an individual, i.e., $\vdash e :^{int} I$
%\item $E$ is a concept, i.e., $\vdash E :^{int} C$
% \glec{where $I$ and $C$ are the non-terminals from the grammar}
%\item $e$ has concept $E$, i.e., $\vdash e \isa E$
%\end{itemize}
%\end{frame}
%
%\begin{frame}\frametitle{Side Note: Propositions as Types}
%\begin{blockitems}{If our grammar has proof terms as well, we can}
%\item write $\vdash p:f$ if $p$ is a proof of proposition $f$
%\item have variables $x:f$ to make the (named) assumption that $f$ holds
%\end{blockitems}
%
%The usual notation is then the abbreviation
%\[\Gamma,\; f \tb\text{for exists $p$ such that} \tb \Gamma,\;p:f\]
%\glec{sufficient when not working with proof terms}
%\end{frame}


\section{Kinds of Typing: Extrinsic and Intrinsic}

\begin{frame}\frametitle{Breakout Question}
Is this an improvement over BOL?
\begin{commgrammar}
\gcomment{Declarations}\\
\gprod{D}{\kw{individual}\; i: C}{typed atomic individual}\\
\galtprod{\kw{concept}\; c}{atomic concept}\\
\galtprod{\kw{relation}\; r\sq C\times C}{typed atomic relation}\\
\galtprod{\kw{property}\; p\sq C\times T}{typed atomic property}\\
\end{commgrammar}
\glec{rest as before}
\end{frame}

\begin{frame}\frametitle{Actually, when is a language an improvement?}
 \lec{orthogonal, often mutually exclusive criteria}
\begin{blockitems}{Trade-off for syntax design}
  \item expressivity: easy to express knowledge
    \glec{e.g., big grammar, complex type system}
  \item simplicity: easy to implement/interpret
    \glec{e.g., few, carefully chosen productions, types}
 \end{blockitems}

\begin{blockitems}{Semantics}
\item specification, implementation, documentation
\end{blockitems}

\begin{blockitems}{Intended users}
  \item skill level
  \item prior experience with related languages
  \item amount of training needed
  \item innovation height, differential evaluation against existing languages
\end{blockitems}
\end{frame}

\begin{frame}\frametitle{Actually, when is a language an improvement? (2)}
\begin{blockitems}{Support software ecosystem}
  \item optional tool support: IDEs, debuggers, heuristic checkers, alternative implementations, interpreter/REPL
  \item many/large well-crafted vocabularies and package managers to find them
  \item integrations with other languages: translations, common run-time platforms, foreign function interface
\end{blockitems}

\begin{blockitems}{Long-term plans: re-answer the above question but now}
  \item maintainability: syntax was changed, everything to be redone
  \item backwards compatibility: support for legacy input
  \item scalability: expressed knowledge content has reached huge sizes
\end{blockitems}
\end{frame}

\begin{frame}\frametitle{General Idea}
\begin{blockitems}{A \textbf{type system} for a syntax consists of}
 \item some non-terminals $\ExpSym$, whose words are called $\ExpSym$-\textbf{expressions},
% \item the non-terminal producing an expression is its \textbf{intrinsic type}
%  \begin{itemize}
%  \item untyped: all expressions have same intrinsic type
%  \item typed by grammar: non-terminals are the intrinsic types
%  \end{itemize}
   \glec{coarse, context-free, classification into disjoint sets}
 \item for some symbols $\ExpSym$
  \begin{itemize}
   \item set of types: $\TpSym$-expressions for a non-terminal $\TpSym$
   \item typing relation $\Gamma\vdash^L_V e:T$ between $\ExpSym$-expressions $e$ and $\TpSym$-expressions $T$
  \end{itemize}
  \glec{fine, context-sensitive, classification into disjoint or overlapping sets}
\end{blockitems}

\begin{blockitems}{Examples}
\item BOL: non-terminals $\ExpSym$ for expressions are $C,I,R,P,F$ \\
   $I$-expressions typed by $C$-expressions
    \glec{overlapping, types undecidable/difficult to check}
\item SFOL: non-terminals $\ExpSym$ for expressions are $Y,T,F$ \\
 $T$-expressions typed by $Y$-expressions
   \glec{disjoint, types easy to infer}
\end{blockitems}
\end{frame}

\begin{frame}\frametitle{Church vs. Curry Typing}
\begin{center}
\footnotesize
\begin{tabular}{l|ll}
& intrinsic & extrinsic \\
\hline
$\lambda$-calculus by & Church & Curry \\
type is & carried by object & given by environment \\
typing is a & function objects $\to$ types & relation objects $\times$ types \\
objects have & unique type & any number of types \\
types interpreted as & disjoint sets & unary predicates \\
\hline
type given by & part of declaration & additional axiom \\
 \tb example               &  \kw{individual} "WuV":"course"  & \kw{individual} "Wuv",\\
                           &                                  & "WuV" \texttt{is-a} "course"\\
\hline
examples   & SFOL, SQL & OWL, Scala, English \\
           & most logics, functional PLs & ontology, OO, \\
           &                             & natural languages \\
           & many type theories & set theories
\end{tabular}
\end{center}
\end{frame}

\begin{frame}\frametitle{Type Checking}
\begin{center}
\footnotesize
\begin{tabular}{l|ll}
& intrinsic & extrinsic \\
\hline
type is & carried by object & given by environment \\
typing is a & function objects $\to$ types & relation objects $\times$ types \\
objects have & unique type & any number of types \\
\hline
type given by & part of declaration & additional axiom \\
 \tb example               &  \kw{individual} WuV: course  & \kw{individual} Wuv,\\
                           &                                  & WuV \texttt{is-a} course\\
\hline
type inference for $x$ & uniquely infer $A$ from $x$ & find minimal $A$ with $x:A$ \\
type checking & inferred=expected & prove $x:A$ \\
subtyping $A<:B$ & cast from $A$ to $B$ & $x:A$ implies $x:B$ \\
typing decidable & yes unless too expressive & no unless restricted \\
typing errors & static (compile-time) & dynamic (run-time)\\
\hline
advantages & easy & flexible \\
           & unique type inference & allows subtyping \\
\end{tabular}
\end{center}
\end{frame}

\begin{frame}\frametitle{Examples: Curry-Typing in BOL}
\begin{center}
\footnotesize
\begin{tabular}{l|lll}
Semantics  & objects & types & typing relation\\
\hline
absolute & individuals $I$ & concepts $C$ & $i \isa C$\\
SFOL & terms of type $\iota$  & predicates $C\sq\iota$ & $c(i)$\\
%  SQL & table Individuals & tables containing ids & id of i in table $c$ \\
%  Scala & String & hash sets of strings & $c$.contains($i$) \\
English & proper nouns & common nouns & "$i$ is a $C$"
\end{tabular}
\end{center}
\end{frame}

\begin{frame}\frametitle{Examples}
\begin{center}
\begin{tabular}{l|llll}
System & typing & objects & types & typing relation\\
\hline
%pure Church & one per type & none & none \\
%pure Curry & one for all expressions & types $T$ & $:$ \\
%\hline
any & Church & expressions & non-terminals & derived from\\
BOL & Curry  & individuals & concepts & $\isa$\\
SFOL & Church & terms & types & $:$ \\
set theory & Curry & sets & sets & $\in$ \\
OO & Curry & instances & classes & $\mathtt{isInstanceOf}$\\
\end{tabular}
\end{center}
\end{frame}

\begin{frame}\frametitle{Subtyping}
Subtyping works best with Curry Typing
\begin{itemize}
 \item explicit subtyping as in $\N<:\Z$
 \item comprehension/refinement as in $\{x:\N|x\neq 0\}$
 \item operations like union and intersection on types
 \item inheritance between classes, in which case subclass = subtype
 \item anonymous record types as in $\{x:\N,y:\Z\}<:\{x:\N\}$
\end{itemize}
\end{frame}

\section{Kinds of Types}

\begin{frame}\frametitle{Question}
What kind of types are there?
\end{frame}

\begin{frame}\frametitle{Abstract vs. Concrete Types}
\textbf{Concrete} type: values are
\begin{itemize}
\item given by their internal form,
\item defined along with the type, typically built from already-known pieces.
\end{itemize}
\lec{product types, enumeration types, collection types}
\lec{main example: concrete (inductive/algebraic) data types}

\textbf{Abstract} type: values are
\begin{itemize}
\item given by their externally visible properties,
\item defined in any environment that understands the type definition.
\end{itemize}
\lec{structures, records, classes, aggregation types}
\lec{main example: abstract data types}
\end{frame}

\begin{frame}\frametitle{Non-Recursive vs. Recursive Types}
\textbf{Non-Recursive} types 
\begin{itemize}
\item given by some expressions
\item can be anonymous
\item values given directly
\end{itemize}
\lec{integers, lists of strings, \ldots}

\textbf{Recursive} types
\begin{itemize}
\item definition of the type must refer to the type itself\\
so type must have a name
\item type typically defines other named operations
\item values obtained by fixed-point constructions
\end{itemize}
\lec{optional property of concrete and abstract data types}
\end{frame}

\begin{frame}\frametitle{Atomic vs. Complex Types}
\textbf{Atomic} type
\begin{itemize}
\item given by its name
\item values are a set
\end{itemize}
\lec{integers, strings, booleans, \ldots}

\textbf{Complex} types
\begin{itemize}
\item arise by applying type symbol to arguments\\
\item separate set of values for each tuple of arguments
\end{itemize}
\lec{two kinds of complex types (next slide)}
\end{frame}

\begin{frame}\frametitle{Type operators vs. Dependent type Families}
Both are complex: take arguments and return a type

\textbf{Type operators} take \emph{only type arguments}, e.g.,
 \begin{itemize}
 \item type operator $\times$
 \item takes two types $A,B$
 \item returns type $A\times B$
 \end{itemize}

\textbf{Dependent types} take \emph{also value arguments}, e.g.,
 \begin{itemize}
 \item dependent type operator $vector$
 \item takes natural number $n$, type $A$
 \item returns type $A^n$ of $n$-tuples over $A$
 \end{itemize}
\lec{dependent types much more complicated, less uniformly used}
\lec{harder to starndardize}
\end{frame}


\begin{frame}\frametitle{Built-in vs. User-Defined Types}
\textbf{Built-in} types
\begin{itemize}
\item syntax and semantics fixed by language designer
\item part of grammar, implementation, etc.
\item usually concrete, atomic, non-recursive
\end{itemize}
\lec{typical: integers, strings, lists}
\glec{sometimes also called primitive or basic types}

\textbf{User-Defined} types
\begin{itemize}
\item declared by users in vocabulary
\item standard syntax prescribed by grammar, possibly customizable
\item semantics given by operations and their axioms
\end{itemize}
\lec{anything the language can axiomatize}
\glec{usually difficult to axiomatize recursive properties}
\end{frame}

\section{Non-Recursive Data Types}


\begin{frame}\frametitle{Common Built-in Types}
\begin{blockitems}{Typical (quasi-)primitive types}
 \item natural numbers (= $\N$)
 \item arbitrary precision integers (= $\Z$)
 \item fixed precision integers (32 bit, 64 bit, \ldots)
 \item floating point (float, double, \ldots)
 \item Booleans
 \item characters (ASCII, Unicode)
 \item strings
\end{blockitems}

Observation:
\begin{itemize}
\item essentially the same in every language
 \lec{including whatever language used for semantics}
\item semantics by translation trivial
\end{itemize}
\end{frame}

\begin{frame}\frametitle{Less Common Types}
\begin{blockitems}{Problem}
 \item quickly encounter primitive types not supported by common languages
 \item need to encode them using existing types
  \lec{typically as strings, ints, or prodcuts/lists thereof}
\end{blockitems}

\begin{blockitems}{Examples}
\item date, time, color, location on earth
\item graph, function
\item picture, audio, video
\item physical quantities ($1m$, $1in$, etc.)
\item gene, person
\end{blockitems}
\end{frame}

\begin{frame}\frametitle{Specifying a Type}
\begin{blockitems}{Components}
 \item the type \glec{eg: 'int'}
 \item values of the type \glec{eg: 0, 1, -1, \ldots}
 \item string encoding \lec{injective function from values to strings}
 \item operations on type \glec{eg: addition, multiplication, \ldots}
\end{blockitems}

\begin{blockitems}{Examples}
\item IEEE floating point numbers
\item ISO 8601 for date/time
\end{blockitems}
\end{frame}

\begin{frame}\frametitle{Type Operators}
As for types, but taking $n$ type argument.

\begin{blockitems}{Component}
 \item the type \glec{eg: 'list'}
 \item arity $n$ \glec{eg: $1$}
 \item for argument types $A_1,\ldots,A_n$ with string encodings $E_1,\ldots,E_n$
  \begin{itemize}
  \item values of the type \lec{eg: $[a_1,\ldots,a_n]$ for $a_i$ values of $A_1$}
  \item string encoding \lec{eg: "[" $E(a_1)$ "," \ldots "," $E_1(a_n)$ "]"}
  \item operations on the type \glec{eg: concatenation, element access, \ldots}
  \end{itemize}
\end{blockitems}

No good standards, but part of most languages
\end{frame}

\begin{frame}\frametitle{Basic Atomic Types}
\begin{blockitems}{typical in IT systems}
 \item fixed precision integers (32 bit, 64 bit, \ldots)
 \item IEEE float, double
 \item Booleans
 \item Unicode characters
 \item strings \glec{could be list of characters but usually bad idea}
\end{blockitems}

\begin{blockitems}{typical in math}
 \item natural numbers (= $\N$)
 \item arbitrary precision integers (= $\Z$)
 \item rational, real, complex numbers
 \item graphs, trees
\end{blockitems}
\lec{clear: language must be modular, extensible}
\end{frame}

\begin{frame}\frametitle{Advanced Atomic Types}

\begin{blockitems}{general purpose}
 \item date, time, color, location on earth
 \item picture, audio, video
\end{blockitems}

\begin{blockitems}{domain-specific}
 \item physical quantities ($1m$, $1in$, etc.)
 \item gene, person
 \item semester, course id, \ldots
\end{blockitems}

\lec{clear: language must be modular, extensible}
\end{frame}

\begin{frame}\frametitle{Collection Data Types}
\begin{blockitems}{Homogeneous Collection Types}
 \item sets
 \item multisets (= bags)
 \item lists
 \lec{all unary type operators, e.g. $list\;A$ is type of lists over $A$}
 \item fixed-length lists (= Cartesian power, vector $n$-tuple)
  \glec{dependent type operator}
\end{blockitems}

\begin{blockitems}{Heterogeneous Collection Types}
 \item lists
 \item fixed-length lists (= Cartesian power, $n$-tuple)
 \item sets
 \item multisets (= bags)
 \lec{all atomic types, e.g., $list$ is type of lists over any objects}
\end{blockitems}
\end{frame}

\begin{frame}\frametitle{Aggregation Data Types}
\begin{blockitems}{Products}
 \item Cartesian product of some types $A\times B$ \\
 values are pairs $(x,y)$ 
 \glec{numbered projections $_1$, $_2$ --- order relevant}
 \item labeled Cartesian product (= record) $\{a: A, b: B\}$ \\
 values are records $\{a=x, b=y\}$
  \glec{named projections $a$, $b$ --- order irrelevant}
\end{blockitems}

\begin{blockitems}{Disjoint Unions}
 \item disjoint union of some types $A\uplus B$\\
  values are $inj_1(x)$, $inj_2(y)$
  \glec{numbered injections $_1$, $_2$ --- order relevant}
 \item labeled disjoint union $a(A)|b(B)$ \\
  values are constructor applications $a(x)$, $b(y)$
  \glec{named injections $a$, $b$ --- order irrelevant}
\end{blockitems}

\glec{labeled disjoint unions uncommon}
\glec{but recursive labeled disjoint union = inductive data type}
\end{frame}

\begin{frame}\frametitle{A Basic Language for Data Types}
Let BDL be given by
\begin{commgrammar}
\gcomment{Types}\\
\gprod{T}{int \bnfalt float \bnfalt string \bnfalt bool}{base types}\\
\galtprod{\cn{list}\,T}{homogeneous lists}\\
\galtprod{\rep{(\ID:T)}}{record types}\\
\galtprod{\ldots}{additional types}\\
\gcomment{Data}\\
\gprod{D}{(64\, bit\, integers)}{}\\
\galtprod{(IEEE\, double)}{}\\
\galtprod{"(Unicode\, strings)"}{}\\
\galtprod{\cn{true} \bnfalt \cn{false}}{}\\
\galtprod{\rep{D}}{lists}\\
\galtprod{\rep{(\ID=D)}}{records}\\
\galtprod{\ldots}{constructors for additional types}\\
\end{commgrammar}
\end{frame}


\section{Recursive Concrete Data Types}

\begin{frame}\frametitle{Motivation}
Idea
\begin{itemize}
\item describe infinite type in finite way
\item describe words derived from grammars
\item exploit inductive structure to catch all values
\end{itemize}

Name: usually called \textbf{inductive} data type, especially when recursive
\end{frame}

\begin{frame}\frametitle{Examples}
Natural numbers $Nat$ given by
\begin{itemize}
\item $zero$
\item $succ(n)$ for every $n:Nat$
\end{itemize}

Lists $list\,A$ over type $A$ given by
\begin{itemize}
\item empty list $nil$
\item $cons(a,l)$ for every $a:A$ and $l:list\,A$
\end{itemize}

Arithmetic expressions $E$ given by
\begin{itemize}
\item natural number $literal(n)$ for $n:Nat$
\item sum $plus(e,f)$ for every $e,f:E$
\item product $times(e,f)$ for every $e,f:E$
\end{itemize}
\end{frame}

\begin{frame}[fragile]\frametitle{Rigorous Definition}
Let $T$ be the set of types that are known in the current context.

An \textbf{inductive data type} is given by
\begin{itemize}
 \item a name $n$, called the \textbf{type},
 \item a set of \textbf{constructors} each consisting of
 \begin{itemize}
  \item a name
  \item a list of elements of $T\cup\{n\}$, called the \textbf{argument} types
 \end{itemize} 
\end{itemize}

Notation:
\begin{lstlisting}
inductive n = c(A,...) | ...

inductive Nat = zero | succ(Nat)
inductive E = Number(Nat) | sum(E,E) | times(E,E)
\end{lstlisting}
\end{frame}

\begin{frame}[fragile]\frametitle{Induction Principle}
The values of the inductive type are exactly the ones that can be built by the constructors.
\begin{itemize}
\item No junk: the constructors are jointly-surjective
 \begin{itemize}
 \item no other values but union of their images
 \item closed world
 \end{itemize}
\item No confusion: the constructors are jointly-injective in the following sense
 \begin{itemize}
  \item each constructor is an injective function
  \item images of the constructor are pairwise disjoint
 \end{itemize}
\end{itemize}

Inductive definition: define function out of inductive type by giving one case per constructor
\lec{pattern matching}
\begin{itemize}
\item total because jointly-surjective
\item well-defined because jointly-injective (no overlap between cases)
\end{itemize}
\end{frame}

\begin{frame}[fragile]\frametitle{Special case: No recursion}
A concrete data types without recursive constructor arguments are called a \textbf{labeled union}.
They are isomorphic to the union of the products of the constructor arguments.

Example:
\begin{lstlisting}
inductive Value = Number(Nat) | true | false
inductive Product(A,B) = Pair(A,B)
\end{lstlisting}

A concrete data types without any constructor arguments is called an \textbf{enumeration}.
They have exactly one element per constructor.

Example:
\begin{lstlisting}
inductive Boolean = true | false
inductive Color = red | blue | green
\end{lstlisting}
\end{frame}

\begin{frame}[fragile]\frametitle{Generalization: Mutual Induction}
Multiple inductive types whose definitions refer to each other.

Example:
\begin{lstlisting}
inductive E = literal(Nat) | sum(E,E) | times(E,E) | ifte(F,E,E)
inductive F = equal(E,E) | less(E,E)
\end{lstlisting}
\end{frame}

\begin{frame}\frametitle{BDL Extended with Named Inductive Types}\label{def:bdl+cdt}
\begin{commgrammar}
\gprod{V}{\rep{Decl}}{Vocabularies}\\
\gprod{Decl}{\kw{inductive}\,t\,\{\rep{(\ID:\rep{T}\to t)}\}}{type definitions}\\
\gcomment{Types}\\
\gprod{T}{\ldots}{as before}\\
\galtprod{t}{reference to a named type}\\
\gcomment{Data}\\
\gprod{D}{\ldots}{as before}\\
\galtprod{\ID(\rep{D})}{constructor application}\\
\end{commgrammar}
\end{frame}

\section{Recursive Abstract Data Types}

\begin{frame}\frametitle{Breakout Question}
What do the following have in common?
\begin{itemize}
\item Java class
\item SQL schema for a table
\item logical theory (e.g., Monoid)
\end{itemize}
\onslide<2>{all are (essentially) abstract data types}
\end{frame}

\begin{frame}[fragile]\frametitle{Motivation}
Recall subject-centered representation of assertion triples:

\begin{lstlisting}
individual "FlorianRabe"
  is-a "instructor" "male"
  "teach" "WuV" "KRMT"
  "age" 40
  "office" "11.137"
\end{lstlisting}

Can we use types to force certain assertions to occur together?
\begin{itemize}
\item Every instructor should teach a list of courses.
\item Every instructor should have an office.
\end{itemize}
\end{frame}

\begin{frame}[fragile]\frametitle{Motivation}
Inspires \textbf{subject-centered types}, e.g.,

\begin{lstlisting}
concept instructor
  teach course$^*$
  age: int
  office: string

individual "FlorianRabe": "instructor"
  is-a "male"
  teach "WuV" "KRMT"
  age 40
  office "11.137"
\end{lstlisting}

Incidental benefits:
\begin{itemize}
\item no need to declare relations/properties separately
\item reuse relation/property names \\ distinguish via qualified names: \lstinline|instructor.age|
\end{itemize}
\end{frame}

\begin{frame}[fragile]\frametitle{Motivation}
Natural next step: inheritance

\begin{lstlisting}
concept person
  age: int
  
concept male <: person

concept instructor <: person
  teach course$^*$
  office: string

individual "FlorianRabe": "instructor" $\sqcap$ "male"
  "teach" "WuV" "KRMT"
  "age" 40
  "office" "11.137"
\end{lstlisting}

\lec{our language quickly gets a very different flavor}
\end{frame}

\begin{frame}\frametitle{Examples}
Prevalence of abstract data types:

\begin{center}
\begin{tabular}{l|ll}
aspect & language & abstract data type \\
\hline
ontologization & UML & class \\
concretization & SQL & table schema \\
computation & Scala & class, interface \\
deduction & various & theory, specification, module, locale \\
 & & if recursive: coinductive type \\
narration & various & emergent feature
\end{tabular}
\end{center}

\lec{same idea, but may look very different across languages}
\end{frame}

\begin{frame}\frametitle{Examples}
\begin{center}
\begin{tabular}{l|ll}
aspect & type & values \\
\hline
computation & abstract class & instances of implementing classes \\
concretization & table schema & table rows \\
deduction & theory & models
\end{tabular}
\end{center}

Values depend on the environment in which the type is used:
\begin{itemize}
\item class defined in one specification language (e.g., UML), \\
 implementations in programing languages Java, Scala, etc.
 \lec{available values may depend on run-time state}
\item theory defined in logic,\\
 models defined in set theories, type theories, programming languages
 \lec{available values may depend on philosophical position}
\end{itemize}
\end{frame}

\begin{frame}\frametitle{Definition}
Given some type system, an \textbf{abstract data type} (ADT) is defined by
  \[\kw{class}\, a = \{c_1:T_1[=t_1],\ldots,c_n:T_n[=t_n]\}\]
  where
  \begin{itemize}
  \item $c_i$ are distinct names
  \item $T_i$ are types
  \item $t_i$ are optional definitions; if given, $t_i:T_i$ required
  \end{itemize}

\begin{blockitems}{Recursion}
\item general case: $a$ may occur in the $T_i$
\item if non-recursive: called a record type
\end{blockitems}
\end{frame}

\begin{frame}\frametitle{BDL Extended with Named ADTs}\label{def:bdl+adt}
\begin{commgrammar}
\gprod{V}{\rep{Decl}}{Vocabularies}\\
\gprod{Decl}{\kw{class}\,t\,\{\rep{\ID:T}\}}{ADT definitions}\\
\gcomment{Types}\\
\gprod{T}{\ldots}{as before}\\
\galtprod{t}{reference to a named ADT}\\
\gcomment{Data}\\
\gprod{D}{\ldots}{as before}\\
\galtprod{t\{\rep{(\ID=D)}\}}{ADT elements}\\
\end{commgrammar}
\end{frame}

\begin{frame}[fragile]\frametitle{Inheritance}
Generalized ADT definition:

\begin{lstlisting}
abstract class $a$ extends $a_1,\ldots,a_m$ {
  $c_1$: $T_1$
  $\vdots$
  $c_n$: $T_n$
}
\end{lstlisting}

Terminology:
\begin{itemize}
\item $a$ \textbf{inherits} from $a_i$
\item $a_i$ are \textbf{super}-X or \textbf{parent}-X of $a$ where $X$ is whatever the language calls its ADTs (e.g., X=class)
\end{itemize}
\end{frame}

\begin{frame}\frametitle{Flattening}
Given ADT with inheritance as above, define \textbf{flattening} by
 \[\flt{a}=\flt{a_1}\cup \flt{a_m}\cup \{c_1:T_1,\ldots,c_n:T_n\}\]
 where duplicate field names are handled as follows
  \begin{itemize}
   \item same name, same type, same or omitted definition: merge
    \glec{details may be much more difficult}
   \item otherwise: ill-formed
  \end{itemize}
\end{frame}

\begin{frame}\frametitle{Subtleties}
We gloss over several major issues:
\begin{itemize}
\item How exactly do we merge duplicate field names? Does it always work?
 \lec{implement abstract methods, override, overload} 
\item Is recursion allowed, i.e., can I define an ADT $a=A$ where $a$ occurs in $A$?
 \lec{common in OO-languages: use $a$ in the types of its fields}
\item What about ADTs with type arguments?
 \lec{e.g., generics in Java, square-brackets in Scala}
\item Is mutual recursion between fields in a flat type allowed?
 \lec{common in OO-languages}
\item Is * commutative? What about dependencies between fields?
\end{itemize}
\lec{no unique answers}
\lec{incarnations of ADTs subtly different across languages}
\end{frame}

\begin{frame}\frametitle{Breakout question}
When using typed concrete data,\\
how to fully realize abstract data types
\begin{itemize}
\item nesting: ADTs occurring as field types
\item inheritance between ADTs
\item mixins
\end{itemize}
\end{frame}

\begin{frame}\frametitle{ADTs in Databases}
\begin{blockitems}{Nesting: field $a:A$ in ADT $B$}
\item field types must be base types, $a:A$ not allowed
\item allow $ID$ as additional base type
\item use field $a:ID$ in table $B$
\item store value of $b$ in table $A$
\end{blockitems}

\begin{blockitems}{Inheritance: $B$ inherits from $A$}
\item add field $parent_A$ to table $B$
\item store values of inherited fields of $B$ in table $A$
\end{blockitems}
\lec{general principle: all objects of type $A$ stored in same table}

\begin{blockitems}{Mixin: $A*B$}
\item essentially join of tables $A$ and $B$ on common fields
\item some subtleties depending on ADT flattening
\end{blockitems}
\end{frame}

\section{Subtyping}

\begin{frame}\frametitle{Subtyping}
\begin{itemize}
 \item relatively easy if all data types disjoint
 \item better with subtyping
 \lec{open problem how to do it nicely}
\end{itemize}

\begin{blockitems}{Subtyping Atomic Types}
 \item $\N <: \Z$
 \item ASCII $<:$ Unicode
\end{blockitems}

\begin{blockitems}{Subtyping Complex Types}
 \item covariance subtyping (= vertical subtyping)
 \glec{same for disjoint unions}
  \[A <: A' \impl list\,A <: list\, A'\]
  \[A_i <: A_i' \impl \{\ldots, a_i:A_i,\ldots\} <: \{\ldots, a_i:A'_i,\ldots\}\]
 \item structural subtyping (= horizontal subtyping)
  \[\{a:A,b:B\} :> \{a:A,b:B,c:C\}\]
  \[a(A)|b(B) <: a(A)|b(B)|c(C)\]
\end{blockitems}
\end{frame}

\begin{frame}\frametitle{Exercise 10}
As a single group, write a joint document that specifies a choice of built-in datatypes for BOL and SFOL.
They should be helpful in general and for the university ontology in particular.

For each new type, you have to give
\begin{itemize}
\item name
\item values
\item string encoding of each value (for printing, import/export)
\end{itemize}

Add the types and their values to your BOL and SFOL implementations and adjust your translation to translates values to themselves.
(Note: Not all types will be supported by theorem provers; so your TPTP printer may have to be partial.)
\end{frame}
