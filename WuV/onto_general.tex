\section{General Principles}\label{sec:onto:principles}

\paragraph{Motivation}
An ontology is an abstract representation of the main concepts in some domain.
Here \emph{domain} refers to any area of the real world such as mathematics, biology, diseases and medications, human relationships, etc.
Many examples can be found at \url{https://bioportal.bioontology.org/}, including the Gene ontology one of the biggest.

Contrary to the other four aspects, ontological knowledge representations do not aim at capturing the entire semantics of the domain objects.
Instead, they focus on defining unique identifiers for those objects and describing some of their properties and relations to each other.

We use the word \textbf{ontologization} to refer to the process of organizing the knowledge of a domain in ontologies.

Ontologies are most valuable when they are \emph{standardized} (either sanctioned through a formal body or a quasi-standard because everyone uses it).
A standard ontology allows everybody in the domain to use the identifiers defined by the ontology in a way that avoids misunderstandings.
Thus, in the simplest form, an ontology can be seen as a dictionary defining the technical terms of a domain.
For example, the Gene ontology defines identifier \texttt{GO:0000001} to have the formal name "mitochondrion inheritance" and the informal definition "The distribution of mitochondria, including the mitochondrial genome, into daughter cells after mitosis or meiosis, mediated by interactions between mitochondria and the cytoskeleton.".

\paragraph{Ontology Languages}
An ontology is written in \textbf{ontology language}.
Common ontology languages are
\begin{compactitem}
 \item description logics such as ALC,
 \item the W3C ontology language OWL, which is the standard ontology languages of the semantic web,
 \item the entity-relationship model, which focuses on modeling rather than formal syntax,
 \item modeling languages like UML, which is the main ontology language used in software engineering.
\end{compactitem}

Ontology languages are not committed to a particular domain --- in the Tetrapod model, they correspond to programming languages and logics, which are similarly uncommitted.
Instead, an ontology language is a formal language that standardizes the syntax of how ontologies can be written as well as their semantics.

\paragraph{Ontologies}
The details of the syntax vary between ontology languages.
But as a general rule, every \textbf{ontology} declares
\begin{compactitem}
 \item \textbf{individual} --- concrete objects that exist in the real world, e.g., "Florian Rabe" or "WuV"
 \item \textbf{concept} --- abstract groups of individuals, e.g., "instructor" or "course"
 \item \textbf{relation} --- binary relations between two individuals, e.g., "teach"
 \item \textbf{properties} --- binary relations between an individuals and a concrete value (a number, a date, etc.), e.g., "creditValue"
 \item \textbf{concept assertions} --- the statement that a particular individual is an instance of a particular concept
 \item \textbf{relation assertions} --- the statement that a particular relation holds about two individuals
 \item \textbf{property assertions} --- the statement that a particular individual has a particular value for a particular property
 \item \textbf{terminological axioms} --- statements about relations between concepts, typically in the form subconcept of statements like "instructor" $\sqsubseteq$ "person"
\end{compactitem}

All assertions can be understood and spoken as subject-predicate-object \textbf{triples} as follows:
\begin{center}
\begin{tabular}{l|lll}
Assertion & \multicolumn{3}{c}{Triple} \\
          & Subject & Predicate & Object \\
\hline
concept assertion  & "Florian Rabe" & \texttt{is-a} & "instructor" \\
relation assertion & "Florian Rabe" & "teach" & "WuV" \\
property assertion & "WuV" & "creditValue" & 7.5 \\
\end{tabular}
\end{center}
This uses a special relation \texttt{is-a} between individuals and concepts.
Some languages group \texttt{is-a} with the other binary relations between individuals for simplicity although it is technically a little different.

The possible values of properties must be fixed by the ontology language.
Typically, it includes at least standard types such as integers, floating point numbers, and strings.
But arbitrary extensions are possible such as dates, RGB-colors, lists, etc.
In advanced languages, it is possible that the ontology even introduces its own basic types and values.

Ontologies are often divided into two parts:
\begin{compactitem}
 \item The \textbf{abstract} part contains everything that holds in general independent of which individuals: concepts, relations, properties, and terminological axioms.
 It describes the general rules how the worlds works without committing to a particular set of inhabitants of the world.
 This part is commonly called the \textbf{TBox} (T for terminological).
 \item The \textbf{concrete} part contains everything that depends on the choice of individuals: individuals and assertion axioms.
 It populates the world with inhabitants.
 This part is commonly called the \textbf{ABox} (A for assertional).
\end{compactitem}

A separate division into two parts is the following:
\begin{compactitem}
 \item The \textbf{signature} part contains everything that introduces a \textbf{named entity}: individuals, concepts, relations, and properties.
 \item The \textbf{theory} part contains everything that describes which statements about the named entities are true: assertions and axioms.
\end{compactitem}


\paragraph{Synonyms}
Because these principles pervade all formal languages, many competing synonyms are used in different domains.
Common synonyms are:
\begin{center}
\begin{tabular}{l|llll|l}
 Here       & OWL      & Description logics & ER model & UML & semantics via logics\\
\hline
 individual & instance & individual & entity & object, instance & constant\\
 concept    & class    & concept &  entity-type & class & unary predicate\\
 relation   & object property & role & role & association & binary predicate \\
 property   & data property   & (not common) & attribute & field of base type & binary predicate\\
\end{tabular}
\end{center}

In particular, the individual-concept relation occurs everywhere and is known under many names:
\begin{center}
\begin{tabular}{l|ll}
 domain & individual & concept \\
\hline
type theory, logic & constant, term & type \\
set theory  & element & set \\
database    & row & table \\
philosophy\footnote{as in \url{https://plato.stanford.edu/entries/object/}} & object & property \\
grammar & proper noun & common noun \\
\end{tabular}
\end{center}

%%%%%%%%%%%%%%%%%%%%%%%%%%%%%%%%%%%%%%%%%%%%%
\section{A Basic Ontology Language}\label{sec:bolsyn}

\newcommand{\galtprodg}[2]{\galtprod{\color{gray}#1}{\color{gray}#2}}

\begin{figure}[hbt]
\begin{commgrammar}
\gcomment{Vocabularies: Ontologies}\\
\gprod{O}{\rep{D}}{}\\
\gcomment{Declarations}\\
\gprod{D}{\kw{individual}\; i}{atomic individual}\\
\galtprod{\kw{concept}\; c}{atomic concept}\\
\galtprod{\kw{relation}\; r}{atomic relation}\\
\galtprod{\kw{property}\; p: T}{atomic property}\\
\galtprod{\kw{axiom}\; F}{axiom}\\
\gcomment{Formulas}\\
\gprod{F}{C \Equiv C}{concept equality}\\
\galtprod{C \sqsubseteq C}{concept subsumption}\\
\galtprod{I\; \texttt{is-a}\; C}{concept instance}\\
\galtprod{I\; R\; I}{relation instance}\\
\galtprod{I\; P\; V}{property instance}\\
\gcomment{Individual expressions}\\
\gprod{I}{i}{atomic individuals}\\
\gcomment{Concept expressions}\\
\gprod{C}{c}{atomic concepts}\\
\galtprod{\top}{universal concept}\\
\galtprod{\bot}{empty concept}\\
\galtprod{C \sqcup C}{union of concepts}\\
\galtprod{C \sqcap C}{intersection of concepts}\\
\galtprod{\forall R.C}{universal relativization}\\
\galtprod{\exists R.C}{existential relativization}\\
\galtprod{\dom R}{domain of a relation}\\
\galtprod{\rng R}{range of a relation}\\
\galtprod{\dom P}{domain of a property}\\
\gcomment{Relation expressions}\\
\gprod{R}{r}{atomic relations}\\
\galtprod{R \cup R}{union of relations}\\
\galtprod{R \cap R}{intersection of relations}\\
\galtprod{R ; R}{composition of relations}\\
\galtprod{R^*}{transitive closure of a relation}\\
\galtprod{R^{-1}}{dual relation}\\
\galtprod{\Delta_C}{identity relation of a concept}\\
\gcomment{Property expressions}\\
\gprod{P}{p}{atomic properties}\\
\gcomment{Identifiers}\\
\gprod{i,c,r,p}{\text{alphanumeric string}}{}\\
\gcomment{Basic types and values}\\
\gprod{T}{\itg \bnfalt \float \bnfalt \bool \bnfalt \strg}{types}\\
\gprod{V}{\text{(omitted)}}{values}
\end{commgrammar}
\caption{Syntax of BOL}\label{fig:bol}
\end{figure}

\clearpage


We could study practical ontology languages like ALC or OWL now.
But those feature a lot of other details that can block the view onto the essential parts.
Therefore, we first define a basic ontology language ourselves in order to have full control over the details.

\begin{definition}[Syntax of BOL]\label{def:bolsyn}
A BOL-ontology is given by the grammar in Fig.~\ref{fig:bol}.
It is well-formed if
\begin{compactitem}
 \item no identifier is declared twice,
 \item every property assertion assigns a value of the type required by the property declaration,
 \item every reference to an atomic individual/concept/relation/property is declared as such.
\end{compactitem}
\end{definition}

The above grammar exhibits some general structure that we find throughout formal KR languages.
In particular, an ontology consists of \textbf{named declarations} of four different kinds of entities as well as some assertions and axioms about them.
Each entity declaration clarifies which kind it is (in our case by starting with a keyword) and introduces a new entity identifier.
For each kind, there are complex expressions.
These are anonymous and built inductively; their base cases are references to the corresponding identifiers.
Sometimes (in our case: individuals and properties), the references are the only expressions of the kind.
Sometimes (in our case: concepts and relations), there can be many productions for complex expressions.
The complex expressions are used to build axioms; in our case, these are the three kinds of assertions and other formulas.

\begin{remark}[Formulas vs. Assertions]\label{rem:bol:ass}
In Fig.~\ref{fig:bol}, the three productions in {\color{gray}gray} are duplicated: they occur both as assertions and as formulas.

We could remove the three productions for assertions and treat them as special cases of axioms.
But We keep the duplication here because assertions are often treated differently from the other axioms.
They are grouped with the individuals in the ABox whereas the other axioms are seen as part of the TBox.
Moreover, when used as assertions, they may have to be interpreted differently than when used as formulas as we will see in Ch.~\ref{sec:bolsem}.

Alternatively, we could remove the three productions in gray.
But then we would lose the ability to talk about formulas that are not true.
That will become relevant in Ch.~\ref{sec:bolquery}.
\end{remark}

\paragraph{Axioms}
The role of the axioms is two-fold:
\begin{compactitem}
 \item They can be used to perform \textbf{consequence closure}: a formula may expresses a closure operation that defines assertions that are automatically added to the ontology.
 That can be difficult as some kind of exhaustive reasoning is needed.
 For example, if there is a subconcept axiom "instructor" $\sqsubseteq$ "person" and a concept assertion "FlorianRabe" is-a "instructor", we have to add the implied concept assertion "Florian Rabe" is-a "person".
 \item They can be used to perform \textbf{consistency conditions} that must not violated by the ontology.
 For example, ontologies may contain contradictory assertions or violations of uniqueness constraints such as a person should only have one father or fathers should be male.
 The axioms exclude such cases.
 That should succeed if the assertions are already consequence-closed.
\end{compactitem}
But spelling out how that works is part of the semantics, not the syntax.

\begin{example}\label{ex:bol}
We give a simple ontology that could be used to represent knowledge in the context of a university:

\begin{lstlisting}
individual FlorianRabe
individual WuV
concept person
concept male
concept instructor
concept course
relation teach
property creditValue: float
FlorianRabe is-a instructor $\sqcap$ male
WuV is-a course
FlorianRabe teach WuV
WuV creditValue 7.5
male $\sqsubseteq$ person
instructor $\sqsubseteq$ person
dom teach $\sqsubseteq$ instructor
rng teach $\sqsubseteq$ course
dom creditValue $\Equiv$ course
course $\sqsubseteq$ $\exists$ teach$^{-1}$ instructor
\end{lstlisting}

The axioms are meant to state that males and instructors are persons, teaching is done by instructors to courses, exactly the courses have credits, and (the last axiom) every course is taught by at least one instructor.
Whether they actually do mean that, depends on the semantics.

The consequence closure (as defined by the semantics) should add the assertion \verb|FlorianRabe is-a person|.
Alternatively, if we use the axioms for consistency checking, we should add that assertion from the beginning.
Otherwise, the axioms would not be true.

If we use axioms for the consequence closure, we can even omit the two concept assertions --- they should be inferred using the domain and range axioms for the relation.

The assertion \lstinline|FlorianRabe is-a instructor $\sqcap$ male| could also be split into two assertions
\lstinline|FlorianRabe is-a instructor| and \lstinline|FlorianRabe is-a male|.
That will be important as some semantics might have difficulties handling all cases.
So it can be helpful to use a variant that does not need $\sqcap$ operator.
\end{example}

%%%%%%%%%%%%%%%%%%%%%%%%%%%%%%%%%%%%%%%%%%%%%
\section{Representing Ontologies as Triples}\label{sec:onto:triple}

It is common to represent an entire ontology as a set of subject-predicate-object triples.
That makes handling ontologies very simple and efficient.
This is the preferred representation of the semantic web.

However, while, e.g., relation assertions are naturally triples, not all declarations are, and some tricks may be necessary.

\paragraph{Inferring the Entity Declarations}
The entity declarations are not naturally triples.
But we can usually infer them from the assertions as follows: any identifier that occurs in a position where an entity of a certain kind is expected is assumed to be declared as an entity for that kind.

For example, the individuals are what occurs as the subject of a concept, relation, or property assertion or as the object of a relation assertion.
It is conceivable that there are individuals that occur in none of these.
But that is unusual because they would be disconnected from everything in the ontology.

If we give TBox and ABox together, this inference approach usually works well.
But if we only give a TBox, this would often not allow inferring all entities.
The only place where they could occur in the TBox is in the axioms, and it is quite possible to have concept, relation, and property declarations that are not used in the axioms.
In fact, it is not unusual not to have any axioms.

\paragraph{Special Predicates}
To turn declarations into triples, we can use reflection, i.e., the process of talking about our language constructs as if they were data.

Reflection requires introducing some built-in entities that represent the features of the language.
In the semantic web area, this is performed using the following entities:
\begin{compactitem}
 \item "rdfs:Resource": a built-in concept of which all individuals are an instance and thus of which every concept is a subconcept
 \item "rdf:type": a special predicate that relates an entity to its type:
  \begin{compactitem}
   \item an individual to its concept (corresponding to \texttt{is-a} above)
   \item other entities to their special type (see below)
  \end{compactitem}
 \item "rdfs:Class": a special class to be used as the type of classes
 \item "rdf:Property": a special class to be used as the type of properties
 \item "rdfs:subClassOf": a special relation that relates a subconcept to a superconcept
% \item "rdfs:subPropertyOf": a special relation that relates a relation to one that it implies
 \item "rdfs:domain": a special relation that relates a relation to the concepts of its subjects
 \item "rdfs:range": a special relation that relates a relation/property to the concept/type of its objects
\end{compactitem}
Here "rdf" and "rdfs" refer to the RDF (Resource Description Framework) and RDFS (RDF Schema) namespaces, which correspond to W3C standards defining those special entities.

Thus, we can represent many and in particular the most important entity declarations as triples:
\begin{center}
\begin{tabular}{l|lll}
Assertion & \multicolumn{3}{c}{Triple} \\
          & Subject & Predicate & Object \\
\hline
individual & individual & "rdf:type" & "rdfs:Resource" \\
concept  & concept & "rdf:type" & "rdf:Class" \\
relation & relation & "rdf:type" & "rdf:Property" \\
property & property & "rdf:type" & "rdf:Property" \\
concept assertion  & individual & "rdf:type" & concept \\
relation assertion & individual & relation & individual \\
property assertion & individual & property & value \\
\hline
\multicolumn{4}{l}{for special forms of axioms}\\
$c\sqsubseteq d$ & $c$ & "rdfs:subClassOf" & $d$ \\
%$r\sqsubseteq s$ & $r$ & "rdfs:subPropertyOf" & s \\
$\dom\,r\Equiv c$ & $r$ & "rdfs:domain" & $c$ \\
$\rng\, r\Equiv c$ & $r$ & "rdfs:range" & $c$ \\
\end{tabular}
\end{center}

This is subject to the restriction that only atomic concepts and relations can be handled.
For example, only concept assertions can be handled that make an individual an instance of an \emph{atomic} concept.
This is particularly severe for axioms, where complex expressions occur most commonly in practice.
Here, the special relations allow capturing the most common axioms as triples.

\paragraph{Problems}
Reflection is subtle and can easily lead to inconsistencies.
We can see this in how the approach of RDF(S) special entities breaks the semantics via FOL.

For example, it treats classes both as concepts (when they occur as the object of a concept assertion) and as individuals (when they occur as subject or object of a "rdfs:subClassOf" relation assertion).
Similarly, "rdfs:Class" is used both as an individual and as a class.
In fact, the standard prescribes that "rdfs:Class" is an instance of itself.

In practice, this is handled pragmatically by using ontologies that make sense.
A formal way to disentangle this is to assume that there are two variants of "rdfs:Class", one as an individual and one as a class.
The translation must then translate "rdfs:Class" differently depending on how it is used.

It would be better if RDFS were described in a way that is consistent under the implicitly intended FOL semantics.
But the more pragmatic approach has the advantage of being more flexible.
For example, being able to treat every class, relation, or property also as an individual makes it easy to annotate metadata to them.
Metadata is a set of properties such as "rdfs:seeAlso" or "owl:versionInfo", whose subjects can be any entity.


\paragraph{Subject-Centered Representations}
When giving a set of triples, there are usually a lot of triples with the same subject.
For example, we could use a simple concrete syntax with one triple per line and whitespace separating subject, predicate, and object:
\begin{lstlisting}
"FlorianRabe" is-a "instructor"
"FlorianRabe" is-a "male"
"FlorianRabe" "teach" "WuV"
"FlorianRabe" "teach" "KRMT"
"FlorianRabe" "age" 40
"FlorianRabe" "office" "11.137"
\end{lstlisting}

It is more human-friendly to group these triples in such a way that the subject only has to be listed once.
For example, we could use a concrete syntax like this, where the subject occurs first and then predicate-object pairs occur on indented lines:
\begin{lstlisting}
"Florian Rabe"
  is-a "instructor"
  is-a "male"
  "teach" "WuV"
  "teach" "KRMT"
  "age" 40
  "office" "11.137"
\end{lstlisting}

If the same predicate occurs with multiple values, we can group those as well.
For example, we could give the objects for the same predicates as a list following the predicate:
\begin{lstlisting}
"Florian Rabe"
  is-a "instructor" "male"
  "teach" "WuV" "KRMT"
  "age" 40
  "office" "11.137"
\end{lstlisting}

Concrete syntaxes based on the triple representation of ontologies will usually adopt some kind of structure like this.
The details may vary.
