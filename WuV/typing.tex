\section{Intrinsic vs. Extrinsic Typing}

\subsection{Overview}

We write $x:A$ to say that $x$ has type $A$.
There are two fundamentally different methods for introducing the types $A$, which are summarized in the following table:

\begin{center}
\begin{tabular}{l|ll}
& intrinsic & extrinsic \\
\hline
goes back to $\lambda$-calculus by & Church & Curry \\
general idea & objects carry their type with them & types are designated by the environment \\
typing is a & function from objects to types & relation between objects and types \\
objects have & unique type & any number of types \\
types often interpreted as & disjoint sets & unary predicates on a universal set \\
\hline
type inference for $x$ & uniquely infer $A$ from $x$ & try to find minimal $A$ such that $x:A$ \\
type checking & compare inferred and expected type & prove $x:A$ \\
subtyping $A<:B$ & mimicked by casting from $A$ to $B$ & defined by $x:A$ implies $x:B$ for all $x$ \\
typing decidable & yes unless too expressive & no unless expressivity restricted \\
typing errors are detected & usually statically (compile-time) & dynamically (run-time)\\
\hline
type of name introduced as & part of declaration & additional axiom \\
 \tb example               &  \kw{individual} "WuV":"course"  & \kw{individual} "Wuv", \; "WuV" \texttt{is-a} "course"\\
\hline
advantages & easy & flexible \\
           & unique type inference & allows subtyping \\
\hline
examples   & SFOL, SQL & OWL, Scala, English \\
           & most logics, functional PLs & ontology, OO, natural languages \\
           & many type theories & set theories
\end{tabular}
\end{center}

\begin{example}[Extrinsically Typed Ontology Language]
In BOL, the objects are the individuals, the types are the concepts, and \texttt{is-a} is the typing relation between them.
The typing is extrinsic:
\begin{compactitem}
 \item Individuals and their concept assertions are introduced in separate declarations.
 \item An individual may be an instance of any number of concepts.
 \item There is no primary concept that could be returned as the inferred type of an individual.
 \item Concepts are subject to subtyping $C\sqsubseteq C'$.
 \item Whether an individual is an instance of a concept, must be checked by reasoning about the \texttt{is-a} relation.
\end{compactitem}

Therefore, all semantics must interpret individuals as elements of a universal collection, and types as unary predicates on that.
Specifically, we have
\begin{center}
\begin{tabular}{l|lll}
semantics in  & universal collection & unary predicate & typing relation $i$ \texttt{is-a} $c$\\
\hline
FOL & type $\iota$  & predicate $c\sq\iota$ & $c(i)$ true\\
SQL & table Individuals & table containing ids & id of i in table $c$ \\
Scala & String & hash set of strings & $c$.contains($i$) \\
English & proper nouns & common nouns & "$i$ is a $c$" is true
\end{tabular}
\end{center}

We can also think of relations as objects.
However, BOL cannot express relation types at all, and there is no intrinsic typing.
Instead, the domain and range of a relation $r$ are given extrinsically via axioms about $\dom\,r$ and $\rng\,r$.
Like for individuals that allows flexibility as the same relations may have multiple types.
\end{example}

\begin{example}[Intrinsically Typed Ontology Language]
We could define TOL, a typed ontology language that arises as a variant of BOL.
The main differences would be
\begin{compactitem}
 \item Individuals are declared with a concept that serves as their type: \kw{individual} $i:C$.
 \item Concept assertions are dropped. They are now part of the individual declarations.
 \item Relations are declared with two concepts for their domain $D$ and range $R$: \kw{relation} $r<:D\times R$.
 \item Properties are declared with a concept for their domain $C$: \kw{property} $p<:C\times T$.
\end{compactitem}

TOL would make many ontologies more concise.
For example, we could simply write
\begin{lstlisting}
concept instructor
concept course
individual FlorianRabe : instructor
teach <: instrctor $\times$ course
\end{lstlisting}

However, we would lose flexibility.
If we want to add the concept "male", it would be difficult to make {\FR} have both types.
We might be able to remedy that by allowing intersections and declaring \lstinline|individual FlorianRabe: instructor $\sqcap$ male|.
But even then, we would have to commit to the type of each individual right away --- we cannot add different concept assertions for the same individual in different places, a common occurrence in building large ontologies.

Allowing $\sqcap$ would also introduce subtyping.
If we are careful in the design of TOL, that may still result in an elegant scalable language.
In particular, typing may remain decidable (depending on what other operations we allow).
But if we go too far, it may end up so complex that it would have been easier to go with extrinsic typing.

That is why we use intrinsic typing only in two related places in BOL:
\begin{compactitem}
 \item The base types and values use an intrinsic type system (whose details we omitted).
 \item The range of properties is given intrinsically by a base type.
\end{compactitem}
\end{example}

\begin{remark}[Subtyping]
Languages with subtyping usually have to use extrinsic type systems.
Typical sources of subtyping are
\begin{compactitem}
 \item explicit subtyping as in $\N<:\Z$
 \item comprehension/refinement as in $\{x:\N|x\neq 0\}$
 \item operations like union and intersection on types
 \item inheritance between classes, in which case subclass = subtype
 \item anonymous record types as in $\{x:\N,y:\Z\}<:\{x:\N\}$
\end{compactitem}
\end{remark}

\subsection{Combined Definition}

Neither intrinsic nor extrinsic typing is strictly better than the other.
The choice of type system is a very difficult trade-off when designing a language.

Many practical languages even combine both methods.
In that case, an intrinsic system is used for the most important high-level types and an extrinsic system is used to refine (some of) the high-level types:

\begin{definition}[Type System]
A \textbf{type system} consists of
\begin{compactitem}
 \item a collection, whose elements are called objects,
 \item a collection, whose elements are called intrinsic types,
 \item a function assigning to every object $x$ its intrinsic type $I$, in which case we write $x:I$,
 \item for some intrinsic types $I$
  \begin{compactitem}
   \item an intrinsic type $E_I$
   \item a relation $\in_I$ between objects with intrinsic types $I$ and $E_I$, called the extrinsic typing relation for $I$.
  \end{compactitem}
\end{compactitem}
\end{definition}

We can now recover the intuitions from above as special cases:
\begin{compactitem}
 \item A purely intrinsic type system is one in which $E_I$ and $\in_I$ are not given for any $I$.
 Thus, only objects and their intrinsic types remain.
 \item A purely extrinsic type system has two intrinsic types, namely $O$ (for objects) and $E_O$ (for types).
 $\in_O$ is the extrinsic typing relation between objects and types.
\end{compactitem}

\begin{example}
We can think of BOL as a combined type system.
The objects are all complex expressions.
The intrinsic types are the non-terminals $I$, $C$, $R$, $P$, and $F$, which separate the objects into the five kinds of individuals, concepts, relations, properties, and formulas.

An extrinsic typing relation exists only for $I$: we have $E_I=C$ and $\in_I$ is the \texttt{is-a} relation.
\end{example}

\begin{example}
In set theory, only a few intrinsic types are used for the high-level grouping of objects.
These include at least $set$ and $prop$.
Objects of these intrinsic types are called sets and propositions.
Some set theories also use an intrinsic type $class$.
Moreover, types like $set\to prop$ can be allowed as the types of unary predicates on sets.

Extrinsic typing is used only for the type $set$: we have $E_{set}=set$ and $\in_{set}$ is the usual elementhood relation between sets.
\end{example}

%%%%%%%%%%%%%%%%%%%%%%%%%%%%%%%%%%%%%%%%%%%%%%%%%%%%
\section{Abstract and Concrete Data Types}

The general thrust of type systems is to shift more and more information into an increasingly complex type system.
This is part of a trade-off: the more the type system can do,
\begin{compactitem}
 \item the more requirements can be expressed and violations thereof detected statically,
 \item the more complex the type system and its documentation and implementation become.
\end{compactitem}

Concrete and abstract data types have proved to be a particularly interesting trade-off on this expressivity-simplicity spectrum and are --- in one way or another --- part of many type systems.

\subsection{Abstract vs. Concrete Types}

The words \emph{abstract} and \emph{concrete} do not have standard definitions for types.
I like the intuitions described below.

A type is called \textbf{concrete} if its values are
\begin{compactitem}
\item given by their internal form,
\item defined along with the type, typically built from already-known pieces.
\end{compactitem}
In other words, a concrete type is given by an \emph{extensional} description of its objects.

\begin{example}
Simple products are concrete types.

They are introduced by (among other rules)
\begin{compactitem}
 \item $A\times B$ is a type if $A$ and $B$ are
 \item values of type $A\times B$ are of the form $(a,b)$ for $a:A$ and $b:B$.
\end{compactitem}
\end{example}

\begin{example}
\textbf{Enumerations} are concrete types that explicitly list a finite set of distinct values.

They can also be seen as the special case of inductive data types in which all constructors take $0$ arguments.
\end{example}

\begin{example}
Inductive data types as seen in Def.~\ref{def:idt} are concrete types.
Their values are formed by applying constructors to other values.

I like calling them \textbf{concrete data types}.
\end{example}

A type is called \textbf{abstract} if its values are
\begin{compactitem}
\item given by their externally visible properties,
\item defined in any environment that understands the type definition.
\end{compactitem}
In other words, an abstract type is given by an \emph{intensional} description of its objects.

This is the case for \textbf{abstract data types}.

%%%%%%%%%%%%%%%%%%%%%%%%%%%%%%%%%%%%%%%%%%%%%%%%%%%%%%%%
\subsection{Concrete Data Types}

Concrete data types are the same as inductive data types and were already defined rigorously in Sect.~\ref{sec:idt}.

It is in the nature of concrete types that they do not depend on the situation in which they are used.
The definition of the type once and for all defines the possible values.
Thus, they do not differ (or differ at most subtly) between languages.
Often the only real difference is whether a language has concrete data types or not.
Languages that do not natively provide them may differ substantially in how they simulate them though.

%%%%%%%%%%%%%%%%%%%%%%%%%%%%%%%%%%%%%%%%%%%%%%%%%%%%%%%%
\subsection{Abstract Data Types}

Contrary to concrete types, the values of an abstract data types depend on what features the language provides to construct objects that satisfy the abstract requirements.
Therefore, the treatment differs widely across languages.
The following table gives an overview:

\begin{center}
\begin{tabular}{l|ll}
aspect & language & abstract data type \\
\hline
ontologization & UML & class \\
concretization & SQL & table schema \\
computation & Scala & class, interface \\
deduction & various & theory, specification, module, locale \\
narration & various & emergent feature
\end{tabular}
\end{center}

\begin{example}[Classes]
A UML class is an abstract data type.
Its values are the instances of implementing classes.

A UML class only defines what methods should be available.
How they are implemented by specific values of the type is left to the programming languages.

Thus, different programming languages could have different values for the same abstract data type.
They certainly look different, e.g., in Java and Scala implementations of the same UML class.
But the languages might also be fundamentally different in expressivity, e.g., a Turing-complete programming language might have strictly more values for the same abstract data type than a non-Turing-complete one.

Moreover, which instances actually exist changes during the run time of the program.
If we take this into account, the values of the abstract data type are not even fixed within a programming language.
\end{example}

\begin{example}[Schemas]
An SQL table schema is an abstract data type.
Its values are the rows.

The schema only defines what types the columns of a table have.
Different database systems might theoretically provide different ways to build rows for the table.

However, this does not happen in practice because SQL table columns are typed by base types, which have the same values across database systems.
This would be different if we allowed table columns to have function types.
\end{example}

\begin{example}[Theories]
A logical theory (e.g., Monoid) is an abstract data type.
Its values are the models of the theory (e.g., for Monoid: $(\N,+,0)$ or $(\N,*,1)$).

The theory only defines what operations a model must provide (for Monoid: binary operation and neutral element) and which axioms it must satisfy (for Monoid: associativity, neutrality).
How we build the models is left open.

We usually build models in mathematical language and naively assume that fixes the models once and for all.
But that is too naive: depending on which mathematical foundation we use (e.g., set theory with or without axioms of choice), we can build different models.
Moreover, we can also build models in type theories (which underly many deduction systems such as Coq or Isabelle).
We can even build them in programming languages, e.g., by implementing theories as classes (typically moving the axioms into comments).

The choice of language substantially changes what the values of the abstract data type are.
\end{example}

In the sequel we introduce a fairly general definition that subsumes many practical languages.

\begin{definition}[Abstract Data Type]
Consider an arbitrary type system.

An \textbf{abstract data type} (ADT) is
\begin{compactitem}
\item a \textbf{flat} type of the form
  \[\{c_1:T_1[=t_i],\ldots,c_n:T_n[=t_i]\}\]
  where the $c_i$ are distinct names, the $T_i$ are types, and the $t_i$ are optional and wherever given must have type $T_i$, or
\item a \textbf{mixin} type of the form $A_1*A_2$ for ADTs $A_i$.
\end{compactitem}

We say that a type system has \textbf{internal ADT}s if all ADTs are types (and thus may in particular occur as the $T_i$ in a record type).
\end{definition}

The intuition of a mixin $A*B$ is that we merge the fields of $A$ and $B$.
However, this union dependent: if $B$ is flat, its fields may refer to fields introduced in $A$.

The most important special case of an ADT are classes:

\begin{definition}[Class]
A class definition defines an ADT abbreviation of the form
\[a = a_1*\ldots*a_m*\{c_1:T_1,\ldots,c_n:T_n\}\]
where the $a_i$ are names of previously defined ADTs.

We call the $a_i$ the \textbf{superclasses} or \textbf{parent classes} and say that $a$ inherits from the $a_i$.
We call the $c_i$ the \textbf{fields} or \textbf{members} of $a$.
\end{definition}

In an OO-language, a class definition is more commonly written somehow like
\begin{lstlisting}
abstract class $a$ extends $a_1$ with $\ldots$ with $a_m$ {
  $c_1$: $T_1$
  $\vdots$
  $c_n$: $T_n$
}
\end{lstlisting}
The details can vary, and special care must be taken in programming languages where initialization may have side effects.

Flat ADTs are the standard case, and all mixin ADTs can be simplified into flat ones.
This can be seen as a semantics in the sense that the language of flat and mixin ADT is translated to the language of flat ADTs.
\begin{definition}[Mixin Semantics]\label{def:mixinflat}
The \textbf{flattening} $\flt{A}$ of an ADT $A$ is defined as follows:
\begin{compactitem}
 \item if $A$ is flat: $\flt{A}=A$
 \item if $A$ is of the form $A_1*\ldots*A_n$:
 $\flt{A}$ arises by concatenating the fields of all $\flt{A_i}$ where duplicate field names are handled as follows:
  \begin{compactitem}
   \item if the same field (same name, types equal, definitions equal or both absent) occurs more than once, only the first occurrence is kept,
   \item if the fields $c:T_1[=t_i]$ and $c:T_2[=t_2]$ occur for inequal types $T_i$, $A$ is ill-formed,
   \item if the fields $c:T=t_1$ and $c:T=t_2$ occur for inequal objects $t_i$, $A$ is ill-formed,
   \item if the fields $c:T=t$ and $c:T$ occur, only the defined one is kept $(\ast)$.
  \end{compactitem}
\end{compactitem}
\end{definition}

\begin{remark}[Dependency Between Fields]
Our definition sweeps a very important but subtle detail under the rug: in a flat ADT with a field $c:T=t$, may $T$ and/or $t$ refer to fields declared later?
We sketch a few possible answers.

In the simplest case, we forbid such forward references.
Then ADTs are very well-behaved.
But we have a problem with the case $(\ast)$ in Def.~\ref{def:mixinflat}: if $c:T$ occurs before $c:T=t$, we cannot simply drop the former because intermediate fields may refer to $c$.
A straightforward solution would be to declare the ADT to be ill-formed.
But unfortunately, this case is very important in practice --- it occurs whenever $c:T$ is declared in an abstract class and $c:T=t$ in a concrete class implementing it.

A more common solution is to allow the fields to be mutually recursive.
Consider a flat ADT with fields $\Gamma, c:T[=t], \Delta$ where $\Gamma$ and $\Delta$ are lists of fields.
Let $\Gamma'$ and $\Delta'$ arise by dropping all definitions.
Then we require that
\begin{compactitem}
 \item $T$ must be a well-formed type in context $\Gamma'$.
 Thus, the types may only refer to previous fields.
 \item $t$ must have type $T$ in context $\Gamma',c:T,\Delta'$.
 Thus, the definitions may be mutually recursive.
\end{compactitem}
This makes the case $(\ast)$ work.
But it comes at the price of recursion, which allows writing non-terminating fields (a feature in a programming language, but potentially undesirable in other settings).

Even so, the mutual-recursion solution is problematic in the presence of dependent types.
Here, dropping definitions is not always allowed:
$T$ might be well-formed in context $\Gamma$, but $\Gamma'$ might not even be a well-formed context at all.
Because OO-languages are usually not dependently-typed, this is not an issue in most settings.
\end{remark}

%%%%%%%%%%%%%%%%%%%%%%%%%%%%%%%%%%%%%%%%%%
\section{Abstract Data Types in Ontologies}

\subsection{Motivation}

Recall the subject-centered representation of individuals described in Sect.~\ref{sec:onto:triple}.
Here we introduce an individual together with all assertions of which it is the subject as in
\begin{lstlisting}
individual "FlorianRabe"
  is-a "instructor" "male"
  "teach" "WuV" "KRMT"
  "age" 40
  "office" "11.137"
\end{lstlisting}

It is often desirable to use types to force the presence of such assertions.
We might wish require that every instructor teaches a list of things, and has an office.
Moreover, we can use types to specify the objects of the respective assertions: we can specify that only courses are taught and that the office is a string.
Rather than the relations with subjects "FlorianRabe" just happening to be around as well, the type system would now force their existence and the type of the object.
Forgetting to give such an assertion or giving it with the wrong object could be detected statically (i.e., without applying the semantics) and flagged as a typing error.

This leads to the idea of \textbf{subject-centered types}.
This could looks as follows:
\begin{lstlisting}
concept instructor
  teach course$^*$
  age: int
  office: string

individual "FlorianRabe": "instructor"
  is-a "male"
  "teach" "WuV" "KRMT"
  "age" 40
  "office" "11.137"
\end{lstlisting}
Now the type "instructor" forces the presence of a list of taught courses (The $^*$ is meant to indicate a list.), an integer for the age, and a string for the office.

We can now see that, in fact, every person should have an age, and not just every instructor.
Because every instructor is meant to be a person, we could try to capture this as well to avoid redundancy.
Moreover, every male is meant to be a person, too.

That leads to the idea of \textbf{modular types}.
This could look as follows:

\begin{lstlisting}
concept person
  age: int
  
concept male <: person

concept instructor <: person
  teach course$^*$
  office: string

individual "FlorianRabe": "instructor" $\sqcap$ "male"
  "teach" "WuV" "KRMT"
  "age" 40
  "office" "11.137"
\end{lstlisting}

Incidentally, that eliminates the need to independently declare relations and properties.
Instead, we can treat their occurrences inside the concept definitions as their declarations.

That has the added benefit that two relations/properties of the same name declared in different concepts can be distinguished and can have different types.
%For example, the relation "person.parent" could relate between a subject of type "person" and an object of type "person", and the relation

\subsection{Database Schemas as Typed Ontologies}

It is very typical that optimized systems are developed for one of the corner aspects.
These systems then subsume an ontology language.
From the Tetrapod perspective, it is instructive to tease apart the ontology.

In particular, what is typically called an ontology follows the semantic web community.
Tetrapodally, they are working on the combination of ontology and concrete data.
The former defines the rules of the world (TBox), the latter populates the world (ABox).

We can see the same pattern in the relational database community.
Tetrapodally, the SQL schema defines the ontology, and the database holds the concrete data.
Fig.~\ref{fig:semweb-reldb} shows the analogy in detail.

\begin{figure}[hbt]
\begin{center}
\begin{tabular}{l|ll}
  & semantic web & relational databases \\
\hline
ontology aspect & TBox of ontology & SQL schema \\
conceptual model & knowledge graph & set of tables \\
concrete data aspect & ABox of ontology & SQL database \\
concrete data storage & set of triples & set of rows of the tables \\
concrete data formats & RDF & CSV \\
concrete data tool & triple store & database implementation \\
typing & soft/Curry & hard/Church\\
query language & SPARQL & SQL SELECT query \\
openness of world & tends to be open & tends to be closed \\
\end{tabular}
\caption{The Two Kinds of Concrete Data Systems}\label{fig:semweb-reldb}
\end{center}
\end{figure}

We can then understand relational databases as an ontology+concrete data approach that focuses on abstract data types.
Each table is one such type, and each column is a field of the type.
The database holds the concrete data, i.e., the rows of the tables.

For example, the SQL schema might declare an ADT by
\begin{lstlisting}
TABLE person(id: ID, name: string, age: int)
TABLE male(id: Ref person)
TABLE instructor(id: ID, parent: Ref person, teach: course$^*$, office: string)
\end{lstlisting}
Here we write \lstinline|Ref $T$| for the type \lstinline|ID| of identifiers if we want to emphasize that it should hold an identifier that is present in the table $T$.

Concepts like \lstinline|male| can be represented as single-column tables holding the identifiers of individuals that are in the concept.
Inheritance can be represented by using two separate tables for parent and child type with the latter holding a special column \lstinline|parent| holding an identifier in the parent table.
For example, concrete data could be inserted as

\begin{lstlisting}
INSERT VALUES (id = X, name = "FlorianRabe", age = 40) INTO person
INSERT VALUES X INTO male
INSERT VALUES (id = Y, parent = X, teach = ["WuV", "KRMT"], office = "11.137")
  INTO instructor
\end{lstlisting}

Usually, relations where the same subject can have multiple objects are not stored as a list-valued column but as a separate two-column table.
For example, we would have
\begin{lstlisting}
TABLE instructor(id: ID, parent: Ref person, office: string)
TABLE course(id: ID, name: string)
TABLE teach(subject: Ref person, object: Ref Course)
INSERT VALUES (id = W, name = "WuV") INTO course
INSERT VALUES (id = K, name = "KRMT") INTO course
INSERT VALUES (subect = X, object = W) INTO teach
INSERT VALUES (subect = X, object = K) INTO teach
\end{lstlisting}


\subsection{Exercise 3}

The topic of Exercise 3 is to build a relational database schema for a univis-like system.

%\section{A Typed Ontology Language}
%
%\begin{figure}[hbt]
%\begin{commgrammar}
%\gcomment{Ontologies}\\
%\gprod{O}{\rep{D}}{}\\
%\gcomment{Declarations}\\
%\gprod{D}{\kw{adt}\; \ID\; \kw{extends}\;\rep{\ID} \{\rep{D}\}}{ADT definition}\\
%\galtprod{\kw{obj}\; \ID\; \kw{implements}\;\ID \{\rep{D}\}}{object definition}\\
%\galtprod{\kw{val}\; \ID : A \opt{= E}}{value declaration/definition}\\
%\galtprod{F}{axioms}\\
%\gcomment{Formulas}\\
%\gprod{F}{E:A}{typing}\\
%\galtprod{A<:A}{subtyping}\\
%\galtprod{E\doteq_A E}{equality at a type}\\
%\gcomment{Type expressions}\\
%\gprod{A}{\ID}{ADTs}\\
%\galtprod{T}{base types}\\
%\gcomment{Value expressions}\\
%\gprod{E}{\ID}{objects}\\
%\galtprod{V}{base values}\\
%\gcomment{Identifiers}\\
%\gprod{\ID}{\text{alphanumeric string}}{}\\
%\gcomment{Basic types and values}\\
%\gprod{T}{\itg \alt \float \alt \bool \alt \strg}{types}\\
%\gprod{V}{\text{(omitted)}}{values}
%\end{commgrammar}
%\caption{Syntax of TOL}\label{fig:tol}
%\end{figure}

