\section{Language Layers}


\begin{frame}\frametitle{Layers of Language Design}
\begin{tabular}{l|ll}
Layer & Specified by & Implemented by \\\hline
Syntax & &\\
\tb Context-Free & grammar & AST+parser+printer \\
\tb Context-Sensitive & inference system & type checker \\
Semantics & \multicolumn{2}{l}{inference system, interpretation, or translation}\\
\hline
Pragmatics & human preferences & human judgment \\
\end{tabular}
\bigskip

KRP = syntax + semantics
\end{frame}

\begin{frame}\frametitle{Layered Processing}
Data is processed in phases
\begin{enumerate}
\item data representation format, e.g., string, JSON, XML, binary
\item parsed --- well-formed syntax tree
\item type-checked by traversal of the syntax tree --- well-typed syntax tree
\item computation by traversal of well-typed AST --- semantics
\end{enumerate}
\end{frame}

\begin{frame}\frametitle{Possible Errors}
\begin{tabular}{l|l}
Layer & Error \\\hline
CFS & not derivable from grammar \\
CSS  & symbols not used as declared, other conditions\\
Sem. & ambiguous/undefined semantics \\
\hline
Pragmatics & not useful \\
\end{tabular}
\end{frame}


\begin{frame}\frametitle{Typical Errors by Layer}
In a programming language:

\begin{center}
\begin{tabular}{l|lll}
Layer & Expression & Issue & Explanation \\\hline
CFS & $1/$ & syntax error & argument missing\\
CSS & $1/"2"$ & typing error & wrong type\\
Sem. & $1/0$ & run-time error & undefined semantics \\
Pragm. & $1/1$ & code review & unnecessarily complex\\
\end{tabular}
\end{center}
\end{frame}

\begin{frame}\frametitle{Typical Errors by Layer}
In a logic:

\begin{center}
\begin{tabular}{l|lll}
Layer & Expression & Issue & Explanation \\\hline
CFS & $\forall x$ & not well-formed & body missing\\
CSS & $\forall x.P(y)$ & not well-typed & $y$ not declared\\
Sem. & the $x\in \N$ with $x<0$ & not well-defined & no such $x$ exists \\
Pragm. & $\exists x.x\neq x$ & not useful & no model exists\\
\end{tabular}
\end{center}
\end{frame}

\section{Context-Free Grammars}

\begin{frame}\frametitle{The Chomsky Hierarchy}
\begin{itemize}
\item CH-0, regular grammars: 
 \begin{itemize}
  \item equivalent to regular expressions and finite automata
  \item not used much as grammars
 \end{itemize}
\item CH-1, context-free grammars (CFGs) \lec{our focus}
\item CH-2, context-sensitive grammars
 \begin{itemize}
   \item important as languages, but awkward as grammars
   \item instead: type system determines subset of context-free language
 \end{itemize}
\item CH-3, unrestricted grammars
 \begin{itemize}
   \item Turing-complete, theoretically important
   \item not used much as grammars
 \end{itemize}
\end{itemize}
\end{frame}

\begin{frame}\frametitle{Definitions}
\begin{itemize}
\item An alphabet is a set of symbols.
\item A word is a list of symbols from the alphabet.
\item A production is pair of words.
 \begin{itemize}
 \item A production is written $lhs::=rhs$.
 \item Multiple productions for the same left-hand side are abbreviated $lhs::=rhs_1 \,|\,\ldots\,|\, rhs_n$.
 \item Right-hand side may also use regular expressions like $^*$ for repetition and $[]$ for optional parts.
 \end{itemize}
\item A CFG is a set of productions where $lhs$ is a single symbol.
 \begin{itemize}
 \item If there is a production $N::=rhs$, $N$ is called non-terminal, otherwise terminal.
 \item If a word contains non-terminal symbols, it is called non-terminal, otherwise terminal.
 \end{itemize}
\item A syntax tree is a tree with nodes are labeled with productions $N::=rhs$ where the non-terminals in rhs are exactly the lhs's of the children.
\item The word produced by a syntax tree is read off by exhaustively replacing every lhs with the respective rhs.
\end{itemize}
\end{frame}

\begin{frame}\frametitle{Example: Syntax of Arithmetic Language}
\begin{commgrammar}
\gcomment{Numbers}\\
\gprod{N}{0\bnfalt 1}{literals}\\
\galtprod{N+N}{sum}\\
\galtprod{N*N}{product}\\
\gcomment{Formulas}\\
\gprod{F}{N\doteq N}{equality}\\
\galtprod{N\leq N}{ordering by size}\\
\end{commgrammar}
\end{frame}

\section{Implementing CFGs via Inductive Data Types}

\begin{frame}\frametitle{Correspondence}
\begin{center}
\begin{tabular}{l|l}
CFG & IDT \\
\hline
non-terminal & type \\
production & constructor \\
non-terminal on left of production & return type of constructor \\
non-terminals on right of production & arguments types of constructor \\
terminals on right of production & notation of constructor\\
words derived from non-terminal $N$ & expressions of type $N$
\end{tabular}
\end{center}
\end{frame}

\begin{frame}\frametitle{Classes of Languages}
Functional languages:
\begin{itemize}
\item pure: ML, Haskell
\item with OO: F\#, Scala
\end{itemize}
\lec{inductive types are primitive}

OO-languages:
\begin{itemize}
\item C\#, Java, C++
\end{itemize}
\lec{inductive types simulated via classes}

Untyped languages:
\begin{itemize}
\item Python, Javascript
\end{itemize}
\lec{inductive types simulated ad hoc}
\end{frame}

\begin{frame}\frametitle{Implementing the Example}
Done interactively. See the examples in the repository. See also the notes.
\end{frame}

\section{Context-Sensitive Syntax}

\begin{frame}\frametitle{Vocabularies and Declarations}
Generic structure of a context-sensitive language
\begin{itemize}
 \item a vocabulary is a list of declarations
  \begin{itemize}
  \item named: type/function/predicate symbol etc.
  \item unnamed: axioms etc.
  \item structural: inclusion/import, datatype definitions
  \end{itemize}
 \item named declarations introduce atomic objects of different kinds
 \item for each kind, a non-terminal for complex expressions of that kind
 \item references to names introduced by declarations are base cases of expressions
\end{itemize}
\end{frame}

\begin{frame}\frametitle{Example: Typed Expressions}
\small
\begin{commgrammar}
\gcomment{Vocabularies}\\
\gprod{Voc}{\rep{Decl}}{list of declarations}\\
\gcomment{Declarations}\\
\gprod{Decl}{id:\rep{Type}\to Type}{typed function symbols}\\
\galtprod{id:\rep{Type}\to Form}{typed predicate symbols}\\
\gcomment{Types}\\
\gprod{Type}{Nat \bnfalt String}{base types}\\
\gcomment{Expressions}\\
\gprod{Expr}{0\bnfalt 1\bnfalt Expr+Expr \bnfalt Expr*Expr}{as before}\\
\galtprod{id(\rep{Expr})}{application of a function symbol}\\
\gcomment{Formulas}\\
\gprod{Form}{Expr\doteq Expr \bnfalt Expr\leq Expr}{as before}\\
\galtprod{id(\rep{Expr})}{application of a predicate symbol}\\
\end{commgrammar}
\end{frame}

\begin{frame}\frametitle{Example: Vocabularies and Expressions}
Example vocabulary $V$ containing the following declarations:
\begin{itemize}
\item $fib:Nat \to Nat$
\item $length: String \to Nat$
\item $mod: Nat,Nat \to Nat$
\item $prime: Nat \to Form$
\end{itemize}

Example expressions relative to $V$
\begin{itemize}
\item expressions: $fib(0)$, $mod(fib(fib(1)),1+1)$
\item formulas: $fib(0)=0$, $prime(fib(1))$
\end{itemize}
\end{frame}

\begin{frame}\frametitle{Primitive vs. Declared}
\begin{blockitems}{Primitive}
 \item built into the language
 \item assumed to exist a priori \lec{fundamentals of nature}
 \item fixed semantics (usually interpreted by identity function)
 \end{blockitems}
 
\begin{center}
\begin{tabular}{l|ll}
& primitive & declared \\
\hline
introduced by & language designer & user \\
introduced in & grammar & vocabulary $V$ \\
visible in & all vocabularies & $V$ only \\
semantics given & explicitly & implicitly \\
\tb\ldots by & translation function & axioms \\
\end{tabular}
\end{center}
\lec{more expressive declarations $\to$ fewer primitives needed}
\lec{paradoxical: more complex language may have simpler grammar}
\end{frame}

\begin{frame}\frametitle{Quasi-Primitive = Declared in standard library}
\begin{blockitems}{Standard library: a vocabulary $StdLib$}
 \item present in every language
  \glec{empty vocabulary by default}
 \item one fixed vocabulary
  \begin{itemize}
  \item implicitly included into every other vocabulary
  \item implicitly fixed by any translation between vocabularies
  \end{itemize}
Combination of advantages
\begin{itemize}
\item from the user's perspective: like a primitive
\item from the theory's/system's perspective: no special treatment
\end{itemize}
\end{blockitems}

\begin{blockitems}{Examples}
\item sufficiently expressive languages
 \begin{itemize}
 \item push many primitive objects to standard library \glec{never all}
 \item simplifies language, especially when defining operations
 \end{itemize}
 \lec{strings in C, BigInteger in Java, inductive type for $\N$}
\item inexpressive languages
\begin{itemize}
\item many primitives \lec{SQL, spreadsheet software}
\item few (quasi)-primitives \lec{few operations available in OWL}
\end{itemize}
\end{blockitems}
\end{frame}

\begin{frame}\frametitle{Example: Removing Built-in Operations}
Grammar without built-in operations
\small
\begin{commgrammar}
\gprod{Voc}{\rep{Decl}}{list of declarations}\\
\gprod{Decl}{id:\rep{Type}\to Type}{typed function symbols}\\
\galtprod{id:\rep{Type}\to Form}{typed predicate symbols}\\
\gprod{Type}{Nat \bnfalt String}{base types}\\
\gprod{Expr}{id(\rep{Expr})}{application of a function symbol}\\
\gprod{Form}{id(\rep{Expr})}{application of a predicate symbol}\\
\end{commgrammar}

Standard library:
\begin{itemize}
\item $0:Nat$, $1:Nat$, $sum: Nat\,Nat\to Nat$, $product: Nat\,Nat\to Nat$,
\item $equals: Nat\,Nat\to FORM$, $lesseq: Nat\,Nat\to FORM$
\end{itemize}
\end{frame}

\begin{frame}\frametitle{Example: Removing more Primitive Operations}
If we add type declarations, we can remove $Nat$ as well
\small
\begin{commgrammar}
\gprod{Voc}{\rep{Decl}}{list of declarations}\\
\gprod{Decl}{id:\rep{Type}\to Type}{typed function symbols}\\
\galtprod{id:\rep{Type}\to FORM}{typed predicate symbols}\\
\galtprod{id:TYPE}{type symbols}\\
\gprod{Type}{id}{reference to a type symbol}\\
\gprod{Expr}{id(\rep{Expr})}{application of a function symbol}\\
\gprod{Form}{id(\rep{Expr})}{application of a predicate symbol}\\
\end{commgrammar}

\glec{Note: $Type$ and $Form$ are non-terminals, $TYPE$ and $FORM$ are not}

Add to default vocabulary: $Nat: TYPE$, $String: TYPE$
\end{frame}

\begin{frame}\frametitle{Context-Sensitivity}
A reference to a declared name must respect the way in which it was declared in the vocabulary
\glec{examples errors relative to $V$ above}
\begin{itemize}
 \item occur in a position where an expression of the right kind is expected
  \glec{example error: $prime(1)=1$}
 \item be applied to the right number of arguments
  \glec{example error: $prime(1,1)$}
 \item if a type system is used
  \begin{itemize}
  \item arguments must have the right types
    \glec{$length(1)$}
  \item return type must match what is expected
    \glec{$fib(1)$ if a string is expected}
  \end{itemize}
\end{itemize}
\end{frame}

\begin{frame}\frametitle{Syntax Traversal}
Context-free traversal
\begin{itemize}
\item one function for each non-terminal/inductive type \glec{mutually recursive}
\item for each such function, one case for each production/constructor
\item for each such case, one recursive call for each non-terminal on the rhs/constructor argument
\end{itemize}

Context-sensitive traversal: as above but
\begin{itemize}
\item functions take extra argument for vocabulary
\item cases for identifier references look up the respective declaration in the vocabulary
\end{itemize}
\end{frame}

\begin{frame}\frametitle{Type Checker}
Syntax checker
\begin{itemize}
 \item context-sensitive traversal where all functions return booleans
 \item case for vocabulary checks each declaration relative to the preceding vocabulary
 \item cases for identifier references check correct use of identifier
\end{itemize}

Type checker: as above but additionally
\begin{itemize}
 \item functions for typed expression additionally take expected type as argument
 \item cases for identifiers references
   \begin{itemize}
   \item check each argument against declared input type
   \item compare output to expected type
   \end{itemize}
\end{itemize}
\end{frame}

\begin{frame}\frametitle{Exercise 4}
Individually, using any programming language, implement the AST for the BOL language.
Allow for integers and strings as basic types.

Implement a type-checker for BOL.
BOL is untyped, and not much type-checking is needed.
The main check needed is that all property assertion use values according to the property type.
\end{frame}
