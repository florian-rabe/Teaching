\section{Typed First-Order Logic}

\begin{frame}\frametitle{Syntax: Declarations}
\begin{commgrammar}
\gcomment{Vocabularies: theories}\\
\gprod{T}{\rep{D}}{}\\
\gcomment{Declarations}\\
\gprod{D}{\kw{type}\; y}{type declaration}\\
\galtprod{\kw{fun}\; f:\rep{Y}\to Y}{function symbol declaration}\\
\galtprod{\kw{pred}\; p\sq\rep{Y}}{predicate symbol declaration}\\
\galtprod{\kw{axiom}\;F}{axiom}\\
\gcomment{Identifiers}\\
\gprod{y,f,p,x}{\text{alphanumeric string}}{}\\
\end{commgrammar}
\end{frame}

\begin{frame}\frametitle{Syntax: Expressions}
\begin{commgrammar}
\gcomment{type expressions}\\
\gprod{Y}{y}{atomic type} \\
\gcomment{term expressions}\\
\gprod{E}{f(\rep{E})}{function symbol applied to arguments} \\
\galtprod{x}{term variables} \\
\gcomment{formula expressions}\\
\gprod{F}{p(\rep{E})}{predicate symbol applied to arguments} \\
\galtprod{E\doteq_Y E}{equality of terms at a type} \\
\galtprod{\top \bnfalt \bot \bnfalt \neg F}{truth, falsity, negation} \\
\galtprod{F\wedge F \bnfalt F \vee F}{conjunction, disjunction} \\
\galtprod{F\impl F \bnfalt F\Leftrightarrow F}{implication, equivalence} \\
\galtprod{\forall x:Y.F \bnfalt \exists x:Y.F}{universal, existential quantification} \\
\end{commgrammar}
\end{frame}

\section{Type Systems}

\begin{frame}\frametitle{Overview}
\begin{blockitems}{General Goal}
\item subdivide expressions into groups (called sorts, types, kinds, etc.)
\item written $\vdash_T E:Y$ for expression $E$ of type $Y$ and vocabulary $T$
\end{blockitems}

\begin{blockitems}{Basic Type System}
\item expressions are always subdivided by their non-terminals
\item the types are the non-terminals
\item type of expression immediately clear from expression
\item context only needed to \emph{check} well-formedness of expression, not to \emph{infer} its type
\end{blockitems}

\begin{blockitems}{Refined Type System}
\item for some non-terminals, the expressions are additionally subdivided
\item the types are other expressions
 \glec{$E$ and $Y$ can be from same or different non-terminals}
\item relation between $E$ and $Y$ entirely up to the language
\end{blockitems}
\end{frame}

\begin{frame}\frametitle{Local Variables}
\begin{blockitems}{Type Systems with Variables}
\item $\Gamma\vdash E:Y$ for expression $E$ of type $Y$
\item $\Gamma=x_1:Y_1,\ldots,x_n:Y_n$ declares free variables that may occur in $E$ and $Y$
\item confusingly: $\Gamma$ usually also called \emph{context}
\end{blockitems}

\begin{blockitems}{Choice of $\Gamma$}
\item for BOL: nothing (BOL has no variable binding)
\item for SFOL: variables introduced by $\forall$ and $\exists$
\end{blockitems}
\end{frame}

\begin{frame}\frametitle{Algorithms for Type Systems}
Typing judgment $\Gamma\vdash_T E: Y$
\begin{blockitems}{Type Checking}
\item input: $T$, $\Gamma$, $E$, $Y$
\item output: boolean
\end{blockitems}

\begin{blockitems}{Type Inference}
\item input: $T$, $\Gamma$, $E$
\item output: $Y$
\end{blockitems}

\glec{in practice: also return error messages, e.g., as exceptions}

\begin{blockitems}{Advanced Variants}
\item as above but additionally return $E'$ and $Y'$
\item $E$ and $Y$ have gaps that are filled by the algorithms resulting in $E'$ and $Y'$
\end{blockitems}
\end{frame}

\begin{frame}\frametitle{Implementing a Type-Checker}
\begin{blockitems}{Structure of Syntax}
\item structural level: vocabularies (and morphisms), declarations
\item expressions: some non-terminals are designated as expressions
 \begin{itemize}
 \item usually at least one per declaration kind
 \item usually includes (or can be extended to include) formulas
 \end{itemize}
\end{blockitems}

\begin{blockitems}{Structure of Type-Checker}
\item function called check-$N$ for each non-terminal $N$
 \begin{itemize}
 \item takes context ($T$ and $\Gamma$) and $N$-word
 \item returns Boolean (or error message(s))
 \end{itemize}
\item special case: if $E$-words are typed by $Y$-words, instead
 \begin{itemize}
 \item function check-$E$ takes $T$, $\Gamma$, $E$-word, \emph{and} $Y$-word (expected type)
 \item function infer-$E$ takes $T$, $\Gamma$, $E$-word; returns $Y$-word (inferred type)
 \end{itemize}
\end{blockitems}
\end{frame}

\begin{frame}\frametitle{Type System for SFOL}
We discussed all rules of the SFOL type system.
\end{frame}

\begin{frame}\frametitle{Exercise 6}
Implement the syntax (with printer but not necessarily with parser) and type-checking of SFOL.
\end{frame}

\section{Theory Morphisms}

\begin{frame}\frametitle{Syntax}
\begin{blockitems}{Syntax like for BOL: morphism $m:V\to W$ maps}
\item type symbols to type expressions
 \[y:=Y\]
\item function symbols to function expressions
 \[f:=[x_1,\ldots,x_n]T\]
 \glec{$n$-ary function expression = term with $n$ free variables}
\item predicate symbols to predicate expressions
 \[p:=[x_1,\ldots,x_n]F\]
 \glec{$n$-ary predicate expression = formula with $n$ free variables}
\end{blockitems}
\end{frame}

\begin{frame}[fragile]\frametitle{Example}
$V$:
\begin{lstlisting}[basicstyle={\footnotesize\color{gray}}]
type person
type int
fun age: person $\to$ int
pred sibling $\subseteq$ person person
\end{lstlisting}

$W$:
\begin{lstlisting}[basicstyle={\footnotesize\color{gray}}]
type human
type int
fun minus: int int $\to$ int
fun birthYear: person $\to$ int
fun currentYear: $\to$ int
pred parent $\subseteq$ person person
\end{lstlisting}

One possible morphism $m:V\to W$:
\begin{lstlisting}[basicstyle={\footnotesize\color{gray}}]
person := human
int    := int
age    := [p] minus(currentYear, birthYear(p))
sibling:= [x,y] $\forall$ p:human. parent(p,x) $\equiv$ parent(p,y)
\end{lstlisting}
\end{frame}

\begin{frame}\frametitle{Type-Checking (1)}
\begin{blockitems}{The easy cases --- like for BOL}
\item for $V$-type symbol $y:=Y$ mapped by $y:=Y$ \\
well-typed if
 \[\vdash_W Y: \TYPE\]
i.e., if $Y$ is a type-expression
\item $V$-axiom asserting $F$: mapping of named symbols must be such that $m(F)$ is a theorem
\end{blockitems}
\end{frame}

\begin{frame}\frametitle{Type-Checking (2)}
\begin{blockitems}{Conditions for function/predicate symbols slightly technical}
\item $V$-function symbol $f:Y_1 \ldots Y_n\to Y$ mapped by $f:=[x_1,\ldots,x_n]T$\\
 \[x_1:m(Y_1),\ldots,x_n:m(Y_n)\vdash_W f': m(Y)\]
i.e., well-typed if $T$ is a well-typed term with
 \begin{itemize}
  \item free variables whose types correspond to the arguments of $f$
  \item output corresponding to the output type of $f$
 \end{itemize}
\item $V$-predicate symbol $p\subseteq Y_1\ldots Y_n$ mapped by $p:=[x_1,\ldots,x_n]F$
 \[x_1:m(Y_1),\ldots,x_n:m(Y_n)\vdash_W F: \FORM\]
i.e., well-typed if $F$ is a well-typed formula with free variables whose types correspond to the arguments of $p$
 \glec{like for function symbols but no output type}
\end{blockitems}
\end{frame}


\begin{frame}\frametitle{Homomorphic Extension}
\begin{blockitems}{Given morphism $m:V\to W$}
\item In principle like for BOL: map $V$-expression $E$ to $W$-expression $m(E)$ by replacing every symbol reference with the expression provided by $m$
\item But one subtlety for function/predicate application
\begin{itemize}
 \item if $m$ contains $f:=[x_1,\ldots,x_n]T$:
   \[m(f(t_1,\ldots,t_n))=T[x_1/m(t_1),\ldots,x_n/m(t_n)]\]
  (term $T$ with each variable $x_i$ substituted with $t_i$)
 \item if $m$ contains $p:=[x_1,\ldots,x_n]F$:
   \[m(p(t_1,\ldots,t_n))=F[x_1/m(t_1),\ldots,x_n/m(t_n)]\]
  (formula $F$ with each variable $x_i$ substituted with $t_i$)
\end{itemize}
\end{blockitems}
\end{frame}

\begin{frame}\frametitle{Theory Morphisms Preserve Theorems}
\begin{blockitems}{Given}
\item well-typed theories $V$, $W$
\item well-typed theory morphism $m:V\to W$
\item well-typed $V$-formula $F$ that is a theorem \glec{i.e., consequence of the $V$-axioms}
\end{blockitems}

Then:
\[m(F) \tb\text{is a } W\text{-theorem}\]

\begin{blockitems}{Enables Big Picture Applications}
\item reuse theorems across theories
\item show equivalence of theories
\item find structural connection between seemingly large theories
\item build module system using inheritance and functors
\item build large theories from small systematically
\end{blockitems}
\end{frame}


\section{Semantics}

\begin{frame}\frametitle{Theorems}
\begin{blockitems}{Semantics for an SFOL theory $V$}
\item theorem: a formula that is implied (must be true) by the axioms
\item contradiction: a formula that must be false due to the axioms
\glec{if there is negation: $F$ is contradiction iff $\neg F$ is theorem}
\item $V$ is inconsistent iff all formulas are theorems \\ (equivalently: iff all formulas are contradictions)
\end{blockitems}
\glec{more details on semantics later}

\begin{blockitems}{Kinds of Formulas}
\item ill-formed: cannot be represented in AST, parse error
\item ill-typed: can be represented in AST, but rejected by type-checker
\item well-typed: accepted by type-checker and one of
\begin{itemize}
\item theorem
\item contradiction
\item other
 \glec{usually, most formulas are neither theorem not contradiction}
\end{itemize}
\end{blockitems}
\end{frame}

\begin{frame}\frametitle{Implementing Semantics}
\begin{blockitems}{Automated Theorem Prover (ATP)}
\item input:
 \begin{itemize}
 \item an SFOL-theory $V$
 \item a well-typed $V$-formula $F$ (called the conjecture)
 \end{itemize}
\item output: $F$ is theorem/$F$ is contradiction/timeout
\end{blockitems}

\begin{blockitems}{Theorem status undecidable for most logics}
\item theorem provers try proving/refuting until interrupted
\item typical: proof/refutation in a few seconds/minutes or never
\end{blockitems}

\glec{many ATPs for SFOL: Vampire, E, Spass, \ldots}
\end{frame}

\begin{frame}\frametitle{TPTP: A Standard Concrete Syntax}
\begin{blockitems}{SFOL Syntax Standardization}
\item basically only one choice for abstract syntax of SFOL
\item but lots of options for concrete syntax
\glec{recall: concrete syntax = grammar with all the terminal symbols}
\end{blockitems}

\begin{blockitems}{TPTP Concrete Syntax}
\item designed by Sutcliffe for CASC competition
\glec{annual competition of SFOL ATPs}
\item gradually became de facto standard syntax for SFOL
\item today: every SFOL ATP supports TPTP as input syntax
\item various extensions of TPTP for other logics
\end{blockitems}

\glec{Online interface for SFOL ATPs at \url{http://www.tptp.org}}
\end{frame}

\begin{frame}[fragile]\frametitle{TPTP Syntax: Declarations}
\begin{blockitems}{Theory}
\item text file containing one line per declaration/axiom/conjecture
\end{blockitems}

\begin{blockitems}{Declarations}
\item named: line \lstinline|tff(decl_N, decl, N: TYPE).| where
\begin{itemize}
\item \lstinline|N| is the name
\item \lstinline|TYPE| is
 \begin{itemize}
 \item \lstinline[mathescape=false]|$tType| for a type declaration
 \item \lstinline|Y1*...*Yn>Y| for a function declaration
 \item \lstinline[mathescape=false]|Y1*...*Yn>$o| for a predicate declaration
 \end{itemize}
\end{itemize}
\item axiom: line \lstinline|tff(N, axiom, F).| where
\begin{itemize}
\item \lstinline|N| is some name
\item \lstinline|F| is the formula
\end{itemize}
\end{blockitems}
\end{frame}

\begin{frame}[fragile]\frametitle{TPTP Syntax: Expressions}
\begin{blockitems}{Expressions}
\item \lstinline|![X:TYPE]:F| and \lstinline|?[X:TYPE]:F| for quantifiers
\item \lstinline!F & G!, \lstinline!F | G!, \lstinline!F => G!, \lstinline!~F! for connectives
\item \lstinline|s(T1,...,Tn)| for function/predicate applications
\end{blockitems}

\begin{blockitems}{Identifiers}
\item variables must start with upper-case letter
\item type/function/predicate symbols must start with a lower-case letter
\item remaining characters alphanumeric
\end{blockitems}

\begin{blockitems}{Conjecture (the formula to prove)}
\item like an axiom
\item but with \lstinline|conjecture| instead of \lstinline|axiom|
\end{blockitems}
\end{frame}

%See \url{https://www.tptp.org/} for the TPTP resources, including the grammar and a web interface to theorem provers.
%The basic syntax is
%\begin{lstlisting}
%fof(id, type, name : the type).
%fof(id, axiom, the formula).
%fof(id, conjecture, the formula).
%\end{lstlisting}
%where formulas are formed using \lstinline|![X:TYPE]:F|, \lstinline|?[X:TYPE]:F|, \lstinline|F&F|, \lstinline!F|F!, \lstinline|F=>F|, \lstinline|~F|

\begin{frame}\frametitle{Exercise 7}
Extend your implementation of SFOL as follows:
\begin{itemize}
\item Add a printer for printing SFOL theories and expressions that produces strings in TPTP syntax.
\item Add a function that takes a theory and a conjecture and produces the corresponding TPTP input. Use an ATP to (try to) prove the conjecture.
\item Optionally:
\begin{itemize}
 \item Add theory morphisms to the AST.
 \item Implement the homomorphic extension --- a function that takes a morphism $m:V\to W$ and a $V$-expression $E$ and returns the $W$-expression $m(E)$.
 \glec{actually one function each for terms, types, formulas}
 \item Add a check method for morphisms that calls the homomorphic extension, your type checker, and a theorem prover as needed.
 \item Check an example theory morphism.
 %Run your checker on some example inputs $E$ to test if you are indeed correctly translating well-typed $S$-expressions to well-typed $T$-expressions.
\end{itemize}
\end{itemize}
\end{frame}
