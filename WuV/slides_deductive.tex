\section{Typed First-Order Logic}

\begin{frame}\frametitle{Syntax}
We give an overview of first-order logic as the main example of the deductive corner of the Tetrapod.
The grammar is given in Section 6.3 of the notes.
\end{frame}

\begin{frame}\frametitle{Exercise 6}
Implement the syntax (with printer but not necessarily with parser) and type-checking of SFOL.
\end{frame}


\section{Type Systems}

\begin{frame}\frametitle{Overview}
\begin{blockitems}{General Goal}
\item subdivide expressions into groups (called sorts, types, kinds, etc.)
\item written $\vdash_V E:T$ for expression $E$ of type $T$ and vocabulary $V$
\end{blockitems}

\begin{blockitems}{Basic Type System}
\item expressions are always subdivided by their non-terminals
\item the types are the non-terminals
\item type of expression immediately clear from expression
\item context only needed to \emph{check} well-formedness of expression, not to \emph{infer} its type
\end{blockitems}

\begin{blockitems}{Refined Type System}
\item for some non-terminals, the expressions are additionally subdivided
\item the types are other expressions
 \glec{$E$ and $T$ can be from same or different non-terminals}
\item relation between $E$ and $T$ entirely up to the language
\end{blockitems}
\end{frame}

\begin{frame}\frametitle{Local Variables}
\begin{blockitems}{Type Systems with Variables}
\item $\Gamma\vdash E:T$ for expression $E$ of type $T$
\item $\Gamma=x_1:Y_1,\ldots,x_n:Y_n$ declares free variables that may occur in $E$ and $T$
\item confusingly: $\Gamma$ usually also called \emph{context}
\end{blockitems}

\begin{blockitems}{Choice of $\Gamma$}
\item for BOL: nothing (BOL has no variable binding)
\item for SFOL: variables introduced by $\forall$ and $\exists$
\end{blockitems}
\end{frame}

\begin{frame}\frametitle{Algorithms for Type Systems}
Judgment $\Gamma\vdash_V E: T$
\begin{blockitems}{Type Checking}
\item input: $V$, $\Gamma$, $E$, $T$
\item output: boolean
\end{blockitems}

\begin{blockitems}{Type Inference}
\item input: $\Gamma$, $V$, $E$
\item output: $T$
\end{blockitems}

\glec{in practice: also return error messages, e.g., as exceptions}

\begin{blockitems}{Advanced Variants}
\item as above but additionally return $E'$ and $T'$
\item $E$ and $T$ have gaps that are filled by the algorithms resulting in $E'$ and $T'$
\end{blockitems}
\end{frame}

\begin{frame}\frametitle{Implementing a Type-Checker}
\begin{blockitems}{Structure of Syntax}
\item structural level: vocabularies (and morphisms), declarations
\item expressions: some non-terminals are designated as expressions
 \begin{itemize}
 \item usually at least one per declaration kind
 \item usually includes (or can be extended to include) formulas
 \end{itemize}
\end{blockitems}

\begin{blockitems}{Structure of Type-Checker}
\item function check-$N$ for each non-terminal $N$
 \begin{itemize}
 \item takes context ($V$ and $\Gamma$) and $N$-word
 \item returns error message(s)
 \end{itemize}
\item whenever $N$-words are typed by $Y$-words, instead
 \begin{itemize}
 \item function check-$N$ takes $V$, $\Gamma$, $N$-word, \emph{and} $Y$-word (expected type)
 \item function infer-$N$ takes $V$, $\Gamma$, $N$-word; returns $Y$-word (inferred type)
 \end{itemize}
\end{blockitems}
\end{frame}


\begin{frame}\frametitle{Implementing Semantics}
\begin{blockitems}{Automated Theorem Prover (ATP)}
\item input: an SFOL-theory $T$ and a well-typed formula $F$ (called the conjecture)
\item output: $F$ is theorem/$F$ is contradiction/timeout
\end{blockitems}

\begin{blockitems}{Theorem status undecidable for most logics}
\item theorem provers try proving/refuting until interrupted
\item typical: proof/refutation in a few seconds/minutes or never
\end{blockitems}

\glec{many ATPs for SFOL: Vampire, E, Spass, \ldots}
\glec{ATPs compete in the annual CASC competition}
\end{frame}

\begin{frame}\frametitle{TPTP: A Standard Concrete Syntax}
\glec{recall: concrete syntax = grammar with all the terminal symbols}
\begin{blockitems}{TPTP}
\item designed by Sutcliffe for CASC competition
\item gradually became de facto standard
\item today: every SFOL ATP supports TPTP as input syntax
\item various extensions of TPTP for other logics
\end{blockitems}

\glec{Online interface for SFOL ATPs at \url{http://www.tptp.org}}
\end{frame}

\begin{frame}\frametitle{TPTP Syntax}
\glec{recall: concrete syntax = grammar with all the terminal symbols}
\begin{blockitems}{TPTP}
\item designed by Sutcliffe for CASC competition
\item gradually became de facto standard
\item today: every SFOL ATP supports TPTP as input syntax
\item various extensions of TPTP for other logics
\end{blockitems}

\glec{Online interface for SFOL ATPs at \url{http://www.tptp.org}}
\end{frame}

\begin{frame}[fragile]\frametitle{TPTP Syntax: Declarations}
\begin{blockitems}{Theory}
\item text file containing one line per declaration/axiom/conjecture
\end{blockitems}

\begin{blockitems}{Declarations}
\item named: line \lstinline|tff(decl_N, decl, N: TYPE).| where
\begin{itemize}
\item \lstinline|N| is the name
\item \lstinline|TYPE| is
 \begin{itemize}
 \item \lstinline[mathescape=false]|$tType| for a type declaration
 \item \lstinline|Y1*...*Yn>Y| for a function declaration
 \item \lstinline[mathescape=false]|Y1*...*Yn>$o| for a predicate declaration
 \end{itemize}
\end{itemize}
\item axiom: line \lstinline|tff(N, axiom, F).| where
\begin{itemize}
\item \lstinline|N| is some name
\item \lstinline|F| is the formula
\end{itemize}
\end{blockitems}
\end{frame}

\begin{frame}[fragile]\frametitle{TPTP Syntax: Expressions}
\begin{blockitems}{Expressions}
\item \lstinline|![X:TYPE]:F| and \lstinline|?[X:TYPE]:F| for quantifiers
\item \lstinline!F & G!, \lstinline!F | G!, \lstinline!F => G!, \lstinline!~F! for connectives
\item \lstinline|s(T1,...,Tn)| for function/predicate symbol applications
\end{blockitems}

\begin{blockitems}{Identifiers}
\item variables must start with upper-case letter
\item type/function/predicate symbol names must start with a lower-case letter
\item remaining characters alphanumeric
\end{blockitems}

\begin{blockitems}{Conjecture (the formula to prove)}
\item like an axiom
\item but with \lstinline|conjecture| instead of \lstinline|axiom|
\end{blockitems}
\end{frame}

\begin{frame}\frametitle{Exercise 7}
Extend your implementation of SFOL as follows:
\begin{itemize}
\item Add theory morphisms to the AST and add a check method for them
\item Implement a function that takes a morphism $m:S\to T$ and an $S$-expression $E$ (one function each for terms, types, formulas) and returns the homomorphic translation $m(E)$.
Run your checker on some example inputs $E$ to test if you are indeed correctly translating well-typed $S$-expressions to well-typed $T$-expressions.
\item Add a printer for printing SFOL theories and expressions in TPTP syntax.
\item Add a function that takes a theory and a conjecture and produces the corresponding TPTP input. Use an ATP to (try to) prove the conjecture.
\end{itemize}
\end{frame}
