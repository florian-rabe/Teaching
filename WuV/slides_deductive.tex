\section{Typed First-Order Logic}

\begin{frame}\frametitle{Syntax}
We give an overview of first-order logic as the main example of the deductive corner of the Tetrapod.
The grammar is given in Section 6.3 of the notes.
\end{frame}

\begin{frame}\frametitle{Exercise 6}
Implement the syntax (with printer but not necessarily with parser) and type-checking of SFOL.
\end{frame}


\section{Type Systems}

\begin{frame}\frametitle{Overview}
\begin{blockitems}{General Goal}
\item subdivide expressions into groups (called sorts, types, kinds, etc.)
\item written $\vdash_V E:T$ for expression $E$ of type $T$ and vocabulary $V$
\end{blockitems}

\begin{blockitems}{Basic Type System}
\item expressions are always subdivided by their non-terminals
\item the types are the non-terminals
\item type of expression immediately clear from expression
\item context only needed to \emph{check} well-formedness of expression, not to \emph{infer} its type
\end{blockitems}

\begin{blockitems}{Refined Type System}
\item for some non-terminals, the expressions are additionally subdivided
\item the types are other expressions
 \glec{$E$ and $T$ can be from same or different non-terminals}
\item relation between $E$ and $T$ entirely up to the language
\end{blockitems}
\end{frame}

\begin{frame}\frametitle{Local Variables}
\begin{blockitems}{Type Systems with Variables}
\item $\Gamma\vdash E:T$ for expression $E$ of type $T$
\item $\Gamma=x_1:Y_1,\ldots,x_n:Y_n$ declares free variables that may occur in $E$ and $T$
\item confusingly: $\Gamma$ usually also called \emph{context}
\end{blockitems}

\begin{blockitems}{Choice of $\Gamma$}
\item for BOL: nothing (BOL has no variable binding)
\item for SFOL: variables introduced by $\forall$ and $\exists$
\end{blockitems}
\end{frame}

\begin{frame}\frametitle{Algorithms for Type Systems}
Judgment $\Gamma\vdash_V E: T$
\begin{blockitems}{Type Checking}
\item input: $V$, $\Gamma$, $E$, $T$
\item output: boolean
\end{blockitems}

\begin{blockitems}{Type Inference}
\item input: $\Gamma$, $V$, $E$
\item output: $T$
\end{blockitems}

\glec{in practice: also return error messages, e.g., as exceptions}

\begin{blockitems}{Advanced Variants}
\item as above but additionally return $E'$ and $T'$
\item $E$ and $T$ have gaps that are filled by the algorithms resulting in $E'$ and $T'$
\end{blockitems}
\end{frame}


\begin{frame}\frametitle{Implementing a Type-Checker}
\begin{blockitems}{Structure of Syntax}
\item structural level: vocabularies (and morphisms), declarations
\item expressions: some non-terminals are designated as expressions
 \begin{itemize}
 \item usually at least one per declaration kind
 \item usually includes (or can be extended to include) formulas
 \end{itemize}
\end{blockitems}

\begin{blockitems}{Structure of Type-Checker}
\item function check-$N$ for each non-terminal $N$
 \begin{itemize}
 \item takes context ($V$ and $\Gamma$) and $N$-word
 \item returns error message(s)
 \end{itemize}
\item whenever $N$-words are typed by $Y$-words, instead
 \begin{itemize}
 \item function check-$N$ takes $V$, $\Gamma$, $N$-word, \emph{and} $Y$-word (expected type)
 \item function infer-$N$ takes $V$, $\Gamma$, $N$-word; returns $Y$-word (inferred type)
 \end{itemize}
\end{blockitems}
\end{frame}
