\documentclass{beamer}

\usepackage{url}
\usepackage{multicol}
\usepackage{amsmath}
\usepackage{amssymb}

\usepackage{../macros/basics}
\usepackage{../macros/basics-slides}

\begin{document}

\title{How to give a talk?}
\author{Florian Rabe}
\institute{University Erlangen-Nuremberg}
\date{2019}
\begin{frame}
    \titlepage
    
The following presents several versions of the same slide from one of my talks.
Think about how the formating (but not the content) change between slides in a way that improves the slide.
\end{frame}

%\section{General}
%
%\begin{frame}\frametitle{Advice for Students}
%\begin{itemize}
%  \item Loosely edited list of advice
%  \url{https://www.authorea.com/43655-Advice-for-Students}
%  \item Apparently very helpful
%  \item Neither sound nor complete
%  \item Includes section on talks
%  \item Any feedback/additions welcome
%\end{itemize}
%\end{frame}

\section{Example: Final Preparation of a Slide}

\begin{frame}\frametitle{MathML vs. TPTP: Logical Similarities}
If we abstract from
\begin{itemize}
  \item concrete syntax
  \item intended purpose
  \item user community
  \item tool support
\end{itemize}
the languages are quite similar:

MathML
\begin{itemize}
 \item MathML objects: constants, variables, application, arbitrary binding
 \item all operators introduced/specified in content dictionaries
\end{itemize}
TPTP (since expansion towards higher-order logic)
\begin{itemize}
 \item TPTP formulas: constants, variables, application, built-in binders $\forall\exists\lambda\Pi\Sigma$
 \item logic-related operators built-in, specified in various language references
  \lec{no fixed type systems, no fixed calculus}
 \item other operators introduced/specified in TPTP files
\end{itemize}
\end{frame}

\begin{frame}\frametitle{MathML vs. TPTP: Logical Similarities}
If we abstract from
\begin{itemize}
  \item concrete syntax
  \item intended purpose
  \item user community
  \item tool support
\end{itemize}
the languages are quite similar:

\begin{blockitems}{MathML}
 \item MathML objects: constants, variables, application, arbitrary binding
 \item all operators introduced/specified in content dictionaries
\end{blockitems}
\begin{blockitems}{TPTP (since expansion towards higher-order logic)}
 \item TPTP formulas: constants, variables, application, built-in binders $\forall\exists\lambda\Pi\Sigma$
 \item logic-related operators built-in, specified in various language references
  \lec{no fixed type systems, no fixed calculus}
 \item other operators introduced/specified in TPTP files
\end{blockitems}
\end{frame}

\begin{frame}\frametitle{MathML vs. TPTP: Logical Similarities}
If we abstract from
\begin{itemize}
  \item concrete syntax
  \item intended purpose
  \item user community
  \item tool support
\end{itemize}
the languages are quite similar:

\begin{blockitems}{MathML}
 \item constants, variables, application, arbitrary binding
 \item all operators introduced/specified in content dictionaries
\end{blockitems}
\begin{blockitems}{TPTP (since expansion towards higher-order logic)}
 \item constants, variables, application, built-in binders $\forall\exists\lambda\Pi\Sigma$
 \item most operators introduced/specified in TPTP files
 \item built-in logic-related operators, specified in various language references
  \lec{no fixed type systems, no fixed calculus}
\end{blockitems}
\end{frame}

\begin{frame}\frametitle{MathML vs. TPTP: Logical Similarities}
If we abstract from
\begin{itemize}
  \item concrete syntax
  \item intended purpose
  \item user community
  \item tool support
\end{itemize}
the languages are quite similar:

\begin{blockitems}{MathML}
 \item constants, variables, application, \hlA{arbitrary binding}
 \item \hlA{all} operators introduced/specified in content dictionaries
\end{blockitems}
\begin{blockitems}{TPTP (since expansion towards higher-order logic)}
 \item constant, variables, application, \hlA{built-in binders $\forall\exists\lambda\Pi\Sigma$}
 \item \hlA{most} operators introduced/specified in TPTP files
 \item \hlA{built-in logic-related operators}, specified in various language references
  \lec{no fixed type systems, no fixed calculus}
\end{blockitems}
\end{frame}

\begin{frame}\frametitle{MathML vs. TPTP: Logical Similarities}
If we abstract from
\begin{itemize}
  \item concrete syntax
  \item intended purpose
  \item user community
  \item tool support
\end{itemize}
the languages are quite similar:

\begin{blockitems}{MathML}
 \item constants, variables, application, \hlA{arbitrary binding}
 \item \hlA{all} operators introduced/specified in \hlA{content dictionaries}
\end{blockitems}
\begin{blockitems}{TPTP (since expansion towards higher-order logic)}
 \item constant, variables, application, \hlA{built-in binders $\forall\exists\lambda\Pi\Sigma$}
 \item \hlA{most} operators introduced/specified in \hlA{TPTP files}
 \item \hlA{built-in logic-related operators}
   \begin{itemize}
     \item semantics left open
       \lec{no fixed type systems, no fixed calculus}
     \item specified in various language references
       \lec{fof, tff, thf, thf1, \ldots}
    \end{itemize}
\end{blockitems}
\end{frame}

\begin{frame}\frametitle{MathML vs. TPTP: Logical Similarities}
Languages are quite similar if we abstract from
\vspace{-1em}
\begin{multicols}{2}
\begin{itemize}
  \item concrete syntax
  \item intended purpose
\end{itemize}
\begin{itemize}
  \item user community
  \item tool support
\end{itemize}
\end{multicols}

\begin{blockitems}{MathML}
 \item constants, variables, application, \hlA{arbitrary binding}
 \item \hlA{all} operators introduced/specified in \hlA{content dictionaries}
\end{blockitems}
\begin{blockitems}{TPTP (since expansion towards higher-order logic)}
 \item constant, variables, application, \hlA{built-in binders $\forall\exists\lambda\Pi\Sigma$}
 \item \hlA{most} operators introduced/specified in \hlA{TPTP files}
 \item \hlA{built-in logic-related operators}
   \begin{itemize}
     \item semantics left open
       \lec{no fixed type systems, no fixed calculus}
     \item various dialects
       \lec{fof, tff, thf, thf1, \ldots}
   \end{itemize}
\end{blockitems}
\end{frame}
\end{document}