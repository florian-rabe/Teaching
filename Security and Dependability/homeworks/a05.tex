\documentclass[a4paper]{article}

\usepackage[course={Secure and Dependable Systems},number=5,date=2017-03-28,duedate=2017-04-06,solutions]{../../myhomeworks}

\newcounter{chapter} % needed for dependencies of mylecturenotes
\usepackage[root=../..]{../../mylecturenotes}
\usepackage{../../macros/algorithm}

\begin{document}

\header

\begin{problem}{Verification: Class Invariants}{2}
Argue informally but rigorously why the formula in the stack example from the notes is a strong\footnote{I had forgotten to put the word \emph{strong} here. So that part was not required for the solution.} class invariant.

\begin{solution}
For a fresh instance, the class invariant holds because both sides of the equation are $0$.

Later, every assignment to the variable $elements$ is immediately followed by an assignment to the variable $size$.
In both cases, $size$ changes by the same amount as the length of $elements$.
Thus, the class invariant is preserved.
\end{solution}
\end{problem}

\begin{problem}{Verification: Class Invariants}{3}
Solve one of the following (using pseudo-code, our example formal system, or a programming language):

\begin{enumerate}
\item Give a useful weak class invariant for the following class:

\begin{acode}
\aclass{Date}{year:\Int, month:\Int, day:\Int}{}{
 \afunI[Date]{yesterday}{}{\ldots}\\
 \afunI[Date]{tomorrow}{}{\ldots}
}
\end{acode}
You can assume that there are no leap years.

Implement the methods such that the class invariant is preserved.

\item Give a useful strong class invariant for the following class:
\begin{acode}
\aclassA{PriorityQueue}{}{}{
 \akey{private} \amval{data}{List[\Int]}{Nil}\\
 \afunI[{Option[\Int]}]{dequeue}{}{\aifelseI{data==Nil}{None}{Some(data.head)}}\\
 \afunI[\Unit]{enqueue}{x:\Int}{\ldots}
}
\end{acode}

Implement $insert$ such that the class invariant is preserved.
\end{enumerate}

\begin{solution}
\begin{compactenum}
 \item The class invariant should be (or imply) $year!=0 \wedge 1\leq month\leq 12 \wedge 1\leq day\leq numDays(month)$ for a function $numDays(d:\Int):\Int$ that returns the number of days in the respective month.
 
 The implementation part is straightforward.
 \item Assuming that the priority queue should always return the greatest value, the class invariant should be (or imply) $\amatchI{data}{\acase{Nil}{\true},\acase{Cons(hd,tl)}{\afa{i}{0\leq i< length(tl)\impl hd\geq get(tl,i)}}}$, where $get$ retrieves an element from a list.
 The solution that always returns the least value is also correct.

 The implementation part is straightforward.
\end{compactenum}
Other class invariants were possible. Their usefulness is judged based on whether invalid instances are eliminated. Depending on the usefulness, partial credit was given.

One subproblem had to be solved. If both were solved, the one with the best result was used for grading.
\end{solution}
\end{problem}

\begin{problem}{Verification: Pure Functions}{3}
Using the definitions from the notes, formally prove $m+n==n+m$ by induction.

To discharge the subgoals, you may use the theorems $zero\_left$ etc.

\begin{solution}
We apply the induction rule to the proof goal $m+n==n+m$ to the variable $n$, which yields two subgoals:
\[
\rul{\iscons{}{m:nat}{}{m+zero==zero+m} \tb \iscons{}{m:nat, n:nat, p: \aproof{m+n==n+m}}{}{m+succ(n)==succ(n)+m}}
    {\iscons{}{m:nat,n:nat}{}{m+n==n+m}}
\]

The left subgoal can be discharged by using $zero\_right$, $zero\_left$, and reflexivity of $==$ to evaluate
\[m+zero==zero+m\tb\rewrites \tb m==m \tb\rewrites\tb \true.\]

The right subgoal can be discharged by using $succ\_right$, $succ\_left$, $p$, and reflexivity of $==$ to evaluate
 \[m+succ(n)==succ(n)+m\tb\rewrites \tb succ(m+n)==succ(n+m)\tb \rewrites\]
 \[\tb succ(m+n)==succ(m+n)\tb\rewrites\tb \true.\]
\end{solution}
\end{problem}

\begin{problem}{Proof Assistants: Practice}{4}
Install Isabelle or Coq (see the links in the lecture notes).

Write a simple pure recursive function and verify that it meets its specification using the tool.
Generate executable code from your function.
\medskip

Submit a reasonable combination of screen shots, shell logs, system output etc. that demonstrates you completed the task.

You may use any example that is already part of the available documentation or tutorials.
But you have to prove that you actually installed the system and ran the verification.
For example, you can copy an example from the tutorial, rename the function to your name, and then run the verification.
\end{problem}

\end{document}
