No system is perfectly secure---a brute-force attack is always possible.
Therefore, a central idea of cryptography is that waiting for a polynomial amount of time (in the sense of complexity theory) is feasible but waiting for a super-polynomial amount of time is not.
Then a system is often defined as secure if the best possible attack has super-polynomial complexity.

\begin{definition}[Polynomial and Negligible Functions]
 A function $f:\N\to\R^+$ is called
  \begin{compactitem}
   \item \textbf{polynomial} if $f\in O(p)$ for some polynomial $p$.
   \item \textbf{super-polynomial} if $f\nin O(p)$ for every polynomial $p$ (i.e., if it is not polynomial).
   \item \textbf{negligible} if $f\in O(1/|p(x)|)$ for every polynomial $p:\N\to\R^+$.
  \end{compactitem}
\end{definition}

Super-polynomial functions increase faster than any polynomial increases.
Negligible functions are somewhat dual: they decrease faster than any polynomial increases.
In cryptography, we can think of polynomial/super-polynomial/negligible functions as normal/infeasible/trivial.

\begin{theorem}
We have the following closure properties:
\begin{compactitem}
 \item The sum of super-polynomial/polynomial/negligible functions is super-polynomial/polynomial/negligible again.
 \item The product of a super-polynomial/negligible function with a polynomial function is super-polynomial/negligible again.
 \item A function that is greater/smaller than a super-polynomial/negligible function is super-polynomial/negligible again.
\end{compactitem}
\end{theorem}

\begin{definition}[Polynomial Time]
 An algorithm $A$ is called \textbf{polynomial time} if its worst-case time complexity of $A$ for input of size $n$ is a polynomial function.
\end{definition}

\begin{definition}[Probabilistic Algorithm]
 A \textbf{probabilistic algorithm} is like an algorithm except that it may return different results when called multiple times for the same input.
\end{definition}

\begin{example}[Fermat Primality Test]
Many prime number tests are derived from Fermat's Little Theorem \ref{thm:math:fermatlittle}.
A states that a necessary but not sufficient property for $p$ being prime is that $x^p\Equiv_p x$ for all $a\in \Z$.

To use this as a primality test, we randomly choose $3$ numbers $1<x<p-1$ and test the relation.
If it holds for all three $x$, we return true.

$x^p\Equiv_p x$ is unlikely to hold for random $x$.
So if it succeeds three times, $p$ is probably but not definitely a prime number.

This algorithm is probabilistic in the following sense:
\begin{compactitem}
 \item It needs randomness to work and may return different values for the same input.
 \item Its result is not necessarily correct.
\end{compactitem}

The reason why we choose $1<x<p-1$ is that
\begin{compactitem}
 \item Values outside of $\Z_p$ are irrelevant because we take them modulo $p$ anyway.
 \item The values $x=0$ and $x=1$ are useless because the property anyway holds for them.
 \item The value $x=p-1$ (which is congruent to $-1$) is useless because the property anyway holds if $p$ is odd. (And if $p$ is not odd, we already know it is not a prime.)
\end{compactitem}
\end{example}

In most situations, an algorithm that may return different results for the same input is simply broken and not an algorithm at all.
However, occasionally, probabilistic algorithms are very useful.
Cryptography is an example because we often \emph{need} to make a random choice to prevent an attacker from predicting what we did.
For example, it is often unavoidable for the attacker to know what encryption scheme we use---but that is acceptable if we choose a random key.
Therefore, encryption schemes often consist of a probabilistic key generation algorithm and deterministic encryption/decryption algorithms that take the key as input.
Another application in cryptography is \emph{padding} a message: to prevent leaking the information how long a message is, we can pad every message to a fixed length by adding random data.
In that case, the encryption algorithm is also probabilistic.

Independent of applications in cryptography, probabilistic algorithms are also of great help in situations where (i) a deterministic algorithm has a large complexity but (ii) a probabilistic algorithm that sometimes returns false results is much faster.
If it is easy to test whether a result is correct, we can simply re-run the probabilistic algorithm until it finds a correct result.
That is sometimes used in key generation algorithms, e.g., to find a large prime number.
%Alternatively, if we cannot test if a result is correct, we may be able to run the probabilistic algorithm multiple times in order to increase the probability that a result is correct.

\begin{definition}[Probabilistic Polynomial Time]
 An algorithm is called \textbf{probabilistic polynomial time} (PPT) if it is probabilistic and polynomial.
\end{definition}

One-way-functions (OWF) are functions that are easy to compute but hard to invert:

\newcommand{\Prob}[2]{\mathop{\mathrm{Pr}}_{#1}[#2]}

\begin{notation}
For a function $f:X\to\Bool$, we write $\Prob{x\in X}{f(x)}$ for the probability that $f(x)$ is true when $x$ is chosen uniformly from $X$.
\end{notation}

\begin{definition}[One-Way-Function]
 A polynomial function $f:\Sigma^*\to \Sigma^*$, is a \textbf{one-way-function} if for every probabilistic algorithm $A(n\in\N,y\in \Sigma^*)\to\Sigma^*$, whose complexity is polynomial in $n$, the function
 \[n\tb\mapsto\tb \Prob{x\in\Sigma^n}{f(A(n,f(x)))=f(x)}\]
 is negligible.
\end{definition}
Intuitively, OWFs are super-polynomially hard to invert.
More precisely, any $A$ that attempts to guess (i.e., it is probabilistic) an $x\in\Sigma^n$ such that $y$ and $A(n,y)$ behave the same way under $f$ succeeds with negligible probability.

\begin{example}
It is not known if one-way functions exist.

Some functions that are commonly believed to be OWFs are discrete exponentiation and multiplication.
For example, we do not know if there is a polynomial factoring algorithm.
\end{example}

Pseudo-random-generators (PRG) are functions that can be used to generate numbers that are essentially random:
\begin{definition}[Pseudo-Randomness]
 A polynomial function $R:\Sigma^*\to\Sigma^*$ is a \textbf{pseudo-randomness-generator}
 \begin{compactitem}
   \item $R$ maps input of the same length to output of the same length, i.e., there is an $l:\N\to\N$ such that $R$ maps $\Sigma^n\to\Sigma^{l(n)}$ for every $n$.
   \item for every PPT $A(x\in\Sigma^*)\in\Bool$ the function
  \[n \tb\mapsto \tb \left|\,\Prob{x\in\Sigma^n}{A(R(x))}-\Prob{x\in\Sigma^{l(n)}}{A(x)}\,\right|\]
  is negligible.
 \end{compactitem}
\end{definition}

Intuitively, $R$ is a PRG if it maps $\Sigma^*\to\Sigma^*$ such that no PPT $A$ can tell the difference between $R(x)$ and randomly chosen $x$ with non-negligible probability.
The condition about the length of the output of $R$ is only needed so that we can take the probability on the right-hand side over a finite set.

PRGs and OWFs are intricately linked together:
\begin{theorem}[PRG Theorem]
Given a PRG, we can define an OWF, and vice versa.
\end{theorem}

%%This part should probably be moved to the appendix
%\begin{definition}[hard-core bit]
% Let $f$ be a one-way function. Now the function $b:\{0,1\}^*\to\{0,1\}$ is called a \emph{hard-core bit} for $f$ if $b$ is computable in polynomial time and there exists a negligible function $neg(n)$ such that for any $n\in\N$ and any PPT $A$:
% $$\mathrm{Pr}\left[A(f(x),1)=b(x)\right]\leq\frac{1}{2}+neg(n), $$ where $x\in\{0,1\}$ uniformly random. 
%\end{definition}
%
%The next step is to find one-way functions (assuming that there are one-way functions at all) for which we can build hard-core bits. 
%\begin{theorem}
% Let $f$ be a one-way function. Define $g:\{0,1\}^{2n}\to\{0,1\}^*$ as $g(x\circ y)=f(x)\circ y$, where $\left|x\right|=\left|y\right|=n$. Then $g$ is a one-way function with hard-core bit $$b(x,y)=\bigoplus_{i=1}^n x_i\land y_i=\sum_{i=1}^{n}x_iy_i (\mod 2).$$
%\end{theorem}
%If now $f$ was a one-way permutation, we can use the above construction to get an additional pseudo-random bit, so we already have a pseudo-random generator.
