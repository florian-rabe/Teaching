\subsection{Using Asymmetric Encryption}

An important application of asymmetric encryption is authentication.

For example, an agent $A$ who wants to prove her identity can generate a key and publish the public key.
To authenticate $A$, we generate a random string, encrypt it with the public key, and ask for its decryption.
Because only $A$ can decrypt, this is sufficient to authenticate.

Another authentication scheme uses a digital signature.
Here the private key is used for encryption and the public one for decryption. (Note that for RSA the distinction between encryption and decryption key is insubstantial anyway.)
$A$ sends a message together with its encryption, and the recipient decrypts and compares.
A better scheme arises if $A$ first applies a cryptographic hash to the message and sends the original message and the encrypted hash value.

\subsection{Multi-Factor Authentication}

The idea of multi-factor authentication is to use the conjunction of multiple authentication tests.
This is particularly useful if two factors are used that depend on fundamentally different systems: in that case authentication is still safe even if one of the systems is compromised.

The most common application is to authenticate a user via password first, then ask for a second one-time password.
Usually the one-time password is sent to the user as a text message or (in older systems) as a list of one-time passwords on paper by mail.
Even if the user's password is compromised, the authentication remains secure (and can be used to reset the original password).
However, to be truly secure, the user must never type his password on the phone.