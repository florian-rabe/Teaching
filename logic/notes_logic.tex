\documentclass{book}
\usepackage[top=2cm,bottom=2cm,left=2cm,right=2cm]{geometry}
\usepackage{graphicx}
\usepackage{amsmath,amssymb,amsthm}
\usepackage{xkeyval,paralist}
\usepackage{bm} %% bold face math symbols
\usepackage{listings}
\usepackage{tikz}
%\usetikzlibrary{shapes}
\usetikzlibrary{arrows}

\usepackage{stmaryrd}\newcommand{\contra}{\lightning}

\usepackage{rotating} \newcommand{\sw}[1]{\begin{sideways}#1\end{sideways}}

\usepackage{ded}

\usepackage{../mylecturenotes}
\usepackage{../macros}

\renewcommand{\P}[2]{\Pi #1:#2.}

\usepackage[bookmarks=true, linkcolor=blue, citecolor=blue, urlcolor=blue, colorlinks=true, breaklinks=true, bookmarksopen=true,bookmarksopenlevel=1,bookmarksnumbered]{hyperref} % this needs to go last

\title{Integrated Lectures Notes on Logic}
\author{Florian Rabe}
\date{2008-2022}

\begin{document}
\maketitle

\tableofcontents
\newpage

This is a collection of notes from my undergraduate and graduate lectures on logic.

Standard textbooks for logic are \cite{fol_gallier}, \cite{fol_smullyan}, and \cite{intrologicenderton} (look for the respective latest edition).
\cite{andrews_truthproof} is an interesting textbook that emphasizes higher-order logic and proof theory.

\part{First-Order Logic}\label{part:fol}

\chapter{Syntax}\label{sec:syn}
  \input{fol-syntax}
  
\chapter{Context-Sensitive Constraints}\label{sec:infsys}
  \section{Inference Systems}
     \input{inference_systems}
  \section{Inference Systems for Well-Formed Syntax}
     \input{fol_syntax_inference_systems}

\chapter{Model-Theoretic Semantics}\label{sec:mt}
  \input{fol-models}

\chapter{Proof-Theoretic Semantics}\label{sec:pt}
  \input{fol-proofs}

\chapter{The Relation between Proof and Model Theory}
  \input{fol-proofs_models}

\chapter{Computability}
  \input{compute}

\chapter{Closures and Galois Connections}
  \input{closure}

\chapter{Induction}
  \input{induction}

\chapter{Morphisms}
  \input{fol-morphisms}

\chapter{Universal Model Theory}\label{sec:unimod}
\footnote{This chapter can be skipped.}
  \input{model-categories}

\part{Higher-Order Logic}\label{part:hol}

\chapter{Syntax}
  \input{hol-syntax}

\chapter{Model-Theoretic Semantics}
  \input{hol-models}
  
\chapter{Proof-Theoretic Semantics}
  \input{hol-proofs}

\chapter{The Relation between Proof and Model Theory}
  \input{hol-proofs_models}

\part{Foundations of Mathematics}\label{sec:found}

\chapter{General Ideas}
  \input{found-general}

\chapter{First-Order Set Theory}
  \input{found-set}
  
\chapter{Higher-Order Logic}
  \input{found-hol}

\chapter{Implemented Foundations}
  \input{found-formal}

\part{Logic Translations}\label{sec:trans}

\chapter{Translating First-Order to Higher-Order Logic}\label{sec:folhol}
  \input{translations-folhol}

\chapter{An Abstract Definition of a Logic Translation}\label{sec:trans:abs}
  \input{translations-abs}
  
\chapter{Model Theories as Logic Translations}
  \input{translations-modasmorph}

\part{A Proof-Theoretical Logical Framework}\label{part:lf}

\chapter{LF for Everybody}\label{sec:lffe}
 \input{lf-for-everybody}

\chapter{Syntax of LF}
 \input{lf-syntax}

\chapter{Semantics of LF}
 \input{lf-semantics}

\chapter{Representing Formal Languages in LF}
  \input{lf-logics}
  
\chapter{The LF Implementation Twelf}\label{sec:twelf}
  \input{lf-twelf}

%\part{Summary}
%
%\chapter{Abstract Properties of Type Theories and Logics}
%  \input{summary-abs}

%\chapter{Logic Translations}
%\begin{center}
%\begin{tabular}{|c|c|c|c|c|}
%\hline
%       &             & \multicolumn{3}{|c|}{Behavior under} \\
%Symbol & Declared in & Logic translation & Morphism & Substitution \\ \hline
%logical symbol & logic $L$ & translated & fixed & fixed \\
%function, predicate symbol & $L$-signature $\Sigma$ & part of translated signature & translated & fixed \\
%variable & $\Sigma$-context $\Gamma$ & part of translated context &  part of translated context      & translated \\
%\hline
%\end{tabular}
%\end{center}

\appendix

\chapter{Mathematical Preliminaries}

\input{../math/all}

\bibliographystyle{alpha}
\bibliography{../../../../../frabe/Program_Data/Latex/bib/rabe,../../../../../frabe/Program_Data/Latex/bib/pub_rabe,../../../../../frabe/Program_Data/Latex/bib/systems,../../../../../frabe/Program_Data/Latex/bib/institutions,../../../../../frabe/Program_Data/Latex/bib/historical}

\end{document}