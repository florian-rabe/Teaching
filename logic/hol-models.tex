\section{The Naive Semantics}

The semantics of HOL is defined similarly to the semantics of FOL: We fix interpretations of the logical symbols, use models to interpret the constants, and use assignments to interpret the variables.

However, because the definitions of types and signatures are mutually recursive, we cannot define models, assignments, and the interpretation function $\semm{-}{I,\alpha}$ separately anymore. Instead, we use a mutual induction.

We start with an overview. For a $\Sigma$-model $I$ and an assignment $\alpha$ for a $\Sigma$-context $\Gamma$, the relation between syntax and semantics is as follows:
\begin{center}
\begin{tabular}{|c|c|c|}
\hline
Expression          & Syntax                           & Semantics \\ \hline
type $A$            & $\istype{\Sigma}{A}$             & $\semm{A}{I}\in \Set$ \\
term $t$            & $\oftype{\Sigma}{\Gamma}{t}{A}$  & $\semcm{\Gamma}{t}{I,\alpha}\in \semm{A}{I}$ \\
formula $F$         & $\oftype{\Sigma}{\Gamma}{F}{o}$  & $\semcm{\Gamma}{F}{I,\alpha}\in \semm{o}{I}=\{0,1\}$ \\
\hline
\end{tabular}
\end{center}

Note that because variables do not occur in types, the semantics of types only depends on the model and not on the assignment.

\begin{definition}[Models, Assignments, Semantics of STT]
A \emph{model} $I$ of an STT-signature $\Sigma$ is defined as follows:
\begin{itemize}
	\item If $\Sigma=\cdot$, then $I$ is the empty tuple.
	\item If $\Sigma=\Sigma_0,a:\TYPE$, then $I$ is a pair $(I_0,a^I)$ where $I_0$ is a model of $\Sigma_0$ and $a^I$ is a set.
	\item If $\Sigma=\Sigma_0,c:A$, then $I$ is a pair $(I_0,c^I)$ where $I_0$ is a model of $\Sigma_0$ and $c^I\in \semm{A}{I}$.
\end{itemize}

An assignment $\alpha$ from a $\Sigma$-context $\Gamma$ into a $\Sigma$-model $I$ is defined as follows:
\begin{itemize}
	\item If $\Gamma=\cdot$, then $\alpha$ is the empty function.
	\item If $\Gamma=\Gamma_0,x:A$, then $\alpha$ is a function $\alpha_0 \cup\{x/u\}$ where $\alpha_0$ is an assignment for $\Gamma_0$ and $u\in \semm{A}{I}$.
\end{itemize}

For a signature $\Sigma$, a context $\Gamma$, a model $I$, and an assignment $\alpha$, the semantics of types and terms in context $\Gamma$ is defined as follows:
\begin{itemize}
	\item $\semm{a}{I}=a^I$,
	\item $\semm{A\arr B}{I}=(\semm{B}{I})^{\semm{A}{I}}$,
	\item $\semcm{\Gamma}{c}{I,\alpha}=c^I$,
	\item $\semcm{\Gamma}{x}{I,\alpha}=\alpha(x)$,
	\item $\semcm{\Gamma}{f\;t}{I,\alpha}= \semcm{\Gamma}{f}{I,\alpha}(\semcm{\Gamma}{t}{I,\alpha})$,
	\item $\semcm{\Gamma}{\lam{x}{A}t}{I,\alpha}= \{(u,\semcm{\Gamma,x:A}{t}{I,\alpha\cup\{x/u\}}) \;:\; u\in \semm{A}{I}\}$.
\end{itemize}
\end{definition}


Then, to define the semantics of HOL, we only have to fix interpretations for the symbols in $\Sigma_{HOL}$. In Def.~\ref{def:hol:models}, we will only fix the interpretations of $o$ and $\doteq_A$; the interpretation of the remaining symbols is induced by the abbreviations from Sect.~\ref{hol:syntax:abbrevs}. Then we proof in Lem.~\ref{lem:hol:models} that these abbreviations do indeed yield the intended interpretations for them.

\begin{definition}[Semantics of HOL]\label{def:hol:models}
We abbreviate $\Boole=\{0,1\}$.
A HOL-model of the HOL-signature $\Sigma$ is an STT-model of $\Sigma$ such that (i) all base types are interpreted as non-empty sets, and (ii) the symbols of $\Sigma_{HOL}$ are interpreted as follows:
\begin{itemize}
	\item $o^I=\Boole$,
	\item $\doteq_A^I \;\in\; \big(\Boole^{\semm{A}{I}}\big)^{\semm{A}{I}} \tb\msuchthat\mforall a,b\in\semm{A}{I}\tb \doteq_A^I(a)(b)\;=\;\cas{1\mifc a=b\\ 0 \mothw}$
\end{itemize}
For $I\in\Mod^{HOL}(\Sigma)$ and $F\in\Sen^{HOL}(\Sigma)$, we put
 \[I\md F \tb\miff\tb \semm{F}{I}=1.\]
As for FOL, this induces the model-theoretic consequence.
\end{definition}

\begin{lemma}\label{lem:hol:models}
Using the abbreviations from Sect.~\ref{hol:syntax:abbrevs}, we have for all $u,v\in\Boole$ and all $\phi:\semm{A}{I}\arr\Boole$:
\begin{itemize}
	\item $\true^I=1$,
	\item $\false^I=0$,
	\item $\neg^I(u)=1-u$,
  \item $\wedge^I(u)(v)=\min\{u,v\}$,
  \item $\vee^I(u)(v)=\max\{u,v\}$,
	\item $\impl^I(u)(v)=\cas{1\mifc u\leq v \\ 0 \mothw,}$
	\item $(\forall_A)^I(\phi)=\min\limits_{a\in \semm{A}{I}}\phi(a)
	  = \cas{1\mifc \phi(a)=1 \mfor\mall a\in\semm{A}{I}\\ 0 \mothw,}$
	\item $(\exists_A)^I(\phi) = \max\limits_{a\in \semm{A}{I}}\phi(a)
	  = \cas{1\mifc \phi(a)=1 \mfor\msome a\in\semm{A}{I}\\ 0 \mothw.}$
\end{itemize}
\end{lemma}
\begin{proof}
We prove the case for $\neg$ as example and leave the rest as an exercise.

Expanding the definition of $\neg$:
\[\neg^I(u)=\semcm{\cdot}{\neg}{I}(u)=\semcm{\cdot}{\lam{x}{o}x\doteq_o \false}{I}(u)\]
Applying the semantics of $\lambda$-abstraction:
\[=\Bigg\{\Big(v,\semcm{x:o}{x\doteq_o\false}{I,x/v}\Big)\;:\;v\in \semm{o}{I}\Bigg\} \;(u)\]
Applying the function $\Bigg\{\ldots\Bigg\}$ to the argument $u$:
\[=\semcm{x:o}{x\doteq_o\false}{I,x/u}\]
Applying the semantics of application twice:
\[=\semcm{x:o}{\doteq_o}{I,x/u}\big(\semcm{x:o}{x}{I,x/u}\big)\big(\semcm{x:o}{\false}{I,x/u}\big)\]
Applying the semantics of variables (to $x$) and constants (to $\doteq_o$ and $\false$):
\[=\;\doteq_o^I(u,\false^I)\]
Applying the semantics of $\doteq_o$ and assuming we have proved $\false^I=0$ already:
\[=\bcas{1\mifc u=\false^I \\ 0 \mothw}=1-u\]
\end{proof}

\begin{remark}
The restriction (i) in Def.~\ref{def:hol:models} implies that all types are interpreted as non-empty sets.
\end{remark}

\begin{remark}
Just like for the syntax, we find that the semantics of HOL is conceptually more complicated as for FOL -- mutual recursion for models and semantics -- but technically simpler than FOL.
\end{remark}