A binary relation between sets $A$ and $B$ is a subset $\#\sq A\times B$.
We usually write $(x,y)\in\#$ as $x\# y$.

\subsection{Classification}

\begin{definition}[Properties of Relations]\label{def:math:rel}
We say that $\#$ is \ldots if the following holds:
 \begin{compactitem}
 \item functional: for all $x\in A$, there is at most one $y\in B$ such that $x\# y$.
 \item total: for all $x\in A$, there is at least one $y\in B$ such that $x\# y$.
 \item injective: for all $y\in B$, there is at most one $x\in A$ such that $x\# y$.
 \item surjective: for all $y\in B$, there is at least one $x\in A$ such that $x\# y$.
 \end{compactitem}

Moreover, $\#$ is \ldots if it is \ldots:
 \begin{compactitem}
 \item a partial function: functional
 \item a (total) function: functional and total
 \item bijective: total, functional, injective, and bijective
 \end{compactitem}

If a $\#$ is functional, we write $\#(x)$ for the $y\in B$ such that $x\# y$ if such a (necessarily unique) $y$ exists.
\end{definition}

Because relations are the subsets of some set (namely $A\times B$), they admit all operations of subsets:
\begin{compactitem}
 \item empty relation: for no $x\in A$, $y\in B$ we have $x\# y$
 \item universal relation: for all $x\in A$, $y\in B$ we have $x\# y$
 \item for two relations $r$ and $s$:
  \begin{compactitem}
   \item union of $r$ and $s$: $x\,(r\cup s)\,y$ iff $x\,r\,y$ or $x\,r\,y$ 
   \item intersection of $r$ and $s$: $x\,(r\cup s)\,y$ iff $x\,r\,y$ and $x\,r\,y$
  \end{compactitem}
\end{compactitem}

Moreover, if $r\sq s$ (if $x\,r\,y$, then also $x\,s\,y$), we say that $s$ is a refinement of $r$.

\subsection{Operations on Relations}

\begin{definition}[Identity]\label{def:math:reliden}
We define the identity relation $\Delta_A$ between $A$ and $A$ by: 
\[x\,\Delta_A\,y \tb\miff\tb x=y\]
i.e., $\Delta_A=\{(x,x):x\in A\}$.
\end{definition}

\begin{theorem}[Identity]\label{thm:math:reliden}
The identity is functional, total, injective, and surjective.
\end{theorem}

\begin{definition}[Composition]\label{def:math:relcomp}
Given relations $r$ between $A$ and $B$ and $s$ between $B$ and $C$, we define the relation $rs$ between $A$ and $C$ by
\[x\,(rs)\,z \tb\miff\tb x\,r\,y \mand y\,s\,z \mforsome y\in B\]
\end{definition}

Composition $rs$ is also written as $r;s$ or $s\circ r$.

\begin{theorem}[Composition]\label{thm:math:relcomp}
If two relations are both functional/total/injective/surjective, then so is their composition.
\end{theorem}

\begin{theorem}[Properties of Composition]\label{def:math:relcat}
Composition is associative whenever defined.

Identity is a neutral element for composition whenever defined.
\end{theorem}

\begin{definition}[Powers]\label{def:math:relcomp}
Given a relation $r$ between $A$ and $A$ and a natural number $n$, we define $X^n$ inductively by
\[r^0=\Delta_A\]
\[r^{n+1}=X^n r\]
We also define
\[r^{-n}=(r^{-1})^n\]
\end{definition}

\begin{definition}[Dual Relation]\label{def:math:reldual}
For every relation $\#$, the relation $\#^{-1}$ is defined by $x\,\#^{-1}\, y$ iff $y\,\#\,x$.

$\#^{-1}$ is called the \textbf{dual} of $\#$.
\end{definition}

\begin{theorem}[Dual Relation]\label{thm:math:reldual}
If a relation is functional/total/injective/surjective, then its dual is injective/surjective/functional/total, respectively.

In particular, the dual of a bijective function is a bijective function, which we call its \textbf{inverse}.
\end{theorem}

\begin{theorem}[Properties of Dualization]\label{thm:math:reldualprop}
For all relations $r$, we have $(r^{-1})^{-1}=r$.

For the identity relation, we have $\Delta_A^{-1}=\Delta_A$.

For all composed relations, we have $(rs)^{-1}=s^{-1}r^{-1}$.

For all powers of relations, we have $(r^n)^{-1}=(r^{-1})^n=r^{-n}$.
\end{theorem}

\subsection{Point-Free Formulations}

A statement about relations between $A$ and $B$ is called \emph{point-free} if no elements $x\in A$ or $y\in B$ are mentioned explicitly.
Point-free formulations take some getting used to but are more compact and eventually easier to work with.

\begin{theorem}[Point-Free Formulations]
For a relation $r$ between $A$ and $B$, the following are equivalent:
\begin{ctabular}{|l|l|}
\hline
functional & $r^{-1}r\sq\Delta_B$\\
total      & $rr^{-1}\supseteq\Delta_A$\\
injective  & $rr^{-1}\sq\Delta_A$\\
surjective & $r^{-1}r\supseteq\Delta_B$\\
\hline
\end{ctabular}

For a relation $r$ on $A$, the following\footnotemark are equivalent:
\begin{ctabular}{|l|l|}
\hline
reflexive & $\Delta_A\sq r$\\
symmetric & $r^{-1}=r$\\
transitive & $rr\sq r$\\
reflexive and transitive & $\Delta_A\sq r$ and $rr=r$\\
\hline
\end{ctabular}
\end{theorem}

\footnotetext{The concepts on the left are defined in Def.~\ref{def:math:binrel}.}

Moreover, we have the following equalities for a function $f$\footnote{See Def.~\ref\label{def:kernel}.}:
\[\ker\,f = f\,f^{-1} \tb\tb \image\,f\times \image\,f = f^{-1}\,f\]