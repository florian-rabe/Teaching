   \footnote{This section is under development and should be considered unpublished.}

\section{History}

% caesar, one-time pad, substitution + transpostions + products thereof, etc.
One of the oldest known encryption algorithms is the so called Caesar cipher. It is said that he used it for communication with his army. It is a very simple character-wise substitution cipher. The idea is to substitute letters for each others. In this very simple case the alphabet has been shifted by 3 letters in a cyclic way. Thus, an a would be encrypted to an e, b would be f, c would be g, d would be h,\ldots, w becomes a, x becomes b and z becomes d. 
\begin{example}\ 
  \begin{lstlisting}
  	meet me after the toga party
  	phhw ph diwhu wkh wrjd sduwb
  \end{lstlisting}
\end{example}
Let us make this algorithm mathematically a bit more precise. Firstly, we represent each letter by a number $1, 2, 3,\ldots , 26$ and the key $k$ another number from 1 to 26. Now for each letter $l$, we can compute its image $\phi(l)$ under the cipher $\phi$ as $phi(l):\Equiv l+3\modop 26$.

In general the alphabet can be shifted by any arbitrary number $k$ of letters from 1 to 26 (or whatever the size of the alphabet is). This is obviously, a very weak cipher. Here, an attacker can easily try out all 26 possible combinations and thus find the key and break the cipher (brute-force-attack). 

We can generalize this approach of substituting the letters individually by different letters in the alphabet. This is called a monoalphabetic cipher. The key now contains the images of each individual letter in our alphabet i.e. it is the ciphertext of the plaintext ``abcdefghijklmnopqrstuvwxyz''. In order to be able to unambiguously decrypt a ciphertext, we require every letter to have a different image. There are still $26!\approx 4\times 10^{26}\approx 2^{88}$ possible keys, so a brute force attack seems very hard and this looks like a very strong cipher, at first sight. 

Monoalphabetic substitution ciphers can however be attacked very efficiently. This is because, they don't obfuscate certain patterns in the plain-text, since they encrypt the same plaintext with the same ciphertext every time. In particular the same letter is always encrypted with the same letter in the ciphertext. 
Thus, the frequencies of occurrences of letters in the ciphertext can be correlated to the expected frequencies of occurrences of the plaintext. Then, the substitution can be guessed. This attack works for all monoalphabetic ciphers. 

So it is useful, to look at polyalphabetic ciphers i.e. cipher that encrypt entire blocks of plaintext. One very simple polyalphabetic cipher is the so called Vigin\`e{}re-cipher, a generalisation of Caesar's cipher, where blocks of subsequent letters are encrypted using Ceasar's cipher for each letter in the block using a different sub-key. 

This can be attacked by first looking for repetitive patterns in the ciphertext (in order to guess the length of the key) and attacking the decomposed Caesar encrypted subsequences individually. 
This works well if the length of the message is significantly longer than the length of the key. 

One notable special case arises, when the length of the key equals the length of the message. In this case we call the cipher a one-time-pad. Then the encryption is absolutely secure, since any ciphertext can be decrypted to any arbitrary plaintext of same length, given the right key. It is vital now that the key is never used twice, otherwise the encryption can be broken by encrypting one ciphertext with the other yielding the encryption of the first \emph{plain}-text with the second. This already gives an attacker potentially very useful information (especially as parts of the plaintext might be known or guessed easily) and might also allow attacks recovering the key. Therefore, this encryption is known as one-time-pad. In practice however, this algorithm is not very useful itself, since it requires the secure transfer of a key as long as the message itself. This can however, occasionally be useful. 

A different approach to encrypt a message is to rearrange the letters of the message instead of substituting them. Such a cipher is called a transposition cipher. One very old example is the rail-fence transposition cipher. Here a message is spelled out diagonally over a fence. Then, the ciphertext is read of row-by-row. 
\begin{example}\ 
  \begin{lstlisting}
    m e   e a t r t e t g   a t
     e t m   f e   h   o a p r y
  \end{lstlisting}
\end{example}
%Another interesting polyalphabetic cipher is the playfair cipher, probably beyond the scope of this class

In any case, transposition ciphers don't really obfuscate plaintext patterns either, for instance the number of occurrences of the letters remain unchanged by the cipher. 
We can now try to compose ciphers, to make them more secure. These ciphers (resulting from composition of other ciphers) are called product ciphers. If we compose a substitution cipher with a substitution cipher, we obtain a substitution cipher again, leaving the same structural weaknesses. If we compose two transposition ciphers, we will again yield a transposition cipher. If however, we compose a substitution and a transposition, we will yield a much harder cipher. This is the key idea of modern (symmetric) cryptography. 


\section{Preliminaries}

\begin{definition}[polynomial-time algorithm]
 An algorithm $A$ is called polynomial time algorithm iff there exists $k\in \mathbb{N}$ such that the maximum number $t$ of operations $A$ has to perform for input of length $n$ satisfies $t(n)\in O(n^k)$.
\end{definition}

\begin{definition}[A probabilistic polynomial-time algorithm]
 A probabilistic polynomial-time algorithm (PPT) is a polynomial-time algorithm that might be non-deterministic. If it is deterministic we will just call it polynomial time algorithm.
\end{definition}

\begin{definition}
 A function $f:\,\mathbb{N}\to\mathbb{R}$ is called negligible iff $$\forall k\in\mathbb{N}.\,\exists N_k\in\mathbb{N}.\,\forall n\geq N_k:\,n\left|f(n)\right|<1.$$
\end{definition}

One important cryptographic primitive is the so called one-way function (OWF). Using these one-way-functions one can build provably secure symmetric encryption algorithms. However, it is currently not known whether there exist any one-way function. In the following we will especially consider one-way permutations (OWP), bijective OWFs with identical input and output length. 

\begin{definition}[One way function]
 A function $f:\{0,1\}^*\to \{0,1\}^*$, is called a one-way function iff, $f$ is a PPT and for any natural number $n$ and any PPT $A$, there is a negligible function $neg$ such that: $$\mathrm{Pr}\left[f(A(f(x),1^n))=f(x)\right]\leq neg,$$ where $x$ is chosen uniformly random.
\end{definition}
Intuitively, OWFs are simply hard to invert PPTs. Some functions that are commonly believed to be OWFs are modular exponentiation and the multiplication of big prime numbers. 
%\vdots\ \\
%One way permutations\\\vdots\ \\

Another important concept is the so called Pseudo-random generator (PRG or PRNG). 
\begin{definition}[PRGs]
 A function $H:\{0,1\}^*\to \{0,1\}^*$ is called PRG iff:
 \begin{itemize}
  \item $H$ is a PPT
  \item There exists a PPT the so called \emph{length extension function} $l:\mathbb{N}\to\mathbb{N}$, s.t. $\forall n\in\mathbb{N}.\,l(n)>n\land \left|H(x)\right|=l(\left|x\right|), \forall x\in\{0,1\}^*$
  \item There is a negligible function $neg$ s.t. for any PPT $A$, we have:
  $$\left|\mathrm{Pr}\left[A(H(U_n))=1\right]-\mathrm{Pr}\left[D(U_{l(n)})=1\right]\right|<neg(n),$$ where $U_k$ denotes a uniformly random element of $\{0,1\}^k$. 
 \end{itemize}
\end{definition}

It can be shown that given any OWF, we can build a PRG. %TODO: Insert a proof for at least OWPs using hard core bits
%This part should probably be moved to the appendix
\begin{definition}[hard-core bit]
 Let $f$ be a one-way function. Now the function $b:\{0,1\}^*\to\{0,1\}$ is called a \emph{hard-core bit} for $f$ if $b$ is computable in polynomial time and there exists a negligible function $neg(n)$ such that for any $n\in\\mathbb{N}$ and any PPT $A$:
 $$\mathrm{Pr}\left[A(f(x),1)=b(x)\right]\leq\frac{1}{2}+neg(n), $$ where $x\in\{0,1\}$ uniformly random. 
\end{definition}

The next step is to find one-way functions (assuming that there are one-way functions at all) for which we can build hard-core bits. 
\begin{theorem}
 Let $f$ be a one-way function. Define $g:\{0,1\}^{2n}\to\{0,1\}^*$ as $g(x\circ y)=f(x)\circ y$, where $\left|x\right|=\left|y\right|=n$. Then $g$ is a one-way function with hard-core bit $$b(x,y)=\bigoplus_{i=1}^n x_i\land y_i=\sum_{i=1}^{n}x_iy_i (\mod 2).$$
\end{theorem}
If now $f$ was a one-way permutation, we can use the above construction to get an additional pseudo-random bit, so we already have a pseudo-random generator.


\section{Hashing}

\subsection{MDx}

\subsection{SHA-x}


\section{Symmetric Encryption}
\subsection{General Definition}

%\vdots polynomial time algorithm\\\vdots\ \\probabilistic algorithm\\\vdots\ \\negligible functions\\\vdots \\encryption scheme\\\vdots
\begin{definition}
 An encryption scheme is an ordered triple $(G, E, D)$, where $G$ and $E$ are PPT algorithms and $D$ is a polynomial time algorithm iff for any \emph{security parameter} $n\in\mathbb{N}$:
 \begin{itemize}
  \item The \emph{key generation algorithm} $G$, takes as input $1$ and uses randomness to chose a \emph{key $k$} from a set of possible keys $K_n$ (the key space)
  \item The \emph{encryption algorithm} $E$, takes as input a \emph{message} $m\in P_n$ ($P_n$ is called the \emph{plaintext space} for the security parameter $n$) and uses $k$ to compute an encrypted message $c\in C_n$ ($C_n$ is called the \emph{ciphertext space} for $n$). If $E$ uses the key $k$ on the message $m$ and outputs $s$, we write $E_k(m)=c$.
  \item The \emph{decryption algorithm} $D$, takes an encrypted message and uses the key $k$ to decipher it (deterministically). So we have $$\forall n\in\mathbb{N},\,\forall m\in P_n,\,\forall k \in K_n.\,D_k(E_k(m))=m.$$
 \end{itemize}
\end{definition}

There are various different notions of ``security'' of encryption. We will discuss the notions of %guess indistinguishability (guess ind.), 
computational indistinguishability (comp. ind.) and security against a chosen Plain-text attack (ind. CPA). 
%\begin{definition}[guess indistinguishable]
% We will call an encryption scheme $(E,D,k)$ random indistinguishable iff for any PPT $A$, for any message $m$ of length $n$ and $e_2\in \{0,1\}^n$ chosen uniformly random:
% \[\exists neg.\,\mathrm{Pr}\left[A(e_i)=1\right],\] where $e_1:=E(m_1), i\in\{0,1\}$, uniformly random. 
%\end{definition}
\begin{definition}[guess indistinguishable]
  We will call an encryption scheme $(E,D,k)$ computationally indistinguishable iff for any PPT $A$, for any two messages $m_1, m_2$ of length $n$ and $i\in \{0,1\}$, chosen uniformly random, there is a negligible function $neg.$ s.t.:
  \[(\mathrm{Pr}\left[A(E(m_0))=1\right]-\mathrm{Pr}\left[A(E(m_1))=1\right])\leq \frac{1}{2}+neg(n), \] where $neg(n)$ is a negligible function.
\end{definition}
\begin{definition}[indistinguishability under chosen-plaintext attack (IND\_CPA)]\ \\
  This is a stronger condition than guess indistinguishability. 
  Let again $(E, D, k)$ denote our encryption scheme. 
  Now the attacker $A$ is allowed to query an oracle with arbitrary (finite) many pairs of messages $(p_0, p_1)$ and receives $E(m_i)$, for $i$ uniformly random in $\{0,1\}$. Then $A$ is asked to distinguish the real message from a faked one, given the ciphertext of one of them (chosen uniformly random). If this can only be done with a probability that is negligibly greater than $\frac{1}{2}$, the encryption scheme $(E, D, k)$ is called indistinguishable under chosen-plaintext attack or secure against a chosen-plaintext attack (IND\_CPA secure). 
  
  In other words an encryption scheme $(E, D, k)$ is IND\_CPA secure iff it is computationally indistinguishable against an attacker that can additionally query the above oracle, before having to break the cipher. 
  
  Since the attacker could simply chose to not use the oracle, it follows that any encryption scheme that is IND\_CPA secure is also computationally indistinguishable, so this is indeed a stronger condition. 
\end{definition}
%TODO: Insert the definition of IND_CPA

\subsection{Specific Schemes}

Given a PRG, we can iterate it on its own output to get an arbitrarily long pseudo-random output. Now we can construct an encryption scheme by simply taking the key as input to a pseudo random generator and xoring the message with the output of the pseudo-random generator. The decryption can be done by simply encrypting the ciphertext a second time. The resulting encryption scheme can already be shown to be computational indistinguishable, although not necessarily ind. CPA secure. We will now try to improve the security to also reach at least ind. CPA security. 

\subsubsection{Feistel Ciphers}

For this we can use a so called Feistel network.
%Improving a comp. ind. encryption scheme to an ind. CPA secure scheme using a Feistel-network.
\begin{definition}[A Feistel cipher]
 Let $k$ be any natural number (the number of \emph{rounds}). Let $f_{k_i}$ be a family of functions\protect\footnote{If possible pseudorandom functions and possibly one-way functions.} ($f$ is the so called \emph{round function}) of output length $n$ indexed by the sequence of \emph{round keys} $k_1, k_2, \ldots, k_n$. 
 Then the following encryption algorithm $E_k$ is called Feistel cipher. %network with $k$ iterations (for some odd $k$) based on a pseudo random generator $f_{k_i}$ and round keys $k_1, k_2, \ldots, k_n$ is an encryption scheme $(G,E,D)$, defined by:
 %For any odd $k\in\mathbb{N}$ we call an encryption scheme $(G, E, D)$ a Feistel network with $k$ iterations and $P$-Box $f$, iff for some \emph{round keys} $k_1, k_2, \ldots, k_n$ used as input for the pseudo random generator $f_{k_i}$. 
 \begin{itemize}
  \item Fix a message $m=:x_1\circ x_2$, where $\left|x_1\right|=\left|x_2\right|=n$
  \item Define the sequences $L_1, L_2, \ldots, L_n$ and $R_1, R_2, \ldots, R_n$ by $L_1:=x_1, R_1:=x_2$ and $L_{n+1}:=R_n, R_{n+1}:=L_n\oplus f_{k_n}(R_n)$. Finally define $E_k:x_1\circ x_2\to L_k\circ R_k$. 
 \end{itemize}
 Now we can define the corresponding decryption algorithm $D_k$ just like $e_k$, but with the reversed order of round keys:
 \begin{itemize}
   \item Fix a ciphertext $c=:x_1\circ x_2$, where $\left|x_1\right|=\left|x_2\right|=n$
   \item Define the sequences $L_1, L_2, \ldots, L_n$ and $R_1, R_2, \ldots, R_n$ by $L_1:=x_1, R_1:=x_2$ and $L_{n+1}:=R_n, R_{n+1}:=L_n\oplus f_{k_{k-n}}(R_n)$. Finally define $D_k:x_1\circ x_2\to L_k\circ R_k$. 
 \end{itemize}
\end{definition}
Feistel ciphers have been shown to fulfill several notions of security assuming that the round function is actually pseudo random. For instance Feistel networks with at least $3$ rounds are ind. CPA secure and for more rounds they fulfill even stronger notions of security. %TODO: Check and clearify the exact model (3 seems sufficient under some assumptions, but 4 is definitely better (and already fulfills stronger notions)) and perhaps mention some other results. %see \url{https://link.springer.com/chapter/10.1007/978-3-540-45146-4\_30}


 In practice many symmetric encryption schemes are based either on Feistel networks or on substitution-permutation-networks. 
 \begin{definition}[Substitution-permutation-network]
  A substitution-permutation-networks is a series of linked substitutions (\emph{$S$-Boxes}) and permutations \emph{$P$-Boxes} of blocks in an encryption algorithm. 
  The corresponding encryption algorithm splits the message into several boxes, which are typically fed into the $S$-Boxes, then the result is fed into the $P$-Box (and at some point the round key is used, for instance xored with the result as is AES). The later mentioned AES-encryption algorithm is one example of an encryption algorithm build around a substitution-permutation-network.
 \end{definition}
 Substitution-permutation-network and Feistel networks using $S$-Boxes are quite similar, but there are also some noticeable differences. Ciphers based on substitution-permutation-networks can be better parallelized, but Feistel ciphers can use any pseudo random function (for instance any one-way function) and is therefore limited to invertible ($P$-Boxes). %Also the Feistel networks can be adapted to ciphers not using blocks (for instance it is used in OAEP).
%TODO:Insert some additional material about block modes.

\subsubsection{AES} %Using the wikipedia article
AES is in principle a relatively simple substitution-permutation network based on simple matrix operation. There exist three versions of AES for different key-sizes a 128 -bit version, a 192-bit version and the strongest 256-bit version. They differ mainly by the key size and the number of rounds. They all encrypt 128-bit block of data using a substitution-permutation network.\par
\begin{table}[ht]\centering
  \begin{tabular}{|c|c|c|c|}
  	\hline key size & 128-bit & 192-bit & 256-bit \\ 
  	\hline round number & 10 & 12 & 14 \\ 
  	\hline 
  \end{tabular}
  \caption{AES versions and their number of rounds}
\end{table} 
In the following, let $n$ denote the number of rounds of the AES cipher. 
AES consists of the key expansion, where the key is expanded into 128-bit subkeys that are used in the individual rounds of the substitution-permutation network of AES. The block of data to be encrypted is represented as a \emph{state}, a $4\times 4$ matrix of bytes.  
AES encryption uses four basic operations that are run in every round. They iteratively change the \emph{state} based on the \emph{round key}, the subkey for that specific round. In every round, the following operations are applied in this order (except for the last round, where mix-columns is omitted):
\begin{compactenum}
  \item Sub-bytes: Here every byte of the state is substituted according to the 8-bit S-box. 
  \item Shift-row: Now the bytes in every row of the state are cyclically left-shifted. The $i$-th row is shifted by $i-1$ bytes. 
  \item Mix-columns: This is the most interesting of the permutation steps. This time the columns of the state are transformed. The columns, are interpreted as a column-vector. The entries of this vector (the bits inside the columns) are viewed as polynomials of degree $7$ with bits as coefficients. These polynomials can be added using component-wise xor and can be multiplied using polynomail multiplicatin modulo the polynomial $x^8+x^4+x^3+x+1$. Now the transformation of each column can be represented as a left-multiplication with the matrix $$\left(\begin{matrix}
    2&3&1&1\\1&2&3&1\\1&1&2&3\\3&1&1&2
  \end{matrix}\right).$$
  \item Add-round-key: Finally the state is xored with the round-key   
\end{compactenum}
We can see that AES is a substitution-permutation-network, where the first step is the applied substitution and the second, third and fourth step are permutation, so is their composition. Hence, any round of AES is a substitution followed by a permutation. 

Now we already mostly understand how AES works, except for the key-expansion algorithm. The final algorithm is now: 
\begin{compactenum}
  \item Key expansion: Expand the key to $n+1$ subkeys or round-keys. 
  \item Initial round:  Apply Add-round-key to the state using the first subkey
  \item Rounds: Run $n-1$ rounds on the state, where each $i$-th round consists of:
  \begin{compactenum}
  	\item Sub-bytes
  	\item Shift-row
  	\item Mix-columns
  	\item Add-round-key using the next ($i+1$-th) round-key
  \end{compactenum}
  \item The final round (without mix-columns)
  \begin{compactenum}
  	\item Sub-bytes
  	\item Shift-rows
  	\item Add-round-key using the last $n+1$-th round-key
  \end{compactenum}
\end{compactenum}

Since each of the operations applied to the state vector can be inverted, it is clear how AES encrypted ciphertext can be decrypted again: Firstly, the key is expanded to the round-keys. Then, the rounds are inverted one by one, using the round-keys in inverse order. The rounds can be inverted by applying the inverse operation of the operation in any round in reverse order. In case of add-round-key, the inverse operation is the same operation (xor is inverse to itself). In case mix-columns, the inverse linear transformation must be applied, in case of shift rows, we use a right-shift instead of a left-shift and sub-bytes can be inverted using the inverse substitution table. 
% substitution, one-time pad, DES

\section{Asymmetric Encryption}

The idea behind RSA is that if $N=p\cdot q$ for large prime numbers $p$ and $q$, it is very difficult to compute $p$ and $q$ from $n$.

\subsection{RSA}

\paragraph{Setup}
Choose two large primes $p$ and $q$ (typically of roughly equal size).
Put $N=p\cdot q$.

Now put $n=(p-1)(q-1)$. (Actually, any common multiple of the two numbers is fine.)
Note that $n=\phi(N)$.
Pick $e\in Z_n$ such that there is a $d\in\Z_n$ with $e\cdot d\Equiv_n 1$.
Such a $d$ exists if $\gcd(e,n)=1$ and is easy to compute (see Thm.~\ref{thm:math:extendedeuclid}).

The keys are defined as follows:
\begin{compactitem}
 \item public information (encryption key): $N$ and $e$
 \item private information (decryption key): $n$, $d$, $p$, and $q$
\end{compactitem}
Among the private information, only $N$ and $d$ are needed later on.
So $n$, $p$, and $q$ can be forgotten.
But they have to remain private---$p$ (or $q$) is enough to compute $n$ and $d$.

Different keys are often compared by their size.
That size is the number of bits in $N$.

\paragraph{Encryption}
Messages are numbers $x\in F_N$.
For example, we can choose the largest $k$ such that $2^k<N$ and use $k$-bit messages.

Encryption and decryption are functions $\Z_N\to \Z_N$ given by
\begin{compactitem}
 \item encryption: $x\mapsto x^e\modop N$
 \item decryption: $x\mapsto x^d\modop N$
\end{compactitem}

These are indeed inverse to each other:

\begin{theorem}
For all $x\in Z_N$, we have $(x^d)^e\Equiv_N (x^e)^d \Equiv_N x$.
\end{theorem}
\begin{proof}
In general, because $N=p\cdot q$ for prime numbers $p$ and $q$, we have that $x\Equiv_N y$ iff $x\Equiv_p y$ and $x\Equiv_q y$.

So we have to show that $x^{de}\Equiv_p x$.
(We also have to show the same result for $q$, but the proof is the same.)
We distinguish two cases:
\begin{compactitem}
\item $p|x$: Then trivially $x^{de}\Equiv_p x\Equiv_p 0$.
\item Otherwise. Then $p$ and $x$ are coprime.\\
   By construction of $e$ and $d$ and using Thm.~\ref{thm:math:extendedeuclid}, we have $k\in\N$ such that $e\cdot d+k\cdot n=1$.
   Thus, we have to show $x^{de}=x\cdot (x^{p-1})^{k\cdot(q-1)}\Equiv_p x$.
   That follows from $x^{p-1}\Equiv_p 1$ as known from Thm.~\ref{thm:math:fermatlittle}.
\end{compactitem}
\end{proof}

\paragraph{Attacks}
To break RSA, $d$ has to be computed.
There are $3$ natural ways to do that:
\begin{compactitem}
 \item Factor $N$ into $p$ and $q$. Then compute $d$ easily.
 \item Compute $n$ using $n=\phi(N)$ (which may be easier than finding $p$ and $q$). Then compute $d$ easily.
 \item Find $d$ such that $e\cdot d\Equiv_n 1$ (which may be easier than finding $n$).
\end{compactitem}
Currently these are believed to be equally hard.

It is believed that there is no algorithm for factoring $N$ that is polynomial in the number of bits of $N$.
That is no proved.
There are hypothetical machines (e.g., quantum computers) that can factor $N$ polynomially.

Note that checking if $N$ can be factored (without producing the factors) is polynomial, and practical algorithms exist (in particular, the AKS algorithms).
That is important to find the large prime number $p$ and $q$ efficiently.
\medskip

If there is indeed no polynomial algorithm, factoring relies on brute-force attacks that find all prime numbers $k<\sqrt{N}$ and test $k|N$.
Therefore, larger keys are harder than break to smaller ones.
Because of improving hardware, the key size that is considered secure grows over time.

Keys of size $1024$ are considered secure today, but because security is a relative term, keys of size $2048$ are often recommended. 
%It is quetionable that 1024-bit rsa is really secure, even though it has not yet been publicly broken;
%(see for instance https://en.wikipedia.org/wiki/RSA_%28cryptosystem%29#Integer_factorization_and_RSA_problem or https: //www.schneier.com/blog/archives/2007/05/307digit_number.html).
Larger keys are especially important if data is needed to remain secure far into the future, when faster hardware will be available.

\section{Authentication}

\section{Key Generation and Distribution}

   % security protocol correctness, such as key management protocol correctness,
   % e.g., going back to the work by Burrows, Abadi, and Needham on a logic of authentication (https://doi.org/10.1145/77648.77649)
   % Its an application of logic to a concrete problem space where errors are obviously bad.
