Static analysis is a collective term for methods that do not execute the program.
It is performed by external tools that analyze the source code to detect faults.

Usually static analysis works with the source code.
But in principle it can work on the binary as well, e.g., some tools work on Java byte code.

Due to undecidability, detecting faults requires insight and careful case-by-case analysis, which is expensive.
Therefore, it has become very successful to focus on classes of faults and then to systematically find their occurrences.
We can think of these as individual heuristics, each hunting for a specific common bug, e.g., the ones discussed in Setc.~\ref{sec:sd:commonbugs}).
State-of-the-art tools check for hundreds of such fault classes.
Depending on the fault class, the checks may exhibit false-negatives (not all instances are detected) or false-positives (correct code is falsely marked).

The above definition includes compilers, and indeed compilation, in particular type-checking, can be seen as the most basic kind of static analysis.
But, more narrowly, static analysis usually refers to additional checks that go beyond type-checking.

Some compilers include some basic such checks.
More checks are implemented by dedicated analysis tools that are employed when the compiler reports no more faults.
Some of these tools can be integrated with the IDEs to report the results in the same way as compiler errors.

An overview of static analysis tools for various programming languages can be found at \url{https://en.wikipedia.org/wiki/Static_code_analysis}
For example, findbugs is a very popular tool working on Java byte code.
It checks for over $400$ different faults.

%\paragraph{Coding Style}
%% (indentation, documentation)
%
%\paragraph{Array and List Bounds}
%Often the length is statically (i.e., at compile-time, without executing) known.
%For example, if an array is created via $x=Array[\Int](n)$, we expect for-loop that read or write to $x$ to run from $0$ to $n-1$.
%
%If the bounds are different, that can be used to generate a warning.
%
%This is only a heuristic, of course.
%For the general case, we have to prove that the index $i$ in $x[i]$ is between $0$ and $n-1$.
%There are tools that indeed try to prove that.
%
%\paragraph{Uninitialized Memory}
%
%\paragraph{Null Pointers}
%Analysis tools can inspect every dereferencing of a pointer $x$ and check if there is a previous assignment that assigns a non-null value to $x$.
%
%\paragraph{Unreachable Code}
%
%\paragraph{Non-Exhaustive Match}
%
%\paragraph{Casting}