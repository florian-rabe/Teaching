\documentclass{book}

%\usepackage{graphicx}
%\usepackage{xkeyval}
%\usepackage{multirow}
%\usepackage{bm} %% bold face math symbols
\usepackage{listings}
%\usepackage{tikz}
%\usetikzlibrary{shapes}
%\usetikzlibrary{arrows}
%\usepackage{stmaryrd}%\newcommand{\contra}{\lightning}
%\usepackage{rotating} \newcommand{\sw}[1]{\begin{sideways}#1\end{sideways}}

%\usepackage{algorithm}
%\usepackage{ded}
\usepackage{../mylecturenotes}

\title{Lectures Notes on Secure and Dependable Systems}
\author{Florian Rabe}
\date{2017}


\begin{document}
\maketitle

\tableofcontents
\newpage

\part{Introduction} % one week

 \chapter{Meta-Remarks}
  \begin{center}
\textbf{Important stuff that you should read carefully!}
\end{center}

\paragraph{State of these notes}
I constantly work on my lecture notes.
Therefore, keep in mind that:
\begin{compactitem}
\item I am developing these notes in parallel with the lecture---they can grow or change throughout the semester.
\item These notes are neither a subset nor a superset of the material discussed in the lecture.
\item Unless mentioned otherwise, all material in these notes is exam-relevant (in addition to all material discussed in the lectures).
\end{compactitem}
\medskip

\paragraph{Collaboration on these notes}
I am writing these notes using LaTeX and storing them in a git repository on GitHub at \url{https://github.com/florian-rabe/Teaching}.
Familiarity with LaTeX as well as Git and GitHub is not part of this lecture. But it is essential skill for you.
Ask in the lecture if you have difficulty figuring it out on your own.
\medskip

As an experiment in teaching, I am inviting all of you to collaborate on these lecture notes with me.
\medskip

By forking and by submitting pull requests for this repository, you can suggest changes to these notes.
For example, you are encouraged to:
\begin{compactitem}
\item Fix typos and other errors.
\item Add examples and diagrams that I develop on the board during lectures.
\item Add solutions for the homeworks if I did not provide any (of course, I will only integrate solutions after the deadline).
\item Add additional examples, exercises, or explanations that you came up or found in other sources.
 If you use material from other sources (e.g., by copying an diagram from some website), make sure that you have the license to use it and that you acknowledge sources appropriately!
\end{compactitem}
The TAs and I will review and approve or reject the changes.
If you make substantial contributions, I will list you as a contributor (i.e., something you can put in your CV).
\medskip

Any improvement you make will not only help your fellow students, it will also increase your own understanding of the material.
Therefore, I can give you up to $10\%$ bonus credit for such contributions.
(Make sure your git commits carry a user name that I can connect to you.)
Because this is an experiment, I will have to figure out the details along the way.

\paragraph{Other Advice}
I maintain a list of useful advice for students at \url{https://svn.kwarc.info/repos/frabe/Teaching/general/advice_for_students.pdf}.
It is mostly targeted at older students who work in individual projects with me (e.g., students who work on their BSc thesis).
But much of it is useful for you already now or will become useful soon.
So have a look.

\paragraph{Skipped Chapters}
These notes were originally prepared for my 2nd semester CS course at Jacobs University in Spring 2017.
In that course, the following chapters were skipped or treated only very superficially: \ref{sec:ad:finiteds}, \ref{sec:ad:numbers}, \ref{sec:ad:option}, \ref{sec:ad:functions}, \ref{sec:ad:unions}, \ref{sec:ad:parallel}, \ref{sec:ad:prot}, \ref{sec:ad:random}, \ref{sec:ad:quantum}.
In the other chapters, almost all material was covered; only a few subsections were skipped.

 \chapter{Challenges}
  % systems are not secure and dependable
  % we don't know how to build them
  % harder to teach than algos, logic, etc
  % works for engineers, book how they know it
  % will hopefully change during your careers
  % typical: low cost vs. safe
  % this course: overview of current methods with different degrees of adoption in practice 
   \section{Major Disasters Caused by Programming Errors}

All damage estimates are relative to the time of the failure and not adjusted to inflation.

Note that for security problems, the size of the damage is naturally unknown because attacks will typically remain secret.
Only the cost of updating the software can be estimated, but that value is often small.

\paragraph{Mariner 1}
The 1962 rocket launch of Mariner 1 failed causing damage of around \$20 million.

The cause was a programming error, where a mathematical formula in the specification was misread.

Details: \url{https://en.wikipedia.org/wiki/Mariner_1}

\paragraph{Therac-25}
Between 1985 and 1987, the Therac-25 machine for medical radiation therapy caused death and/or serious injury in at least $6$ cases.
Patients received a radiation overdose because the high intensity energy beam was administered while using the protection meant for the low intensity beam.

The cause was that the hardware protection was discontinued relying exclusively on software to prevent a mismatch of beam and protection configuration.
But the software had always been buggy due to a systemic failures in the software engineering process including complex systems (code written in assembly, machine had its own OS), lack of software review, insufficient testing (overall system could not be tested), bad documentation (error codes were not documented), and bad user interface (critical safety errors could be manually overridden, thus effectively being warnings).

Details: \url{https://en.wikipedia.org/wiki/Therac-25}

\paragraph{Patriot Rounding Error}
In 1991 during the Gulf war, a US Patriot anti-missile battery failed to track an incoming Iraqi Scud missile resulting the death of 28 people.

The cause was a rounding error in the floating point computation used for analyzing the missile's path.
The software had to divide a large integer (number of $0.1s$ clock cycles since boot $100$ hours ago) by $10$ to obtain the time.
This was done using a floating-point multiplication by $0.1$---but $0.1$ is off by around $0.000000095$ when chopped to a $24$-bits binary float.
The resulting time was off by $0.3$ seconds, which combined with the high speed of Scud missile led to a serious miscalculation of the flight path.

Details: \url{http://www-users.math.umn.edu/~arnold/disasters/patriot.html}

\paragraph{Ariane 5}
In 1996, the first launch of an Ariane 5 rocket (overall development cost \$$7$ billion) failed, and the rocket had to be destructed after launch.
Both the primary and the backup system had shut down, each trying to transfer control to the other, after encountering the same behavior, which they interpreted as a hardware error.

The cause was an overflow exception in the alignment system caused by converting a $64$-bit float to a $16$-bit integer, which was not caught and resulted in the display of diagnostic data that the autopilot could not interpret.
The programmers were away of the problem but had falsely concluded that no conversion check was needed (and therefore omitted the check to speed up processing).
Their conclusion had been made based on Ariane 4 flight data that turned out to be inappropriate for Ariane 5.

The faulty component was not even needed for flight and was only kept active for a brief time after launch for convenience and in order to avoid changing a running system.

Details: \url{http://www-users.math.umn.edu/~arnold/disasters/ariane5rep.html}

\paragraph{Intel Pentium Bug}
In 1994, it was discovered that the Intel Pentium processor (at the time widely used in desktop computers) wrongly computed certain floating point divisions.
The cost of replacing the CPUs was estimated at about \$$400$ million.

The error occurred in about 1 in 9 billion divisions.
For example, $4195835.0/3145727.0$ yielded $1.333 739 068 902 037 589$ instead of $1.333 820 449 136 241 000$.

The cause was a bug in the design of the floating point unit's circuit.

\paragraph{Kerberos Random Number Generator}
From 1988 to 1996, the network authentification protocol Kerberos used a mis-designed random number generation algorithm.
The resulting keys were so predictable that brute force attacks became trivial although it is unclear if the bug was ever exploited.

The cause was the lack of a truly random seed value for the algorithm.
Moreover, the error persisted across attempted fixes because of process failures (code hard to read, programmers had moved on to next version).

Detail: \url{http://docs.lib.purdue.edu/cgi/viewcontent.cgi?article=2331&context=cstech}

\paragraph{USS Yorktown}
In 1997, critical navigation and weapons hardware on the USS Yorktown was paralyzed at sea for $3$ hours while rebooting machines.

The cause was a blank field in a database that was interpreted as $0$ leading to a division-by-zero.
Special floating point values such as infinity or NaN were not used resulting in an exception.
The exception was handled by neither the software nor the operating systems (Windows NT) thus crashing both.

Details: \url{http://www.cs.berkeley.edu/~wkahan/Boulder.pdf}

\paragraph{Mars Climate Orbiter}
In 1998 the Mars Climate Orbiter was lost causing damage of around \$$300$ million after software had calculated a false trajectory when updating the position of the spacecraft.

The cause was that two components by different manufacturers exchanged physical quantities as plain numbers (i.e., without units).
One component assumed customary units (pound seconds) whereas the other assumed SI units (Newton seconds).
The first component was in violation of the specification of the interface.

\paragraph{Year 2000 and 2038 Problems}
Leading to the year 2000, about \$$300$ billion were spent worldwide to update outdated software that was unable to handle dates with a year of $2000$ or higher.

The cause was that much software was used far beyond the originally envisioned lifetime.
At programming time, especially at times when memory was still scarce, it made sense to use only two digits for the year in a date.
That assumption became flawed when dates over $2000$ had to be handled.

A related problem is expected in the year 2038.
At that point the number of seconds since 1970-01-01, which is the dominant way of storing time on Unix, will exceed the capacity of a $32$-bit integer.
While application software is expected to be updated by then anyway, modern embedded systems may or may not still be in use.

\paragraph{Los Angeles Airport Network Outage}
In 2007, LA airport was partially blocked for $10$ due to a network outage that prevented passenger processing.
About 17,000 passengers were affected.

The cause was a single network card malfunction that flooded the network and propagated through the local area network.

Details: \url{https://www.oig.dhs.gov/assets/Mgmt/OIGr_08-58_May08.pdf}

\paragraph{Debian OpenSSL Random Number Generator}
From 2006 to 2008 Debian's variant of OpenSSL used a flawed random number generator.
This made the generated keys easily predictable and thus compromised.
It is unclear whether this was exploited.

The cause was that two values were used to obtain random input: the process ID and an uninitialized memory field.
Uninitialized memory should never be used but is sometimes used as a convenient way to cheaply obtain a random number in a low-level programming language like C.
The respective line of code had no immediately obvious purpose because it was not commented.
Therefore, it was removed by one contributor after code analysis tools had detected the use of uninitialized memory and flagged it as a potential bug.

Detail: \url{https://github.com/g0tmi1k/debian-ssh}

\paragraph{Knight Capital Trading Software}
In 2012, high-frequency trading company Knight Capital lost about \$$10$ million per minute for 45 minutes trading on the New York Stock Exchange.

The cause was an undisclosed bug in their automatic trading software.
% http://www.bbc.com/news/magazine-19214294

\paragraph{Heartbleed}
From 2012 to 2014, the OpenSSL library was susceptible to an attack that allowed remotely reading out sections of raw physical memory.
The affected sections were random but repeated attacks could piece together large parts of the memory.
The compromised memory sections could include arbitrary critical data such as passwords or encryption keys.
OpenSSL was used not only by many desktop and server applications but also in portable and embedded devices running Linux.
The upgrade costs are very hard to estimate but were put at multiple \$$100$ millions by some experts.

The cause was a bug in the Heartbeat component, which allowed sending a message to the server, which the server echoed back to test if the connection is alive.
The server code did not check whether the given message length $l$ was actually the length of the message $m$.
Instead, it always returned $l$ bytes starting from the memory address of $m$ even if $l$ was larger than the length of $m$.
This was possible because the used low-level programming language (C) let the programmers store $m$ in a memory buffer and then over-read from that buffer.

Details: \url{http://www.theregister.co.uk/2014/04/09/heartbleed_explained/}

\paragraph{Shellshock (Bashdoor)}
From 1998 to 2014, it was possible for any user to gain root access in the bash shell on Unix-based systems.
The upgrade cost is unknown but was generally small because updates were rolled out within $1$ week of publication.
Moreover, in certain server applications that passed data to bash this was possible for arbitrary clients as well.

The cause was the use of unvalidated strings to represent complex data.
Bash allowed storing function definitions as environment variables in order to share function definitions across multiple instances.
The content of these environment variables was trusted because function definitions are meant to be side-effect-free.
However, users could append $; C$ to the value of an environment variable defining a function.
When executing this function definition, bash also executed $C$.

%env x='() { :;}; C bash -c :
%means
%x = 'lambda(). : ; C
%and C is also executed (with root privileges)

Independently, many server applications (including the widely used cgi-bin) pass input provided by remote users to bash through environment variables.
This resulted in input provided by remote clients being passed to the bash parser, which was against the assumptions of the parser.
Indeed, several bugs in the bash parser caused remotely exploitable vulnerabilities.

Details: \url{https://fedoramagazine.org/shellshock-how-does-it-actually-work/}

\paragraph{Apple 'goto fail' Bug}
From 2012 to 2014, Apple's iOS SSL/TLS library falsely accepted faulty certificates.
This left most iOS applications susceptible to impersonation or man-in-the-middle attacks.
Because Apple updated the software after detecting the bug, its cost is unclear.

The immediately cause was a falsely-duplicated line of code, which ended the verification of the certificate instead of moving on to the next check.
But a number of insufficiencies in the code and the software engineering process exacerbated the effect of the small bug.

The code was as follows:

\begin{lstlisting}
static OSStatus SSLVerifySignedServerKeyExchange(
  SSLContext *ctx, bool isRsa, SSLBuffer signedParams,
  uint8_t *signature, UInt16 signatureLen)
{
	OSStatus        err;
	...

	if ((err = SSLHashSHA1.update(&amp;hashCtx, &amp;serverRandom)) != 0)
		goto fail;
	if ((err = SSLHashSHA1.update(&amp;hashCtx, &amp;signedParams)) != 0)
		goto fail;
		goto fail;
	if ((err = SSLHashSHA1.final(&amp;hashCtx, &amp;hashOut)) != 0)
		goto fail;
	...

fail:
	SSLFreeBuffer(&amp;signedHashes);
	SSLFreeBuffer(&amp;hashCtx);
	return err;
\end{lstlisting}

In a better programming language that emphasizes the use of high-level data structures, the bug would likely not have happened or be caught easily.
But even using C, it could have been caught by a variety of measures including unreachable code analysis, indentation style analysis, code coverage analysis, unit testing, or coding styles that enforce braces around one-command blocks.
 
Details: \url{https://www.imperialviolet.org/2014/02/22/applebug.html}
%https://www.dwheeler.com/essays/apple-goto-fail.html

\section{Major Problems Caused By System Updates}

\paragraph{Odyssey Court Software}
In an ongoing crisis since 2016, US county court and California and other states have difficulties using the new Odyssey software for recording and dissemination of court decisions.
This has caused dozens of human rights violations due to erroneous arrests or imprisonment.
This includes cases where people spent 20 days in jail based on warrants that had already been dismissed.

The cause is a tight staffing situation combined with the switch to a new, more modern software system for recording court decisions.
The new software expects uses more high-level data types (e.g., reference to a law instead of string) in many places.
This has led to the erroneous recording of decisions and a backlog of converting old decisions into the new database (including decisions that invalidate decisions that are already in the database).

Details: \url{https://arstechnica.com/tech-policy/2016/12/court-software-glitches-result-in-erroneous-arrests-defense-lawyers-say/}

\paragraph{Other Notable Cases}
This is a selection of failures that did not cause direct damage but led to availability failures on important infrastructure.

In 1990, all AT\&T phone switching centers shut down for 9 hours due to a bug in a software update.
An estimated 75 million phone calls were missed.

In 1999, a faulty software update in the British passport office delayed procedures.
About half a million passports were issued late.

In 2004, the UK's child support agency EDS introduced a software update while restructuring the personnel.
This led to several million people receiving too much or too little money and hundreds of thousands of back-logged cases.

In 2015, the New York Stock Exchange had to pause for $3$ hours for a reboot after a software problem.
700,000 trades had to be canceled.

In 2015, hundreds of flights in the North Eastern US had to be canceled or delayed for several hours.
The cause was a problem with new and behind-schedule computer system installed in air traffic control centers.
% http://www.bbc.com/news/world-us-canada-33950381

\section{Other Notable Problems}

\paragraph{Excel Gene Names}
In 2016, researchers found that about 20\% of papers in genomics journals contain errors in supplementary spreadsheets.

The cause is that Microsoft Excel by default guesses the type of cell data that is entered as a string and converts the string into that type.
This affects gene names like "SEPT2" (Septin 2, converted to the date September 02) or REKIN identifiers like "2310009E13" (converted to the floating point number $2.31E+13$).

Details: \url{https://genomebiology.biomedcentral.com/articles/10.1186/s13059-016-1044-7}

\paragraph{FBI Virtual Case File Project}
In 2005 the Virtual Case File project of the FBI, which had been developed since 2000, was scrapped.
The software was never deployed, but the project resulted in the loss of \$$170$ million of development cost

The cause was systemic failures in the software engineering process including
\begin{compactitem}
 \item poor specification, which caused bad design decisions
 \item repeated specification changes
 \item repeated change in management
 \item micromanagement of software developers
 \item inclusion of many personnel which little training in computer science in key positions
\end{compactitem}
These problems were exacerbated by the planned flash deployment instead of a gradual phasing-in of the new system---a decision that does have advantages but made the systems difficult to test and made it easier for design flaws to creep in.
The above had two negative effects on the code base
\begin{compactitem}
 \item increasing code size due to changing specifications
 \item increasing scope due to continually added features
\end{compactitem}
which exacerbated the management and programming problems.

\paragraph{Failures in Involving Computer-Related Manufacturing}
This is a selection of other notable failures that involve hardware manufacturing.

In 2006, two Airbus plants used incompatible version of CAD software.
This resulted in cables being produced too short to connect.

In 2006, Sony batteries mostly used in Dell notebooks had to be recalled.
The resulting cost was about \$$100$ million.

In 2016, Samsung Galaxy phones had to be recalled due to faulty batteries.

\section{Major Vulnerabilities due to Weak Security}

\subsection{Software and Internet}

% identity theft, two-factor authentication

\paragraph{Operating Systems}
forced updates, i love you, government-sponsored exploits, ios article

\paragraph{Cloud Services}
storage, social networks, synchronization, fappening

\paragraph{Large Institutions}
In 2014, Sony Pictures suffered a major break in (possibly by North Korea to blackmail or punish Sony in relation to the movie \emph{The Interview}) mostly facilitated by unprecedented negligence.
Problems included
\begin{compactitem}
 \item unencrypted storage of sensitive information
 \item password stored in plain text files (sometimes even called ``passwords'' or placed in the same directory as encrypted files)
 \item easily guessable passwords
 \item large number of unmonitored devices
 \item lack of accountability and responsibility for security, ignorance towards recommendations and audits
 \item lack of systematic lesson-learning from previous failures (which included 2011 hacks of Sony PlayStation Network and Sony Pictures that stole account information including unsalted or plain text passwords)
 \item weak IT and information security teams
\end{compactitem}
Stolen data included employee data (including financial data), internal emails, and movies.

% DNC email hack

\paragraph{Web Site Account Data}
Many organizations holding user data employ insufficient security against digital break-ins and insufficient (if any) encryption of user data.
They get hacked routinely.
An overview can be found at \url{https://haveibeenpwned.com/}.

The effects are exacerbated by two effects:
\begin{compactitem}
 \item System administrators are not sufficiently educated about password hashing, often false believing default hash configurations to be secure.
 Thus, hacks often allow inverting the hash function thus exposing passwords in addition to the possibly sensitive user data.
 \item Users are not sufficiently educated about systematically using different passwords on every site.
 Thus, any breach also compromises accounts on any other sites that use the same user name or email address and password.
\end{compactitem}

The following describes a few high-profile cases.
\medskip

In a (estimated) 2008 (only reported in 2016) of myspace, about $360$ million accounts were compromised.
The stolen data included user name, email address, and badly hashed passwords (unsalted SHA1).

In a 2012 hack of linkedin, $160$ million accounts were compromised.
The stolen data included user name, email address, and badly hashed passwords (unsalted SHA1).

In a 2015 hack of Ashley Madison, about $30$ accounts were compromised.
The stolen records included name, email address, hashed password, physical description, and sexual preferences.
Most passwords were hashed securely (using bcrypt for salting and stretching), but about $10$ million passwords were hashed insecurely (using a single MD5 application).
This led to multiple extortion attempts and possibly suicides.
% https://nakedsecurity.sophos.com/2015/09/10/11-million-ashley-madison-passwords-cracked-in-10-days/

In a 2016 hack of the Friend Finder network, about $400$ million accounts were compromised.
The stolen records included name, email address, registration date, and unhashed or badly hashed passwords.
% https://www.leakedsource.com

In two separate hacks in 2013 (only reported in 2016) and 2014 of Yahoo, over $1$ billion user accounts were compromised by presumably state-sponsored actors.
The stolen records included name, email address, phone number, date of birth, and hashed passwords, and in some cases security questions and answers.

\subsection{Dedicated Systems}

Many domains are increasingly using computer technology.
Often this is done by engineers with little training in computer science and even less training in security aspects.
In many cases, the resulting systems are highly susceptible to attacks, spared only by the priorities of potential hackers and terrorists.

\paragraph{Embedded Systems}

\paragraph{Cars}
The upcoming wave of self-driving cars requires the heavy use of experienced software developers and a thorough regulation process.
It is therefore reasonable to hope that security will play a major role in the design and legal regulation.

But even today's traditional cars are susceptible to attacks including remote takeover of locks, wheels, or engine.
The causes are
\begin{compactitem}
 \item not or not properly protected physical interfaces for diagnostics and repair,
 \item permanent internet connections, which are useful for navigation and entertainment, that are not strictly separated from engine controls.
\end{compactitem}
One of the more high-profile benevolent attack demonstrations was described in \url{https://www.wired.com/2015/07/hackers-remotely-kill-jeep-highway/}.

\paragraph{Medical Systems}
Hospitals and manufacturers of medical devices are notoriously easy to hack.

Weaknesses include unchangeable master passwords, unencrypted communication between devices, outdated and non-updateable software running in devices, and outdated or non-existent protection against attackers.
Systemic causes include a highly-regulated release process that precludes fast patching of software and a slow update cycle.

Details:
\url{http://cacm.acm.org/magazines/2015/4/184691-security-challenges-for-medical-devices/fulltext}

See also the Symantec 2016 Healthcare Internet Security Threat Report available at \url{https://www.symantec.com/solutions/healthcare}



  \chapter{Concepts} % one week
    % correctness, dependability, quality, safety, security, privacy
    % Avizienis et al - Fundamental Concepts of Dependability (2001)
       % correctness, dependability, quality, safety, security, privacy
    % Avizienis et al - Fundamental Concepts of Dependability (2001)


% guard against human error; during software development
\part{Dependable Software}

  \chapter{Development Processes} % one week
   % documentation, coding style, versioning, code review, issue-tracking, travis-style testing
   % code analysis: coverage, coding style, indentation, documentation, unreachable code, non-exhaustive match, ill-typed equality
   %   tools: https://en.wikipedia.org/wiki/Static_code_analysis (findbugs for java)
   %          dynamic: https://en.wikipedia.org/wiki/Memory_debugger Valgrind, Purify
        
  \chapter{Common Criteria}

  \chapter{Systematic Testing} %one week
   % overview, unit testing framework, code coverage framework, based on Java libraries
    % overview, unit testing framework, code coverage framework, based on Java libraries


   % database dependency -> peter
   % network dependency -> jürgen

  % \chapter{Design Principles}
  % input validation, no string, high-level data structures
  % buffer bounds, array bounds null pointers
  % typed languages, casting
  % safe by default, minimal access rights
  % expose minimal functionality in interfaces
  % strong, stable specification, documentation
  

% guard against attack
\part{Security}

  \chapter{Institution Security}
   % phishing, social engineering etc., antivirus?

  \chapter{Data Security} % two weeks
    % symmetric key: substitution, one-time pad DES, AES, key generation/distribution
    % asymmetric/public key: RSA
    % authentication: via public key
    % hashing: MD5, SHA-1
    \paragraph{Acknowledgments}
This chapter contains contributions by Colin Rothgang.
J\"urgen Sch\"onw\"alder provided comments on the first version.

\section{History and Introduction}\label{sec:sd:crypto:hist}
The definitions in this chapter are based on \cite{PuMaC2016}. 
This section is based partially on \cite{cryptoNetworkSlides}. 
One of the oldest known encryption algorithms is the so called Caesar cipher. It is said that he used it for communication with his army. It is a very simple character-wise substitution cipher. The idea is to substitute letters for each others. In this very simple case the alphabet has been shifted by 3 letters in a cyclic way. Thus, an a would be encrypted to an e, b would be f, c would be g, d would be h,\ldots, w becomes a, x becomes b and z becomes d. 
\begin{example}\ 
  \begin{lstlisting}
  	meet me after the toga party
  	phhw ph diwhu wkh wrjd sduwb
  \end{lstlisting}
\end{example}
Let us make this algorithm mathematically a bit more precise. Firstly, we represent each letter by a number $1, 2, 3,\ldots , 26$ and the key $k$ another number from 1 to 26. Now for each letter $l$, we can compute its image $\phi(l)$ under the cipher $\phi$ as $phi(l):\Equiv l+3\modop 26$.

In general the alphabet can be shifted by any arbitrary number $k$ of letters from 1 to 26 (or whatever the size of the alphabet is). This is obviously, a very weak cipher. Here, an attacker can easily try out all 26 possible combinations and thus find the key and break the cipher (brute-force-attack). 

We can generalize this approach of substituting the letters individually by different letters in the alphabet. This is called a monoalphabetic cipher. The key now contains the images of each individual letter in our alphabet i.e. it is the ciphertext of the plaintext ``abcdefghijklmnopqrstuvwxyz''. In order to be able to unambiguously decrypt a ciphertext, we require every letter to have a different image. There are still $26!\approx 4\times 10^{26}\approx 2^{88}$ possible keys, so a brute force attack seems very hard and this looks like a very strong cipher, at first sight. 

Monoalphabetic substitution ciphers can however be attacked very efficiently. This is because, they don't obfuscate certain patterns in the plain-text, since they encrypt the same plaintext with the same ciphertext every time. In particular the same letter is always encrypted with the same letter in the ciphertext. 
Thus, the frequencies of occurrences of letters in the ciphertext can be correlated to the expected frequencies of occurrences of the plaintext. Then, the substitution can be guessed. This attack works for all monoalphabetic ciphers. 

So it is useful, to look at polyalphabetic ciphers i.e. cipher that encrypt entire blocks of plaintext. One very simple polyalphabetic cipher is the so called Vigin\`e{}re-cipher, a generalisation of Caesar's cipher, where blocks of subsequent letters are encrypted using Ceasar's cipher for each letter in the block using a different sub-key. 

This can be attacked by first looking for repetitive patterns in the ciphertext (in order to guess the length of the key) and attacking the decomposed Caesar encrypted subsequences individually. 
This works well if the length of the message is significantly longer than the length of the key. 

One notable special case arises, when the length of the key equals the length of the message. In this case we call the cipher a one-time-pad. Then the encryption is absolutely secure, since any ciphertext can be decrypted to any arbitrary plaintext of same length, given the right key. It is vital now that the key is never used twice, otherwise the encryption can be broken by encrypting one ciphertext with the other yielding the encryption of the first \emph{plain}-text with the second. This already gives an attacker potentially very useful information (especially as parts of the plaintext might be known or guessed easily) and might also allow attacks recovering the key. Therefore, this encryption is known as one-time-pad. In practice however, this algorithm is not very useful itself, since it requires the secure transfer of a key as long as the message itself. This can however, occasionally be useful. 

A different approach to encrypt a message is to rearrange the letters of the message instead of substituting them. Such a cipher is called a transposition cipher. One very old example is the rail-fence transposition cipher. Here a message is spelled out diagonally over a fence. Then, the ciphertext is read of row-by-row. 
\begin{example}\ 
  \begin{lstlisting}
    m e   e a t r t e t g   a t
     e t m   f e   h   o a p r y
  \end{lstlisting}
\end{example}
%Another interesting polyalphabetic cipher is the playfair cipher, probably beyond the scope of this class

In any case, transposition ciphers don't really obfuscate plaintext patterns either, for instance the number of occurrences of the letters remain unchanged by the cipher. 
We can now try to compose ciphers, to make them more secure. These ciphers (resulting from composition of other ciphers) are called product ciphers. If we compose a substitution cipher with a substitution cipher, we obtain a substitution cipher again, leaving the same structural weaknesses. If we compose two transposition ciphers, we will again yield a transposition cipher. If however, we compose a substitution and a transposition, we will yield a much harder cipher. This is the key idea of modern (symmetric) cryptography. 


\section{Fundamental Concepts}
No system is perfectly secure---a brute-force attack is always possible.
Therefore, a central idea of cryptography is that waiting for a polynomial amount of time (in the sense of complexity theory) is feasible but waiting for a super-polynomial amount of time is not.
Then a system is often defined as secure if the best possible attack has super-polynomial complexity.

\begin{definition}[Polynomial and Negligible Functions]
 A function $f:\N\to\R^+$ is called
  \begin{compactitem}
   \item \textbf{polynomial} if $f\in O(p)$ for some polynomial $p$.
   \item \textbf{super-polynomial} if $f\nin O(p)$ for every polynomial $p$ (i.e., if it is not polynomial).
   \item \textbf{negligible} if $f\in O(1/|p(x)|)$ for every polynomial $p:\N\to\R^+$.
  \end{compactitem}
\end{definition}

Super-polynomial functions increase faster than any polynomial increases.
Negligible functions are somewhat dual: they decrease faster than any polynomial increases.
In cryptography, we can think of polynomial/super-polynomial/negligible functions as normal/infeasible/trivial.

\begin{theorem}
We have the following closure properties:
\begin{compactitem}
 \item The sum of super-polynomial/polynomial/negligible functions is super-polynomial/polynomial/negligible again.
 \item The product of a super-polynomial/negligible function with a polynomial function is super-polynomial/negligible again.
 \item A function that is greater/smaller than a super-polynomial/negligible function is super-polynomial/negligible again.
\end{compactitem}
\end{theorem}

\begin{definition}[Polynomial Time]
 An algorithm $A$ is called \textbf{polynomial time} if its worst-case time complexity of $A$ for input of size $n$ is a polynomial function.
\end{definition}

\begin{definition}[Probabilistic Algorithm]
 A \textbf{probabilistic algorithm} is like an algorithm except that it may return different results when called multiple times for the same input.
\end{definition}

\begin{example}[Fermat Primality Test]
Many prime number tests are derived from Fermat's Little Theorem \ref{thm:math:fermatlittle}.
A states that a necessary but not sufficient property for $p$ being prime is that $x^p\Equiv_p x$ for all $a\in \Z$.

To use this as a primality test, we randomly choose $3$ numbers $1<x<p-1$ and test the relation.
If it holds for all three $x$, we return true.

$x^p\Equiv_p x$ is unlikely to hold for random $x$.
So if it succeeds three times, $p$ is probably but not definitely a prime number.

This algorithm is probabilistic in the following sense:
\begin{compactitem}
 \item It needs randomness to work and may return different values for the same input.
 \item Its result is not necessarily correct.
\end{compactitem}

The reason why we choose $1<x<p-1$ is that
\begin{compactitem}
 \item Values outside of $\Z_p$ are irrelevant because we take them modulo $p$ anyway.
 \item The values $x=0$ and $x=1$ are useless because the property anyway holds for them.
 \item The value $x=p-1$ (which is congruent to $-1$) is useless because the property anyway holds if $p$ is odd. (And if $p$ is not odd, we already know it is not a prime.)
\end{compactitem}
\end{example}

In most situations, an algorithm that may return different results for the same input is simply broken and not an algorithm at all.
However, occasionally, probabilistic algorithms are very useful.
Cryptography is an example because we often \emph{need} to make a random choice to prevent an attacker from predicting what we did.
For example, it is often unavoidable for the attacker to know what encryption scheme we use---but that is acceptable if we choose a random key.
Therefore, encryption schemes often consist of a probabilistic key generation algorithm and deterministic encryption/decryption algorithms that take the key as input.
Another application in cryptography is \emph{padding} a message: to prevent leaking the information how long a message is, we can pad every message to a fixed length by adding random data.
In that case, the encryption algorithm is also probabilistic.

Independent of applications in cryptography, probabilistic algorithms are also of great help in situations where (i) a deterministic algorithm has a large complexity but (ii) a probabilistic algorithm that sometimes returns false results is much faster.
If it is easy to test whether a result is correct, we can simply re-run the probabilistic algorithm until it finds a correct result.
That is sometimes used in key generation algorithms, e.g., to find a large prime number.
%Alternatively, if we cannot test if a result is correct, we may be able to run the probabilistic algorithm multiple times in order to increase the probability that a result is correct.

\begin{definition}[Probabilistic Polynomial Time]
 An algorithm is called \textbf{probabilistic polynomial time} (PPT) if it is probabilistic and polynomial.
\end{definition}

One-way-functions (OWF) are functions that are easy to compute but hard to invert:

\newcommand{\Prob}[2]{\mathop{\mathrm{Pr}}_{#1}[#2]}

\begin{notation}
For a function $f:X\to\Bool$, we write $\Prob{x\in X}{f(x)}$ for the probability that $f(x)$ is true when $x$ is chosen uniformly from $X$.
\end{notation}

\begin{definition}[One-Way-Function]
 A polynomial function $f:\Sigma^*\to \Sigma^*$, is a \textbf{one-way-function} if for every probabilistic algorithm $A(n\in\N,y\in \Sigma^*)\to\Sigma^*$, whose complexity is polynomial in $n$, the function
 \[n\tb\mapsto\tb \Prob{x\in\Sigma^n}{f(A(n,f(x)))=f(x)}\]
 is negligible.
\end{definition}
Intuitively, OWFs are super-polynomially hard to invert.
More precisely, any $A$ that attempts to guess (i.e., it is probabilistic) an $x\in\Sigma^n$ such that $y$ and $A(n,y)$ behave the same way under $f$ succeeds with negligible probability.

\begin{example}
It is not known if one-way functions exist.

Some functions that are commonly believed to be OWFs are discrete exponentiation and multiplication.
For example, we do not know if there is a polynomial factoring algorithm.
\end{example}

Pseudo-random-generators (PRG) are functions that can be used to generate numbers that are essentially random:
\begin{definition}[Pseudo-Randomness]
 A polynomial function $R:\Sigma^*\to\Sigma^*$ is a \textbf{pseudo-randomness-generator}
 \begin{compactitem}
   \item $R$ maps input of the same length to output of the same length, i.e., there is an $l:\N\to\N$ such that $R$ maps $\Sigma^n\to\Sigma^{l(n)}$ for every $n$.
   \item for every PPT $A(x\in\Sigma^*)\in\Bool$ the function
  \[n \tb\mapsto \tb \left|\,\Prob{x\in\Sigma^n}{A(R(x))}-\Prob{x\in\Sigma^{l(n)}}{A(x)}\,\right|\]
  is negligible.
 \end{compactitem}
\end{definition}

Intuitively, $R$ is a PRG if it maps $\Sigma^*\to\Sigma^*$ such that no PPT $A$ can tell the difference between $R(x)$ and randomly chosen $x$ with non-negligible probability.
The condition about the length of the output of $R$ is only needed so that we can take the probability on the right-hand side over a finite set.

PRGs and OWFs are intricately linked together:
\begin{theorem}[PRG Theorem]
Given a PRG, we can define an OWF, and vice versa.
\end{theorem}

%%This part should probably be moved to the appendix
%\begin{definition}[hard-core bit]
% Let $f$ be a one-way function. Now the function $b:\{0,1\}^*\to\{0,1\}$ is called a \emph{hard-core bit} for $f$ if $b$ is computable in polynomial time and there exists a negligible function $neg(n)$ such that for any $n\in\N$ and any PPT $A$:
% $$\mathrm{Pr}\left[A(f(x),1)=b(x)\right]\leq\frac{1}{2}+neg(n), $$ where $x\in\{0,1\}$ uniformly random. 
%\end{definition}
%
%The next step is to find one-way functions (assuming that there are one-way functions at all) for which we can build hard-core bits. 
%\begin{theorem}
% Let $f$ be a one-way function. Define $g:\{0,1\}^{2n}\to\{0,1\}^*$ as $g(x\circ y)=f(x)\circ y$, where $\left|x\right|=\left|y\right|=n$. Then $g$ is a one-way function with hard-core bit $$b(x,y)=\bigoplus_{i=1}^n x_i\land y_i=\sum_{i=1}^{n}x_iy_i (\mod 2).$$
%\end{theorem}
%If now $f$ was a one-way permutation, we can use the above construction to get an additional pseudo-random bit, so we already have a pseudo-random generator.


\section{Symmetric Encryption}\label{sec:sd:crypto:sym}
\subsection{General Definition}

%\vdots polynomial time algorithm\\\vdots\ \\probabilistic algorithm\\\vdots\ \\negligible functions\\\vdots \\encryption scheme\\\vdots
\begin{definition}
 An encryption scheme is an ordered triple $(G, E, D)$, where $G$ and $E$ are PPT algorithms and $D$ is a polynomial time algorithm iff for any \emph{security parameter} $n\in\mathbb{N}$:
 \begin{itemize}
  \item The \emph{key generation algorithm} $G$, takes as input $1$ and uses randomness to chose a \emph{key $k$} from a set of possible keys $K_n$ (the key space)
  \item The \emph{encryption algorithm} $E$, takes as input a \emph{message} $m\in P_n$ ($P_n$ is called the \emph{plaintext space} for the security parameter $n$) and uses $k$ to compute an encrypted message $c\in C_n$ ($C_n$ is called the \emph{ciphertext space} for $n$). If $E$ uses the key $k$ on the message $m$ and outputs $s$, we write $E_k(m)=c$.
  \item The \emph{decryption algorithm} $D$, takes an encrypted message and uses the key $k$ to decipher it (deterministically). So we have $$\forall n\in\mathbb{N},\,\forall m\in P_n,\,\forall k \in K_n.\,D_k(E_k(m))=m.$$
 \end{itemize}
\end{definition}

There are various different notions of ``security'' of encryption. We will discuss the notions of %guess indistinguishability (guess ind.), 
computational indistinguishability (comp. ind.) and security against a chosen Plain-text attack (ind. CPA). 
%\begin{definition}[guess indistinguishable]
% We will call an encryption scheme $(E,D,k)$ random indistinguishable iff for any PPT $A$, for any message $m$ of length $n$ and $e_2\in \{0,1\}^n$ chosen uniformly random:
% \[\exists neg.\,\mathrm{Pr}\left[A(e_i)=1\right],\] where $e_1:=E(m_1), i\in\{0,1\}$, uniformly random. 
%\end{definition}
\begin{definition}[guess indistinguishable]
  We will call an encryption scheme $(E,D,k)$ computationally indistinguishable iff for any PPT $A$, for any two messages $m_1, m_2$ of length $n$ and $i\in \{0,1\}$, chosen uniformly random, there is a negligible function $neg.$ s.t.:
  \[(\mathrm{Pr}\left[A(E(m_0))=1\right]-\mathrm{Pr}\left[A(E(m_1))=1\right])\leq \frac{1}{2}+neg(n), \] where $neg(n)$ is a negligible function.
\end{definition}
\begin{definition}[indistinguishability under chosen-plaintext attack (IND\_CPA)]\ \\
  This is a stronger condition than guess indistinguishability. 
  Let again $(E, D, k)$ denote our encryption scheme. 
  Now the attacker $A$ is allowed to query an oracle with arbitrary (finite) many pairs of messages $(p_0, p_1)$ and receives $E(m_i)$, for $i$ uniformly random in $\{0,1\}$. Then $A$ is asked to distinguish the real message from a faked one, given the ciphertext of one of them (chosen uniformly random). If this can only be done with a probability that is negligibly greater than $\frac{1}{2}$, the encryption scheme $(E, D, k)$ is called indistinguishable under chosen-plaintext attack or secure against a chosen-plaintext attack (IND\_CPA secure). 
  
  In other words an encryption scheme $(E, D, k)$ is IND\_CPA secure iff it is computationally indistinguishable against an attacker that can additionally query the above oracle, before having to break the cipher. 
  
  Since the attacker could simply chose to not use the oracle, it follows that any encryption scheme that is IND\_CPA secure is also computationally indistinguishable, so this is indeed a stronger condition. 
\end{definition}
%TODO: Insert the definition of IND_CPA

\subsection{Specific Schemes}

Given a PRG, we can iterate it on its own output to get an arbitrarily long pseudo-random output. Now we can construct an encryption scheme by simply taking the key as input to a pseudo random generator and xoring the message with the output of the pseudo-random generator. The decryption can be done by simply encrypting the ciphertext a second time. The resulting encryption scheme can already be shown to be computational indistinguishable, although not necessarily ind. CPA secure. We will now try to improve the security to also reach at least ind. CPA security. 

\subsubsection{Feistel Ciphers}

For this we can use a so called Feistel network.
%Improving a comp. ind. encryption scheme to an ind. CPA secure scheme using a Feistel-network.
\begin{definition}[A Feistel cipher]
 Let $k$ be any natural number (the number of \emph{rounds}). Let $f_{k_i}$ be a family of functions\protect\footnote{If possible pseudorandom functions and possibly one-way functions.} ($f$ is the so called \emph{round function}) of output length $n$ indexed by the sequence of \emph{round keys} $k_1, k_2, \ldots, k_n$. 
 Then the following encryption algorithm $E_k$ is called Feistel cipher. %network with $k$ iterations (for some odd $k$) based on a pseudo random generator $f_{k_i}$ and round keys $k_1, k_2, \ldots, k_n$ is an encryption scheme $(G,E,D)$, defined by:
 %For any odd $k\in\mathbb{N}$ we call an encryption scheme $(G, E, D)$ a Feistel network with $k$ iterations and $P$-Box $f$, iff for some \emph{round keys} $k_1, k_2, \ldots, k_n$ used as input for the pseudo random generator $f_{k_i}$. 
 \begin{itemize}
  \item Fix a message $m=:x_1\circ x_2$, where $\left|x_1\right|=\left|x_2\right|=n$
  \item Define the sequences $L_1, L_2, \ldots, L_n$ and $R_1, R_2, \ldots, R_n$ by $L_1:=x_1, R_1:=x_2$ and $L_{n+1}:=R_n, R_{n+1}:=L_n\oplus f_{k_n}(R_n)$. Finally define $E_k:x_1\circ x_2\to L_k\circ R_k$. 
 \end{itemize}
 Now we can define the corresponding decryption algorithm $D_k$ just like $e_k$, but with the reversed order of round keys:
 \begin{itemize}
   \item Fix a ciphertext $c=:x_1\circ x_2$, where $\left|x_1\right|=\left|x_2\right|=n$
   \item Define the sequences $L_1, L_2, \ldots, L_n$ and $R_1, R_2, \ldots, R_n$ by $L_1:=x_1, R_1:=x_2$ and $L_{n+1}:=R_n, R_{n+1}:=L_n\oplus f_{k_{k-n}}(R_n)$. Finally define $D_k:x_1\circ x_2\to L_k\circ R_k$. 
 \end{itemize}
\end{definition}
Feistel ciphers have been shown to fulfill several notions of security assuming that the round function is actually pseudo random. For instance Feistel networks with at least $3$ rounds are ind. CPA secure and for more rounds they fulfill even stronger notions of security. %TODO: Check and clearify the exact model (3 seems sufficient under some assumptions, but 4 is definitely better (and already fulfills stronger notions)) and perhaps mention some other results. %see \url{https://link.springer.com/chapter/10.1007/978-3-540-45146-4\_30}


 In practice many symmetric encryption schemes are based either on Feistel networks or on substitution-permutation-networks. 
 \begin{definition}[Substitution-permutation-network]
  A substitution-permutation-networks is a series of linked substitutions (\emph{$S$-Boxes}) and permutations \emph{$P$-Boxes} of blocks in an encryption algorithm. 
  The corresponding encryption algorithm splits the message into several boxes, which are typically fed into the $S$-Boxes, then the result is fed into the $P$-Box (and at some point the round key is used, for instance xored with the result as is AES). The later mentioned AES-encryption algorithm is one example of an encryption algorithm build around a substitution-permutation-network.
 \end{definition}
 Substitution-permutation-network and Feistel networks using $S$-Boxes are quite similar, but there are also some noticeable differences. Ciphers based on substitution-permutation-networks can be better parallelized, but Feistel ciphers can use any pseudo random function (for instance any one-way function) and is therefore limited to invertible ($P$-Boxes). %Also the Feistel networks can be adapted to ciphers not using blocks (for instance it is used in OAEP).
%TODO:Insert some additional material about block modes.

\subsubsection{AES} %Using the wikipedia article
AES is in principle a relatively simple substitution-permutation network based on simple matrix operation. There exist three versions of AES for different key-sizes a 128 -bit version, a 192-bit version and the strongest 256-bit version. They differ mainly by the key size and the number of rounds. They all encrypt 128-bit block of data using a substitution-permutation network.\par
\begin{table}[ht]\centering
  \begin{tabular}{|c|c|c|c|}
  	\hline key size & 128-bit & 192-bit & 256-bit \\ 
  	\hline round number & 10 & 12 & 14 \\ 
  	\hline 
  \end{tabular}
  \caption{AES versions and their number of rounds}
\end{table} 
In the following, let $n$ denote the number of rounds of the AES cipher. 
AES consists of the key expansion, where the key is expanded into 128-bit subkeys that are used in the individual rounds of the substitution-permutation network of AES. The block of data to be encrypted is represented as a \emph{state}, a $4\times 4$ matrix of bytes.  
AES encryption uses four basic operations that are run in every round. They iteratively change the \emph{state} based on the \emph{round key}, the subkey for that specific round. In every round, the following operations are applied in this order (except for the last round, where mix-columns is omitted):
\begin{compactenum}
  \item Sub-bytes: Here every byte of the state is substituted according to the 8-bit S-box. 
  \item Shift-row: Now the bytes in every row of the state are cyclically left-shifted. The $i$-th row is shifted by $i-1$ bytes. 
  \item Mix-columns: This is the most interesting of the permutation steps. This time the columns of the state are transformed. The columns, are interpreted as a column-vector. The entries of this vector (the bits inside the columns) are viewed as polynomials of degree $7$ with bits as coefficients. These polynomials can be added using component-wise xor and can be multiplied using polynomail multiplicatin modulo the polynomial $x^8+x^4+x^3+x+1$. Now the transformation of each column can be represented as a left-multiplication with the matrix $$\left(\begin{matrix}
    2&3&1&1\\1&2&3&1\\1&1&2&3\\3&1&1&2
  \end{matrix}\right).$$
  \item Add-round-key: Finally the state is xored with the round-key   
\end{compactenum}
We can see that AES is a substitution-permutation-network, where the first step is the applied substitution and the second, third and fourth step are permutation, so is their composition. Hence, any round of AES is a substitution followed by a permutation. 

Now we already mostly understand how AES works, except for the key-expansion algorithm. The final algorithm is now: 
\begin{compactenum}
  \item Key expansion: Expand the key to $n+1$ subkeys or round-keys. 
  \item Initial round:  Apply Add-round-key to the state using the first subkey
  \item Rounds: Run $n-1$ rounds on the state, where each $i$-th round consists of:
  \begin{compactenum}
  	\item Sub-bytes
  	\item Shift-row
  	\item Mix-columns
  	\item Add-round-key using the next ($i+1$-th) round-key
  \end{compactenum}
  \item The final round (without mix-columns)
  \begin{compactenum}
  	\item Sub-bytes
  	\item Shift-rows
  	\item Add-round-key using the last $n+1$-th round-key
  \end{compactenum}
\end{compactenum}

Since each of the operations applied to the state vector can be inverted, it is clear how AES encrypted ciphertext can be decrypted again: Firstly, the key is expanded to the round-keys. Then, the rounds are inverted one by one, using the round-keys in inverse order. The rounds can be inverted by applying the inverse operation of the operation in any round in reverse order. In case of add-round-key, the inverse operation is the same operation (xor is inverse to itself). In case mix-columns, the inverse linear transformation must be applied, in case of shift rows, we use a right-shift instead of a left-shift and sub-bytes can be inverted using the inverse substitution table. 

\section{Asymmetric Encryption}\label{sec:sd:crypto:asym}
The basic idea of asymmetric encryption is that different keys are used for encryption and decryption.
An important requirement for security is that the decryption key cannot be computed from the encryption key.

This has the practical advantage that the---not security-critical---encryption key can be made public.

\subsection{Schemes}

We only need to make a minor modification to the Def.~\ref{def:sd:symscheme}:

\begin{definition}[Asymmetric Encryption Scheme]\label{def:sd:asymscheme}
 An \textbf{asymmetric encryption scheme} is an encryption scheme $(\Sigma,K, G, E, D)$, where
  \begin{compactitem}
   \item $K_n=K^e_n\times K^d_n$
   \item if $k=(k^e,k^d)$, then $E_k$ depends only on $k^e$ and $D_k$ depends only $k^d$
  \end{compactitem}
\end{definition}
Intuitively, the keys are pairs of an encryption and a decryption key.

\subsection{Schemes based on Modular Arithmetic}

We fix the alphabet to be $\Sigma=\{0,1\}$ and identify the elements of $\Sigma^n$ and with natural numbers in binary representation.

\subsubsection{Modular Arithmetic as a Cipher}

To encrypt plaintexts from $\Sigma^n$, we proceed in two steps:
\begin{compactenum}
 \item We embed the plaintext into $\Z_N$ for certain $N>2^n$.
 \item We apply a cipher function $\Z_N\to\Z_N$.
\end{compactenum}

$N$ is part of the key.
The number of bits in $N$ is usually seen as the size of the key.
This size must be greater than $n$.

The embedding introduces randomness to make sure that encrypting the same plaintext multiple times yields different ciphertexts.
The cipher function is the main source of security.

\subsubsection{Mode of Operation and Padding}

\paragraph{No Mode of Operation}
In principle, we could use the same modes of operation as for block ciphers.
However, that is not common:
\begin{compactitem}
 \item Asymmetric encryption has the advantage that the encryption key can be made public.
 To exploit that advantage, it is practical not to change the key very often.
 In that case, using the same key for many blocks of many long message increases the chance of attacks.
 \item Asymmetric encryption tends to be more complex than symmetric encryption.
 Therefore, it is not practical to use asymmetric encryption for many long messages.
\end{compactitem}
Instead, it makes more sense to use \textbf{hybrid encryption}: use an asymmetric scheme only to transmit the key for a symmetric scheme.
In that case, it is not necessary to be able to send arbitrarily long messages with the same fixed key.
Instead, a fixed message length suffices as long as it can hold the symmetric key.

\paragraph{Padding}
For the same reason as with symmetric schemes, some randomness must be introduced into the encryption function to make sure that repeatedly mapping the same plaintext yields different ciphertexts.
That is the role of the \textbf{padding}, which embeds the plaintext from $\Z_{2^n}$ into $\Z_N$.

The basic idea is to pick $N>2^{n+i}$, append $i$ $0$s to the plaintext and then apply some cipher $\Z_{2^{n+i}}\to\Z_{2^{n+1}}$ (e.g., a substitution-permutation network with a randomly chosen parameter).

The details of padding are very difficult and subject to active research.
The most important example is Optimal Asymmetric Encryption Padding (OAEP).
Multiple padding schemes based on OAEP have been standardized in RFC 447: Public-Key Cryptography Standards (PKCS) \#1.

\subsubsection{RSA}

\paragraph{Overview}
RSA is a family of cipher functions based on modular arithmetic.

It was developed in the 1970s inspired by ideas by Diffie and Hellman.
It is named after the authors of the algorithm (Rivest, Shamir, Adleman).
A related algorithm was developed earlier by the UK secrete service but remained classified until the 1990s.

The basic ideas is to use $N=p\cdot q$ for large prime numbers $p$ and $q$.
Becuase it is (assumed to be) very difficult to compute $p$ and $q$ from $N$, $p$ and $q$ remain private even if $N$ is public.

\paragraph{Key Generation}
To compute $G(n)$, we first randomly choose two large primes $p$ and $q$ (typically of roughly equal size) such that $p\cdot q>2^{n+i}$ where $i$ is the desired number of padding bits.
We put $N=p\cdot q$.

Secondly, we put $m=(p-1)(q-1)$. (Actually, any common multiple of $p-1$ and $q-1$ is fine.)
Note that $m=\phi(N)$.
Then we pick $e\in \Z_m$ such that $\gcd(e,m)=1$ and compute the $d\in\Z_m$ with $e\cdot d\Equiv_m 1$.
Such a $d$ exists because $\gcd(e,m)=1$ and is easy to compute (see Thm.~\ref{thm:math:extendedeuclid}).

The keys are now defined as follows:
\begin{compactitem}
 \item public key (encryption key): $(N,e)$
 \item private key (decryption key): $(N,d)$
\end{compactitem}
$m$, $p$, and $q$ are not needed for encryption or decryption but must remain private: $p$ (or $q$) is enough to compute $m$ from $N$, and $m$ is enough to compute $d$ from $e$.
So we delete $m$, $p$, and $q$ after generating the key.

Note that to choose $p$ and $q$ efficiently, it is important to have access to a fast primality test.

\paragraph{Encryption and Decryption}
The cipher function and its inverse are the functions $\Z_N\to \Z_N$ defined by
\begin{compactitem}
 \item encryption: $x\mapsto x^e\modop N$
 \item decryption: $x\mapsto x^d\modop N$
\end{compactitem}

Both can be computed efficiently, e.g., using square-and-multiply.

These are indeed inverse to each other:

\begin{theorem}
For all $x\in \Z_N$, we have $(x^d)^e\Equiv_N (x^e)^d \Equiv_N x$.
\end{theorem}
\begin{proof}
In general, because $N=p\cdot q$ for prime numbers $p$ and $q$, we have that $x\Equiv_N y$ iff $x\Equiv_p y$ and $x\Equiv_q y$.

So we have to show that $x^{de}\Equiv_p x$.
(We also have to show the same result for $q$, but the proof is the same.)
We distinguish two cases:
\begin{compactitem}
\item $p|x$: Then trivially $x^{de}\Equiv_p 0\Equiv_p x$.
\item Otherwise. Then $p$ and $x$ are coprime.\\
   By construction of $e$ and $d$ and using Thm.~\ref{thm:math:extendedeuclid}, we have $k\in\Z$ such that $e\cdot d+k\cdot m=1$.
   Using $m=(p-1)(q-1)$, we obtain $x^{de}=x\cdot (x^{p-1})^{-k\cdot(q-1)}$.
   That yields $x^{de}\Equiv_p x$ by using $x^{p-1}\Equiv_p 1$ as known from Thm.~\ref{thm:math:fermatlittle}.
\end{compactitem}
\end{proof}

\paragraph{Attacks}
To break RSA, $d$ has to be computed.
There are $3$ natural ways to do that:
\begin{compactitem}
 \item Factor $N$ into $p$ and $q$. Then compute $d$ easily.
 \item Compute $m$ using $m=\phi(N)$ (which may be easier than finding $p$ and $q$). Then compute $d$ easily.
 \item Find $d$ such that $e\cdot d\Equiv_m 1$ (which may be easier than finding $m$).
\end{compactitem}
Currently these are believed to be equally hard.

It is believed that there is no algorithm for factoring $N$ that is polynomial in the number of bits of $N$.
That is not proved.
There are hypothetical machines (e.g., quantum computers) that can factor $N$ polynomially.

Note that checking if $N$ can be factored (without producing the factors) is polynomial, and practical algorithms exist (in particular, the AKS algorithm).
Incidentally, that is important to find the large prime number $p$ and $q$ efficiently.
\medskip

If there is indeed no polynomial algorithm, factoring relies on brute-force attacks that find all prime numbers $k<\sqrt{N}$ and test $k|N$.
Therefore, larger keys are harder than break to smaller ones.
Because of improving hardware, the key size that is considered secure grows over time.

Keys of size $1024$ are considered secure today, but because security is a relative term, keys of size $2048$ are often recommended. 
%It is quetionable that 1024-bit rsa is really secure, even though it has not yet been publicly broken;
%(see for instance https://en.wikipedia.org/wiki/RSA_%28cryptosystem%29#Integer_factorization_and_RSA_problem or https: //www.schneier.com/blog/archives/2007/05/307digit_number.html).
Larger keys are especially important if data is needed to remain secure far into the future, when faster hardware will be available.


\section{Hashing}\label{sec:sd:crypto:hash}
This topic was covered in a presentation by a student.

\section{Authentication}\label{sec:sd:crypto:auth}
\subsection{Using Asymmetric Encryption}

An important application of asymmetric encryption is authentication.

For example, an agent $A$ who wants to prove her identity can generate a key and publish the public key.
To authenticate $A$, we generate a random string, encrypt it with the public key, and ask for its decryption.
Because only $A$ can decrypt, this is sufficient to authenticate.

Another authentication scheme uses a digital signature.
Here the private key is used for encryption and the public one for decryption. (Note that for RSA the distinction between encryption and decryption key is insubstantial anyway.)
$A$ sends a message together with its encryption, and the recipient decrypts and compares.
A better scheme arises if $A$ first applies a cryptographic hash to the message and sends the original message and the encrypted hash value.

\subsection{Multi-Factor Authentication}

The idea of multi-factor authentication is to use the conjunction of multiple authentication tests.
This is particularly useful if two factors are used that depend on fundamentally different systems: in that case authentication is still safe even if one of the systems is compromised.

The most common application is to authenticate a user via password first, then ask for a second one-time password.
Usually the one-time password is sent to the user as a text message or (in older systems) as a list of one-time passwords on paper by mail.
Even if the user's password is compromised, the authentication remains secure (and can be used to reset the original password).
However, to be truly secure, the user must never type his password on the phone.

%\section{Key Generation and Distribution}
   % security protocol correctness, such as key management protocol correctness,
   % e.g., going back to the work by Burrows, Abadi, and Needham on a logic of authentication (https://doi.org/10.1145/77648.77649)
   % Its an application of logic to a concrete problem space where errors are obviously bad.


   % database security -> peter
   % network security -> jürgen, slides 269-324

   % differential privacy (wikipedia or http://www.cis.upenn.edu/~aaroth/courses/privacyF11.html)

%%  hands-on micro-projects in state-of-the-art tools (e.g., Isabelle for program verification)

% verify formally often after software development
\part{Formal Foundations}

  \chapter{Automata} % one week
  
  \chapter{Functional Programming} % one week
  
  \chapter{Logic} % one week
  % sat, hol, LTL, CTL

\part{Formal Methods for Software Engineering}
  % static program analysis:  control/data flow analysis, abstract interpretation

  \chapter{Model Checking} % two weeks
      % static program analysis:  control/data flow analysis, abstract interpretation

   % cegar, computational tree logic, ...
   % blast, frama-c

   % cegar, computational tree logic, ...
   % blast, frama-c

  \chapter{Program Verification} % two weeks
   % Hoare calculus, dynamic logic, theorem proving

%\part{Summary}
%
%\chapter{}
%  \input{summary}

\part{Appendix}

\appendix

\chapter{Mathematical Preliminaries}\label{sec:math}

\section{Power Sets}\label{sec:math:powerset}

\input{\currfiledir powerset}

\section{Relations and Functions}\label{sec:math:relfun}

\input{\currfiledir relfun}

\section{Binary Relations on a Set}\label{sec:math:binrel}

\input{\currfiledir binrel}

\section{Binary Functions on a Set}\label{sec:math:binop}

\input{\currfiledir binop}

\section{The Integer Numbers}\label{sec:math:int}

\input{\currfiledir integers}

\section{Size of Sets}\label{sec:math:setsize}

\input{\currfiledir setsize}

\section{Important Sets and Functions}\label{sec:math:sets}

\input{\currfiledir sets}


\bibliographystyle{alpha}
\bibliography{../../Program_Data/Latex/bib/rabe,../../Program_Data/Latex/bib/pub_rabe,../../Program_Data/Latex/bib/systems,../../Program_Data/Latex/bib/institutions,../../Program_Data/Latex/bib/historical}

\end{document}