\subsection{General Definition}

%\vdots polynomial time algorithm\\\vdots\ \\probabilistic algorithm\\\vdots\ \\negligible functions\\\vdots \\encryption scheme\\\vdots
\begin{definition}
 An encryption scheme is an ordered triple $(G, E, D)$, where $G$ and $E$ are PPT algorithms and $D$ is a polynomial time algorithm iff for any \emph{security parameter} $n\in\mathbb{N}$:
 \begin{itemize}
  \item The \emph{key generation algorithm} $G$, takes as input $1$ and uses randomness to chose a \emph{key $k$} from a set of possible keys $K_n$ (the key space)
  \item The \emph{encryption algorithm} $E$, takes as input a \emph{message} $m\in P_n$ ($P_n$ is called the \emph{plaintext space} for the security parameter $n$) and uses $k$ to compute an encrypted message $c\in C_n$ ($C_n$ is called the \emph{ciphertext space} for $n$). If $E$ uses the key $k$ on the message $m$ and outputs $s$, we write $E_k(m)=c$.
  \item The \emph{decryption algorithm} $D$, takes an encrypted message and uses the key $k$ to decipher it (deterministically). So we have $$\forall n\in\mathbb{N},\,\forall m\in P_n,\,\forall k \in K_n.\,D_k(E_k(m))=m.$$
 \end{itemize}
\end{definition}

There are various different notions of ``security'' of encryption. We will discuss the notions of %guess indistinguishability (guess ind.), 
computational indistinguishability (comp. ind.) and security against a chosen Plain-text attack (ind. CPA). 
%\begin{definition}[guess indistinguishable]
% We will call an encryption scheme $(E,D,k)$ random indistinguishable iff for any PPT $A$, for any message $m$ of length $n$ and $e_2\in \{0,1\}^n$ chosen uniformly random:
% \[\exists neg.\,\mathrm{Pr}\left[A(e_i)=1\right],\] where $e_1:=E(m_1), i\in\{0,1\}$, uniformly random. 
%\end{definition}
\begin{definition}[guess indistinguishable]
  We will call an encryption scheme $(E,D,k)$ computationally indistinguishable iff for any PPT $A$, for any two messages $m_1, m_2$ of length $n$ and $i\in \{0,1\}$, chosen uniformly random, there is a negligible function $neg.$ s.t.:
  \[(\mathrm{Pr}\left[A(E(m_0))=1\right]-\mathrm{Pr}\left[A(E(m_1))=1\right])\leq \frac{1}{2}+neg(n), \] where $neg(n)$ is a negligible function.
\end{definition}
\begin{definition}[indistinguishability under chosen-plaintext attack (IND\_CPA)]\ \\
  This is a stronger condition than guess indistinguishability. 
  Let again $(E, D, k)$ denote our encryption scheme. 
  Now the attacker $A$ is allowed to query an oracle with arbitrary (finite) many pairs of messages $(p_0, p_1)$ and receives $E(m_i)$, for $i$ uniformly random in $\{0,1\}$. Then $A$ is asked to distinguish the real message from a faked one, given the ciphertext of one of them (chosen uniformly random). If this can only be done with a probability that is negligibly greater than $\frac{1}{2}$, the encryption scheme $(E, D, k)$ is called indistinguishable under chosen-plaintext attack or secure against a chosen-plaintext attack (IND\_CPA secure). 
  
  In other words an encryption scheme $(E, D, k)$ is IND\_CPA secure iff it is computationally indistinguishable against an attacker that can additionally query the above oracle, before having to break the cipher. 
  
  Since the attacker could simply chose to not use the oracle, it follows that any encryption scheme that is IND\_CPA secure is also computationally indistinguishable, so this is indeed a stronger condition. 
\end{definition}
%TODO: Insert the definition of IND_CPA

\subsection{Specific Schemes}

Given a PRG, we can iterate it on its own output to get an arbitrarily long pseudo-random output. Now we can construct an encryption scheme by simply taking the key as input to a pseudo random generator and xoring the message with the output of the pseudo-random generator. The decryption can be done by simply encrypting the ciphertext a second time. The resulting encryption scheme can already be shown to be computational indistinguishable, although not necessarily ind. CPA secure. We will now try to improve the security to also reach at least ind. CPA security. 

\subsubsection{Feistel Ciphers}

For this we can use a so called Feistel network.
%Improving a comp. ind. encryption scheme to an ind. CPA secure scheme using a Feistel-network.
\begin{definition}[A Feistel cipher]
 Let $k$ be any natural number (the number of \emph{rounds}). Let $f_{k_i}$ be a family of functions\protect\footnote{If possible pseudorandom functions and possibly one-way functions.} ($f$ is the so called \emph{round function}) of output length $n$ indexed by the sequence of \emph{round keys} $k_1, k_2, \ldots, k_n$. 
 Then the following encryption algorithm $E_k$ is called Feistel cipher. %network with $k$ iterations (for some odd $k$) based on a pseudo random generator $f_{k_i}$ and round keys $k_1, k_2, \ldots, k_n$ is an encryption scheme $(G,E,D)$, defined by:
 %For any odd $k\in\mathbb{N}$ we call an encryption scheme $(G, E, D)$ a Feistel network with $k$ iterations and $P$-Box $f$, iff for some \emph{round keys} $k_1, k_2, \ldots, k_n$ used as input for the pseudo random generator $f_{k_i}$. 
 \begin{itemize}
  \item Fix a message $m=:x_1\circ x_2$, where $\left|x_1\right|=\left|x_2\right|=n$
  \item Define the sequences $L_1, L_2, \ldots, L_n$ and $R_1, R_2, \ldots, R_n$ by $L_1:=x_1, R_1:=x_2$ and $L_{n+1}:=R_n, R_{n+1}:=L_n\oplus f_{k_n}(R_n)$. Finally define $E_k:x_1\circ x_2\to L_k\circ R_k$. 
 \end{itemize}
 Now we can define the corresponding decryption algorithm $D_k$ just like $e_k$, but with the reversed order of round keys:
 \begin{itemize}
   \item Fix a ciphertext $c=:x_1\circ x_2$, where $\left|x_1\right|=\left|x_2\right|=n$
   \item Define the sequences $L_1, L_2, \ldots, L_n$ and $R_1, R_2, \ldots, R_n$ by $L_1:=x_1, R_1:=x_2$ and $L_{n+1}:=R_n, R_{n+1}:=L_n\oplus f_{k_{k-n}}(R_n)$. Finally define $D_k:x_1\circ x_2\to L_k\circ R_k$. 
 \end{itemize}
\end{definition}
Feistel ciphers have been shown to fulfill several notions of security assuming that the round function is actually pseudo random. For instance Feistel networks with at least $3$ rounds are ind. CPA secure and for more rounds they fulfill even stronger notions of security. %TODO: Check and clearify the exact model (3 seems sufficient under some assumptions, but 4 is definitely better (and already fulfills stronger notions)) and perhaps mention some other results. %see \url{https://link.springer.com/chapter/10.1007/978-3-540-45146-4\_30}


 In practice many symmetric encryption schemes are based either on Feistel networks or on substitution-permutation-networks. 
 \begin{definition}[Substitution-permutation-network]
  A substitution-permutation-networks is a series of linked substitutions (\emph{$S$-Boxes}) and permutations \emph{$P$-Boxes} of blocks in an encryption algorithm. 
  The corresponding encryption algorithm splits the message into several boxes, which are typically fed into the $S$-Boxes, then the result is fed into the $P$-Box (and at some point the round key is used, for instance xored with the result as is AES). The later mentioned AES-encryption algorithm is one example of an encryption algorithm build around a substitution-permutation-network.
 \end{definition}
 Substitution-permutation-network and Feistel networks using $S$-Boxes are quite similar, but there are also some noticeable differences. Ciphers based on substitution-permutation-networks can be better parallelized, but Feistel ciphers can use any pseudo random function (for instance any one-way function) and is therefore limited to invertible ($P$-Boxes). %Also the Feistel networks can be adapted to ciphers not using blocks (for instance it is used in OAEP).
%TODO:Insert some additional material about block modes.

\subsubsection{AES} %Using the wikipedia article
AES is in principle a relatively simple substitution-permutation network based on simple matrix operation. There exist three versions of AES for different key-sizes a 128 -bit version, a 192-bit version and the strongest 256-bit version. They differ mainly by the key size and the number of rounds. They all encrypt 128-bit block of data using a substitution-permutation network.\par
\begin{table}[ht]\centering
  \begin{tabular}{|c|c|c|c|}
  	\hline key size & 128-bit & 192-bit & 256-bit \\ 
  	\hline round number & 10 & 12 & 14 \\ 
  	\hline 
  \end{tabular}
  \caption{AES versions and their number of rounds}
\end{table} 
In the following, let $n$ denote the number of rounds of the AES cipher. 
AES consists of the key expansion, where the key is expanded into 128-bit subkeys that are used in the individual rounds of the substitution-permutation network of AES. The block of data to be encrypted is represented as a \emph{state}, a $4\times 4$ matrix of bytes.  
AES encryption uses four basic operations that are run in every round. They iteratively change the \emph{state} based on the \emph{round key}, the subkey for that specific round. In every round, the following operations are applied in this order (except for the last round, where mix-columns is omitted):
\begin{compactenum}
  \item Sub-bytes: Here every byte of the state is substituted according to the 8-bit S-box. 
  \item Shift-row: Now the bytes in every row of the state are cyclically left-shifted. The $i$-th row is shifted by $i-1$ bytes. 
  \item Mix-columns: This is the most interesting of the permutation steps. This time the columns of the state are transformed. The columns, are interpreted as a column-vector. The entries of this vector (the bits inside the columns) are viewed as polynomials of degree $7$ with bits as coefficients. These polynomials can be added using component-wise xor and can be multiplied using polynomail multiplicatin modulo the polynomial $x^8+x^4+x^3+x+1$. Now the transformation of each column can be represented as a left-multiplication with the matrix $$\left(\begin{matrix}
    2&3&1&1\\1&2&3&1\\1&1&2&3\\3&1&1&2
  \end{matrix}\right).$$
  \item Add-round-key: Finally the state is xored with the round-key   
\end{compactenum}
We can see that AES is a substitution-permutation-network, where the first step is the applied substitution and the second, third and fourth step are permutation, so is their composition. Hence, any round of AES is a substitution followed by a permutation. 

Now we already mostly understand how AES works, except for the key-expansion algorithm. The final algorithm is now: 
\begin{compactenum}
  \item Key expansion: Expand the key to $n+1$ subkeys or round-keys. 
  \item Initial round:  Apply Add-round-key to the state using the first subkey
  \item Rounds: Run $n-1$ rounds on the state, where each $i$-th round consists of:
  \begin{compactenum}
  	\item Sub-bytes
  	\item Shift-row
  	\item Mix-columns
  	\item Add-round-key using the next ($i+1$-th) round-key
  \end{compactenum}
  \item The final round (without mix-columns)
  \begin{compactenum}
  	\item Sub-bytes
  	\item Shift-rows
  	\item Add-round-key using the last $n+1$-th round-key
  \end{compactenum}
\end{compactenum}

Since each of the operations applied to the state vector can be inverted, it is clear how AES encrypted ciphertext can be decrypted again: Firstly, the key is expanded to the round-keys. Then, the rounds are inverted one by one, using the round-keys in inverse order. The rounds can be inverted by applying the inverse operation of the operation in any round in reverse order. In case of add-round-key, the inverse operation is the same operation (xor is inverse to itself). In case mix-columns, the inverse linear transformation must be applied, in case of shift rows, we use a right-shift instead of a left-shift and sub-bytes can be inverted using the inverse substitution table. 