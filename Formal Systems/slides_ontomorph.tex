\section{Ontology Morphisms}

\begin{frame}\frametitle{Idea}
\begin{blockitems}{Intuition of morphism $m$}
\item connects two ontologies, written $m:V\to W$
\item maps $V$-symbols to $W$-expressions
\item extends homomorphically to ma $V$-expressions to $W$-expressions
 \glec{replace every symbol with its assignment}
 \glec{like substitutions for contexts}
\end{blockitems}

\begin{blockitems}{Purpose}
\item extend $V$ with entirely new declarations \\
  special case of $W=V,E$, and identity morphism $V\to W$
\item extend the vocabulary with definitions \\
   special case $m:V,E\to V$, and $m$ maps new symbols to definitions
\item ontology evolution: $V$ is old ontology, $W$ new, $m$ interprets $V$ in $W$
\item transfer legacy content from old to new ontology
\end{blockitems}
\end{frame}

\begin{frame}[fragile]\frametitle{Exercise 2}
We write ontologies for sex and gender.

Write two ontologies for
\begin{itemize}
\item cis-normative world view with just men and women
\item trans-inclusive world view that accommodates sex and gender
\end{itemize}
and relate them with ontology morphisms.
\end{frame}

\begin{frame}\frametitle{Side Note: Knowledge Representation is Apolitical}
Knowledge representation makes no judgment about which ontologies or morphisms are fair, moral, politically correct, etc.

\begin{blockitems}{It can only judge practicality, e.g.,}
\item well-formedness and consistency
\item decidability, efficiency of querying
\item simplicity, e.g., measured by
\begin{itemize}
\item number of declarations or the size of expressions
\item number of axioms about each symbol
\end{itemize}
\item existence and simplicity of morphisms
\end{blockitems}

Languages must allow for expressing whichever knowledge/opinion the user has.

Only users can judge if an ontology is correct.
\end{frame}

\begin{frame}\frametitle{BOL Morphisms Formally}
Syntax: Extend grammar with vocabulary morphisms
\begin{commgrammar}
\gprod{M}{\rep{A}: O\to O}{morphisms}\\
\gprod{A}{i\mapsto I}{individual assignment}\\
\galtprod{c\mapsto C}{concept assignment}\\
\galtprod{r\mapsto R}{relation assignment}\\
\galtprod{p\mapsto P}{property assignment}
\end{commgrammar}

Well-formedness for $M:O\to O'$:
\begin{itemize}
\item one assignment $\ID\mapsto E$ for each declaration $\ID$ of $O$
\item $E$ must be an $O'$-expression of the right kind
 \begin{itemize}
 \item individual symbols to individual expressions
 \item concept symbols to concept expressions
 \item relation symbols to relation expressions
 \item property symbols of type $V$ to property expressions of type $V$
 \item what about assertions and axioms? \glec{see below}
 \end{itemize}
\end{itemize}
\end{frame}

\begin{frame}\frametitle{Homomorphic Extension}
Given morphism $m:O\to O'$, define
\begin{itemize}
\item mapping $\ov{m}$ from $O$-expressions $E$ to $O'$-expressions $\ov{m}(E)$ by
\item replacing every $O$-symbol $s$ in $E$ \\ with the expression $s\mapsto E$ provided by $m$.
\end{itemize}
\lec{Notation: $m(E)$ instead of $\ov{m}(E)$}
\bigskip

Well-defined mapping because morphisms must contain exactly one assignment for every $O$-symbol.
\end{frame}


\begin{frame}\frametitle{BOL Morphisms: What about Axioms?}
A morphism $m:O\to O'$ is well-formed if
\begin{itemize}
\item for every axiom/assertion $F$ in $O$,
\item we have that $m(F)$ is a theorem of $O'$.
\end{itemize}
\bigskip

Theorem: Morphisms preserve truth
\begin{itemize}
\item if $\vdash_O E:E'$ then $\vdash_{O'} m(E):m(E')$
\item if $\vdash_O F$ then $\vdash_{O'} m(F)$
\end{itemize}
\bigskip

Mapping axioms works best if
\begin{itemize}
\item every axiom/assertion has a name
\item new expression kind for proofs \glec{given by derivations of some absolute deductive semantics}
 \glec{axioms = proof symbols = atomic proofs}
\item morphisms contain assignments $a\mapsto P$ of axiom $a$ to proof $P$
\end{itemize}
\end{frame}

\begin{frame}[fragile]\frametitle{Example/Exercise 2}
\begin{blockitems}{Assume $CisNormative$ is BOL vocabulary containing}
\item concepts $\cn{man}$, $\cn{woman}$
\item axioms $\cn{man}\sqcup\cn{woman}\Equiv \top$ and $\cn{man}\sqcap\cn{woman}\Equiv \bot$
\end{blockitems}
\glec{simplified cis-normative world view}

\begin{blockitems}{and $TransFriendly$ contains \hfill\small one way to accommodate transgender people}
\item concepts $\cn{sexmale}$, $\cn{sexfemale}$, $\cn{cis}$, $\cn{trans}$
\item appropriate axioms
\end{blockitems}

\begin{blockitems}{Now have ontology morphism $CisNormative \to TransFriendly$}
\item morphism $\cn{gendermatters}$
 \begin{itemize}
 \item $\cn{man}\mapsto (\cn{sexmale}\sqcap\cn{cis})\sqcup (\cn{sexfemale}\sqcap\cn{trans})$
 \item $\cn{woman}\mapsto (\cn{sexfemale}\sqcap\cn{cis})\sqcup (\cn{sexmale}\sqcap\cn{trans})$
 \end{itemize}
\item alternative morphism $\cn{sexmatters}$
 \begin{itemize}
 \item $\cn{man}\mapsto \cn{sexmale}$
 \item $\cn{woman}\mapsto \cn{sexfemale}$
 \end{itemize}
\end{blockitems}
\end{frame}

\begin{frame}\frametitle{Prevalence of Morphisms}
Deduction
\begin{itemize}
\item algebraic hierarchy, e.g., $Monoid\to Group$
\item theory $\to$ model, e.g., $Group\to Integer$
\end{itemize}
Computation
\begin{itemize}
\item class extension
\item interface implementation
\item type class instances
\item functor
\item API adapters
\end{itemize}
Concrete data
\begin{itemize}
\item between tables: database views
\item between schemas: database migration
\end{itemize}

General: module systems for building large vocabularies
\end{frame}
