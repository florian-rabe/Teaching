\begin{frame}\frametitle{Recall: Ontologies}
\begin{blockitems}{Main ideas}
\item Ontology abstractly describes concepts and relations
\item Tool maintains concrete data set
\item Focus on efficiently
  \begin{itemize}
  \item identifying (i.e., assign names)
  \item representing
  \item processing
  \item querying
  \end{itemize}
  large sets of concrete data
\end{blockitems}

\begin{blockitems}{Recall: TBox-ABox distinction}
  \item TBox: general parts, abstract, fixed
   \lec{main challenge: correct modeling of domain}
  \item ABox: concrete individuals and assertions about them, growing
   \lec{main challenge: aggregate them all}
\end{blockitems}
\end{frame}

\begin{frame}\frametitle{Concrete Data}
\begin{blockitems}{Concrete is}
\item Base values: integers, strings, booleans, etc.
\item Collections: sets, multisets, lists (always finite)
\item Aggregations: tuples, records (always finite)
\item User-defined concrete data: enumerations, inductive types
\item Advanced objects: finite maps, graphs, etc.
\end{blockitems}

\begin{blockitems}{Concrete is not}
\item Uninterpreted symbols
\item Variables (free or bound)
 \glec{$\lambda$-abstraction, quantification}
\item Symbolic expressions
 \lec{formulas, algorithms}
% Exceptions:
%  \begin{itemize}
%  \item expressions of inductive type
%  \item application of built-in functions
%  \item queries that return concrete data
%  \end{itemize}
\end{blockitems}
\end{frame}

\begin{frame}\frametitle{Two Approaches to Representing Concrete Data}
\begin{blockitems}{\emph{Curry}-typed ontology languages (e.g., BOL, OWL)}
\item Representation based on \emph{knowledge graph}
\item Ontology written in BOL-like language
\item Data maintained as \emph{set of triples}
  \glec{tool = triple store}
\item Typical language/tool design
 \begin{itemize}
 \item ontology and query language \emph{separate}
  \glec{e.g., OWL, SPARQL}
 \item triple store and query engine integrated
  \glec{e.g., Virtuoso tool}
 \end{itemize}
\end{blockitems}

\begin{blockitems}{\emph{Church}-typed languages (e.g., SQL, UML)}
\item Representation based on \emph{abstract data types}
\item Ontology written as set of related ADTs \glec{SQL database schema}
\item Data maintained as \emph{tables}
  \glec{tool = (relational) database}
\item Typical language/tool design
 \begin{itemize}
 \item ontology and query language \emph{integrated}
  \glec{e.g., SQL}
 \item table store and query engine integrated
  \glec{e.g., SQLite tool}
 \end{itemize}
\end{blockitems}
\end{frame}

\begin{frame}\frametitle{Evolution of Approaches}
\begin{blockitems}{Our usage is non-standard}
 \item Common
  \begin{itemize}
  \item ontologies = untyped approach, OWL, triples, SPARQL
  \item relational databases = typed approach, tables, SQL
  \end{itemize}
 \item Our understanding: two approaches evolved from same idea
	\begin{itemize}
	\item ontolgoies = Curry-typed ontology + data store
	\item relational database = Church-typed ontology + data store
	\end{itemize}
\end{blockitems}

\begin{blockitems}{Evolution}
\item Typed-untyped distinction minor technical difference
\item Optimization of respective advantages causes speciation
\item Today segregation into different
 \begin{itemize}
 \item jargons
 \item languages, tools
 \item communities, conferences
 \item courses
 \end{itemize}
\end{blockitems}
\end{frame}

\begin{frame}\frametitle{Data structures for Curry-typed concrete data}
\begin{blockitems}{Central data structure = knowledge graph}
\item nodes = individuals $i$
 \begin{itemize}
 \item identifier
 \item sets of concepts of $i$
 \item key-value sets of properties of $i$
 \end{itemize}
\item edges = relation assertions
 \begin{itemize}
 \item from subject to object
 \item labeled with name of relation
 \end{itemize}
\end{blockitems}

\begin{blockitems}{Processing strengths}
\item store: as triple set
\item edit: Protege-style or graph-based
\item visualize: as graph
  \glec{different colors for concepts, relations}
\item query: match, traverse graph structure
\item untyped data simplifies integration, migration
\end{blockitems}
\end{frame}

\begin{frame}\frametitle{Data structures for Church-typed concrete data}
\begin{blockitems}{Central data structure = relational database}
\item tables = abstract data type
\item rows = objects of that type
\item columns = fields of ADT
\item cells = values of fields
\end{blockitems}

\begin{blockitems}{Processing strengths}
\item store: as CSV text files, or similar
\item edit: SQL commands or table editors
\item visualize: as table view
\item query: relational algebra
\item typed data simplifies selecting, sorting, aggregating
\end{blockitems}
\end{frame}

\begin{frame}\frametitle{Identifiers}
\begin{blockitems}{Curry-Typed Knowledge graph}
\item concept, relation, property names given in TBox
\item individual names attached to nodes
\end{blockitems}

\begin{blockitems}{Church-Typed Database}
\item table, column names given in schema
\item row identified by distinguished column (= key) \\
options
 \begin{itemize}
 \item preexistent characteristic column
 \item added upon insertion
  \begin{itemize}
  \item UUID string
  \item incremental integers
  \item concatenation of characteristic list of columns
  \end{itemize} 
 \end{itemize}
\item column/row identifiers formed by qualifying with table name
\end{blockitems}
\end{frame}

\begin{frame}\frametitle{Axioms}
\begin{blockitems}{Curry-Typed Knowledge Graph}
\item traditionally very expressive axioms
\item yields inferred assertions
\item triple store must do consequence closure to return correct query results
\item not all axioms supported by every triple store
\end{blockitems}

\begin{blockitems}{Church-Typed Database}
\item typically no axioms
\item instead consistency constraints, triggers
\item allows limited support for axioms without calling it that way
\item stronger need for users to program the consequence closure manually
\end{blockitems}
\end{frame}

\begin{frame}\frametitle{Open/Closed World}
\begin{itemize}
\item Question: is the data complete?
 \begin{itemize}
 \item closed world: yes
 \item open world: not necessarily
 \end{itemize}
\item Dimensions of openness
 \begin{itemize}
  \item existence of individual objects
  \item assertions about them
 \end{itemize}
\item Sources of openness
  \begin{itemize}
  \item more exists but has not yet been added
  \item more could be created later
  \end{itemize}
\item Orthogonal to typed/untyped distinction, but in practice
 \begin{itemize}
 \item knowledge graphs use open world
 \item databases use closed world
 \end{itemize}
\end{itemize}
\lec{Open world is natural state, closing adds knowledge}
\end{frame}

\begin{frame}\frametitle{Closing the World}
\begin{blockitems}{Derivable consequences}
 \item induction: prove universal property by proving for each object
 \item negation by failure: atomic property false if not provable
 \item term-generation constraint: only nameable objects exist
\end{blockitems}

\begin{blockitems}{Enabled operations}
 \item universal set: all objects
 \item complement of concept/type
 \item defaults: assume default value for property if not otherwise asserted
\end{blockitems}

\begin{blockitems}{Monotonicity problem}
 \item monotone operation: bigger world = more results
 \item examples: union, intersection, $\exists R.C$, join, IN conditions
 \item counter-examples: complement, $\forall R.C$, NOT IN conditions
\end{blockitems}
\lec{technically, non-monotone operations in open world dubious}
\end{frame}

\begin{frame}\frametitle{Summary}
\begin{tabular}{l|ll}
  & semantic web & relational databases \\
\hline
ontology aspect & TBox of ontology & SQL schema \\
conceptual model & knowledge graph & set of tables \\
concrete data aspect & ABox of ontology & SQL database \\
concrete data storage & set of triples & set of rows of the tables \\
concrete data formats & RDF & CSV \\
concrete data tool & triple store & database implementation \\
typing & soft/Curry & hard/Church\\
query language & SPARQL & SQL SELECT query \\
openness of world & tends to be open & tends to be closed \\
\end{tabular}
\end{frame}


\begin{frame}\frametitle{Exercise 9}
Absolute semantics:
\begin{itemize}
\item Via concrete data: Export your ontology to a triple store like Virtuoso and run a concrete query in SPARQL.
\item Via deduction: Export your ontology in OWL format to a reasoner like FaCT++ and prove a theorem.
Potentially, do this by installing a plugin for a reasoner in your ontology IDE.
\end{itemize}

Relative semantics:
\begin{itemize}
\item Via deduction: finish exercise 8 and prove a theorem via an SFOL theorem prover.
\end{itemize}
\end{frame}